\chapter{Esercizi}
\label{ch:esercizi_settimana_4}

\begin{esercizio}
    Sia $G$ un gruppo di ordine 45.
    Dimostrare che $G$ è abeliano.
\end{esercizio}
\begin{soluzione}
    \begin{equation*}
        P \in \Syl_5(G) \allora \ordine{P} = 5
    \end{equation*}

    $P$ è abeliano e ciclico (vedi~\ref{cor:gruppo_ordine_primo}).

    \begin{equation*}
        Q \in \Syl_3(G) \allora \ordine{Q} = 9
    \end{equation*}

    $Q$ è abeliano, perché tutti i gruppi di ordine $p^2$ sono abeliani.

    $P$ e $Q$ hanno ordine coprimo, quindi la loro intersezione contiene solo l'identità
    (vedi~\ref{cor:sottogruppi_ordini_coprimi}).

    \begin{equation*}
        \ordine{PQ} = \dfrac{\ordine{P} \cdot \ordine{Q}}{\ordine{P \cap Q}} = \dfrac{5 \cdot 9}{1} = 45 \allora G = PQ
    \end{equation*}

    Ma non posso concludere che $G$ sia abeliano, in quanto non ho certezza che gli elementi in $P$ commutino con tutti
    gli elementi in $Q$.
    Ci serve qualche altra idea.

    Per il secondo teorema di Sylow sappiamo che:
    \begin{gather*}
        \begin{cases}
            n_5(G) \equiv 1 \mod 5 \\
            n_5(G) \divisore 9
        \end{cases}
        \allora n_5(G) = 1
    \end{gather*}

    Quindi $P$ è l'unico 5-Sylow di $G$, quindi $P$ è un sottogruppo normale.

    Possiamo ora far agire $G$ per coniugio su se stesso.
    Ma, dal momento che $P$ è un sottogruppo normale, se coniugo un elemento $x \in P$ con un qualunque elemento $g \in G$,
    il coniugato $g x g^{-1}$ sta ancora in $P$.
    Quindi posso far agire $G$ per coniugio su $P$.
    Esiste quindi un omomorfismo $\rho$ tale che:
    \begin{align*}
        \rho: G &\longrightarrow \Aut(P) \\
         &\longmapsto (x \overset{\rho_g}{\longmapsto} gxg^{-1})
    \end{align*}

    Il nucleo di questo omomorfismo è dato dal centralizzante di $P$ in $G$:
    \begin{equation*}
        \ker \rho = C_G(P) = \{g \taleche gx = xg \,\forall x \in P\}
    \end{equation*}

    $P$ è un sottogruppo ciclico di ordine 5, quindi $P \isomorfo C_5$.
    Quindi anche gli automorfismo di $P$ sono isomorfi agli automorfismi di $C_5$.
    Per il teorema~\ref{thr:automorfismi_gruppi_ciclici}:
    \begin{equation*}
        \ordine{\Aut P} = \ordine{\Aut C_5} = \phi(5) = 4
    \end{equation*}

    Inoltre, per il corollario~\ref{crl:Omomorfismi_fondamentale_1} al teorema fondamentale degli omomorfismi,
    il gruppo quoziente $G/C_G(P)$ è isomorfo al sottogruppo di $\Aut P$ immagine di $\rho$;
    quindi l'ordine del gruppo quoziente $G/C_G(P)$ deve essere un divisore di 4.

    Lo stesso $G/C_G(P)$ è un gruppo quoziente di $G$, quindi il suo ordine deve dividere l'ordine di $G$, che è 45.

    Ma 4 e 45 sono coprimi, quindi l'ordine del gruppo quoziente $G/C_G(P)$ è 1, quindi $G = C_G(P)$.

    Quindi tutti gli elementi di $G$ commutano con $P$, in particolare tutti gli elementi di $Q$ commutano con $P$.
    Tenendo conto del fatto che anche $Q$ è abeliano, possiamo concludere che $G$ è abeliano.
\end{soluzione}

\begin{esercizio}
    Siano dati $\alpha = (123)$ e $\beta = (147)(258)(369)$, con $\alpha, \beta \in S_9$.

    Studiare il gruppo $G = \gen{\alpha, \beta} \le S_9$, in particolare studiare l'ordine, l'esponente, il centro
    e il numero di classi di coniugio.
\end{esercizio}
\begin{soluzione}
    \textbf{Ordine:}
    Definisco:
    \begin{align*}
        \alpha_1 &= \alpha = (123) \\
        \alpha_2 &= \beta \alpha_1 \beta^{-1} = (456) \\
        \alpha_3 &= \beta \alpha_2 \beta^{-1} = (789) \\
        \alpha_4 &= \beta \alpha_3 \beta^{-1} = (123) = \alpha
    \end{align*}

    Consideriamo il sottogruppo:
    \begin{equation*}
        A = \gen{\alpha_1, \alpha_2, \alpha_3}
    \end{equation*}

    $A$ è normale in $G$: basta far vedere che i generatori di $G$ lo normalizzano.
    Se infatti ci sono i generatori di $G$ $\alpha$ e $\beta$ nel normalizzante di $A$, allora c'è tutto $G$.

    $\alpha$ è nel normalizzante, perché $\alpha = \alpha_1 \in A \le N_G(A)$.

    $\beta$ è nel normalizzante, perché quando coniugo con $\beta$, giro tra loro i generatori di $A$.

    Quindi:
    \begin{equation*}
        A \normale N_G(A) = G
    \end{equation*}

    Visto che $A$ è normale, allora:
    \begin{equation*}
        G = A\gen{\beta}
    \end{equation*}

    I generatori di $A$ sono dei cicli disgiunti, quindi $\alpha_1, \alpha_2, \alpha_3$ commutano tra di loro.
    Quindi:
    \begin{gather*}
        A \isomorfo C_3 \times C_3 \times C_3 \\
        A = \{\alpha_1^{n_1}\alpha_2^{n_2}\alpha_3^{n_3} \taleche 0 \le n_i \le 2\} \\
        \ordine{A} = 27
    \end{gather*}

    $\beta \not\in A$ in quando non è del tipo di ciclo degli elementi di $A$.
    Quindi:
    \begin{equation*}
        \ordine{A \cap \gen{\beta}} = 1
    \end{equation*}

    Possiamo ora calcolare l'ordine di $G$:
    \begin{equation*}
        \ordine{G} = \dfrac{\ordine{A} \cdot \ordine{\gen{\beta}}}{\ordine{A \cap \gen{\beta}}} = \dfrac{27 \cdot 3}{1} = 81
    \end{equation*}

    \bigskip
    \textbf{Centro:}

    Possiamo calcolare il centro di $G$ intersecando il centralizzante di $\alpha$ e il centralizzante di $\beta$:
    \begin{equation*}
        Z(G) = C_G(\alpha) \cap C_G(\beta)
    \end{equation*}

    Il $C_G(\alpha)$ contiene $A$, in quando $\alpha \in A$ e $A$ è abeliano.
    L'indice di $A$ è $\indice{G}{A} = 3$, quindi $C_G(\alpha)$ o è $A$ o è $G$.

    Se fosse $C_G(\alpha) = G$ allora $\alpha \in Z(G)$ ma non è vero perché $\alpha$ non commuta con $\beta$.

    Quindi il $C_G(\alpha) = A$.

    Determiniamo ora quale elementi di $A$ commutano con $\beta$.
    Prendiamo il generico elemento $a = \alpha_1^{n_1}\alpha_2^{n_2}\alpha_3^{n_3} \in A$ e vediamo quando commuta con
    $\beta$:
    \begin{gather*}
        a \beta = \beta a \sse a = \beta a \beta^{-1} \\
        a = \alpha_1^{n_1}\alpha_2^{n_2}\alpha_3^{n_3} = \beta a \beta^{-1} = \alpha_2^{n_1}\alpha_3^{n_2}\alpha_1^{n_3} \\
        \allora n_1 = n_2 = n_3
    \end{gather*}

    Quindi:
    \begin{equation*}
        Z(G) = \gen{\alpha_1\alpha_2\alpha_3}
    \end{equation*}

    Ovvero il centro è un gruppo ciclico di ordine 3.

    \bigskip
    \textbf{Esponente:}

    Il sottogruppo $A$ contiene elementi di ordine 1 e di ordine 3, quindi $\exp(A) = 3$.

    Dal momento che $\indice{G}{A} = 3$ sappiamo che $G/A \isomorfo C_3$.
    Quindi se prendiamo un generico elemento $g \in G$ abbiamo:
    \begin{equation*}
        g^3 \in A \allora (g^3)^3 = 1
    \end{equation*}

    Ogni elemento elevato alla 9 è l'identità.
    Quindi l'esponente è 3 o 9.

    Verifichiamo se ci sono elementi di ordine 9.
    Prendiamo per esempio l'elemento $(\alpha\beta)^3$:
    \begin{align*}
    (\alpha\beta)^3 &= \alpha\beta \alpha\beta \alpha\beta = \\
        &= \alpha\beta \alpha\beta (\beta \beta^{-1}) \alpha\beta = \\
        &= \alpha\beta \alpha\beta^2(\beta^{-1} \alpha\beta) = \\
        &= \alpha\beta (\beta^2\beta^{-2}) \alpha \beta^2(\beta^{-1} \alpha\beta) = \\
        &= \alpha\beta^3(\beta^{-2} \alpha \beta^2)(\beta^{-1} \alpha\beta) = \\
        &= \alpha_1 1 \alpha_3 \alpha_2 = \\
        &\ne 1
    \end{align*}

    Quindi $(\alpha\beta)^3$ ha ordine 3, ovvero $\alpha\beta$ ha ordine 9.

    L'esponente $\exp(G) = 9$.

    \bigskip
    \textbf{Classi di coniugio:}

    Prendiamo un generico $a \in A$.
    Dal momento che $A \le C_G(a)$ allora l'ordine del centralizzante di $a$ è 27 o 81.

    Il centralizzante $C_G(a)$ ha ordine 81 quando a commuta con tutti gli elementi di $G$, ovvero $a \in Z(G)$.
    Questo succede esattamente 3 volte perché $Z(G) = \gen{\alpha_1\alpha_2\alpha_3}$.

    In tutti gli altri casi il centralizzante $C_G(a)$ ha ordine 27.
    Quindi questi elementi $a$ hanno 3 coniugati, 3 come l'indice del centralizzante.

    Ora, $A$ contiene 27 elementi.

    3 elementi di $A$ sono in $Z(G)$, quindi abbiamo 3 classi di coniugio di cardinalità 1.

    Gli altri 24 elementi di $A$ hanno ciascuno 3 coniugati.
    \textbf{Le classi di coniugio di $A$ non escono da $A$, perché $A$ è normale.}
    Quindi abbiamo 8 classi di cardinalità 3.

    Ci sono 54 elementi appartenenti a $G - A$.
    Questo elementi sono del tipo $a\beta, a\beta^{-1}$ con $a \in A$.

    Determiniamo il centralizzante $C_G(a\beta)$.
    Tale centralizzante contiene l'elemento $a\beta$ perché \textbf{ogni elemento centralizza se stesso}.

    Se moltiplico il sottogruppo $A$ per il sottogruppo $\gen{a\beta}$ ottengo l'intero gruppo $G$:
    \begin{equation*}
        G = A\gen{a\beta}
    \end{equation*}

    Per cui un generico elemento $g \in G$ lo posso scrivere come:
    \begin{equation*}
        g = a^* (a\beta)^i \quad \text{con } a^* \in A, i \in \Z
    \end{equation*}

    Quindi abbiamo che:
    \begin{equation*}
        g \in C_G(a\beta) \sse a^* \in C_G(a\beta)
    \end{equation*}

    Infatti:
    \begin{align*}
        g \in C_G(a\beta) &\sse g a \beta = a \beta g \\
        &\sse a^* (a\beta)^i a \beta = a \beta a^* (a\beta)^i \\
        &\sse a^* a\beta = a\beta a^* \\
        &\sse a^* \in C_G(a \beta)
    \end{align*}

    Quindi abbiamo anche che:
    \begin{equation*}
        C_G(a\beta) = \gen{a\beta}C_A(a\beta)
    \end{equation*}

    Infatti:
    \begin{align*}
        C_G(a\beta) &= \{g \taleche ga\beta = a \beta g\} \\
        &= \{a^* (a\beta)^i \taleche a^* (a\beta)^i a \beta = a\beta a^* (a\beta)^i \} \\
        &= \{a^* (a\beta)^i \taleche a^* a \beta = a\beta a^* \} \\
        &= \gen{a\beta} \{a^* \taleche a^* a\beta = a\beta a^* \} \\
        &= \gen{a\beta} C_A(a\beta)
    \end{align*}

    $a \in C_A(a\beta)$ perché commuta com tutto quello che sta in $A$, visto che A è abeliano.
    Quindi possiamo semplificare e dire:
    \begin{equation*}
        C_G(a\beta) = \gen{a\beta} C_A(\beta)
    \end{equation*}

    Ma il centralizzatore $C_A(\beta)$ l'abbiamo calcolato prima: coincide con il centro di $G$:
    \begin{equation*}
        C_G(a\beta) = \gen{a\beta} Z(G)
    \end{equation*}

    Per cui potremo calcolare la cardinalità dei centralizzanti come:
    \begin{equation*}
        \ordine{C_G(a\beta)} = \dfrac{\ordine{a\beta} \cdot \ordine{Z(G)}}{\ordine{\gen{a\beta} \cap Z(G)}}
    \end{equation*}

    Tuttavia di queste cardinalità conosco solo quella del centro $Z(G)$.
    Non conosco l'ordine di $a\beta$: a volte ha ordine 3, a volte ha ordine 9;
    infatti se moltiplico $\alpha\beta$ ottengo un elemento di ordine 9, mentre se moltiplico
    $\alpha_1\alpha_2\alpha3 \cdot \beta$ ottengo un elemento di ordine 3.

    Consideriamo il caso in cui $\ordine{a\beta} = 3$.
    Il sottogruppo $a\beta$ e il sottogruppo $Z(G)$ sono entrambi di ordine 3, e non coincidono,
    perché $Z(G) = \gen{\alpha_1\alpha_2\alpha_3}$ con $\alpha_1\alpha_2\alpha_3 \in A$, mentre $a\beta \not\in A$.
    Quindi la loro intersezione contiene solo l'identità.
    Per cui:
    \begin{equation*}
        \ordine{C_G(a\beta)} = \dfrac{\ordine{a\beta} \cdot \ordine{Z(G)}}{\ordine{\gen{a\beta} \cap Z(G)}} =
        \dfrac{3 \cdot 3}{1} = 9
    \end{equation*}

    Consideriamo il caso in cui $\ordine{a\beta} = 9$.
    Abbiamo visto che un generico elemento di $G$ elevato alla terza appartiene ad $A$: $(a\beta)^3 \in A$.
    Inoltre ogni elemento di $A$ ha ordine 3: $\ordine{(a\beta)^3 = 3}$.
    Inoltre $(a\beta)^3$ commuta con $a\beta$ perché ogni elemento commuta con le proprie potenze.
    Ma allora $(a\beta)^3$ è un elemento di $A$ che commuta con $a\beta$.
    Quindi $(a\beta)^3 \in Z(G)$ e ci sono 3 elementi di $\gen{a\beta}$ appartenenti anche a $Z(G)$.
    Per cui:
    \begin{equation*}
        \ordine{C_G(a\beta)} = \dfrac{\ordine{a\beta} \cdot \ordine{Z(G)}}{\ordine{\gen{a\beta} \cap Z(G)}} =
        \dfrac{9 \cdot 3}{3} = 9
    \end{equation*}

    Sia che $a\beta$ abbia ordine 3, sia che abbia ordine 9, ha sempre 9 coniugati.
    Quindi ho 6 classi di coniugio formate da 9 elementi.
\end{soluzione}