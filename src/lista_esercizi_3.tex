\chapter{Esercizi}
\label{ch:esercizi_settimana_4}

\begin{esercizio}
    Sia $G$ un gruppo di ordine 45.
    Dimostrare che $G$ è abeliano.
\end{esercizio}
\begin{soluzione}
    \begin{equation*}
        P \in \Syl_5(G) \allora \ordine{P} = 5
    \end{equation*}

    $P$ è abeliano e ciclico (vedi~\ref{cor:gruppo_ordine_primo}).

    \begin{equation*}
        Q \in \Syl_3(G) \allora \ordine{Q} = 9
    \end{equation*}

    $Q$ è abeliano, perché tutti i gruppi di ordine $p^2$ sono abeliani.

    $P$ e $Q$ hanno ordine coprimo, quindi la loro intersezione contiene solo l'identità
    (vedi~\ref{cor:sottogruppi_ordini_coprimi}).

    \begin{equation*}
        \ordine{PQ} = \dfrac{\ordine{P} \cdot \ordine{Q}}{\ordine{P \cap Q}} = \dfrac{5 \cdot 9}{1} = 45 \allora G = PQ
    \end{equation*}

    Ma non posso concludere che $G$ sia abeliano, in quanto non ho certezza che gli elementi in $P$ commutino con tutti
    gli elementi in $Q$.
\end{soluzione}