\chapter{Esercizi}
\label{ch:esercizi_settimana_4}

\begin{esercizio}
    Sia $G$ un gruppo di ordine 45.
    Dimostrare che $G$ è abeliano.
\end{esercizio}
\begin{soluzione}
    \begin{equation*}
        P \in \Syl_5(G) \allora \ordine{P} = 5
    \end{equation*}

    $P$ è abeliano e ciclico (vedi~\ref{cor:gruppo_ordine_primo}).

    \begin{equation*}
        Q \in \Syl_3(G) \allora \ordine{Q} = 9
    \end{equation*}

    $Q$ è abeliano, perché tutti i gruppi di ordine $p^2$ sono abeliani.

    $P$ e $Q$ hanno ordine coprimo, quindi la loro intersezione contiene solo l'identità
    (vedi~\ref{cor:sottogruppi_ordini_coprimi}).

    \begin{equation*}
        \ordine{PQ} = \dfrac{\ordine{P} \cdot \ordine{Q}}{\ordine{P \cap Q}} = \dfrac{5 \cdot 9}{1} = 45 \allora G = PQ
    \end{equation*}

    Ma non posso concludere che $G$ sia abeliano, in quanto non ho certezza che gli elementi in $P$ commutino con tutti
    gli elementi in $Q$.
    Ci serve qualche altra idea.

    Per il secondo teorema di Sylow sappiamo che:
    \begin{gather*}
        \begin{cases}
            n_5(G) \equiv 1 \mod 5 \\
            n_5(G) \divisore 9
        \end{cases}
        \allora n_5(G) = 1
    \end{gather*}

    Quindi $P$ è l'unico 5-Sylow di $G$, quindi $P$ è un sottogruppo normale.

    Possiamo ora far agire $G$ per coniugio su se stesso.
    Ma, dal momento che $P$ è un sottogruppo normale, se coniugo un elemento $x \in P$ con un qualunque elemento $g \in G$,
    il coniugato $g x g^{-1}$ sta ancora in $P$.
    Quindi posso far agire $G$ per coniugio su $P$.
    Esiste quindi un omomorfismo $\rho$ tale che:
    \begin{align*}
        \rho: G &\longrightarrow \Aut(P) \\
         &\longmapsto (x \overset{\rho_g}{\longmapsto} gxg^{-1})
    \end{align*}

    Il nucleo di questo omomorfismo è dato dal centralizzante di $P$ in $G$:
    \begin{equation*}
        \ker \rho = C_G(P) = \{g \taleche gx = xg \,\forall x \in P\}
    \end{equation*}

    $P$ è un sottogruppo ciclico di ordine 5, quindi $P \isomorfo C_5$.
    Quindi anche gli automorfismo di $P$ sono isomorfi agli automorfismi di $C_5$.
    Per il teorema~\ref{thr:automorfismi_gruppi_ciclici}:
    \begin{equation*}
        \ordine{\Aut P} = \ordine{\Aut C_5} = \phi(5) = 4
    \end{equation*}

    Inoltre, per il corollario~\ref{crl:Omomorfismi_fondamentale_1} al teorema fondamentale degli omomorfismi,
    il gruppo quoziente $G/C_G(P)$ è isomorfo al sottogruppo di $\Aut P$ immagine di $\rho$;
    quindi l'ordine del gruppo quoziente $G/C_G(P)$ deve essere un divisore di 4.

    Lo stesso $G/C_G(P)$ è un gruppo quoziente di $G$, quindi il suo ordine deve dividere l'ordine di $G$, che è 45.

    Ma 4 e 45 sono coprimi, quindi l'ordine del gruppo quoziente $G/C_G(P)$ è 1, quindi $G = C_G(P)$.

    Quindi tutti gli elementi di $G$ commutano con $P$, in particolare tutti gli elementi di $Q$ commutano con $P$.
    Tenendo conto del fatto che anche $Q$ è abeliano, possiamo concludere che $G$ è abeliano.
\end{soluzione}