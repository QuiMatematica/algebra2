\chapter{Sintesi sui prodotti}

Sia $G$ un gruppo; $H$ e $K$ due sottogruppi di $G$.

\section{Sottogruppo generato da $H$ e $K$}

$\gen{H, K}$ è il sottogruppo generato da $H$ e $K$, ovvero il più piccolo sottogruppo che contiene $H$ e $K$ (figura~\ref{fig:sottogruppo_generato_da_h_e_k}).

\begin{figure}[tp]
	\centering
	\tikz {
		\node (a) at (2,6) {$G$};
		\node (f) at (2,4) {$\gen{H, K}$};
		\node (b) at (0,2) {$H$};
		\node (d) at (4,2) {$K$};
		\draw (a) edge[->] (f);
		\draw (f) edge[->] (b);
		\draw (f) edge[->] (d);
	}
	\caption{Sottogruppo generato da $H$ e $K$.}
	\label{fig:sottogruppo_generato_da_h_e_k}
\end{figure} 

\section{Prodotto di sottogruppi}

\begin{equation}
	HK = \{hk \taleche h \in H \land k \in K\}
\end{equation}

$HK$ è l'insieme degli elementi di $G$ che sono esprimibili come prodotto di un elemento di $H$ per un elemento di $K$.

\begin{teorema}
	Se $H$ e $K$ sono finiti, allora:
	
	\begin{equation}
		\ordine{HK} = \dfrac{\ordine{H}\ordine{K}}{\ordine{H \cap K}}
	\end{equation}
\end{teorema}

$HK$ non è sempre un sottogruppo.

Per esempio:

\begin{gather}
	G = S_3 \\
	H = \gen{(12)} = \{1, (12)\}
	K = \gen{(13)} = \{1, (13)\} 
\end{gather}

L'insieme prodotto contiene:

\begin{gather}
	1 \cdot 1 = 1 \\
	1 \cdot (13) = (13) \\
	(12) \cdot 1 = (12) \\
	(12) \cdot (13) = (132)
\end{gather}

Quindi $\ordine{HK} = 4$, ma 4 non divide $6 = \ordine{S_3}$ quindi per il teorema di Lagrange~\ref{thr:Laterali_Lagrange} $HK$ non può essere un sottogruppo di $S_3$. Infatti mancano:

\begin{gather}
	(132)^{-1} = (123) \\
	(132)\cdot(12) = (23)
\end{gather}

\begin{teorema}
	$HK$ è un sottogruppo se e solo se $HK = KH$.
	
	In particolare questo si verifica quando uno dei due sottogruppi è normale.
\end{teorema}

In generale vale:

\begin{equation}
	HK \subseteq \gen{H, K}
\end{equation}

Infatti:

\begin{gather}
	H = \gen{(12)} \\
	K = \gen{(13)} \\
	HK \subseteq\, \gen{H, K} \,= S_3
\end{gather}

\begin{teorema}
	Se $HK$ è un sottogruppo, allora $HK = \gen{H, K}$.
\end{teorema}

\section{Prodotto diretto}

\begin{equation}
	H \times K = \{(h, k) \taleche h \in H \land k \in K\}
\end{equation}

Il prodotto tra due elementi del prodotto diretto viene effettuata componenti per componente:

\begin{equation}
	(h, k)(h', k') = (hh', kk')
\end{equation}

\begin{teorema}
	G \isomorfo H $\times$ K se e solo se $\begin{cases}
		G=HK \\
		H, K \normale G \\
		H \cap K = 1
	\end{cases}$.
\end{teorema}

Per comodità non si utilizza la notazione $(h, k)$ ma si scrivere $hk$.

\section{Ricapitolando}

Il prodotto di gruppi ci dà un'informazione solo sugli elementi, che possono essere scritti come prodotto di un elemento per ciascun fattore.

Il prodotto diretto è un caso particolare del precedente, in cui abbiamo anche un'informazione sull'operazione, che deve essere svolta componente per componente.

Consideriamo, per esempio, il gruppo $D_n$ in cui abbiamo:

\begin{itemize}
	\item $\gen{\rho}$: il sottogruppo delle rotazioni;
	\item $\gen{\iota}$: il sottogruppo generato da una riflessione.
\end{itemize}

Si ha:

\begin{equation}
	D_n = \gen{\rho}\gen{\iota}
\end{equation}

Quindi ogni elemento di $D_n$ è prodotto di un elemento di $\gen{\rho}$ e di un elemento di $\gen{\iota}$.

Ma:

\begin{equation}
	D_n \not\isomorfo \gen{\rho} \times \gen{\iota}
\end{equation}

Infatti le coppie $(\rho, 1)$ e $(1, \iota)$ commutano:

\begin{gather}
	(\rho, 1)(1, \iota) = (\rho,\iota) \\
	(1, \iota)(\rho, 1) = (\rho,\iota) \\
\end{gather}

Invece gli elementi $\rho$ e $\iota$ non commutano perché:

\begin{equation}
	\iota\rho\iota = \rho^{-1} \quad\Longleftrightarrow\quad \rho\iota = \iota\rho^{-1}
\end{equation}