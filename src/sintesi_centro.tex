\chapter{Sintesi: il centro}
\label{ch:sintesi_centro}

Il \emph{centro} $Z(G)$ di un gruppo $G$ è l'insieme degli elementi di $G$ che commutano con tutti gli elementi di $G$:
\begin{equation*}
    Z(G) = \{g \in G \taleche gx = xg \forall x \in G
\end{equation*}

Il centro $Z(G)$ è l'intersezione dei centralizzanti di tutti gli elementi di $G$:
\begin{equation*}
    Z(G) = \bigcap_{g \in G} C_G(g)
\end{equation*}

Il centro $Z(G)$ è un sottogruppo normale di $G$.

\bigskip
Il centro $Z(G)$ è abeliano, perché tutti gli elementi di $Z(G)$ commutano tra loro.

\bigskip
Il gruppo $G$ è abeliano se e solo se $Z(G) = G$.

\bigskip
Il centro del gruppo diedrale è il sottogruppo ciclico generato dalla rotazione di $\pi$.
Nel caso di $D_n$\dots

\bigskip
Il centro è il nucleo dell'omomorfismo che a ogni elemento $g \in G$ associa l'automorfismo interno $I_g$.

\bigskip
Il centro è il nucleo dell'azione per coniugio.