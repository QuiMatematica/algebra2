\documentclass[a4paper,12pt]{book}
\usepackage[utf8]{inputenc}
\usepackage[T1]{fontenc} 
\usepackage[italian]{babel} 
\usepackage{tabularx}
\usepackage{geometry}
\usepackage{amsmath}
\usepackage{amssymb}
\usepackage{booktabs}
\usepackage{siunitx}
%\usepackage{amsthm}
\usepackage{pgfplots}
\usepackage{mathrsfs}
\usepackage{rotating}
\usepackage[backend=biber,style=alphabetic,sorting=ynt]{biblatex}
\usepackage{mdframed}

% per ultimo:
\usepackage{hyperref}

\pgfplotsset{compat=1.15}
\usetikzlibrary{arrows}
\usetikzlibrary{decorations}

\title{Algebra 2}
\author{Claudio Signorini}
\date{\today}

\newcommand{\gen}[1]{\ensuremath{< \negmedspace #1 \negmedspace >}}
\newcommand{\ordine}[1]{\ensuremath{\lvert #1 \rvert}}
\newcommand{\indice}[2]{\ensuremath{\lvert #1:#2 \rvert}}
\newcommand{\modulo}[1]{\ensuremath{\lvert #1 \rvert}}
\newcommand{\taleche}[0]{\ensuremath{\,:\,}}
\newcommand{\isomorfo}[0]{\ensuremath{\cong}}
\newcommand{\normale}[0]{\ensuremath{\unlhd}}
\newcommand{\divisore}[0]{|}
\newcommand{\sse}[0]{\ensuremath{\Longleftrightarrow}}
\newcommand{\N}[0]{\ensuremath{\mathbb{N}}}
\newcommand{\Z}[0]{\ensuremath{\mathbb{Z}}}
\newcommand{\Q}[0]{\ensuremath{\mathbb{Q}}}
\newcommand{\R}[0]{\ensuremath{\mathbb{R}}}
\newcommand{\MCD}[2]{\ensuremath{ ( #1 , #2 ) }}
\newcommand{\insieme}[1]{\ensuremath{ \{ #1 \} }}
\newcommand{\dividetxt}[0]{\ensuremath{\text{ divide }}}
\newcommand{\eq}[0]{\ensuremath{\sim}}

\DeclareMathOperator{\mcm}{mcm}
\DeclareMathOperator{\Sym}{Sym}
\DeclareMathOperator{\Stab}{Stab}
\DeclareMathOperator{\im}{Im}
\DeclareMathOperator{\Aut}{Aut}
\DeclareMathOperator{\End}{End}
\DeclareMathOperator{\Inn}{Inn}
\DeclareMathOperator{\Alt}{Alt}

\mdfdefinestyle{theoremstyle}{%
	linecolor=blue,linewidth=2pt,%
	frametitlerule=true,%
	frametitlebackgroundcolor=blue!20,
	innertopmargin=\topskip,
}
\mdtheorem[style=theoremstyle]{teorema}{Teorema}[chapter]
\mdtheorem[style=theoremstyle]{corollario}{Corollario}[chapter]
\mdtheorem[style=theoremstyle]{lemma}{Lemma}[chapter]
\mdtheorem[style=theoremstyle]{esercizio}{Esercizio}[chapter]

\newmdenv[linecolor=green,
	linewidth=2pt,
	frametitlerule=true,
	frametitle={Dimostrazione},
	frametitlebackgroundcolor=green!20,
]{dimostrazione}

\newmdenv[linecolor=green,
linewidth=2pt,
frametitlerule=true,
frametitle={Soluzione},
frametitlebackgroundcolor=green!20,
]{soluzione}

%\newtheorem{esercizio}{Esercizio}[chapter]
%\newtheorem{teorema}{Teorema}[chapter]
%\newtheorem{lemma}{Lemma}[chapter]
%\newtheorem{corollario}{Corollario}[chapter]

\definecolor{figura}{rgb}{0.6,0.2,0}

\addbibresource{bibliografia.bib}

\begin{document}
	
	\begin{titlepage}
	\begin{center}
		\vspace*{1cm}
		
		\Huge
		\textbf{Algebra 2}
		
		\vspace{1.5cm}
		
		\LARGE
		\textbf{Claudio Signorini}
		
		\today
		
		\vspace{0.5cm}
		
		\vfill
		
	\end{center}
\end{titlepage}
	
	\part{Conoscenze pregresse}
	\chapter{Teorema di Cayley}

\begin{teorema}
	\label{thr:Cayley}
	Ogni gruppo è isomorfo ad un gruppo di trasformazioni.
\end{teorema}
\begin{dimostrazione}
	Per ogni elemento $a \in G$ definiamo la mappa:
	
	\begin{align}
		a_L : G &\longrightarrow G \\
		x &\longmapsto ax
	\end{align}

	Chiamiamo $a_L$ la \emph{traslazione sinistra} (o \emph{moltiplicazione sinistra}) definita da $a$.
	
	Vogliamo dimostrare che:
	
	\begin{equation}
		G_L = \{a_L \taleche a \in G\}
	\end{equation}

	è un gruppo di trasformazioni. Infatti:
	
	\begin{itemize}
		\item la composizione di due elementi di $G_L$ è interna a $G_L$, infatti:
			\begin{equation}
				(a_L \circ b_L)(x) = abx = (ab)x = (ab)_L(x)
			\end{equation}
		\item l'operazione di composizione delle mappe $a_L$ è associativa, infatti:
			\begin{align}
				[a_L \circ (b_L \circ c_L)](x) &=  a_L((b_L \circ c_L)(x)) = \\
				&= a_L(b_L(c_L(x))) = \\
				&= (a_L \circ b_L)(c_L(x)) = \\
				&= [(a_L \circ b_L) \circ c_L](x)
			\end{align}
		\item $1_L: x \longmapsto 1x$ è l'unità in quanto:
			\begin{gather}
				(1_L \circ a_L)(x) = 1_L(a_L(x)) = 1_L(ax) = 1ax = ax = a_L(x) \\
				(a_L \circ 1_L)(x) = a_L(1_L(x)) = a_L(1x) = a1x = ax = a_L(x)
			\end{gather}
			ovvero:
			\begin{equation}
				1_L \circ a_L = a_L \circ 1_L = a_L
			\end{equation}
		\item ogni elemento $a_L$ ha il proprio inverso, che è $(a^-1)_L$, infatti:
			\begin{gather}
				(a_L \circ (a^{-1})_L)(x) = aa^{-1}x = x = 1_L(x) \\
				((a^{-1})_L \circ a_L)(x) = a^{-1}ax = x = 1_L(x) 
			\end{gather}
		\item gli elementi $a_L$ sono trasformazioni in quanto mappe $G \longrightarrow G$.
	\end{itemize}
	
	Consideriamo adesso la mappa:
	
	\begin{align}
		L: G &\longrightarrow G_L
		a &\longmapsto a_L
	\end{align}

	Questa mappa è un omomorfismo in quanto:
	
	\begin{equation}
		(a_L \circ b_L)(x) = abx = (ab)_L(x)
	\end{equation}

	Inoltre è suriettivo per definizione, ed è iniettivo in quanto, se $a = b$, allora:
	
	\begin{equation}
		a_L(x) = ax = bx = b_L(x) \quad\Longrightarrow\quad a_L = b_L
	\end{equation}

	Quindi $L$ è un isomorfismo e:
	
	\begin{equation}
		G \isomorfo G_L
	\end{equation}
\end{dimostrazione}

\begin{corollario}
	Ogni gruppo finito di ordine $n$ è isomorfo ad un sottogruppo del gruppo simmetrico $S_n$.
\end{corollario}
	\chapter{Laterali di un sottogruppo}

\begin{teorema}[Teorema di Lagrange]
	\label{thr:Laterali_Lagrange}
	L'ordine di un sottogruppo $H$ di un gruppo finito $G$ è un fattore dell'ordine di $G$. Più precisamente:
	
	\begin{equation}
		\ordine{G} = \ordine{H} \cdot \indice{G}{H}
	\end{equation}
\end{teorema}
\begin{dimostrazione}
	Vedi \cite[pag. 52]{jacobson}.
\end{dimostrazione}
	\chapter{Gruppi quozienti}

Si $G$ un gruppo e $N$ un suo sottogruppo normale:

\begin{equation}
	N \normale G
\end{equation}

Definiamo una relazione di congruenza:

\begin{equation}
	a \equiv b \,(\text{mod } N) \quad \text{ se } a^{-1}b \in N
\end{equation}

ovvero se:

\begin{equation}
	b = aN
\end{equation}

ovvero se:

\begin{equation}
	\exists n \in N \taleche b = an
\end{equation}

Vedi \cite[pag. 56]{jacobson} per la dimostrazione che questa relazione è di equivalenza.

Le classi di equivalenza corrispondono ai laterali del sottogruppo $N$, e formano un gruppo in quanto:

\begin{gather}
	(gN)(hN) = ghN \\
	N \text{ è l'unità} \\
	(gN)^{-1} = g^{-1}N
\end{gather}

Definiamo \textbf{gruppo quoziente} di $G$ relativo a $N$ il gruppo formato dalle classi di equivalenza $G/\equiv$. Indichiamo questo gruppo con $G/N$.

Le classi di equivalenza formano una partizione del gruppo $G$, pertanto gli elementi del gruppo quoziente $G/N$ rappresentano a loro volta una partizione di $G$. Uno degli elementi è rappresentato dal sottogruppo normale $N$ (figura~\ref{fig:Gruppi_quozienti_partizione}).

\begin{figure}[tp]
	\centering
	\begin{tikzpicture}[line cap=round,line join=round,>=triangle 45,x=1cm,y=1cm]
		\clip(-4.5,-3.4) rectangle (4.5,3.4);
		\draw [rotate around={0:(0,0)},line width=1pt] (0,0) ellipse (4cm and 3cm);
		\draw [shift={(-11,0)},line width=1pt] plot[domain=-0.27:0.27,variable=\t]({9*cos(\t r)},{9*sin(\t r)});
		\draw [shift={(-1,12)},line width=1pt] plot[domain=4.63:5.12,variable=\t]({12*cos(\t r)},{12*sin(\t r)});
		\draw [shift={(12,3)},line width=1pt]  plot[domain=3.4:3.65,variable=\t]({11*cos(\t r)},{11*sin(\t r)});
		\draw (3.2,2.8) node[anchor=north west] {$G$};
		\draw (-0.5,-1) node[anchor=north west] {$N$};
		\draw (-3.5,0.5) node[anchor=north west] {$g_1N$};
		\draw (-0.5,2) node[anchor=north west] {$g_2N$};
		\draw (2.4,-0.2) node[anchor=north west] {$g_3N$};
	\end{tikzpicture}
	\caption{Gruppo quoziente $G/N$ come partizione del gruppo $G$.}
	\label{fig:Gruppi_quozienti_partizione}
\end{figure}

Ogni gruppo $G \ne 1$ ha due sottogruppi normali: $G$ e $1$. $G$ è detto \textbf{semplice} se ha solo questi sottogruppi normali.
	\chapter{Omomorfismi}

La funzione $\eta: G \longrightarrow G'$ è un \textbf{omomorfismo} se:

\begin{gather}
	\label{eqn:omomorfismo_proprieta} \eta(ab) = \eta(a)\eta(b) \\
	\eta(1) = 1'
\end{gather}

La seconda è conseguenza della prima. Quindi per verificare che una funzione è un omomorfismo ci basta verificare la prima. 

Un omomorfismo suriettivo si chiama \textbf{epimorfismo}.

Un omomorfismo iniettivo si chiama \textbf{monomorfismo}.

Un omomorfismo biiettivo si chiama \textbf{isomorfismo}.

Un omomorfismo di un gruppo su se stesso si chiama \textbf{endomorfismo}.

Un endomorfismo biiettivo si chiama \textbf{automorfismo} (figura~\ref{fig:Omomorfismi_dalle_funzioni_agli_automorfismi}).

\begin{figure}[tp]
	\centering
	\tikz {
		\node (a) at (0,8) {funzione/mappa};
		\node (b) at (3,9) {suriettiva};
		\node (c) at (3,7) {iniettiva};
		\node (d) at (6,8) {biiettiva};
		\draw (a) edge[->] (b); 
		\draw (a) edge[->] (c);
		\draw (b) edge[->] (d);
		\draw (c) edge[->] (d);
		\node (e) at (0,5) {omomorfismo};
		\node (f) at (3,6) {epimorfismo};
		\node (g) at (3,4) {monomorfismo};
		\node (h) at (6,5) {isomorfismo};
		\draw (e) edge[->] (f); 
		\draw (e) edge[->] (g);
		\draw (f) edge[->] (h);
		\draw (g) edge[->] (h);
		\node (i) at (0,3) {endomorfimo};
		\node (j) at (6,3) {automorfismo};
		\draw (a) edge[->] (e);
		\draw (e) edge[->] (i);
		\draw (i) edge[->] (j);
	}
	\caption{Dalle funzioni agli automorfismi}
	\label{fig:Omomorfismi_dalle_funzioni_agli_automorfismi}
\end{figure}

\begin{teorema}
	Siano $\eta$ e $\zeta$ due omomorfismi $G \longrightarrow G'$. Sia $S$ l'insieme dei generatori di $G$.
	
	Se $\eta(s) = \zeta(s) \,\forall s \in S$ allora $\eta = \zeta$.
\end{teorema}
\begin{dimostrazione}
	Vedi \cite[pag. 60]{jacobson}.
\end{dimostrazione}

\begin{corollario}
	Sia $\eta$ un endomorfismo di $G$. Sia $H$ l'insieme degli elementi fissi di $\eta$:
	
	\begin{equation}
		H = \{h \in G \taleche \eta(h) = h\}
	\end{equation}

	$H$ è un sottogruppo di $G$.
\end{corollario}

\begin{teorema}
	Si $\eta: G \longrightarrow G'$ un omomorfismo, e $\zeta: G' \longrightarrow G''$ un altro omomorfismo. Allora $\eta\zeta: G \longrightarrow G''$ è un omomorfismo.
\end{teorema}

Gli automorfismo di un gruppo formano il \textbf{gruppo degli automorfismi}: $\Aut G$. E' un gruppo di trasformazioni.

Gli endomorfismi di un gruppo formano il \textbf{mononoide degli endomorfismi}: $\End G$. E' un monoide di trasformazioni.

Il \textbf{nucleo} $\ker \eta$ di un omomorfismo $\eta$ è l'insieme degli elementi di $G$ tali che la loro immagine è l'identità:

\begin{equation}
	\ker \eta = \{g \in G \taleche \eta(g) = 1'\}
\end{equation}

\section{Teorema fondamentale degli omomorfismi}

\begin{lemma}
	\label{lmm:omomorfismo_iniettivo}
	Un omomorfismo $\eta$ è iniettivo se e solo se $\ker \eta = 1$.
\end{lemma}
\begin{dimostrazione}
	\textbf{Prima parte:}
	
	\begin{equation}
		\eta \text{ iniettivo} \quad\Longrightarrow\quad \ker \eta = 1
	\end{equation}
	
	Poniamo per assurdo che $\ker \eta \ne 1$. Allora:
	
	\begin{equation}
		\exists b \in G, b \ne 1 : \eta(b) = 1'
	\end{equation}

	Ma anche:
	
	\begin{equation}
		\eta(1) = 1'
	\end{equation} 

	Quindi $\eta$ non può essere iniettivo.
	
	\textbf{Seconda parte:}
	
	\begin{equation}
		\ker \eta = 1 \quad\Longrightarrow\quad \eta \text{ iniettivo}
	\end{equation}
	
	Poniamo per assurdo che $\eta$ non sia iniettivo. Allora:
	
	\begin{equation}
		\exists a, b \in G, a \ne b : \eta(a) = \eta(b)
	\end{equation}

	Se $a$ e $b$ sono elementi diversi, allora:
	
	\begin{equation}
		b \ne a \Longrightarrow a^{-1}b \ne a^{-1}a \Longrightarrow a^{-1}b \ne b
	\end{equation}

	Passiamo alle immagini:
	
	\begin{align}
		\eta(a^{-1}b) &= \eta(a)^{-1}\eta(b) &\text{perché $\eta$ è un omomorfismo} \\
		&=\eta(a)^{-1}\eta(a) &\text{perché $\eta(a)=\eta(b)$} \\
		&=1' &\text{per prodotto di inversi}
	\end{align}

	Quindi:
	
	\begin{equation}
		a^{-1}b \in \ker \eta \quad\Longrightarrow\quad \ker \eta \ne 1
	\end{equation}
	
\end{dimostrazione}

\begin{teorema}[Teorema fondamentale degli omomorfismi di gruppi]
	\label{thr:Omomorfismi_fondamentale}
	Dato un omomorfismo $\eta: G \longrightarrow G'$ allora (figure~\ref{fig:Omomorfismi_fondamentale} e~\ref{fig:Omomorfismi_partizione}):
	\begin{itemize}
		\item $\eta(G) \le G'$
		\item $\ker \eta \normale G$
		\item $\bar{\eta}: a \ker \eta \longmapsto \eta(a)$ è iniettiva
	\end{itemize}
\end{teorema}

\begin{figure}[tp]
	\centering
	\tikz {
		\node (a) at (0,3) {$G$};
		\node (b) at (3,3) {$G'$};
		\node (c) at (0,0) {$G/\ker \varphi$};
		\draw (a) edge[->] node[above] {$\eta$} (b); 
		\draw (a) edge[->] node[left] {$\nu$} (c);
		\draw (c) edge[->] node[below right] {$\bar{\eta}$} (b);
	}
	\caption{Teorema fondamentale degli omomorfismi}
	\label{fig:Omomorfismi_fondamentale}
\end{figure}

\begin{figure}[tp]
	\centering
	\begin{tikzpicture}[line cap=round,line join=round,>=triangle 45,x=1cm,y=1cm]
		\clip(-4.5,-3.4) rectangle (4.5,3.4);
		\draw [rotate around={0:(0,0)},line width=1pt] (0,0) ellipse (4cm and 3cm);
		\draw [shift={(-11,0)},line width=1pt] plot[domain=-0.27:0.27,variable=\t]({9*cos(\t r)},{9*sin(\t r)});
		\draw [shift={(-1,12)},line width=1pt] plot[domain=4.63:5.12,variable=\t]({12*cos(\t r)},{12*sin(\t r)});
		\draw [shift={(12,3)},line width=1pt]  plot[domain=3.4:3.65,variable=\t]({11*cos(\t r)},{11*sin(\t r)});
		\draw (3.2,2.8) node[anchor=north west] {$G$};
		\draw (-0.5,-1) node[anchor=north west] {$\ker\eta$};
		\draw (-3.7,0.5) node[anchor=north west] {$g_1\ker\eta$};
		\draw (-0.5,2) node[anchor=north west] {$g_2\ker\eta$};
		\draw (2.2,-0.2) node[anchor=north west] {$g_3\ker\eta$};
	\end{tikzpicture}
	\caption{Gruppo quoziente $G/\ker\eta$ come partizione del gruppo $G$.}
	\label{fig:Omomorfismi_partizione}
\end{figure}

\begin{dimostrazione}
	Vedi \cite[pagg. 61 e 62]{jacobson}.
	
	\textbf{Prima parte. }	
	Per verifica che $\eta(G)$ è un sottogruppo di G' devo verificare che l'insieme è chiuso per il prodotto definito in G'.
	
	Dati due elementi $g_1, g_2 \in G$ qualunque:
	
	\begin{gather}
		\eta(g_1), \eta(g_2) \in \eta(G) \\
		\eta(g_1)\eta(g_2) = \eta(g_1g_2) \in \eta(G)
	\end{gather}

	Quindi il prodotto di due elementi dell'immagine $\eta(G)$ è interno all'immagine stessa: $\eta(G)$ è un sottogruppo di $G'$.
	
	\textbf{Seconda parte.}
	Definiamo ora:
	
	\begin{equation}
		K = \ker \eta = \{k \in G \taleche \eta(k) = 1'\}
	\end{equation}

	Per verificare che $\ker \eta$ è un sottogruppo normale di $G$ mi basta verificare che:
	
	\begin{equation}
		gKg^{-1} = K \quad\forall g \in G
	\end{equation}
		
	$\forall g \in G$ e $\forall k \in K$ risulta:
	
	\begin{align}
		\eta(gkg^{-1}) &= \eta(g)\eta(k)\eta(g^{-1}) = &\text{perché $\eta$ è un omomorfismo} \\
		&= \eta(g)1'\eta(g^{-1}) = &\text{perché $k \in \ker \eta$} \\
		&= \eta(g)\eta(g^{-1}) = &\text{perché 1' è l'identità di $G'$} \\
		&= \eta(gg^{-1}) = &\text{perché $\eta$ è un omomorfismo} \\
		&= \eta(1) = &\text{perché $g$ e $g^{-1}$ sono inversi} \\
		&= 1' &\text{perché $\eta$ è un omomorfismo}
	\end{align}

	Quindi:
	
	\begin{equation}
		gkg^{-1} \in K \quad\Longrightarrow\quad gKg^{-1} = K \quad\Longrightarrow\quad K \normale G
	\end{equation}

	\textbf{Terza parte.}
	Consideriamo $L$ un sottogruppo normale di $G$ contenuto in $K$:
	
	\begin{equation}
		L \normale K \normale G
	\end{equation}

	Formiamo il gruppo quoziente $G/L$ in cui sappiamo che:
	
	\begin{gather}
		\label{eqn:quoziente_L_1} (aL)(bL) = abL \quad \forall a, b \in G \\
		\label{eqn:quoziente_L_2}L \text{ è unità}
	\end{gather}

	Consideriamo la funzione $\nu$ così definita:
	
	\begin{align}
		\nu:\, G &\longrightarrow G/L \\
		a &\longmapsto aL
	\end{align}

	Le formule~\eqref{eqn:quoziente_L_1} e~\eqref{eqn:quoziente_L_2} garantiscono che $\nu$ sia un omomorfismo.

	Dobbiamo verificare che l'omomorfismo indotto da $\eta$:
	
	\begin{align}
		\bar{\eta}:\, G/L &\longrightarrow G' \\
		aL &\longmapsto \eta(a)
	\end{align}
	
	sia \emph{ben definito}, ovvero che il suo comportamento non dipenda dagli elementi scelti per ciascun laterale. Verifichiamo quindi cosa succede se prendiamo due elementi $a$ e $b$ di $G$ diversi ma appartenenti allo stesso laterale:
	
	\begin{align}
		aL &= bL &\text{$a$ e $b$ appartengono allo stesso laterale} \\
		\exists l \in L \taleche b &= aL &\text{per definizione di laterale} \\
		\eta(b) &= \eta(a)\eta(l) &\text{perché $\eta$ è un omomorfismo} \\
		\eta(b) &= \eta(a) 1' &\text{perché $l \in L \subseteq K = \ker \eta$} \\
		\eta(b) &= \eta(a) &\text{perché $1'$ è unità di $G'$}
	\end{align} 
	
	Quindi $\bar{\eta}$ è ben definito.
	
	Quindi:
	
	\begin{equation}
		\forall a \in G : \bar{\eta}\nu(a) = \bar{\eta}(aL) = \eta(a)
	\end{equation}

	Quindi vale il diagramma commutativo di figura~\ref{fig:Omomorfismi_fondamentale_con_L}.

	Determiniamo l'immagine di $\bar{\eta}$.
	
	Se $g' \in \eta(G)$ allora:
	
	\begin{align}
		\exists a \in G \taleche \eta(a) = g' &\quad\text{per definizione di immagine} \\
		\bar{\eta}(aL) = g' &\quad\text{per definizione di $\bar{\eta}$} \\
		g' \in \bar{\eta}(G/L) &\quad\text{per definizione di immagine}
	\end{align}

	Se $g' \in \bar{\eta}(G/L)$ allora:

	\begin{align}
		\exists a \in G \taleche \bar{\eta}(aL) = g' &\quad\text{per definizione di immagine} \\
		\eta(a) = g' &\quad\text{per definizione di $\bar{\eta}$} \\
		g' \in \eta(G) &\quad\text{per definizione di immagine}
	\end{align}

	Quindi:
	
	\begin{equation}
		\eta(G) = \bar{\eta}(G/L)
	\end{equation}
	
	Determiniamo infine il nucleo di $\bar{\eta}$ ovvero:
	
	\begin{align}
		\ker \bar{\eta} = \{aL \taleche \bar{\eta}(aL) = 1'\} & \quad\text{per definizione di nucleo} \\
		\ker \bar{\eta} = \{aL \taleche \eta(a) = 1'\} & \quad\text{per definizione di $\bar{\eta}$} \\
		\ker \bar{\eta} = \{aL \taleche a \in \ker \eta\} & \quad\text{per definizione di nucleo} \\
		\label{eqn:nucleo_eta_segnato} \ker \bar{\eta} = \ker \eta / L &\quad\text{per definizione di laterale}
	\end{align}

	Infatti preso un $k \in \ker \eta$ si ha:
	
	\begin{align}
		\nu(k) = kL &\quad\text{per definizione di $\nu$} \\
		\bar{\eta}(kL) = \eta(k) & \quad\text{per definizione di $\bar{\eta}$} \\
		\bar{\eta}(kL) = 1' &\quad\text{perché $k \in \ker \eta$}
	\end{align}
	
	Invece, preso un $g \not\in \ker \eta$ si ha:
	
	\begin{align}
		\nu(g) = gL &\quad\text{per definizione di $\nu$} \\
		\bar{\eta}(gL) = \eta(g) &\quad\text{per definizione di $\bar{\eta}$} \\
		\bar{\eta}(gL) \ne 1' &\quad\text{perché $g \not\in \ker \eta$}
	\end{align}
	
	Infine, per il lemma~\ref{lmm:omomorfismo_iniettivo}, l'omomorfismo $\bar{\eta}$ è iniettivo se e solo se il suo nucleo è uguale all'unità. L'unità di $G/L$ è $L$, quindi $\bar{\eta}$ è iniettivo se accade che:
	
	\begin{align}
		L = \ker \bar{\eta} & \quad\text{condizione per il lemma} \\
		L = \ker \eta / L & \quad\text{per la formula \eqref{eqn:nucleo_eta_segnato}} \\
		L = \ker \eta & \quad\text{per definizione di laterale}
	\end{align} 
	
\end{dimostrazione}

\begin{figure}[tp]
	\centering
	\tikz {
		\node (a) at (0,3) {$G$};
		\node (b) at (3,3) {$G'$};
		\node (c) at (0,0) {$G/L$};
		\draw (a) edge[->] node[above] {$\eta$} (b); 
		\draw (a) edge[->] node[left] {$\nu$} (c);
		\draw (c) edge[->] node[below right] {$\bar{\eta}$} (b);
	}
	\caption{Diagramma commutativo con il sottogruppo normale $L$}
	\label{fig:Omomorfismi_fondamentale_con_L}
\end{figure}

\begin{corollario}[Primo corollario del teorema fondamentale degli omomorfismi di gruppi]
	\label{crl:Omomorfismi_fondamentale_1}
	Dato un epimorfismo $\eta: G \longrightarrow G'$ allora $\bar{\eta}: a \ker \eta \longmapsto \eta(a)$ è un isomorfismo.
\end{corollario}
\begin{dimostrazione}
	Se infatti $\eta$ è suriettivo, allora $\bar{\eta}$ è iniettivo e suriettivo, quindi è un isomorfismo (immagine~\ref{fig:eta_suriettivo}).
\end{dimostrazione}

\begin{figure}[tp]
	\centering
	\tikz {
		\node (a) at (0,4) {$G$};
		\node (b) at (4,4) {$G'$};
		\node (c) at (0,0) {$G/\ker \eta$};
		\draw (a) edge[->] node[above] {$\eta$} (b);
		\draw (a) edge[->] node[below] {suriettivo} (b); 
		\draw (a) edge[->] node[left] {$\nu$} (c);
		\draw (c) edge[->] node[above left] {$\bar{\eta}$} (b);
		\draw (c) edge[->] node[below right] {iniettivo e suriettivo} (b);
	}
	\caption{Diagramma commutativo con $\eta$ suriettivo}
	\label{fig:eta_suriettivo}
\end{figure}

\begin{corollario}[Secondo corollario del teorema fondamentale degli omomorfismi di gruppi]
	\label{crl:Omomorfismi_fondamentale_2}	Se un gruppo $G$ gode di un omomorfismo $\eta: G \longrightarrow G'$ allora $\exists K \normale G$ tale che $G/K \isomorfo \eta(G)$ dove $K = \ker \eta$.
\end{corollario}

\section{Sintesi}

Una mappa/funzione

\begin{equation}
	\eta: G \longrightarrow G'
\end{equation}

\begin{itemize}
	\item è un omomorfismo se $\eta(ab) = \eta(a)\eta(b)$;
	\item è un endomorfismo se $G' = G \Longrightarrow \eta : G \longrightarrow G$;
	\item è un automorfismo se $\exists \eta^{-1}: G \longrightarrow G$. 
\end{itemize}
	\chapter{Isomorfismi}

\section[Teorema dei due sottogruppi]{Teorema dei due sottogruppi (o secondo teorema di isomorfismo)}

\begin{teorema}[Teorema dei due sottogruppi]
	\label{thr:due_sottogruppi}
	Dati $H \le G$ e $N \normale G$ allora
	
	\begin{equation}
		\dfrac{H}{H \cap N} \isomorfo \dfrac{HN}{N}
	\end{equation}
\end{teorema}
\begin{figure}[tp]
	\centering
	\tikz {
		\node (a) at (2,6) {$G$};
		\node (f) at (2,4) {$HN$};
		\node (b) at (0,2) {$H$};
		\node (d) at (4,2) {$N$};
		\node (e) at (2,0) {$H \cap N$};
		\node [red] (g) at (2,2) {$\isomorfo$};
		\draw (a) edge[->] (f) 
		(f) edge[->] (b)
		(f) edge[->] (d) 
		(b) edge[->] (e)
		(d) edge[->] (e)
		(1,1) edge[->,red] (g)
		(3,3) edge[->,red] (g);
	}
	\caption{Rappresentazione del teorema dei due sottogruppi.}
	\label{fig:Isomorfismi_due_sottogruppi}
\end{figure}
\begin{dimostrazione}
	Consideriamo l'omomorfismo:
	
	\begin{align}
		\pi: G &\longrightarrow G/N \\
		g &\longmapsto gN
	\end{align}

	Consideriamo ora la restrizione $\pi |_H$ per la quale il dominio diventa il sottogruppo $H$:
	
	\begin{align}
		\pi |_H: H &\longrightarrow G/N \\
		h &\longmapsto hN
	\end{align}

	Vogliamo determinare cos'è l'immagine di $H$: appartengono all'immagine i laterali generati da elementi di $H$, ma è possibile che ci siano elementi di $G$ esterni ad $H$ che generano gli stessi laterali:
	
	\begin{align}
		\pi(H) &= \{gN \taleche \exists h \in H, gN = hN \} = \\
		&= \{gN \taleche \exists h \in H, g \in hN\} = \\
		&= \{gN \taleche g \in HN\} = \\
		&= NH/N
	\end{align}

	Determiniamo ora il nucleo di $\pi |_H$:
	
	\begin{align}
		\ker \pi |_H &= \{h \in H \taleche hN = N\} \\
		&= H \cap N
	\end{align}

	Quindi per il corollario~\ref{crl:Omomorfismi_fondamentale_1}  del teorema fondamentale degli omomorfismi:
	
	\begin{equation}
		\dfrac{H}{\ker \pi |_H} \isomorfo \pi(H) \quad\Longrightarrow\quad \dfrac{H}{H \cap N} \isomorfo \dfrac{HN}{N}
	\end{equation}

\end{dimostrazione}

\section[Teorema di corrispondenza]{Teorema di corrispondenza (o terzo teorema di isomorfismo)}

\begin{teorema}[Teorema di corrispondenza]
	\label{thr:corrispondenza}
	Sia dato un epimorfismo $\eta: G \longrightarrow \overline{G}$.
	
	I sottogruppi $H$ di $G$ che contengono $\ker\eta$ sono in corrispondenza biunivoca con i sottogruppi $\eta(H)$ di $\overline{G}$ (figura~\ref{fig:Isomorfismi_corrispondenza}).
	
	Questa corrispondenza preserva le inclusioni, gli indici, la normalità e i quozienti.
	
	Caso particolare: se $N \normale G$, allora posso definire $\eta$ come:
	
	\begin{equation}
		\eta: G \longrightarrow G/N \quad \text{ con } \quad \eta(g) = gN
	\end{equation}
	
	Ogni $H \le G$ con $N \subseteq H$ è isomorfo a $H/N \le G/N$  (figura~\ref{fig:Isomorfismi_corrispondenza_laterali}).
	
	Se anche $H \normale G$ allora:
	
	\begin{equation}
		G/H=(G/N)/(H/N)
	\end{equation}
\end{teorema}

\begin{figure}[tp]
	\centering
	\tikz {
		\node (a) at (2,4) {$G$};
		\node (b) at (0,2) {$H_1$};
		\node (c) at (2,2) {$H_2$};
		\node (d) at (4,2) {$H_3$};
		\node (e) at (2,0) {$\ker \eta$};
		\draw (a) edge[->] (b) 
			(a) edge[->] (c)
			(a) edge[->] (d) 
			(b) edge[->] (e)
			(c) edge[->] (e)
			(d) edge[->] (e);
	}
	$\qquad$
	\tikz {
		\node (a) at (2,4) {$\overline{G} = \eta(G)$};
		\node (b) at (0,2) {$\eta(H_1)$};
		\node (c) at (2,2) {$\eta(H_2)$};
		\node (d) at (4,2) {$\eta(H_3)$};
		\node (e) at (2,0) {$1 = \eta(\ker \eta)$};
		\draw (a) edge[->] (b) 
			(a) edge[->] (c)
			(a) edge[->] (d) 
			(b) edge[->] (e)
			(c) edge[->] (e)
			(d) edge[->] (e);
	}
	\caption{Rappresentazione del teorema di corrispondenza.}
	\label{fig:Isomorfismi_corrispondenza}
\end{figure}

\begin{figure}[tp]
	\centering
	\tikz {
		\node (a) at (2,4) {$G$};
		\node (b) at (0,2) {$H_1$};
		\node (c) at (2,2) {$H_2$};
		\node (d) at (4,2) {$H_3$};
		\node (e) at (2,0) {$N$};
		\draw (a) edge[->] (b) 
		(a) edge[->] (c)
		(a) edge[->] (d) 
		(b) edge[->] (e)
		(c) edge[->] (e)
		(d) edge[->] (e);
	}
	$\qquad$
	\tikz {
		\node (a) at (2,4) {$G/N$};
		\node (b) at (0,2) {$H_1/N$};
		\node (c) at (2,2) {$H_2/N$};
		\node (d) at (4,2) {$H_3/N$};
		\node (e) at (2,0) {$1 = N/N$};
		\draw (a) edge[->] (b) 
		(a) edge[->] (c)
		(a) edge[->] (d) 
		(b) edge[->] (e)
		(c) edge[->] (e)
		(d) edge[->] (e);
	}
	\caption{Rappresentazione del teorema di corrispondenza nel caso particolare.}
	\label{fig:Isomorfismi_corrispondenza_laterali}
\end{figure}
	
	\part{Settimana prima}
	\chapter{Gruppi ciclici}


Un gruppo $G$ si dice \textbf{ciclico} se:

\begin{equation}
	\exists g \in G : \forall x \in G \quad x=g^k \quad \text{con } k \in \Z
\end{equation}

L'elemento $g$ si dice \textbf{generatore} del gruppo $G$ e si scrive:

\begin{equation}
	G = \gen{g}
\end{equation}

Si pongono tre problemi:

\begin{enumerate}
	\item classificare i gruppi ciclici (ovvero elencarli tutti a meno di isomorfismi)
	\item quanti sono e come sono fatti i sottogruppi di un gruppo ciclico?
	\item in quanti modi posso scegliere i generatori di un gruppo ciclico?
\end{enumerate}

\section[Classificazione dei gruppi ciclici]{Classificazione dei gruppi ciclici\footnote{\cite[29 settembre 2020]{lucchini}}}

\begin{teorema}
	\label{thr:Ciclici_isomorfismi_con_z}
	Se $G$ è un gruppo ciclico, allora $G \isomorfo \Z$ oppure $G \isomorfo \Z/m\Z$.
\end{teorema}
\begin{dimostrazione}
	Chiamiamo $g$ il generatore del gruppo ciclico $G$. Consideriamo la funzione:
	
	\begin{align}
		\varphi : (\Z, +) &\longrightarrow G \\
		t &\longmapsto g^t
	\end{align}

	$\varphi$ è un omomorfismo perché:
	
	\begin{equation}
		\varphi(t_1 + t_2) = g^{t^1 + t^2} = g^{t_1} \dot g^{t_2} = \varphi(t_1) \dot \varphi(t_2)
	\end{equation}

	Ovvero $\varphi$ è compatibile con le operazioni interne: la somma in $\Z$ in e il prodotto in $G$.
	
	$\varphi$ è suriettivo: deriva dalla definizione di gruppo ciclico. Quindi è un epimorfismo. Il gruppo $G$ si ottiene come \emph{immagine epimorfa} del gruppo degli interi.
	
	Per il corollario al teorema fondamentale degli omomorfisci \textbf{riferimento} quando abbiamo un omomorfismo suriettivo, allora l'immagine ($G$) è isomorfa al gruppo quoziente del dominio sopra il nucleo:
	
	\begin{equation}
		G \isomorfo \Z/\ker \varphi
	\end{equation}

	Studiamo allora $\ker \varphi$: abbiamo due casi.
	
	\textbf{Caso 1: } $\ker \varphi = \{0\}$, ovvero $\varphi$ è iniettivo.
	
	Quindi $G \isomorfo \Z$, infatti se $G = \gen{g}$ abbiamo:
	
	\begin{gather}
		g^1 = 1 \longleftrightarrow t = 0 \\
		g^{t_1} = g^{t_2} \longleftrightarrow t_1 = t_2
	\end{gather}

	\textbf{Caso 2: } $\ker \varphi \ne \{0\}$
	
	Quindi:
	
	\begin{equation}
		\exists m \ne 0 \land m \in \Z \taleche \varphi(m) = g^m = 1
	\end{equation}

	Se $m \in \ker \varphi$ allora anche $-m \in \ker \varphi$. Infatti il nucleo è un sottogruppo di $\Z$, quindi se un sottogruppo contiene $m$ allora contiene anche $-m$.  Dimostrazione veloce:
	
	\begin{equation}
		g^m = 1 \Longrightarrow g^{-m} = (g^m)^{-1} = 1^{-1} = 1
	\end{equation}

	Quindi possiamo considerare $m$ un numero positivo:

	\begin{equation}
		\exists m \ne 0 \land m \in \N \taleche \varphi(m) = g^m = 1
	\end{equation}

	Ora usiamo la proprietà del minimo dei naturali e scegliamo $m$ come il minimo intero positivo tale che $g^m = 1$; cerchiamo di capire quali sono gli altri interi positivi che appartengono a $\ker \varphi$: sia $t \in \ker \varphi$ e facciamo la divisione per $m$:
	
	\begin{equation}
		t = qm + r \quad \text{con } 0 \le r < m
	\end{equation}

	Si ha che:
	
	\begin{equation}
		1 = g^t = g^{qm + r} = g^{qm} \cdot g^r = (g^m)^q \cdot g^r = 1^q \cdot g^r = 1 \cdot g^r = g^r
	\end{equation}

	Quindi $g^r = 1$. Ma $r$ è un intero non negativo minore di $m$, quindi per non contraddire la scelta di minimalità risulta:
	
	\begin{equation}
		r = 0
	\end{equation}

	Quindi $t$ è un multiplo di $m$, ovvero:
	
	\begin{equation}
		t \in m\Z
	\end{equation}

	Quindi ogni elemento di $\ker \varphi$ è una potenza di $\Z$:
	
	\begin{equation}
		\ker \varphi = m\Z
	\end{equation}

	Quindi $G$ è isomorfo a $\Z/m\Z$:
	
	\begin{equation}
		G \isomorfo \Z/m\Z
	\end{equation}

\end{dimostrazione}

Quindi quando abbiamo un gruppo ciclico $G$ abbiamo due possibilità:

\begin{enumerate}
	\item $G \isomorfo \Z \Longrightarrow$ gruppo ciclico infinito;
	\item $G \isomorfo \Z/m\Z \Longrightarrow$ gruppo ciclico finito: $G = \{1, g, \dots, g^{m-1}\}$.
\end{enumerate}

\section[Sottogruppo generato da un elemento]{Sottogruppo generato da un elemento\footnote{\cite[1 ottobre 2021]{lucchini}}}

Dato un generico gruppo $G$, consideriamo un elemento $x \in G$. Definiamo:

\begin{equation}
	\gen{x} = \{x^m \taleche m \in \Z\}
\end{equation}

Questo è il più piccolo sottogruppo ciclico di $G$ che contiene $x$.

Chiamiamo \textbf{ordine di $x$} l'ordine del sottogruppo generato da $x$:

\begin{equation}
	\ordine{x} = \ordine{\gen{x}}
\end{equation}

Per il teorema~\ref{thr:Ciclici_isomorfismi_con_z} sappiamo che:

\begin{equation}
	x = 
	\begin{cases}
		\infty \Leftrightarrow \gen{x} \isomorfo \Z \Leftrightarrow x^m = 1 \leftrightarrow m = 0 \\
		m \Leftrightarrow \gen{x} \isomorfo \Z/m\Z \Leftrightarrow m \text{ più piccolo intero positivo t.c. } x^m = 1 
	\end{cases}
\end{equation}

\begin{teorema}
	\label{thr:Ciclici_ordine_elementi}
	Dati $G = \gen{g}$ e $t \in \Z$ con $t \ne 0$, allora:
	\footnote{Non è necessario che $G$ sia ciclico, in quanto è comunque ciclico il suo sottogruppo $\gen{g}$.}
	
	\begin{equation}
		\ordine{g^t} = 
		\begin{cases}
			\infty & \text{se } \ordine{g} = \infty \\
			\dfrac{m}{\MCD{m}{t}} & \text{se } \ordine{g} = m
		\end{cases}
	\end{equation} 
\end{teorema}
\begin{dimostrazione}
	\textbf{Caso 1:} $\ordine{g} = \infty$:
	
	Per calcolare l'ordine di $g^t$ verifichiamo per quali valori di $r$ risulta:
	
	\begin{equation}
		(g^t)^r = 1 \Longrightarrow g^{tr} = 1 \Longrightarrow tr = 0
	\end{equation}

	Ma $t \ne 0$ per ipotesi, quindi $r = 0$ per la legge dell'annullamento del prodotto. Abbiamo quindi che:
	
	\begin{equation}
		(g^t)^r = 1 \Longleftrightarrow r = 0
	\end{equation}

	Quindi:
	
	\begin{equation}
		\ordine{g^t} = \infty
	\end{equation}

	\textbf{Caso 2:} $\ordine{g} = m$:
	
	Come nel caso 1 cerchiamo i valori di $r$ tali che:
	
	\begin{equation}
		(g^t)^r = 1 \Longrightarrow g^{tr} = 1 \Longrightarrow m|tr
	\end{equation}

	Chiamiamo $d$ il MCD tra $m$ e $t$:
	
	\begin{equation}
		d = \MCD{m}{t} \Longrightarrow
		\begin{cases}
			m = d m_1 \\
			t = d t_1 \\
			\MCD{m_1}{t_1} = 1 &\text{coprimi}
		\end{cases}
	\end{equation}

	Quindi:
	
	\begin{equation}
		m|tr \Longrightarrow dm_1 | dt_1r \Longrightarrow m_1|t_1r
	\end{equation}

	Ma $t_1$ non ha alcun divisore di $m_1$, quindi:
	
	\begin{equation}
		m_1|r
	\end{equation}

	Quindi:
	
	\begin{equation}
		\ordine{g^t} = m_1 = \dfrac{m}{d} = \dfrac{m}{\MCD{m}{t}}
	\end{equation}

	\textbf{NB: } $t = 0 \Longrightarrow g^t = 1 \Longrightarrow \ordine{g^t} = 1$.
\end{dimostrazione}

\begin{esercizio}
	Dato
	
	\begin{equation}
		x = (13)(245) \in S_5
	\end{equation}
	
	Qual è l'ordine di $x$? Quali sono gli ordini di $x^2$ e di $x^4$?	
\end{esercizio}
\begin{dimostrazione}
	Cerchiamo il valore di $t$ tale che:
	
	\begin{gather}
		x^t = 1 \\
		((13)(245))^t = 1 \\
		(13)^t(245)^t = 1
	\end{gather}

	Posso fare quest'ultimo passaggio perché i due cicli sono disgiunti, quindi commutano.
	
	\begin{gather}
		\begin{cases}
			(13)^t = 1 &\longrightarrow \ordine{(13)} = 2 \longrightarrow 2|t \\
			(245)^t= 1 &\longrightarrow \ordine{(245)}= 3 \longrightarrow 3|t
		\end{cases}	\\
		\Longrightarrow 6|t
		\Longrightarrow \ordine{x} = 6
	\end{gather}

	Per l'ordine di $x^2$ applichiamo il teorema:
	
	\begin{equation}
		\ordine{x^2} = \dfrac{\ordine{x}}{\MCD{\ordine{x}}{2}} = \dfrac{6}{\MCD{6}{2}} = \dfrac{6}{2} = 3
	\end{equation}

	Infatti:
	
	\begin{equation}
		(13)^2(245)^2 = (245)^2 \longrightarrow \ordine{(245)^2} = 3
	\end{equation}

	Anche per l'ordine di $x^4$ applichiamo il teorema:
	
	\begin{equation}
		\ordine{x^4} = \dfrac{\ordine{x}}{\MCD{\ordine{x}}{4}} = \dfrac{6}{\MCD{6}{4}} = \dfrac{6}{2} = 3
	\end{equation}
\end{dimostrazione}

\section[Studiare i sottogruppi di un gruppo ciclico]{Studiare i sottogruppi di un gruppo ciclico\footnote{\cite[1 ottobre 2021]{lucchini}}}

\begin{teorema}
	Se $G$ è ciclico e $H \le G$, allora $H$ è ciclico.
\end{teorema}
\begin{dimostrazione}
	Il sottogruppo 1 è ciclico.
	
	Prendiamo un sottogruppo $H \ne 1$. Chiamiamo $g$ il generatore di $G$.
	
	Se $H \ne 1$ allora:
	
	\begin{gather}
		\exists h \ne 1, h \in H \\
		\exists m \in \Z, m \ne 0 \taleche h = g^m \in H
	\end{gather}

	Ma se $g^m \in H$ anche $(g^m)^{-1} \in H$ perché $H$ è un sottogruppo. Quindi posso supporre $m$ intero positivo. Uso allora la proprietà del minimo e scelgo $m$ come il minimo per cui $g^m \in H$.
	
	Prendo ora un altro elemento $\bar{h} \in H$:
	
	\begin{equation}
		\exists t \taleche \bar{h} = g^t
	\end{equation}

	Diviso $t$ per $m$:
	
	\begin{gather}
		t = mq + r \text{ con } 0 \le r < m \\
		\bar{h} = g^{mq + r} = (g^m)^q \cdot g^r \\
		g^r = (g^m)^{-q} \cdot \bar{h}
	\end{gather}

	Ma $g^m \in H$, quindi anche $(g^m)^{-q} \in H$, inoltre $\bar{h} \in H$, quindi anche $g^r \in H$.
	
	Ma $r < m$, mentre $m$ era il più piccolo intero positivo tale che $g^m \in H$.
	
	Quindi:
	
	\begin{gather}
		r = 0 \\
		m|t \forall \bar{h} = g^t \in H \\
		\bar{h} \in \gen{g^m} \\
		H = \gen{g^m}
	\end{gather}
\end{dimostrazione}

\section{Quanti sono i sottogruppi di un gruppo ciclico?}

Definiamo la funzione $\alpha$:

\begin{gather}
	\alpha: \N \longrightarrow \{H \taleche H \le G = \gen{g}\} \\
	m \longmapsto \gen{g^m}
\end{gather}

$\alpha$ è suriettiva perché ogni sottogruppo si scrive come $H = \gen{g^m}$. Si aggiunge il sottogruppo identico $1 = \gen{g^0}$.

\begin{teorema}
	Se $\ordine{g} = \infty$ allora i sottogruppi di $G = \gen{g}$ sono in corrispondenza biiettiva con i naturali.
\end{teorema}
\begin{dimostrazione}
	Discutiamo l'iniettività di $\alpha$:
	
	\begin{align}
		\gen{g^{m_1}} = \gen{g^{m_2}} &\Longrightarrow \gen{g^{m_1}} \le \gen{g^{m_2}} \\
		&\Longrightarrow g^{m_1} = (g^{m_2})^{t_2}
	\end{align}

	Ma poiché $\ordine{g} = \infty$, allora:
	
	\begin{equation}
		m_1 = m_2 t_2 \Longrightarrow m_2 | m_1
	\end{equation}

	Abbiamo anche l'inclusione opposta da cui ricaviamo che $m_1 | m_2$.
	
	Quindi $m_1 = m_2$.
\end{dimostrazione}

Dato un gruppo ciclico $G = \gen{g}$ e un suo sottogruppo $H = \gen{g^m}$, anche $G/H$ è un sottogruppo di G, quindi è ciclico. Qual è l'ordine di tale sottogruppo? E' la più piccola potenza di $g$ in $H$, ovvero $m$:
\footnote{Sono molto perplesso su questa osservazione. Potrei essere d'accordo nel dire che il gruppo quoziente è isomorfo ad un sottogruppo, ma poi come vado a giustificare tutto il resto?}

\begin{equation}
	\ordine{G/H} = m
\end{equation}

\begin{teorema}
	\label{thr:ciclici_finiti_sottogruppi}
	Se $\ordine{g} = n$ allora i sottogruppi di $G = \gen{g}$ sono in corrispondenza biiettiva con i divisori di $n$.
\end{teorema}
\begin{dimostrazione}
	Consideriamo un generico $H \le G$. Allora esiste un intero positivo $m$ tale che $H = \gen{g^m}$ ed $m$ è il più piccolo intero positivo tale che $g^m \in H$.
	
	C'è un legame tra $m$ e $n$?
	
	Dividiamo $n$ per $m$:
	
	\begin{gather}
		n = mq + r \quad 0 \le r < m \\
		1 = g^n = g^{mq+r} = g^{mq} \cdot g^r \\
		\Longrightarrow g^r = (g^{mq})^{-1}
	\end{gather}

	Ma se $g^m \in H$, anche $g^{mq} \in H$ e anche $(g^{mq})^{-1} \in H$. Quindi:
	
	\begin{equation}
		g^r \in H \Longrightarrow r = 0 \Longrightarrow m|n
	\end{equation}

	Facciamo meglio la mappa:
	
	\begin{align}
		\text{divisori di }n &\longrightarrow \text{sottogruppi di }G \\
		m &\longmapsto \gen{g^m}
	\end{align}

	Anche questa mappa è suriettiva. Ma è anche iniettiva?
	
	\begin{align}
		\ordine{\gen{g^m}} &= \ordine{g^m} & \text{per definizione di ordine} \\
		&= \dfrac{\ordine{g}}{\MCD{\ordine{g}}{m}} & \text{per teorema~\ref{thr:Ciclici_ordine_elementi}} \\
		&= \dfrac{n}{m}
	\end{align}

	Quindi la mappa è iniettiva (figura~\ref{fig:Ciclici_divisori_di_n}).
		
\end{dimostrazione}

\begin{figure}[tp]
	\centering
	\tikz {
		\node (a) at (0,2) {$G = \gen{g}$};
		\node (b) at (0,0) {$H = \gen{g^m}$};
		\node at (2,2) {ordine $n$};
		\node at (2,0) {ordine $\dfrac{n}{m}$};
		\draw (a) edge[-] (b);
		\draw[decorate,decoration={brace,raise=100pt}] (a) -- (b) node[pos=.5,right=103pt,black]{$\indice{G}{H} = m$};
	}
	\caption{Sottogruppo di un gruppo ciclico}
	\label{fig:Ciclici_divisori_di_n}
\end{figure}

Di conseguenza, se un gruppo ciclico è finito, non esistono due sottogruppi con uguale ordine.

\section{Quante scelte abbiamo per i generatori?}

Ovvero, dato $G = \gen{g}$, prendiamo un generico $g^k \in G$: è vero che $\gen{g^k} = G$?

\textbf{NB: } $\gen{g^k} = \gen{g^{-k}}$: se $g^k$ è un generatore, anche $g^{-k}$ è generatore dello stesso gruppo, quindi mi posso occupare dei soli positivi.

\begin{teorema}
	\label{thr:generatori_gruppi_ciclici_infiniti}
	Se $G = \gen{g}$ e $\ordine{g} = \infty$, allora $\{g, g^{-1}\}$ sono gli unici generatori.
\end{teorema}
\begin{dimostrazione}
	Dal momento che $\alpha$ è iniettiva, si ha:
	
	\begin{equation}
		\gen{g^m} = G \Longrightarrow m = 1
	\end{equation} 

	A questo aggiungo anche l'inverso, con $m = -1$.
\end{dimostrazione}

Visto che $G=\gen{g}$ con $\ordine{g} = \infty$ risulta $G \isomorfo \Z$, allora 1 e -1 sono gli unici generatori di $\Z$.

\begin{teorema}
	\label{thr:generatori_gruppi_ciclici_finiti}
	Se $G = \gen{g}$ e $\ordine{g} = n$, allora i generatori di $G$ sono tutti e soli i $G^k$ con $1 \le k < n$ e $\MCD{k}{n} = 1$ (coprimi).
\end{teorema}
\begin{dimostrazione}
	Quando $\gen{g^k} = G$?
	
	\begin{align}
		\gen{g^k} = G &\Longrightarrow \ordine{g^k} = \ordine{G} \\
		&\Longrightarrow \dfrac{n}{\MCD{n}{k}} = n \\
		&\Longrightarrow \MCD{n}{k} = 1
	\end{align}
\end{dimostrazione}

Chiamiamo $\varphi(n)$ la \textbf{funzione di Eulero}, che restituisce il numero di coprimi di $n$ tra gli interi positivi minori di $n$.

\textbf{Il numero di generatori di $G$ con $\ordine{G} = n$ è $\varphi(n)$.}

\begin{esercizio}
	Dato $G = \gen{g}$ con $\ordine{G} = 20$. Quali sono i sottogruppi? Quali i generatori?
\end{esercizio}
\begin{soluzione}
	I sottogruppi sono in corrispondenza con i divisori dell'ordine del gruppo.
	
	I generatori di ogni sottogruppo sono le potenze coprime con l'ordine del sottogruppo.
	
	L'ordine dei sottogruppi è dato dall'ordine del gruppo diviso per la potenza dei generatori.
	
	\begin{center}		
		\begin{tabular}{cccc}
			$\gen{g}$ & ordine 20 & $\varphi(20) = 8$ & $g, g^3, g^7, g^9, g^{11}, g^{13}, g^{17}, g^{19}$ \\
			$\gen{g^2}$ & ordine 10 & $\varphi(10) = 4$ & $g^2, g^6, g^{14}, g^{18}$ \\
			$\gen{g^4}$ & ordine 5 & $\varphi(5) = 4$ & $g^4, g^8, g^{12}, g^{16}$ \\
			$\gen{g^5}$ & ordine 4 & $\varphi(4) = 2$ & $g^5, g^{15}$ \\
			$\gen{g^{10}}$ & ordine 2 & $\varphi(2) = 1$ & $g^{10}$ \\
			$\gen{g^{20}}$ & ordine 1 & $\varphi(1) = 1$ & $g^{20} = 1$ \\
			\bottomrule
			& & 20 &
		\end{tabular}
	\end{center}

	Ad ogni elemento di $G$ abbiamo associato il sottogruppo che genera.
\end{soluzione}

\section{Piccola sintesi}

Dato un gruppo ciclico $C_n$ di ordine $n$:

\begin{itemize}
	\item le potenze divisori dell'ordine $n$ corrispondono ai sottogruppi;
	\item le potenze coprime dell'ordine $n$ corrispondono ai generatori.
\end{itemize}

Consideriamo $C_{12}$:

\begin{center}
	\begin{tabular}{cccc}
		\toprule
		$g^1$ & divisore e coprimo & sottogruppo = $C_{12}$ & generatore \\
		\midrule
		$g^2$ & divisore & sottogruppo & \\
		\midrule
		$g^3$ & divisore & sottogruppo & \\
		\midrule
		$g^4$ & divisore & sottogruppo & \\
		\midrule
		$g^5$ & coprimo & & generatore \\
		\midrule
		$g^6$ & divisore & sottogruppo & \\
		\midrule
		$g^7$ & coprimo & & generatore \\
		\midrule
		$g^8$ & & & \\
		\midrule
		$g^9$ & & & \\
		\midrule
		$g^{10}$ & & & \\
		\midrule
		$g^{11}$ & coprimo & & \\
		\midrule
		$g^{12}$ & divisore & sottogruppo = 1 & \\
		\bottomrule
	\end{tabular}
\end{center}

Sottogruppi:

\begin{center}
	\begin{tabular}{cccc}
		\toprule
		Sottogruppo & ordine & numero generatori & lista generatori \\
		\midrule
		$\gen{g} = C_{12}$ & 12 & $\varphi(12) = 4$ & $g, g^5, g^7, g^{11}$ \\
		$\gen{g^2}$ & 6 & $\varphi(6) = 2$ & $g^2, g^{2 \cdot 5} = g^{10}$ \\
		$\gen{g^3}$ & 4 & $\varphi(4) = 2$ & $g^3, g^{3 \cdot 3} = g^9$ \\
		$\gen{g^4}$ & 3 & $\varphi(3) = 2$ & $g^4, g^{4 \cdot 2} = g^8$ \\
		$\gen{g^6}$ & 2 & $\varphi(2) = 1$ & $g^6$ \\
		$\gen{g^{12}} = 1$ & 1 & $\varphi(1) = 1$ & $g^{12} = 1$ \\
		\bottomrule 
	\end{tabular}
\end{center}

\section{Gruppi privi di sottogruppi non banali}

\begin{teorema}
	I gruppi privi di sottogruppi non banali sono ciclici e di ordine primo.
\end{teorema}
\begin{dimostrazione}
	Prendiamo un elemento $g \in G$ con $g \ne 1$.
	
	Consideriamo $H = \gen{g}$: è un sottogruppo di $G$, non può essere il sottogruppo identico, quindi non può che coincidere con il gruppo $G$. Quindi:
	
	\begin{equation}
		G = H = \gen{g}
	\end{equation}

	Quindi $G$ è un gruppo ciclico.
	
	Se $\ordine{G} = \infty$ allora ha infiniti sottogruppi, che non è accettabile. Quindi $G$ deve avere ordine finito.
	
	Se $\ordine{G} = m$, allora i sottogruppi sono in corrispondenza con i divisori di $m$. Ma noi vogliamo solo 2 sottogruppi, quindi $m$ deve essere un numero primo.
\end{dimostrazione}

\section{Esponente}

Sia dato un gruppo $G$ finito.

Definiamo \textbf{esponente} di $G$ ($\exp(G)$) il più piccolo intero positivo $m$ tale che:

\begin{equation}
	g^m = 1 \,\, \forall g \in G	
\end{equation}
 
Ovvero:

\begin{equation}
	\ordine{g} \text{ divide } \exp(G) \,\, \forall g \in G
\end{equation}

Ovvero:

\begin{equation}
	\exp(G) = \mcm(\ordine{g_0}, \ordine{g_1}, \dots, \ordine{g_n}) \,\, \text{con } g_i \in G
\end{equation}

\begin{esercizio}
	Calcolare $\exp(S_4)$.
\end{esercizio}
\begin{soluzione}
Elenchiamo le tipologie di elementi di $S_4$:

\begin{center}
	\begin{tabular}{ccc}
		unità & 1 & ordine 1 \\
		scambi & (12) & ordine 2 \\
		doppi scambi & (12)(34) & ordine 2 \\
		3-cicli & (123) & ordine 3 \\
		4-cicli & (1234) & ordine 4
	\end{tabular}
\end{center}

Quindi:

\begin{equation}
	\exp(S_4) = \mcm(2, 3, 4) = 12
\end{equation}
\end{soluzione}

\begin{teorema}
	\label{thr:esponente}
	Se $G$ è un gruppo abeliano finito, allora
	
	\begin{equation}
		G \text{ ciclico } \Longleftrightarrow \exp(G) = \ordine{G}
	\end{equation}
\end{teorema}
\begin{dimostrazione}
	Vedi \cite[pag. 46-47]{jacobson}.
\end{dimostrazione}

Chiavi di lettura del precedente teorema:

\begin{gather}
	G \text{ ciclico e finito } (\Longrightarrow G \text{ abeliano}) \Longrightarrow \exp(G) = \ordine{G} \\
	G \text{ abeliano e finito} \land \exp(G) = \ordine{G} \Longrightarrow G \text{ ciclico }
\end{gather}

\begin{teorema}
	Sia $K$ un campo e sia $G$ un sottogruppo del gruppo moltiplicativo $K^*$. Se $G$ è finito, allora $G$ è ciclico.
\end{teorema}
\begin{dimostrazione}
	In un campo la moltiplicazione è commutativa, quindi $G$ è abeliano.
	
	Quindi $G$ è abeliano e finito e posso usare il teoremo~\ref{thr:esponente}.
	
	Chiamiamo:
	
	\begin{gather}
		e = \exp(G) \\
		n = \ordine{G}
	\end{gather}

	In un gruppo finito l'esponente divide l'ordine:\footnote{l'ordine di $G$ e l'ordine di $g$ possono essere diversi $\forall g \in G$.}
	
	\begin{equation}
		e|n \Longrightarrow e \le n
	\end{equation}

	Prendo il polinomio che ha coefficienti nel campo:
	
	\begin{equation}
		x^e - 1 \in K[x]
	\end{equation}

	$e$ è l'esponente, quindi $g^e = 1 \forall g \in G$, quindi il polinomio ammette come radici almeno tutti gli elementi di $g$.
	
	Usiamo il teorema di Ruffini: il numero di radici di un polinomio è minore o uguale al grado, quindi:
	
	\begin{equation}
		n \le e
	\end{equation}

	Quindi:
	
	\begin{equation}
		n = e
	\end{equation}

	Quindi, per il teorema~\ref{thr:esponente}, $G$ è ciclico.
\end{dimostrazione}

\begin{esercizio}
	Se $H$ è un sottogruppo finitamente generato di $(\Q, +)$, allora $G$ è ciclico.
\end{esercizio}
\begin{soluzione}
	Battezziamo i generatori di $H$:
	
	\begin{equation}
		H = \gen{\dfrac{a_1}{b_1}, \dfrac{a_2}{b_2}, \dots, \dfrac{a_r}{b_r}}
	\end{equation}

	Prendiamo:
	
	\begin{equation}
		h = \dfrac{1}{b_1 \cdot b_2 \cdot \dots \cdot b_r}
	\end{equation}

	Quindi:
	
	\begin{equation}
		\dfrac{a_i}{b_i} = h \cdot a_i \cdot b_1 \cdot \dots \cdot b_{i-1} \cdot b_{i+1} \cdot \dots \cdot b_r
	\end{equation}

	Quindi ogni generatore è un multiplo di $h$ (attenzione che stiamo usando una notazione additiva):
	
	\begin{equation}
		\dfrac{a_i}{b_i} \in \gen{h}
	\end{equation}

	Quindi:
	
	\begin{equation}
		H \le \gen{h}
	\end{equation}

	Ma i sottogruppi di gruppi ciclici sono ciclici, quindi $H$ è ciclico.
\end{soluzione}

\section{Teorema cinese del resto}

\begin{teorema}
	Se $G$ è un gruppo ciclico di ordine $n$, allora $G$ è isomorfo ad un prodotto diretto di gruppi ciclici di ordine potenza di primo.
\end{teorema}
\begin{dimostrazione}
	Scomponiamo l'ordine $n$ in fattori primi:
	
	\begin{equation}
		n = p_1^{a_1} \cdot \dots \cdot p_t^{a_t}
	\end{equation}

	Posso quindi considerare la quantità:
	
	\begin{equation}
		q_i = \dfrac{n}{p_i{a_i}}
	\end{equation}

	che corrisponde al prodotto di tutti i fattori tranne quello $i$-esimo.
	
	Se ora battezziamo $g$ il generatore del gruppo ciclico $G$, chiamiamo:
	
	\begin{equation}
		g_i := g^{q_i}
	\end{equation}

	Ciascuno di questi elementi ha ordine:
	
	\begin{equation}
		\ordine{g_i} = \dfrac{n}{q_i} = p_i^{a_i}
	\end{equation}

	Consideriamo allora il prodotto diretto:
	
	\begin{equation}
		\gen{g_1} \times \dots \times \gen{g_t}
	\end{equation}

	Definisco la seguente funzione:
	
	\begin{align}
		\gamma: \gen{g_1} \times \dots \times \gen{g_t} &\longrightarrow G
		(x1, \dots, x_t) \longmapsto x_1 \dots x_t
	\end{align}

	$\gamma$ è l'isomorfismo che cerco?
	
	$\gamma$ è un omomorfismo, infatti:
	
	\begin{align}
		\gamma(x_1, \dots, x_t) &= x1 \cdot \dots \cdot x_t \cdot y_1 \cdot \dots \cdot y_t = \\
		&= x_1y_1 \cdot \dots \cdot x_ty_t =  \\
		&= \gamma(x_1y_1, \dots, x_ty_t) = \\
		&= \gamma((x_1, \dots, x_t) \cdot (y_1, \dots, y_t))
	\end{align}

	Devo dimostrare che $\gamma$ è un isomorfismo. Ma dominio e codominio hanno la stessa cardinalità (infatti la cardinalità del dominio è il prodotto delle cardinalità dei gruppi $\gen{g_i}$). Quindi basta dimostrare l'iniettività o la suriettività.
	
	Dimostriamo la suriettività.
	
	Per come li ho costruiti l'MCD dei numeri $q_i$ è pari ad 1. Allora l'MCD si può scrivere come combinazione lineare a coefficienti interi degli elementi considerati:
	
	\begin{equation}
		1 = \sum_i r_i q_i \quad \text{con } r_i \in \Z
	\end{equation}

	Quindi:
	
	\begin{align}
		g &= g^1 = \\
		&= g^{\sum_i r_i q_i} = \\
		&= \prod_i(g^{q_i})^{r_i} = \\
		&= \prod_i(g_i)^{r_i} = \\
		&= \gamma(g_1^{r_1}, \dots, g_t^{r_t}) \\
		\Longrightarrow g &\in \im(\gamma)
	\end{align}

	Quindi $\gamma$ è suriettiva, quindi anche iniettiva.
\end{dimostrazione}
Questo teorema corrisponde al:

\begin{teorema}[Teorema cinese del resto]
	\begin{equation}
		\Z/m\Z \isomorfo \Z/p_1^{a_1}\Z \times \dots \times \Z/p_t^{a_t}\Z
	\end{equation}
\end{teorema}
	\chapter{Laterali}
\label{ch:laterali}

\section{Laterali destri e sinistri}
\label{sec:laterali}

Dato un sottogruppo $H$ di un gruppo $G$, possiamo definire questa relazione d'equivalenza:
\begin{equation*}
	g_1 \sim g_2 \Longleftrightarrow \exists h \in H \taleche g_2 = g_1 h
\end{equation*}

Le classi di equivalenza di questa relazione si chiamano \textbf{laterali sinistri} e sono del tipo $xH$: prendo un elemento $x$ di $G$ e lo moltiplico per tutti gli elementi di $H$.

Il numero di classi di equivalenza si chiama \textbf{indice} e si indica con $\indice{G}{H}$.

L'indice dei laterali sinistri è uguale all'indice dei laterali destri.

\section{Trasversali}
\label{sec:trasversali}

Un \textbf{trasversale sinistro} è un insieme $T$ di rappresentanti per le classi laterali sinistre.

I laterali sinistri formano una partizione del gruppo, quindi possiamo scrivere:
\begin{equation*}
	G = \bigcup^\circ_{t \in T} tH
\end{equation*}

Un trasversale sinistro non è necessariamente un trasversale destro.

Consideriamo per esempio il gruppo $S_3$ (vedi capitolo~\ref{ch:S3}) e il suo sottogruppo:
\begin{equation*}
	H=\gen{(23)} = \{(1), (23)\}
\end{equation*}

I laterali sinistri di $H$ in $S_3$ sono:
\begin{gather*}
	1H = \{(1), (23)\} = (23)H \\
	(12)H = \{(12), (123)\} = (123)H \\
	(13)H = \{(13), (132)\} = (132)H
\end{gather*}

Mentre i laterali destri di $H$ in $S_3$ sono:
\begin{gather*}
	H1 = \{(1), (23)\} = H(23) \\
	H(12) = \{(12), (132)\} = H(132) \\
	H(13) = \{(13), (123)\} = H(123)
\end{gather*}

Quindi l'insieme
\begin{equation*}
	T = \{(1), (13), (123)\}
\end{equation*}

è un trasversale sinistro, ma non un trasversale destro.

\begin{teorema}
	Se $T$ è un trasversale sinistro, allora $T^{-1}$ è un trasversale destro, dove:
	\begin{equation*}
		T^{-1} = \{t^{-1} \taleche t \in T\}
	\end{equation*}
\end{teorema}
\begin{dimostrazione}
	Partiamo dall'unione disgiunta dei trasversali sinistri e applichiamo l'inverso:
	\begin{gather*}
		G = \bigcup^\circ_{t \in T} tH \\
		G^{-1} = \bigcup^\circ_{t \in T} (tH)^{-1} \\
		G = \bigcup^\circ_{t \in T} H^{-1}t^{-1} \\
		G = \bigcup^\circ_{t \in T} Ht^{-1}
	\end{gather*}

	Dove abbiamo potuto considerare che $G = G^{-1}$ e $H = G^{-1}$ in quanto gruppi.
\end{dimostrazione}

Abbiamo quindi dimostrato che:

\begin{teorema}
	L'indice, sia che lavori a destra sia che lavori a sinistra, è sempre lo stesso.
\end{teorema}

\section{Sottogruppo normale}
\label{sec:sottogruppo_normale}

Un sottogruppo $N \le G$ è \textbf{normale} se si ha:
\begin{equation*}
	gNg^{-1} = N \quad \forall g \in G
\end{equation*}

oppure, che è equivalente:
\begin{equation*}
	gN = Ng \quad \forall g \in G
\end{equation*}

Si indica con:
\begin{equation*}
	N \normale G
\end{equation*}

Se $N$ è un sottogruppo normale, un laterale destro è anche laterale sinistro, e un trasversale destro è anche trasversale sinistro.

\section{Sottogruppo di sottogruppo}
\label{sec:sottogruppo di sottogruppo}

\begin{teorema}
	\label{thr:fattorizzazione_indici}
	Se $K \le H \le G$, allora
	\begin{equation*}
		\indice{G}{K} = \indice{G}{H} \cdot \indice{H}{K}
	\end{equation*}
\end{teorema}
\begin{dimostrazione}
	Chiamo $T$ un trasversale sinistro di $H$ in $G$.
	
	Chiamo $U$ un trasversale sinistro di $K$ in $H$.
	
	Vogliamo trovare un trasversale sinistro di $K$ in $G$:
	\begin{gather*}
		G = \bigcup^\circ_{t \in T} tH \quad\land\quad H = \bigcup^\circ_{u \in U} uK \\
		G = \bigcup^\circ_{t \in T} t \biggl(\bigcup^\circ_{u \in U} uK\biggr) \\
		G = \bigcup^\circ_{t \in T, u \in U} tuK
	\end{gather*}

	Quindi $TU$ è un trasversale sinistro di $K$ in $G$.
	
	Quindi:
	\begin{equation*}
		\indice{G}{K} = \ordine{TU} = \ordine{T} \cdot \ordine{U} = \indice{G}{H} \cdot \indice{H}{K}
	\end{equation*}

	Ovvero: \emph{il sottogruppo intermedio fattorizza l'indice}.
\end{dimostrazione}

\section{Intersezione di sottogruppi}
\label{sec:laterali_intersezione_sottogruppi}

\begin{esercizio}
	Dati due sottogruppi $H$ e $K$ di $G$ (figura~\ref{fig:Laterali_intersezione_di_sottogruppi}):
	\begin{equation*}
		\indice{G}{H \cap K} \le \indice{G}{H} \cdot \indice{G}{K}
	\end{equation*}
\end{esercizio}
\begin{figure}[tp]
	\centering
	\tikz {
		\node (a) at (2,4) {$G$};
		\node (b) at (0,2) {$H$};
		\node (d) at (4,2) {$K$};
		\node (e) at (2,0) {$H \cap K$};
		\draw (a) edge[->] (b) 
		(a) edge[->] (d) 
		(b) edge[->] (e)
		(d) edge[->] (e);
	}
	\caption{Intersezione di sottogruppi.}
	\label{fig:Laterali_intersezione_di_sottogruppi}
\end{figure}
\begin{soluzione}
	Siano $h_1, h_2 \in H$ tali che:
	\begin{equation*}
		h_1(H \cap K) \ne h_2(H \cap K)
	\end{equation*}

	Ovvero $h_1$ e $h_2$ stanno in laterali dell'intersezione diversi.
	\begin{gather*}
		h_2^{-1}h_1 \not\in H \cap K \\
		h_2^{-1}h_1 \not\in K \quad \text{perché } h_1, h_2 \in H
		h_1 K \ne h_2 K
	\end{gather*}

	Quindi $h_1$ e $h_2$ stanno in laterali di $K$ diversi.
	Quindi:
	\begin{equation*}
		\indice{H}{H \cap K} \le \indice{G}{K}
	\end{equation*}

	Quindi, per il teorema~\ref{thr:fattorizzazione_indici}, si ha:
	\begin{gather*}
		\indice{G}{H \cap K} = \indice{G}{H} \cdot \indice{H}{H \cap K} \\
		\indice{G}{H \cap K} \le \indice{G}{H} \cdot \indice{G}{K}
	\end{gather*}

	L'indice dell'intersezione è minore o uguale al prodotto degli indici.
\end{soluzione}

In generale l'uguaglianza non c'è, vedi per esempio la figura~\ref{fig:Laterali_intersezione_di_sottogruppi_esempio_s3}.

\begin{figure}[tp]
	\centering
	\tikz {
		\node (a) at (2,4) {$G = S_3$};
		\node (b) at (0,2) {$H = \gen{(12)}$};
		\node (d) at (4,2) {$K = \gen{(13)}$};
		\node (e) at (2,0) {$H \cap K = 1$};
		\draw (a) edge[->] (b); 
		\draw (a) edge[->] (d);
		\draw (b) edge[->] (e);
		\draw (d) edge[->] (e);
		\draw[decorate,decoration={brace,raise=40pt}] (0,2) -- (0,4) node[pos=.5,right=-55pt,black]{3};
		\draw[decorate,decoration={brace,raise=40pt}] (4,4) -- (4,2) node[pos=.5,right=43pt,black]{3};
		\draw[decorate,decoration={brace,raise=60pt}] (4,4) -- (4,0) node[pos=.5,right=63pt,black]{6};
	}
	\caption{Intersezione di sottogruppi: esempio con $S_3$.}
	\label{fig:Laterali_intersezione_di_sottogruppi_esempio_s3}
\end{figure}

\begin{teorema}
	Se $\indice{G}{H}$ e $\indice{G}{K}$ sono coprimi, allora
	
	\begin{equation}
		\indice{G}{H \cap K} = \indice{G}{H} \cdot \indice{G}{K}
	\end{equation}
\end{teorema}
\begin{dimostrazione}
	Per il teorema~\ref{thr:fattorizzazione_indici} si ha:
	
	\begin{gather}
		\indice{G}{H} \text{ divide } \indice{G}{H \cap K} \\
		\indice{G}{K} \text{ divide } \indice{G}{H \cap K}
	\end{gather}	

	Quindi:
	
	\begin{gather}
		\indice{G}{H} \cdot \indice{G}{K} \text{ divide } \indice{G}{H \cap K} \\
		\indice{G}{H} \cdot \indice{G}{K} \le \indice{G}{H \cap K}
	\end{gather}

	Ma nell'esercizio precedente abbiamo verificato che:
	
	\begin{equation}
		\indice{G}{H} \cdot \indice{G}{K} \ge \indice{G}{H \cap K}
	\end{equation}

	Quindi:
	
	\begin{equation}
		\indice{G}{H} \cdot \indice{G}{K} = \indice{G}{H \cap K}
	\end{equation}
\end{dimostrazione}

\section{Prodotto di sottogruppi}

\begin{teorema}
	\label{thr:prodotto_sottogruppi}
	Siano $H$ e $K$ due sottogruppi di $G$. Definiamo:
	
	\begin{equation}
		HK = \{hk \taleche h \in H, k \in K\}
	\end{equation}

	Si ha:

	\begin{equation}
		HK \le G \Longleftrightarrow HK = KH
	\end{equation}

	In particolare se $N$ è un sottogruppo normale di $G$, allora $NH = HN$ è un sottogruppo di $G$ per ogni $H \le G$.

\end{teorema}
\begin{dimostrazione}
	Partiamo da un controesempio. Se prendiamo i seguenti sottogruppi di $S_3$:
	
	\begin{equation}
		H = \{1, (12)\}, K = \{1, (13)\}
	\end{equation}
	
	otteniamo il prodotto di sottogruppi:
	
	\begin{equation*}
		HK = \{1, (12), (13), (123)\}
	\end{equation*}
	
	Questo insieme ha cardinalità 4, che non è un divisore dell'ordine di $S_3$ (che è 6), quindi,
	per il  teorema di Lagrange~\ref{thr:Laterali_Lagrange}, non è un sottogruppo di $S_3$.
	
	Supponiamo ora che $HK \le G$: allora si ha:
	
	\begin{equation*}
		HK = (HK)^{-1} = K^{-1}H^{-1} = KH
	\end{equation*}
	
	E questo dimostra la prima parte del teorema.
	
	Inoltre se $N$ è un sottogruppo normale si ha:
	
	\begin{equation*}
		nh = hh^{-1}nh = h(h^{-1}nh) = hn^* \quad \text{con } n^* \in N
	\end{equation*}
\end{dimostrazione}

\begin{teorema}
	\label{thr:ordine_prodotto_sottogruppi}
	Se $H$ e $K$ sono finiti, allora:

	\begin{equation*}
		\ordine{HK} = \dfrac{\ordine{H}\ordine{K}}{\ordine{H \cap K}}
	\end{equation*}
\end{teorema}

\begin{esercizio}
	Sia $G$ è un gruppo, $H \le G$, $N \normale G$ e:
	
	\begin{gather}
		HN = G \\
		H \cap N = 1
	\end{gather}

	Ovvero $H$ è il complementare di $N$.
	
	Esiste una funzione $\gamma: H \times N \longrightarrow G$ biiettiva, ma tale funzione non è un isomorfismo.
\end{esercizio}
\begin{soluzione}
	Se $HN = G$ significa che:
	
	\begin{equation*}
		\forall g \in G \, \exists h \in H, k \in K \taleche g = hn
	\end{equation*}

	Presi due elementi $h_1, h_2 \in H$ e due elementi $n_1, n_2 \in N$:
	
	\begin{equation*}
		h_1n_1 = h_2n_2 \Longrightarrow h_2^{-1}h_1 = n_2n_1^{-1}
	\end{equation*}

	Ma $h_2^{-1}h_1$ è un elemento di $H$ e $n_2n_1^{-1}$ è un elemento di $N$. E dal momento che questi due elementi sono in verità lo stesso elemento, allora tale valore sta anche in $H \cap N$. Ma nell'intersezione c'è solo l'unità, quindi:
	
	\begin{gather*}
		h_2^{-1}h_1 = 1 \Longrightarrow h_1 = h_2 \\
		n_2 n_1^{-1} = 1 \Longrightarrow n_1 = n_2
	\end{gather*}

	Questo significa che il prodotto $g = hn$ è unico per ogni elemento $g$.
	
	Quindi la funzione:
	
	\begin{align}
		\gamma: H \times N &\longrightarrow G \\
		(h, n) &\longmapsto hn
	\end{align}

	è suriettiva e iniettiva, quindi è biiettiva.
	
	Tuttavia non è garantito che $\gamma$ sia un isomorfismo.
	Infatti $G$ potrebbe non essere abeliano, $H$ e $K$ potrebbero essere abeliani quindi anche il prodotto $H \times K$ sarebbe abeliano.
	
	Per esempio se considero:
	\begin{gather*}
		G = S_3 \text{ non abeliano} \\
		H = \gen{(12)} \text{ ciclico quindi abeliano} \\
		N = \gen{(123)} \text{ ciclico quindi abeliano}
	\end{gather*}

	abbiamo che:
	\begin{gather*}
		HN = S_3 \\
		H \cap N = 1
	\end{gather*}

	Ma $H \times N$ è abeliano (perché prodotto diretto di gruppi abeliani), mentre $G$ non lo è.
	
	Questo significa che la funzione non rispetta l'operazione di composizione.
	Per esempio:
	
	\begin{gather}
		((12), (123)) \cdot ((12), 1) = (1, (123)) \\
		((12), 1) \cdot ((12), (123)) = (1, (123))
	\end{gather}

	Quindi la composizione di questi elementi del prodotto diretto è commutativa.
	In genere all'interno del prodotto diretto la composizione è commutativa perché ho sempre composizioni con l'unità
	o tra due elementi che sono inversi.
	
	Se passo alle loro immagini ho:
	
	\begin{gather}
		((12), (123)) \longmapsto (12) \cdot (123) = (23) \\
		((12), 1) \longmapsto (12) \cdot 1 = (12)
	\end{gather}

	E:
	
	\begin{gather}
		(12) \cdot (23) = (123) \\
		(23) \cdot (12) = (132)
	\end{gather}

	Quindi tra questi due elementi di $S_3$ il prodotto non è commutativo.
	
\end{soluzione}

\section{Sottogruppi di indice 2}

\begin{teorema}
	\label{thr:sottogruppi_di_indice_2}
	Si $H$ un sottogruppo di indice 2 di un gruppo $G$. Allora $H$ è normale.
\end{teorema}
\begin{dimostrazione}
	Dal momento che l'indice di $H$ è pari a 2, ogni trasversale sinistro contiene due elementi, quindi ci sono due laterali sinistri: $H$ e $xH$.
	
	Questi due insiemi sono disgiunti perché:
	
	\begin{equation}
		h_1x = h_2 \Longrightarrow x = h_1^{-1}h_2 \in H
	\end{equation}

	Quindi:
	
	\begin{equation}
		G = H \overset{\circ}{\cup} xH
	\end{equation}

	Ma lo stesso ragionamento lo possiamo fare anche con i laterali destri e ottenere:
	
	\begin{equation}
		G = H \overset{\circ}{\cup} Hx
	\end{equation}

	Da questo deduciamo che $xH$ e $Hx$ sono lo stesso insieme, quindi $H$ è normale.
\end{dimostrazione}	
	\chapter{Conseguenze del teorema di Cayley}

Se abbiamo un gruppo $G$ finito e un suo sottogruppo $H$, per il teorema di Lagrange~\ref{thr:Laterali_Lagrange} l'ordine di $H$ divide l'ordine di $G$.

Si può invertire questo teorema?

Ovvero: dato un gruppo $G$ finito di ordine $n$ e scelto un $m$ che divide $n$, esiste un $H \le G$ tale che $\ordine{H} = m$?

Con i gruppi ciclici funziona: esiste uno ed un solo sottogruppo per ogni divisore (vedi\ref{thr:ciclici_finiti_sottogruppi}).

Con i gruppi abeliani funziona.

Controesempio: $A_4$ ha ordine 12 ($4!/2$) ma non ha alcun sottogruppo di ordine 6.

\begin{lemma}
	\label{lmm:per_conseguenza_Cayley}
	Sia $G$ un gruppo di ordine $m$ pari. Allora $G$ contiene almeno un elemento di ordine 2.
\end{lemma}
\begin{dimostrazione}
	Sia:
	
	\begin{equation}
		I = \{g \in G \taleche g = g^{-1}\}
	\end{equation}

	Se $g \not\in I$, allora anche $g^{-1} \not\in I$, quindi ci sono due elementi che appartengono alla differenza $G - I$. Questo vale per ogni elemento che non appartiene ad $I$, quindi:
	
	\begin{equation}
		\ordine{I} \equiv \ordine{G} \mod 2
	\end{equation}

	Quindi se $\ordine{G}$ è pari, anche $\ordine{I}$ è pari.
	
	$I$ non è vuoto, in quanto $1 \in I$. Quindi deve contenere almeno un altro elemento $x \ne 1$. Tale elemento ha ordine 2, infatti:
	
	\begin{equation}
		x^2 = xx = xx^{-1} = 1
	\end{equation}
\end{dimostrazione}

\begin{teorema}
	Sia $G$ un gruppo con $\ordine{G} = 2m$, con $m$ dispari. $G$ contiene almeno un sottogruppo di ordine $m$.
\end{teorema}
\begin{dimostrazione}
	Partiamo dal teorema di Cayley~\ref{thr:Cayley}, per il quale:
	\footnote{La notazione usata nei video \cite{lucchini} è diversa da quella usata in \cite{jacobson}.}
	
	\begin{align}
		l: G &\longrightarrow \Sym G \\
		g &\longmapsto l_g \\
		l_g: G &\longrightarrow G \\
		x &\longmapsto gx
	\end{align}

	E dal momento che $l$ è un isomorfismo: 
	
	\begin{equation}
		G \isomorfo l(G) \le \Sym G
	\end{equation}

	Se dimostriamo che $l(G)$ contiene un sottogruppo di ordine $m$, allora questo vale anche per $G$.
	
	Dal momento che $\ordine{G} = 2m$ è pari, allora esiste almeno un elemento $g$ che ha ordine 2. Consideriamo allora l'elemento $l_g \in l(G) \le \Sym G$. Dal momento che è appartiene a $\Sym G$ lo voglio scrivere come prodotto di cicli disgiunti.
	
	Se prendo un qualunque elemento $x_1 \in G$, si ha che:
	
	\begin{equation}
		x_1 \xrightarrow{l_g} gx_1 \xrightarrow{l_g} = ggx_1 = x_1
	\end{equation}

	Quindi la permutazione scambia gli elementi $x_1$ e $gx_1$. Posso quindi scrivere $l_g$ come prodotto di scambi di elementi disgiunti, e dal momento che $\ordine{G} = 2m$, gli scambi sono esattamente $m$:
	
	\begin{equation}
		l_g = (x_1, gx_1)(x_2, gx_2) \dots (x_m, gx_m)
	\end{equation}
	
	$m$ è dispari, quindi $l_g$ è prodotto di un numero dispari di scambi, quindi è una permutazione dispari e:
	
	\begin{equation}
		l_g \not\in \Alt G
	\end{equation}

	Per comodità chiama $G^* = l(G)$.
	
	Abbiamo:
	\begin{gather}
		G^* \le \Sym G \\
		\Alt G \normale \Sym G
	\end{gather}

	Quindi per il teorema~\ref{thr:prodotto_sottogruppi} si ha:
	\begin{equation*}
		G^* \Alt G \le \Sym G
	\end{equation*}
	
	Inoltre $\Alt G$ è contenuto in $G^* \Alt G$ ma non possono essere uguali perché il gruppo prodotto contiene $l_g$, mentre  $l_g \not\in \Alt G$.
	Quindi:
	\begin{equation*}
		\Alt G < G^* \Alt G \le \Sym G
	\end{equation*}

	Sappiamo che:
	\begin{equation*}
		\indice{\Sym G}{\Alt G} = 2
	\end{equation*}

	quindi $\indice{\Sym G}{G^* \Alt G}$ deve dividere 2, ma non può essere 2 perché $\Alt G \ne G^* \Alt G$.
	Quindi:
	\begin{equation*}
		G^* \Alt G = \Sym G
	\end{equation*}

	Allora, grazie al teorema dei due sottogruppi~\ref{thr:due_sottogruppi} (figura~\ref{fig:conseguenza_Cayley}), possiamo calcolare il seguente indice:
	\begin{equation*}
		\indice{G^*}{G^* \cap \Alt G} = \indice{G^* \Alt G}{\Alt G} = \indice{\Sym G}{\Alt G} = 2
	\end{equation*}
	
	Infine:
	\begin{gather*}
		\indice{G^*}{G^* \cap \Alt G} = \dfrac{\ordine{G^*}}{\ordine{G^* \cap \Alt G}} \\
		\ordine{G^* \cap \Alt G} = \dfrac{\ordine{G^*}}{\indice{G^*}{G^* \cap \Alt G}} =
		\dfrac{2m}{2} = m
	\end{gather*}
	
	Quindi $l(G)$ (che ne frattempo abbiamo battezzato $G^*$) possiede il sottogruppo $G^* \cap \Alt G$ di ordine $m$. Quindi anche $G$ (che è isomorfo a $l(G)$) possiede un sottogruppo di ordine $m$.
	
\end{dimostrazione}
\begin{figure}[tp]
	\centering
	\tikz {
		\node (a) at (2,4) {$G^* \Alt G = \Sym G$};
		\node (b) at (0,2) {$G^*$};
		\node (d) at (4,2) {$\Alt G$};
		\node (e) at (2,0) {$G^* \cap \Alt G$};
		\draw (a) edge[->] (b) 
		(a) edge[->] (d) 
		(b) edge[->] (e)
		(d) edge[->] (e);
		\draw[decorate,decoration={brace,raise=20pt}] (0,0) -- (0,2) node[pos=.5,right=-35pt,black]{2};
		\draw[decorate,decoration={brace,raise=20pt}] (4,4) -- (4,2) node[pos=.5,right=23pt,black]{2};
	}
	\caption{Schema dei sottogruppi coinvolti nel teorema.}
	\label{fig:conseguenza_Cayley}
\end{figure}
	
	\chapter{Lista di esercizi}

\begin{esercizio}
	\label{ex:ordine_HK}
	Se $H$ e $K$ sono sottogruppi finiti di un gruppo $G$, allora
	
	\begin{equation}
		\ordine{HK} = \dfrac{\ordine{H}\ordine{K}}{\ordine{H \cap K}}
	\end{equation}
\end{esercizio}
\begin{soluzione}
	Consideriamo la seguente funzione:
	
	\begin{align}
		\gamma : H \times K &\longrightarrow HK \\
		(h, k) &\longmapsto hk
	\end{align}

	La funzione $\gamma$ è suriettiva per definizione. Raramente è iniettiva.
	
	Cosa succede quando abbiamo immagini uguali di coppie diverse?
	
	\begin{gather}
		\gamma(h_1, k_1) = \gamma(h_2, k_2) \\
		h_1k_1 = h_2k_2 \\
		h_2^{-1}h_1 = k_2k_1^{-1}
	\end{gather}

	L'elemento $h_2^{-1}h_1$ appartiene ad $H$, come l'elemento $k_2k_1^{-1}$ appartiene a $K$. Ma abbiamo verificato che questi elementi in verità sono uno stesso elemento che quindi sta sia in $H$ sia in $K$. Quindi sta anche in $H \cap K$. Chiamiamo $x$ tale elemento. Risulta quindi che:
	
	\begin{equation}
		\begin{cases}
			h_1 = h_2x \\
			k_1 = x^{-1}k_2
		\end{cases}
	\end{equation}

	Viceversa, prendendo un elemento $x \in H \cap K$ abbiamo che:
	
	\begin{equation}
		hk = hx \, x^{-1}k
	\end{equation}

	Quindi:
	
	\begin{equation}
		\gamma^{-1}(hk) = \{(hx, x^{-1}k) \taleche x \in H \cap K\}
	\end{equation}

	Quindi la controimmagine $\gamma^{-1}(hk)$ ha cardinalità $\ordine{H \cap K}$, e questo vale per ogni coppia di elementi $h$ e $k$. Ma possiamo scegliere l'elemento $h$ tra $\ordine{H}$ elementi, e possiamo scegliere l'elemento $k$ tra $\ordine{K}$ elementi. Quindi per il principio di moltiplicazione:
	
	\begin{equation}
		\ordine{HK} = \dfrac{\ordine{H \times K}}{\ordine{H \cap K}} = \dfrac{\ordine{H}\ordine{K}}{\ordine{H \cap K}}
	\end{equation}

\end{soluzione}

\begin{esercizio}
	\label{ex:indici_coprimi}
	Sia $G$ un gruppo finito.
	Se $H$ e $K$ sono sottogruppi di $G$ con indici coprimi, allora:
	
	\begin{equation}
		HK = G
	\end{equation}
\end{esercizio}
\begin{soluzione}
	Per quanto visto nell'esercizio~\ref{ex:ordine_HK}, si ha:
	
	\begin{align}
		\ordine{HK} 
		&= \dfrac{\ordine{H}\ordine{K}}{\ordine{H \cap K}} = \\
		&= \dfrac{\ordine{H}\ordine{K}}{\ordine{H \cap K}} \cdot \dfrac{\ordine{G}\ordine{G}}{\ordine{G}\ordine{G}} = \\
		&= \dfrac{\ordine{H}}{\ordine{G}} \cdot \dfrac{\ordine{K}}{\ordine{G}}\cdot\dfrac{\ordine{G}}{\ordine{H \cap K}} \cdot \ordine{G} = \\
		&= \dfrac{\indice{G}{H \cap K}}{\indice{G}{H}\indice{G}{k}}\ordine{G} = \\
		&= \ordine{G}
	\end{align}

	Sappiamo che $HK \subseteq G$, quindi se hanno la stessa cardinalità:
	
	\begin{equation}
		HK = G
	\end{equation}
\end{soluzione}
	
	\part{Settimana seconda}
	\chapter{Isometrie del piano}
\label{cpt:Isometrie}

\section[Isometrie del piano]{Isometrie del piano\footnote{\cite[7 novembre 2021]{lucchini}}}
Le isometrie sono le trasformazioni del piano che preservano le distanze. A noi interessano quelle che fissano l'origine.

E' immediato pensare che le isometrie rappresentano un sottogruppo delle trasformazioni del piano.

Le isometrie che fissano l'origine sono:

\begin{itemize}
	\item le rotazioni di un angolo $\alpha$, che indichiamo con $\rho_\alpha$ (figura~\ref{fig:Isometrie_Rotazione});
	\item le riflessioni attorno ad una retta che forma un angolo $\alpha$ con l'asse delle ascisse, che indichiamo con $\iota_\alpha$ (figura~\ref{fig:Isometrie_Riflessione}).
\end{itemize}

\section[Isometrie come matrici]{Isometrie come matrici\footnote{Approfondimento personale}}

Prendiamo sul piano cartesiano un generico punto $P(x; y)$ e facciamolo ruotare attorno all'origine del piano di un angolo $\alpha$. Otteniamo così un nuovo punto $P(x'; y')$ (figura~\ref{fig:Isometrie_Rotazione}).

\begin{figure}[tp]
	\centering
	\begin{tikzpicture}[line cap=round,line join=round,>=triangle 45,x=1cm,y=1cm]
		\begin{axis}[
			x=1cm,y=1cm,
			axis lines=middle,
			xmin=-1,
			xmax=5,
			ymin=-1,
			ymax=5,
			xtick={-10,11},
			ytick={-7,10},]
			\clip(-10.72,-7.2) rectangle (11.92,10.32);
			\draw [shift={(-1,-1)},line width=1pt,color=green,fill=green,fill opacity=0.1] (0,0) -- (0:1.3) arc (0:26.56505117707799:1.3) -- cycle;
			\draw [shift={(-1,-1)},line width=1pt,color=green,fill=green,fill opacity=0.1] (0,0) -- (26.56505117707799:1.5) arc (26.56505117707799:59.56505117707799:1.5) -- cycle;
			\draw [shift={(-1,-1)},line width=1pt,color=green] (0:1.3) arc (0:26.56505117707799:1.3);
			\draw [shift={(-1,-1)},line width=1pt,color=green] (0:1.17) arc (0:26.56505117707799:1.17);
			\draw [line width=1pt,color=gray] (0,0)-- (4,2);
			\draw [line width=1pt,color=gray] (0,0)-- (2.2654042017516423,3.8558972759509564);
			\begin{scriptsize}
				\draw [fill=blue] (4,2) circle (2.5pt);
				\draw[] (4.3,2.4) node {$P(x; y)$};
				\draw[] (-0.32,0.35) node {$O$};
				\draw[] (1.6,0.4) node {$\theta$};
				\draw [fill=blue] (2.2654042017516423,3.8558972759509564) circle (2.5pt);
				\draw[] (2.6,4.2) node {$P'(x'; y')$};
				\draw[] (1.35,1.2) node {$\alpha$};
			\end{scriptsize}
		\end{axis}
	\end{tikzpicture}
	\caption{Rotazione}
\label{fig:Isometrie_Rotazione}
\end{figure}

Utilizziamo le formule goniometriche che ci permettono di passare dalle coordinate cartesiane alle coordinate polari.

I \emph{moduli} dei punti $P$ e $P'$, ovvero le loro distanze dall'origine, sono uguali. Quindi possiamo scrivere $\overline{OP}$ per intendere indifferentemente i moduli dei due punti.

Chiamiamo $\theta$ l'\emph{argomento} del punto $P$. Se questo punto viene fatto ruotare di un angolo $\alpha$, l'argomento di $P'$ è $\theta + \alpha$.

Le coordinate del punto $P$ possono essere espresse come:

\begin{equation}
x = \overline{OP} \cos \theta
\end{equation}
\begin{equation}
y = \overline{OP} \sin \theta
\end{equation}

Calcoliamo l'ascissa del punto $P'$:

\begin{align}
	x' & = \overline{OP} \cos (\theta + \alpha) = \\
	& = \overline{OP} ( \cos \theta \cos \alpha - \sin \theta \sin \alpha) = \\
	& = \overline{OP} \cos \theta \cos \alpha - \overline{OP} \sin \theta \sin \alpha = \\
	& = x \cos \alpha - y \sin \alpha
\end{align}
 
Calcoliamo l'ordinata del punto $P'$:

\begin{align}
	y' & = \overline{OP} \sin (\theta + \alpha) = \\
	& = \overline{OP} ( \sin \theta \cos \alpha + \cos \theta \sin \alpha) = \\
	& = \overline{OP} \sin \theta \cos \alpha + \overline{OP} \cos \theta \sin \alpha = \\
	& = y \cos \alpha + x \sin \alpha
\end{align}

Quindi possiamo passare dalle coordinate di $P(x; y)$ alle coordinate $P'(x', y')$ con queste equazioni:

\begin{equation}
	\begin{cases}
		x' = x \cos \alpha - y \sin \alpha \\
		y' = y \cos \alpha + x \sin \alpha
	\end{cases}
\end{equation}

E queste equazioni possono essere sintetizzate come prodotto matriciale:

\begin{equation}
	\begin{bmatrix}
		x' \\
		y' 
	\end{bmatrix}
	=
	\begin{bmatrix}
		\cos \alpha & -\sin \alpha \\
		\sin \alpha & \cos \alpha
	\end{bmatrix}
	\begin{bmatrix}
		x \\
		y 
	\end{bmatrix}
\end{equation}

Chiamiamo $\rho_\alpha$ la matrice dell'isometria:

\begin{equation}
	\rho_\alpha =
	\begin{bmatrix}
		\cos \alpha & -\sin \alpha \\
		\sin \alpha & \cos \alpha
	\end{bmatrix}
\end{equation}

Consideriamo ora una riflessione attorno ad una retta $r$ che forma un angolo $\alpha$ con l'asse delle ascisse. Prendiamo sul piano cartesiano un generico punto $P(x; y)$ e facciamolo riflettere nel nuovo punto $P(x'; y')$ (figura~\ref{fig:Isometrie_Riflessione}).

\begin{figure}[tp]
	\centering
	\begin{tikzpicture}[line cap=round,line join=round,>=triangle 45,x=1cm,y=1cm]
		\begin{axis}[
			x=1cm,y=1cm,
			axis lines=middle,
			xmin=-1,
			xmax=5,
			ymin=-1,
			ymax=5,
			xtick={-10,11},
			ytick={-7,10},]
			\clip(-10.72,-7.2) rectangle (11.92,10.32);
			\draw [shift={(-1,-1)},line width=1pt,color=green,fill=green,fill opacity=0.1] (0,0) -- (0:1.3) arc (0:26.56505117707799:1.3) -- cycle;
			\draw [shift={(-1,-1)},line width=1pt,color=green,fill=green,fill opacity=0.1] (0,0) -- (0:2) arc (0:43:2) -- cycle;
			\draw [shift={(-1,-1)},line width=1pt,color=green] (0:1.3) arc (0:26.56505117707799:1.3);
			\draw [shift={(-1,-1)},line width=1pt,color=green] (0:1.17) arc (0:26.56505117707799:1.17);
			\draw [line width=1pt,color=gray] (0,0)-- (4,2);
			\draw [line width=1pt,color=gray,domain=-10.72:11.92] plot(\x,{(-0--3.4099918003124925*\x)/3.656768508095853});
			\draw [line width=1pt,color=gray] (0,0)-- (2.27415399549615,3.850743253551046);
			\draw [line width=1pt,color=gray,dash pattern=on 3pt off 3pt] (2.27415399549615,3.850743253551046)-- (4,2);
			\begin{scriptsize}
				\draw [fill=blue] (4,2) circle (2.5pt);
				\draw[] (4.3,2.4) node {$P(x; y)$};
				\draw [] (0,0) circle (2pt);
				\draw[] (-0.32,0.35) node {$O$};
				\draw[] (1.46,0.3) node {$\theta$};
				\draw[] (2.2,0.75) node {$\alpha$};
				\draw [fill=blue] (2.27415399549615,3.850743253551046) circle (2.5pt);
				\draw[] (2.3,4.29) node {$P'(x'; y')$};
			\end{scriptsize}
		\end{axis}
	\end{tikzpicture}
	\caption{Rotazione}
	\label{fig:Isometrie_Riflessione}
\end{figure}

Anche in questo caso utilizziamo le formule goniometriche che ci permettono di passare dalle coordinate cartesiane alle coordinate polari.

Indichiamo con $\overline{OP}$ i \emph{moduli} dei punti $P$ e $P'$.

Chiamiamo $\theta$ l'\emph{argomento} del punto $P$. Per calcolare l'argomento di $P'$ consideriamo che:

\begin{align}
	\widehat{P'Ox} & = \widehat{P'Or} + \alpha = \\
	& = \widehat{POr} + \alpha = \\
	& = (\alpha - \theta) + \alpha = \\
	& = 2\alpha - \theta
\end{align}

Calcoliamo l'ascissa del punto $P'$:

\begin{align}
	x' & = \overline{OP} \cos (2\alpha - \theta) = \\
	& = \overline{OP} ( \cos 2\alpha \cos \theta + \sin 2\alpha \sin \theta) = \\
	& = \overline{OP} \cos 2\alpha \cos \theta + \overline{OP} \sin 2\alpha \sin \theta = \\
	& = x \cos 2\alpha + y \sin 2\alpha
\end{align}

Calcoliamo l'ordinata del punto $P'$:

\begin{align}
	y' & = \overline{OP} \sin (2\alpha - \theta) = \\
	& = \overline{OP} ( \sin 2\alpha \cos \theta - \cos 2\alpha \sin \theta) = \\
	& = \overline{OP}  \sin 2\alpha \cos \theta + \overline{OP} \cos 2\alpha \sin \theta = \\
	& = x \sin 2\alpha - y \cos 2\alpha
\end{align}

Quindi possiamo passare dalle coordinate di $P(x; y)$ alle coordinate $P'(x', y')$ con queste equazioni:

\begin{equation}
	\begin{cases}
		x' = x \cos 2\alpha + y \sin 2\alpha \\
		y' = x \cos 2\alpha - y \sin 2\alpha
	\end{cases}
\end{equation}

E queste equazioni possono essere sintetizzate come prodotto matriciale:

\begin{equation}
	\begin{bmatrix}
		x' \\
		y' 
	\end{bmatrix}
	=
	\begin{bmatrix}
		\cos 2\alpha & \sin 2\alpha \\
		\sin 2\alpha & -\cos 2\alpha
	\end{bmatrix}
	\begin{bmatrix}
		x \\
		y 
	\end{bmatrix}
\end{equation}

Chiamiamo $\iota_\alpha$ la matrice dell'isometria:

\begin{equation}
	\iota_\alpha =
	\begin{bmatrix}
		\cos 2\alpha & \sin 2\alpha \\
		\sin 2\alpha & -\cos 2\alpha
	\end{bmatrix}
\end{equation}

\section[Il gruppo delle isometrie]{Il gruppo delle isometrie\footnote{\cite[7 novembre 2021]{lucchini}}}

Possiamo ora verifica che le isometrie formano un gruppo in quanto l'operazione di composizione è interna:

\begin{align}
	\rho_\alpha \rho_\beta & = 
	\begin{bmatrix}
		\cos \alpha & -\sin \alpha \\
		\sin \alpha & \cos \alpha
	\end{bmatrix}
	\begin{bmatrix}
		\cos \beta & -\sin \beta \\
		\sin \beta & \cos \beta
	\end{bmatrix} 
	= \notag \\
	& = 
	\begin{bmatrix}
		\cos \alpha \cos \beta -\sin \alpha \sin \beta & - \cos \alpha \sin \beta - \sin \alpha \cos \beta \\
		\sin \alpha \cos \beta + \cos \alpha \sin \beta & - \sin \alpha \sin \beta + \cos \alpha \cos \beta
	\end{bmatrix} 
	= \notag \\
	& = 
	\begin{bmatrix}
		\cos (\alpha + \beta) & - \sin (\alpha + \beta) \\
		\sin (\alpha + \beta) & \cos (\alpha + \beta)
	\end{bmatrix} 
	= \notag \\
	\label{eqn:Isometrie_R_R}
	& = \rho_{\alpha + \beta}
\end{align}

\begin{align}
	\iota_\alpha \iota_\beta & = 
	\begin{bmatrix}
		\cos 2\alpha & \sin 2\alpha \\
		\sin 2\alpha & -\cos 2\alpha
	\end{bmatrix}
	\begin{bmatrix}
		\cos 2\beta & \sin 2\beta \\
		\sin 2\beta & -\cos 2\beta
	\end{bmatrix} 
	= \notag \\
	& = 
	\begin{bmatrix}
		\cos 2\alpha \cos 2\beta +\sin 2\alpha \sin 2\beta & \cos 2\alpha \sin 2\beta - \sin 2\alpha \cos 2\beta \\
		\sin 2\alpha \cos 2\beta - \cos 2\alpha \sin 2\beta & \sin 2\alpha \sin 2\beta + \cos 2\alpha \cos 2\beta
	\end{bmatrix} 
	= \notag \\
	& = 
	\begin{bmatrix}
		\cos (2\alpha - 2\beta) & - \sin (2\alpha - 2\beta) \\
		\sin (2\alpha - 2\beta) & \cos (2\alpha - 2\beta)
	\end{bmatrix} 
	= \notag \\
	\label{eqn:Isometrie_I_I}
	& = \rho_{2\alpha - 2\beta}
\end{align}

\begin{align}
	\rho_\alpha \iota_\beta & = 
	\begin{bmatrix}
		\cos \alpha & -\sin \alpha \\
		\sin \alpha & \cos \alpha
	\end{bmatrix}
	\begin{bmatrix}
		\cos 2\beta & \sin 2\beta \\
		\sin 2\beta & -\cos 2\beta
	\end{bmatrix} 
	= \notag \\
	& = 
	\begin{bmatrix}
		\cos \alpha \cos 2\beta -\sin \alpha \sin 2\beta & \cos \alpha \sin 2\beta + \sin \alpha \cos 2\beta \\
		\sin \alpha \cos 2\beta + \cos \alpha \sin 2\beta & \sin \alpha \sin 2\beta - \cos \alpha \cos 2\beta
	\end{bmatrix} 
	= \notag \\
	& = 
	\begin{bmatrix}
		\cos (\alpha + 2\beta) & \sin (\alpha + 2\beta) \\
		\sin (\alpha + 2\beta) & - \cos (\alpha + 2\beta)
	\end{bmatrix} 
	= \notag \\
	\label{eqn:Isometrie_R_I}
	& = \iota_{\frac{\alpha}{2} + \beta}
\end{align}

\begin{align}
	\iota_\alpha \rho_\beta & = 
	\begin{bmatrix}
		\cos 2\alpha & \sin 2\alpha \\
		\sin 2\alpha & -\cos 2\alpha
	\end{bmatrix}
	\begin{bmatrix}
		\cos \beta & -\sin \beta \\
		\sin \beta & \cos \beta
	\end{bmatrix} 
	= \notag \\
	& = 
	\begin{bmatrix}
		\cos 2\alpha \cos \beta +\sin 2\alpha \sin \beta & - \cos 2\alpha \sin \beta + \sin 2\alpha \cos \beta \\
		\sin 2\alpha \cos \beta - \cos 2\alpha \sin \beta & - \sin 2\alpha \sin \beta - \cos 2\alpha \cos \beta
	\end{bmatrix} 
	= \notag \\
	& = 
	\begin{bmatrix}
		\cos (2\alpha - \beta) & \sin (2\alpha - \beta) \\
		\sin (2\alpha - \beta) & -\cos (2\alpha - \beta)
	\end{bmatrix} 
	= \notag \\
	\label{eqn:Isometrie_I_R}
	& = \iota_{\alpha-\frac{\beta}{2}}
\end{align}

Se quindi chiamo $R$ l'insieme di tutte le rotazioni e $I$ l'insieme di tutte le riflessioni ottengo:

\begin{equation}
	G = R \cup I
\end{equation}

E $G$ è un gruppo in quanto la composizioni delle isometrie è ancora un'isometria.

Inoltre dalla \eqref{eqn:Isometrie_R_R}, visto che la composizione di due rotazioni è ancora una rotazione, possiamo affermare che $R$ è un sottogruppo di $G$:

\begin{equation}
	R \le G
\end{equation}

Prendiamo ora una riflessione privilegiata, ovvero quella rispetto all'asse delle ascisse:

\begin{equation}
	\iota = \iota_0
\end{equation}

Dalla \eqref{eqn:Isometrie_R_I} ricaviamo che:

\begin{equation}
	\rho_\alpha \iota = \iota_{\frac{\alpha}{2}}
\end{equation}

Quindi attraverso la composizione con $\iota$, ad ogni rotazione corrisponde una riflessione con angolo dimezzato e, viceversa, ad ogni riflessione corrisponde una rotazione con angolo doppio. Quindi abbiamo tante riflessioni quante rotazioni, e non ci sono rotazioni che sono alche riflessioni nè viceversa.

Quindi possiamo affermare che possiamo ottenere l'insieme delle riflessioni come prodotto tra l'insieme delle rotazioni e $\iota$:

\begin{equation}
	I = R\iota
\end{equation}

Inoltre l'unione tra $R$ e $I$ che definisce il nostro gruppo $G$ è un'unione disgiunta a meno dell'unità:

\begin{equation}
	G = R \overset{\circ}{\cup} R\iota
\end{equation}

Inoltre avendo tante rotazioni quante riflessioni, risulta che $R$ è un sottogruppo di \emph{indice} 2.

\begin{equation}
	|G:R| = 2
\end{equation}

Per il teorema \textbf{TODO} i sottogruppi di indice 2 sono normali, quindi $R$ è un sottogruppo normale di $G$.

\begin{equation}
	R \unlhd G
\end{equation}

Dalle formule \eqref{eqn:Isometrie_I_R} e \eqref{eqn:Isometrie_R_I} capiamo che se compongo una riflessione con una rotazione ottengo una riflessione, e se compongo questa con un'altra riflessione ottengo una rotazione. Quindi se coniugo una rotazione con una riflessione ottengo uno rotazione.
Inoltre nel coniugare posso considerare che l'inversa di una riflessione è sempre la riflessione stessa, quindi posso coniugare semplicemente con la composizione:

\begin{equation}
	\iota_\beta \rho_\alpha \iota_\beta
\end{equation}

Andiamo a calcolare il risultato di questo coniugio:

\begin{align}
	\iota_\beta \rho_\alpha \iota_\beta 
	& = \iota_{\beta-\frac{\alpha}{2}} \iota_\beta = \notag \\
	& = \rho_{2(\beta-\frac{\alpha}{2}) - 2\beta} = \notag \\
	& = \rho_{2\beta-\alpha - 2\beta} = \notag \\
	\label{eqn:Isometrie_coniugio_rho}
	& = \rho_{-\alpha}
\end{align}

Ora posso dimenticare tutte le formule di prima e ricordare solo la \eqref{eqn:Isometrie_coniugio_rho}.

Il gruppo $G$ degli isomorfismi ha quindi queste caratteristiche:

\begin{equation}
	R \unlhd G
\end{equation}
\begin{equation}
	R\iota \le G
\end{equation}
\begin{equation}
	R \cap R\iota = 1
\end{equation}

Quindi per il teorema \textbf{TODO} il gruppo $G$ si può ottenere come prodotto dei sottogruppi $R$ e $R\iota$:

\begin{equation}
	G = R \, R\iota
\end{equation}

Quindi ogni elemento di $G$ si scrive in un solo modo: come prodotto di una rotazione che sta in $R$ con una riflessione che sta in $R\iota$.

\begin{equation}
	G = R\gen{\iota}
\end{equation}

Ogni elemento può essere scritto nella forma:

\begin{equation}
	g = \rho_\alpha \iota^\epsilon \quad\text{ con } \epsilon \in \{0,1\}
\end{equation}

Verifichiamo cosa succede se moltiplico due isometrie scritte in questo modo. Ne prendo 2 qualsiasi:

\begin{equation}
	g_1 = \rho_{\alpha_1} \iota^{\epsilon_1}
\end{equation}
\begin{equation}
	g_2 = \rho_{\alpha_2} \iota^{\epsilon_2}
\end{equation}

E li moltiplico:

\begin{align}
	g_1g_2 
	& = \rho_{\alpha_1} \iota^{\epsilon_1} \rho_{\alpha_2} \iota^{\epsilon_2} \notag \\
	& = \rho_{\alpha_1} \iota^{\epsilon_1} \rho_{\alpha_2}  \iota^{\epsilon_1}  \iota^{\epsilon_1} \iota^{\epsilon_2} \notag \\
	& = \rho_{\alpha_1} (\iota^{\epsilon_1} \rho_{\alpha_2}  \iota^{\epsilon_1})  \iota^{\epsilon_1} \iota^{\epsilon_2} \notag 
\end{align}

Si presentano quindi due casi

\begin{equation}
	g_1g_2 =
	\begin{cases}
		\rho_{\alpha_1} \rho_{\alpha_2} \iota^{\epsilon_2} = \rho_{\alpha_1 + \alpha_2} \iota^{\epsilon_2} & \text{se } \epsilon_1 = 0 \\
		\rho_{\alpha_1} \rho_{-\alpha_2} \iota \iota^{\epsilon_2} = \rho_{\alpha_1 - \alpha_2} \iota^{\epsilon_2 + 1} & \text{se } \epsilon_1 = 1 		
	\end{cases}
\end{equation}

\section[Ordine degli elementi]{Ordine degli elementi\protect\footnote{\cite[7 novembre 2021]{lucchini}}}

Per descrivere il sottogruppo $R$ utilizziamo il toro:

\begin{equation}
	T = \{c \in \mathbb{C} \taleche \modulo{c} = 1\}
\end{equation}

Consideriamo la mappa:

\begin{equation}
	R \longrightarrow T: \rho_\alpha \longmapsto e^{i\alpha}
\end{equation}

E' un isomorfismo perché:

\begin{equation}
	\rho_\alpha \rho_\beta = e^{i\alpha} e^{i\beta} = e^{i(\alpha + \beta)} = \rho_{\alpha + \beta}
\end{equation}

Quindi:

\begin{equation}
	R \isomorfo T
\end{equation}

Qual è l'ordine di una rotazione?

\begin{align}
	(\rho_\alpha)^n = 1 & \sse (e^{i\alpha})^n = 1 \\
	& \sse \alpha n = k\, 2 \pi \text{ con } k \in \Z \\
	& \sse \alpha = \dfrac{k}{n}\,2\pi \text{ con } k \in \Z \\
	& \sse \alpha \in \Q \, 2\pi
\end{align}

Per cui la torsione del sottogruppo $R$ è:

\begin{equation}
	\label{eqn:isometrie_T_di_R}
	T(R) = \{\rho_\alpha \taleche \alpha \in \Q\,2\pi\}
\end{equation}

Ovvero l'immagine tramite l'isomorfismo di $\rho_\alpha$ è una radice $n$-esima di 1 per un $n$ opportuno.

Al contrario:

\begin{equation}
	\ordine{\rho_\alpha} = \infty \sse \alpha \not\in \Q\,2\pi
\end{equation}

Quindi se voglio un elemento di ordine infinito, mi basta sceglierlo con $\alpha \not\in \Q\,2\pi$.

Dalla \eqref{eqn:isometrie_T_di_R} risulta che:

\begin{equation}
	\alpha = \dfrac{2\pi}{n} \Longrightarrow \ordine{\rho_\alpha} = n
\end{equation}

Quindi qualunque sia l'ordine che mi interessa, trovo sempre un elemento di quell'ordine che, nel toro, corrisponde ad una radice $n$-esima di 1.

Se ora prendiamo una rotazione $\rho_\alpha$ qualsiasi e la componiamo con $\iota$ otteniamo una riflessione, e tutte le riflessioni hanno ordine 2 in quanto sono l'inverso di se stesse. Ma se noi componiamo due riflessioni:

\begin{equation}
	\iota \cdot \iota\rho_\alpha = \rho_\alpha 
\end{equation}

L'ordine di $\rho_\alpha$ dipende dall'angolo $\alpha$. Quindi \emph{il prodotto di due elementi di ordine 2 può avere ordine infinito o un qualunque altro ordine si voglia}.

\section[Centro del gruppo delle isometrie]{Centro del gruppo delle isometrie\protect\footnote{\cite[7 novembre 2021]{lucchini}}}

Cerchiamo gli elementi $g \in Z(G)$: questi elementi possono essere riflessioni o rotazioni. Divido il problema in 2 parti.

\textbf{Caso 1: $g \in \iota R$}: è una riflessione

Se $g$ è una riflessione allora:

\begin{equation}
	g \rho_\alpha g = \rho_{-\alpha} \Longrightarrow g \rho_\alpha = \rho_{-\alpha} g
\end{equation}

Ma se $g$ è nel centro, allora deve commutare con tutti gli elementi di $G$, in particolare:

\begin{equation}
	g\rho_\alpha = \rho_\alpha g
\end{equation}

Quindi deve risultare:

\begin{gather}
	\rho_{-\alpha} = \rho_\alpha \\
	\alpha \in \{0, \pi\}
\end{gather}

Quindi basta scegliere un $\alpha \not\in \{0, \pi\}$ e ho un elemento che non commuta con $g$. Quindi \emph{le riflessioni non appartengono al centro}.

\textbf{Caso 2: $g \in R$}: è una rotazione.

Una rotazione commuta con tutte le altre rotazioni. Devo però trovare le rotazioni che commutano anche con le riflessioni. Chiamando $g = \rho_\alpha$ rifaccio il ragionamento di prima:

\begin{gather}
	\iota \rho_\alpha \iota = \rho_{-\alpha} \Longrightarrow \iota \rho_\alpha = \rho_{-\alpha} \iota \\
	\rho_\alpha \in Z(G) \Longrightarrow \iota\rho_\alpha = \rho_\alpha \iota \\
	\rho_{-\alpha} = \rho_\alpha \\
	\alpha \in \{0, \pi\}
\end{gather}

Quindi gli unici elementi che stanno in $Z(G)$ sono l'identità e la rotazione $\rho_\pi$:

\begin{equation}
	\label{eqn:Isometrie_Z_G}
	Z(G) = \gen{\rho_\pi} \isomorfo C_2
\end{equation}

\section[Gruppi diedrali]{Gruppi diedrali\protect\footnote{\cite[7 novembre 2021]{lucchini}}}

Prendiamo un pentagono regolare e costruiamo il gruppo delle isometrie che mandano il pentagono in se stesso (figura~\ref{fig:Isometrie_Pentagono}).
\footnote{
	Solitamente si disegna il poligono con un vertice sul semiasse positivo delle ascisse (e il centro nell'origine degli assi). In questo caso la riflessione "base" $\iota$ rappresenta una riflessione che tiene fermo tale vertice e scambia quelli del primo e secondo quadrante con quelli del terzo e quarto quadrante.
	
	Tuttavia è possibile disegnare il poligono anche in altre posizioni.
	Se $n$ è dispari, si può disegnare il poligono con un vertice sul semiasse negativo delle ascisse. Mentre, se $n$ è dispari, si può disegnare il poligono ovviamente simmetrico rispetto all'asse delle ascisse ma con i lati che tagliano l'asse delle $x$. In questo caso la riflessione $\iota$ va a scambiare tutti i vertici: quelli di ordinata positiva con quelli di ordinata negativa.
}

\begin{figure}[tp]
	\centering
	\begin{tikzpicture}[line cap=round,line join=round,>=triangle 45,x=1cm,y=1cm]
		\begin{axis}[
			x=2cm,y=2cm,
			axis lines=middle,
			xmin=-2.5,
			xmax=2.5,
			ymin=-1.5,
			ymax=1.5,
			xtick={-3,3},
			ytick={-2,2},]
			\clip(-2.5,-1.5) rectangle (2.5,1.5);
			\fill[line width=2pt,color=figura,fill=figura,fill opacity=0.1] (1,0) -- (0.30901699437494745,0.9510565162951535) -- (-0.8090169943749471,0.5877852522924731) -- (-0.8090169943749472,-0.5877852522924729) -- (0.30901699437494723,-0.9510565162951534) -- cycle;
			%\draw [line width=2pt] (0,0) circle (2cm);
			\draw [line width=2pt,color=figura] (1,0)-- (0.30901699437494745,0.9510565162951535);
			\draw [line width=2pt,color=figura] (0.30901699437494745,0.9510565162951535)-- (-0.8090169943749471,0.5877852522924731);
			\draw [line width=2pt,color=figura] (-0.8090169943749471,0.5877852522924731)-- (-0.8090169943749472,-0.5877852522924729);
			\draw [line width=2pt,color=figura] (-0.8090169943749472,-0.5877852522924729)-- (0.30901699437494723,-0.9510565162951534);
			\draw [line width=2pt,color=figura] (0.30901699437494723,-0.9510565162951534)-- (1,0);
			\begin{scriptsize}
				\draw (2.4,-.1) node {$x$};
				\draw (-0.1,1.4) node {$y$};
			\end{scriptsize}
		\end{axis}
	\end{tikzpicture}
	\caption{Pentagono regolare di riferimento}
	\label{fig:Isometrie_Pentagono}
\end{figure}

Quali isometrie ci sono?

Abbiamo le rotazioni pari a $\frac{2\pi}{5}$, $2\frac{2\pi}{5}$, $\dots$, $4\frac{2\pi}{5}$:

\begin{gather}
	\rho = \rho_{\frac{2\pi}{5}} \\
	\ordine{\rho} = 5
\end{gather}

Inoltre abbiamo le riflessioni. Chiamiamo $\iota$ la riflessione lungo l'asse delle ascisse.

\begin{gather}
	D_5 = \gen{\rho, \iota} \\
	\ordine{D_5} = 10 \\
	D_5 \isomorfo \gen{\rho}\gen{\iota}
\end{gather}

Vale con qualunque poligono di $n$ lati: otteniamo il \textbf{gruppo diedrale} $D_n$, di grado $n$ e ordine $2n$ (attenzione alle notazioni che cambiano di autore in autore).

Possiamo battezzare i vertici del pentagono con i numeri (figura~\ref{fig:Isometrie_Pentagono_con_numeri}).

\begin{figure}[tp]
	\centering
	\begin{tikzpicture}[line cap=round,line join=round,>=triangle 45,x=1cm,y=1cm]
		\begin{axis}[
			x=2cm,y=2cm,
			axis lines=middle,
			xmin=-2.5,
			xmax=2.5,
			ymin=-1.5,
			ymax=1.5,
			xtick={-3,3},
			ytick={-2,2},]
			\clip(-2.5,-1.5) rectangle (2.5,1.5);
			\fill[line width=2pt,color=figura,fill=figura,fill opacity=0.1] (1,0) -- (0.30901699437494745,0.9510565162951535) -- (-0.8090169943749471,0.5877852522924731) -- (-0.8090169943749472,-0.5877852522924729) -- (0.30901699437494723,-0.9510565162951534) -- cycle;
			%\draw [line width=2pt] (0,0) circle (2cm);
			\draw [line width=2pt,color=figura] (1,0)-- (0.30901699437494745,0.9510565162951535);
			\draw [line width=2pt,color=figura] (0.30901699437494745,0.9510565162951535)-- (-0.8090169943749471,0.5877852522924731);
			\draw [line width=2pt,color=figura] (-0.8090169943749471,0.5877852522924731)-- (-0.8090169943749472,-0.5877852522924729);
			\draw [line width=2pt,color=figura] (-0.8090169943749472,-0.5877852522924729)-- (0.30901699437494723,-0.9510565162951534);
			\draw [line width=2pt,color=figura] (0.30901699437494723,-0.9510565162951534)-- (1,0);
			\begin{scriptsize}
				\draw (1.05,0.15) node {1};
				\draw (0.4,1.1) node {2};
				\draw (-0.9,0.7) node {3};
				\draw (-0.9,-0.7) node {4};
				\draw (0.4,-1.1) node {5};
				\draw (2.4,-.1) node {$x$};
				\draw (-0.1,1.4) node {$y$};
			\end{scriptsize}
		\end{axis}
	\end{tikzpicture}
	\caption{Pentagono regolare di riferimento con vertici numerati}
	\label{fig:Isometrie_Pentagono_con_numeri}
\end{figure}


Posso descrivere ogni isometria rispetto al comportamento sui vertici.

Se considero la rotazione di $\frac{2\pi}{5}$ ottengo la permutazione $(12345)$, mentre se considero la rotazione $\iota$ ottengo la permutazione $(25)(34)$. Quindi possiamo dire:

\begin{equation}
	D_5 \isomorfo \gen{(12345), (25)(34)} \le S_5
\end{equation}

\section[Centro dei gruppi diedrali]{Centro dei gruppi diedrali\footnote{\cite[8 novembre 2021]{lucchini}}}

\begin{esercizio}
	Descrivere $Z(D_n)$.
\end{esercizio}

	Ricordando~\eqref{eqn:Isometrie_Z_G} possiamo dire:
	
	\begin{equation}
		Z(D_n) \subseteq Z(G) = \gen{\rho_\pi}
	\end{equation}

	Ma $\rho_\pi$ potrebbe non appartenere a $D_n$.
	
	Se $\rho_\pi \in D_n$, allora in $D_n$ c'è una rotazione di ordine 2. Ma se $n$ è dispari non ci sono rotazioni di ordine 2. Quindi succede quando $n$ è pari:
	
	\begin{equation}
		\rho_\pi \in D_n \Longleftrightarrow n \text{ è pari}
	\end{equation}

\section[Sottogruppi normali dei gruppi diedrali]{Sottogruppi normali dei gruppi diedrali\footnote{\cite[8 novembre 2021]{lucchini}}}

\begin{esercizio}
	Descrivere i sottogruppi normali di $D_n$.
\end{esercizio}

	Dimostriamo innanzitutto che:
	
	\begin{equation}
		H \le \gen{\rho} \Longrightarrow H \normale D_n
	\end{equation}

	Prendiamo un elemento
	
	\begin{equation}
		g = \rho^\alpha \iota^\beta
	\end{equation}

	con $\alpha$ ridotto a modulo $n$ e $\beta$ ridotto a modulo 2.
	
	$H$ è sottogruppo di un gruppo ciclico, quindi è ciclico:
	
	\begin{equation}
		H = \gen{\rho^k}
	\end{equation}

	Facciamo il coniugio del generatore con l'elemento $g$:
	
	\begin{align}
		g \rho^k g^{-1} 
		& = \rho^\alpha \iota^\beta \rho^k (\rho^\alpha \iota^\beta)^{-1} \\
		& = \rho^\alpha \iota^\beta \rho^k \iota^\beta \rho^{-\alpha}
	\end{align}

	Il valore di $\iota^\beta \rho^k \iota^\beta$ dipende dal valore di $\beta$:
	
	\begin{equation}
		\iota^\beta \rho^k \iota^\beta = 
		\begin{cases}
			\rho^k & \text{se } \beta = 0 \\
			\rho^{-k} & \text{se } \beta = 1
		\end{cases}
	\end{equation}

	Quindi:
	
	\begin{equation}
		g \rho^k g^{-1} = \rho^\alpha \rho^{\pm k} \rho^{-\alpha} 
	\end{equation}

	E poiché le rotazioni commutano:
	
	\begin{equation}
		g \rho^k g^{-1} = \rho^{\pm k}
	\end{equation}
	
	Quindi se coniugo $\rho^k$ o trovo $\rho^k$ oppure trovo il suo inverso $\rho^{-k}$. Quindi $H$ è normale in $D_n$.
	
	Quindi tutti i sottogruppi di $\gen{\rho}$ sono normali.
	
	Verifichiamo ora se ce ne sono altri.
	
	Sia:
	
	\begin{equation}
		H \normale D_n \text{ con } H \not\subseteq \gen{\rho}
	\end{equation}

	Ovvero in $H$ c'è almeno una riflessione:
	
	\begin{equation}
		\exists k : \rho^k \iota = \iota^* \in H
	\end{equation}

	Consideriamo ora l'isometria $\rho \iota^* \rho^{-1} \iota^*$. La componente $\rho \iota^* \rho^{-1}$ appartiene ad $H$ perché $\iota^* \in H \normale D_n$. Anche $\iota^*$ appartiene ad $H$, quindi:
	
	\begin{gather}
		\rho \iota^* \rho^{-1} \iota^* \in H \\
		\rho (\iota^* \rho^{-1} \iota^*) \in H \\
		\rho \rho \in H \\
		\rho^2 \in H \\
		\gen{\rho^2} \le H
	\end{gather}

	Dobbiamo ora distinguere i due casi: $n$ dispari e $n$ pari.
	
	\textbf{Caso 1: $n$ dispari}
	
	Se $n$ è dispari allora
	
	\begin{equation}
		(n; 2) = 1
	\end{equation}

	quindi anche $\rho^2$ è generatore di $\gen{\rho}$:
	
	\begin{equation}
		\ordine{\rho^2} = \ordine{\rho} \quad\Longleftrightarrow\quad \gen{\rho^2} = \gen{\rho}
	\end{equation}
	 
	Quindi $H$ contiene $\rho$ e siccome contiene anche la riflessione $\iota^* = \rho^k \iota$ allora contiene anche $\iota$. Quindi $H = D_n$.

	Quindi l'unico sottogruppo normale che non ho ancora trovato e che non è contenuto in $\gen{\rho}$ è $D_n$. 

	\textbf{Caso 2: $n$ pari}

	Chiamo:
	
	\begin{equation}
		N = \gen{\rho^2}
	\end{equation}

	Usiamo il \textbf{Teorema di corrispondenza} (vedi~\ref{thr:Isomorfismi_corrispondenza}),
	il quale ci dice che esiste una corrispondenza tra i sottogruppi normali di $D_n$ che contengono $N$ e i sottogruppi normali di $\dfrac{D_n}{N}$ (figura~\ref{fig:Isometrie_diedrali_normali}).
	
	\begin{figure}[tp]
		\centering
		\tikz {
			\node (a) at (2,4) {$D_n = \gen{\rho, \iota}$};
			\node (b) at (0,2) {?};
			\node (c) at (2,2) {?};
			\node (d) at (4,2) {$\dots$};
			\node (e) at (2,0) {$N = \gen{\rho^2}$};
			\draw (a) edge[->] (b) 
			(a) edge[->] (c)
			(a) edge[->] (d) 
			(b) edge[->] (e)
			(c) edge[->] (e)
			(d) edge[->] (e);
		}
		$\qquad$
		\tikz {
			\node (a) at (2,4) {$D_n/N$};
			\node (b) at (0,2) {?};
			\node (c) at (2,2) {?};
			\node (d) at (4,2) {$\dots$};
			\node (e) at (2,0) {$1 = N/N$};
			\draw (a) edge[->] (b) 
			(a) edge[->] (c)
			(a) edge[->] (d) 
			(b) edge[->] (e)
			(c) edge[->] (e)
			(d) edge[->] (e);
		}
		\caption{Corrispondenze tra i sottogruppi normali dei gruppi diedrali (1).}
		\label{fig:Isometrie_diedrali_normali}
	\end{figure}

	Ora, dal momento che $D_n = \gen{\rho, \iota}$ abbiamo che:
	\footnote{
		Se $n$ è pari, il sottogruppo normale $\gen{\rho^2}$ partiziona il gruppo $D_n$ nei seguenti laterali:
		\begin{align*}
			1\gen{\rho^2} & = \{1, \rho^2, \rho^4, \dots, \rho^{n-2}\} \\
			& = 1\gen{\rho^2} = \rho^2\gen{\rho^2} = \dots = \rho^{n-2}\gen{\rho^2} \\
			& = \text{ insieme delle rotazioni pari} \\
			\rho\gen{\rho^2} & = \{\rho, \rho^3, \dots, \rho^{n-1}\} \\
			& = \rho\gen{\rho^2} = \rho^3\gen{\rho^2} = \dots = \rho^{n-1}\gen{\rho^2} \\
			& = \text{ insieme delle rotazioni dispari} \\
			\iota\gen{\rho^2} & = \{\iota, \rho^2\iota, \rho^4\iota, \dots, \rho^{n-2}\iota\} \\
			& = \iota\gen{\rho^2} = \rho^2\iota\gen{\rho^2} = \dots = \rho^n\iota\gen{\rho^2} \\
			& = \text{ insieme delle riflessioni pari} \\
			\rho\iota\gen{\rho^2} & = \{\rho\iota, \rho^3\iota, \rho^5\iota, \dots, \rho^{n-1}\iota\} \\
			& = \rho\iota\gen{\rho^2} = \rho^3\iota\gen{\rho^2} = \dots = \rho^{n-1}\iota\gen{\rho^2} \\
			& = \text{ insieme delle riflessioni dispari}
		\end{align*}
		Se $n$ è dispari, si hanno i seguenti laterali:
		\begin{align*}
			1\gen{\rho^2} & = \{\rho, \rho^2, \dots, \rho^n\} \\
			& = 1\gen{\rho^2} = \rho\gen{\rho^2} = \dots = \rho^{n-1}\gen{\rho^2} \\
			& = \text{ insieme delle rotazioni} \\
			\iota\gen{\rho^2} & = \{\iota, \rho\iota, \rho^2\iota, \dots, \rho^{n-1}\iota\} \\
			& = \iota\gen{\rho^2} = \rho\iota\gen{\rho^2} = \dots = \rho^{n-1}\iota\gen{\rho^2} \\
			& = \text{ insieme delle riflessioni} \\
		\end{align*}
	}
	
	\begin{gather}
		\dfrac{D_n}{N} = \gen{\rho N, \iota N} \\
		\ordine{\rho N} = 2 \\
		\ordine{\iota N} = 2
	\end{gather}

	Quindi $\dfrac{D_n}{N}$ è di ordine 4,
	\footnote{In \cite[p. 48]{garonzi} c'è un'altra interessante dimostrazione dell'ordine di $D_n/N$: \\
	\[\ordine{D_n : \gen{\rho^2}} = \ordine{D_n : \gen{\rho}} \cdot \ordine{\gen{\rho} : \gen{\rho^2}} = 4\]}
	abeliano \textbf{(TODO: perché?)}, non ciclico (infatti contiene 2 elementi di ordine 2), quindi:
	
	\begin{equation}
		\dfrac{D_n}{N} \isomorfo C_2 \times C_2
	\end{equation}
	
	$C_2 \times C_2$ ha 4 sottogruppi, tutti normali perché abeliano \textbf{(TODO: perché?)}
	
	Quindi $\dfrac{D_n}{N}$ ha 4 sottogruppi: l'identità di ordine 1 e 3 sottogruppi di ordine 2, che corrispondono a:
	\footnote{
		$\rho N$ è l'insieme delle rotazioni dispari, e la composizione di due rotazioni dispari fa una rotazione pari, ovvero un elemento del laterale $1 N$ che in $D_n/N$ si comporta da unità. 
		
		$\iota N$ è l'insieme delle riflessioni pari, e la composizione di due riflessioni pari è una rotazione pari, ovvero un elemento del laterale $1 N$ che in $D_n/N$ si comporta da unità. 
		
		$\rho\iota N$ è l'insieme delle riflessioni dispari, e la composizione di due riflessioni dispari è una rotazione pari, ovvero un elemento del laterale $1 N$ che in $D_n/N$ si comporta da unità.
	}
	
	\begin{equation}
		\gen{\rho N}, \gen{\iota N}, \gen{\rho\iota N}
	\end{equation}

	Tornando indietro con il teorema di corrispondenza trovo che i tre sottogruppi normali che contengono $\gen{\rho^2}$ sono (figura~\ref{fig:Isometrie_diedrali_normali_c2_c2}):
	
	\begin{equation}
		\gen{\rho^2, \rho} = \gen{\rho}, \gen{\rho^2, \iota}, \gen{\rho^2, \rho\iota}
	\end{equation} 
	
	\begin{sidewaysfigure}
		\centering
		\tikz {
			\node (a) at (2,4) {$D_n = \gen{\rho, \iota}$};
			\node (b) at (0,2) {$\gen{\rho}$};
			\node (c) at (2,2) {$\gen{\rho^2, \iota}$};
			\node (d) at (4,2) {$\gen{\rho^2, \rho\iota}$};
			\node (e) at (2,0) {$N = \gen{\rho^2}$};
			\draw (a) edge[->] (b) 
			(a) edge[->] (c)
			(a) edge[->] (d) 
			(b) edge[->] (e)
			(c) edge[->] (e)
			(d) edge[->] (e);
		}
		$\qquad$
		\tikz {
			\node (a) at (2,4) {$D_n/N$};
			\node (b) at (0,2) {$\gen{\rho N}$};
			\node (c) at (2,2) {$\gen{\iota N}$};
			\node (d) at (4,2) {$\gen{\rho\iota N}$};
			\node (e) at (2,0) {$1 = \gen{1N} = N/N$};
			\draw (a) edge[->] (b) 
			(a) edge[->] (c)
			(a) edge[->] (d) 
			(b) edge[->] (e)
			(c) edge[->] (e)
			(d) edge[->] (e);
		}
		$\qquad$
		\tikz {
			\node (a) at (2,4) {$C_2 \times C_2$};
			\node (b) at (0,2) {$\gen{a, 1}$};
			\node (c) at (2,2) {$\gen{1, b}$};
			\node (d) at (4,2) {$\gen{a, b}$};
			\node (e) at (2,0) {$1$};
			\draw (a) edge[->] (b) 
			(a) edge[->] (c)
			(a) edge[->] (d) 
			(b) edge[->] (e)
			(c) edge[->] (e)
			(d) edge[->] (e);
		}
		\caption{Corrispondenze tra i sottogruppi normali dei gruppi diedrali (2).}
		\label{fig:Isometrie_diedrali_normali_c2_c2}
	\end{sidewaysfigure}

	Quindi i sottogruppi normali di $D_n$ sono:
	
	\begin{enumerate}
		\item i sottogruppi di $\gen{\rho}$
		\item il gruppo $D_n$
		\item \emph{se $n$ è pari:} i sottogruppi $\gen{\rho}$, $\gen{\rho^2, \iota}$, $\gen{\rho^2, \rho\iota}$
	\end{enumerate}

	I gruppi $\gen{\rho^2, \iota}$ e $\gen{\rho^2, \rho\iota}$ sono gruppi diedrali di grado $\frac{n}{2}$ e corrispondono alle isometrie di poligoni con un numero dimezzato di lati.

\section[Sottogruppi di $D_6$]{Sottogruppi di $D_6$\footnote{\cite[8 novembre 2021]{lucchini}}}

\begin{esercizio}
	Elencare i sottogruppi di $D_6$
\end{esercizio}

Sottogruppo delle rotazioni:

\begin{equation}
	R = \gen{\rho} \isomorfo C_6
\end{equation}

Sottogruppi di $R$, in corrispondenza ai divisori di 6:

\begin{equation}
	1, \quad \gen{\rho^3} \isomorfo C_2, \quad \gen{\rho^2} \isomorfo C_2
\end{equation}

Cerchiamo poi i sottogruppi $H \le G$ con $H \not\subseteq R$.

Sappiamo che $R$ è normale, quindi $HR$ è un sottogruppo di $D_6$ (vedi~\ref{thr:prodotto_sottogruppi}).

$R$ è di indice 2 e $H$ non è contenuto in $R$, quindi $HR = D_6$.
\footnote{
Non trovando spiegazioni migliori, io la giustifico così.
$R$ è di indice 2. $H$ non è contenuto in $R$, quindi contiene almeno un elemento che non appartiene ad $R$. 
Questo elemento è presente anche in $HR$, quindi $HR$ ha cardinalità maggiore di $R$, ovvero ha indice minore (visto che cardinalità e indici sono inversamente proporzionali). L'unico indice minore di 2 è 1, quindi $HR$ coincide con $D_6$.
}

Quindi per il teorema dei due sottogruppi (vedi~\ref{thr:due_sottogruppi}) sappiamo che (figura~\ref{fig:Isometrie_due_sottogruppi}):

\begin{figure}[tp]
	\centering
	\tikz {
		\node (a) at (2,4) {$D_6 = HR$};
		\node (b) at (0,2) {$H$};
		\node (d) at (4,2) {$R$};
		\node (e) at (2,0) {$H \cap R$};
		\draw (a) edge[->] (b) 
		(a) edge[->] (d) 
		(b) edge[->] (e)
		(d) edge[->] (e);
	}
	\caption{Teorema dei due sottogruppi applicato a $D_6$.}
	\label{fig:Isometrie_due_sottogruppi}
\end{figure}


\begin{equation}
	\dfrac{H}{H \cap R} \isomorfo \dfrac{HR}{R} \isomorfo C_2 
\end{equation}

Quindi $H$ è generato da $H \cap R$ e da un altro elemento che sta fuori. Visto che $H \cap R$ contiene solo rotazioni, affinché $H$ non sia un sottogruppo delle rotazioni è necessario che l'elemento da aggiungere come generatore sia un'opportuna rotazione. Quindi per conoscere $H$ partiamo dalla varie possibilità di $H \cap R$
	
\[
	\begin{array}{cc}
		\toprule
		H \cap R & H \\
		\midrule
		\gen{\rho} \isomorfo C_6 & \gen{\rho, \iota} = D_6 \\
		\midrule
		\gen{\rho^2} \isomorfo C_3 & \gen{\rho^2, \iota} \isomorfo D_3 \isomorfo S_3 \\
			& \gen{\rho^2, \iota\rho} \isomorfo D_3 \isomorfo S_3 \\
		\midrule
		\gen{\rho^3} \isomorfo C_2 & \gen{\rho^3, \iota} \isomorfo C_2 \times C_2 \\
			& \gen{\rho^3, \iota\rho} \isomorfo C_2 \times C_2 \\
			& \gen{\rho^3, \iota\rho^2} \isomorfo C_2 \times C_2 \\
		\midrule
		1 & \gen{\iota} \isomorfo C_2 \\
			& \gen{\iota\rho} \isomorfo C_2 \\
			& \gen{\iota\rho^2} \isomorfo C_2 \\
			& \gen{\iota\rho^3} \isomorfo C_2 \\
			& \gen{\iota\rho^4} \isomorfo C_2 \\
			& \gen{\iota\rho^5} \isomorfo C_2 \\
		\bottomrule
	\end{array}
\]

Nota che $\iota\rho$ non è contenuto in $\gen{\rho^2, \iota}$.

Invece non devo aggiungere $\gen{\rho^2, \iota\rho^2}$ perché $\iota\rho^2 \in \gen{\rho^2, \iota}$.

	\chapter{Torsione}

Da un gruppo $G$ (infinito), si chiama \textbf{torsione} l'insieme degli elementi di ordine finito:

\begin{equation}
	T(G) = \{g \in G \taleche \ordine{g} \text{ finito}\}
\end{equation}

Generalmente la torsione non è un sottogruppo.

\begin{esercizio}
	Se $G$ è abeliano, allora:
	
	\begin{equation}
		T(G) \le G
	\end{equation}
\end{esercizio}
\begin{soluzione}
	Prendiamo due elementi $x, y \in T(G)$. Se sono di ordine finito significa che:
	
	\begin{gather}
		\exists m \ne 0 \taleche x^m = 1 \\
		\exists n \ne 0 \taleche y^n = 1
	\end{gather}

	Vogliamo domandarci se anche $xy$ è un elemento di $T(G)$. Risulta:
	
	\begin{align}
		(xy)^{mn} &= x^{mn}y^{mn} && \text{ammesso perché $G$ è abeliano} \\
		&= (x^m)^n(y^n)^m && \text{per le proprietà delle potenze} \\
		&= 1^n1^m && \text{per come ho scelto $x$ e $y$} \\
		&= 1 \cdot 1 && \text{per proprietà dell'unità} \\
		&= 1
	\end{align}

	Anche $xy \in T(G)$, quindi $T(G)$ è chiuso per l'operazione di moltiplicazione, quindi $T(G)$ è un sottogruppo.
	
	Inoltre, dal momento che $G$ è abeliano, $T(G)$ è un sottogruppo normale:
	
	\begin{equation}
		T(G) \normale G
	\end{equation}
\end{soluzione}

\begin{esercizio}
	\label{ex:torsione_del_quoziente}
	Se $G$ è abeliano, allora:
	
	\begin{equation}
		T(G/T(G)) = \{T(G)\}
	\end{equation}
\end{esercizio}
\begin{soluzione}
	L'insieme $T(G/G(T))$ contiene i laterali di ordine finito, ovvero i laterali per i quali esiste una potenza che è uguale all'identità del gruppo quoziente. E l'identità del gruppo quoziente è, in questo caso, $T(G)$.
	
	Quindi abbiamo:
	
	\begin{equation}
		T(G/T(G)) = \{gT(G) \taleche \exists n \taleche (gT(G))^n = T(G)\}
	\end{equation}
	
	Distribuiamo la potenza e teniamo presente che la potenza di ogni elemento della torsione appartiene alla torsione stessa (visto che la torsione è un sottogruppo di $G$), quindi:
	
	\begin{equation}
		T(G/T(G)) = \{gT(G) \taleche \exists n \taleche g^nT(G) = T(G)\}
	\end{equation}

	Per le proprietà dei laterali, questo significa che $g^n$ appartiene a $T(G)$:
	
	\begin{equation}
		T(G/T(G)) = \{gT(G) \taleche \exists n \taleche g^n \in T(G)\}
	\end{equation}

	Se $g^n$ appartiene alla torsione significa che esiste una sua potenza che è uguale all'identità:
	
	\begin{equation}
		T(G/T(G)) = \{gT(G) \taleche \exists n \exists m \taleche (g^n)^m = 1\}
	\end{equation}

	Ma questo significa che anche $g$ appartiene alla torsione:
	
	\begin{equation}
		T(G/T(G)) = \{gT(G) \taleche g \in T(G)\}
	\end{equation}

	E se facciamo i laterali con elementi appartenenti al sottogruppo normale, otteniamo il sottogruppo stesso:
	
	\begin{equation}
		T(G/T(G)) = \{T(G)\}
	\end{equation}
\end{soluzione}

Questo teorema significa che, \emph{dato un gruppo $G$ con una torsione $T(G)$, se costruiamo il gruppo quoziente $G/T(G)$ questo non ha torsione se non l'elemento identico $T(G)$}. 

\begin{esercizio}
	Considerare il gruppo additivo del campo $\R$. Dimostrare che:
	
	\begin{equation}
		T(\R/\Z) = \Q/\Z
	\end{equation}
	
	E dimostrare che:
	
	\begin{equation}
		T(\R/\Q) = \{\Q\}
	\end{equation}
\end{esercizio}
\begin{soluzione}
	I laterali di $\R$ contenuti in $\R/\Z$ hanno forma $x + \Z$, con $x \in \R$. Esempi di laterali sono:
	
	\begin{align}
		0 + \Z &= \{0; 0 \pm 1; 0 \pm 2; \dots\} \\
		\frac{1}{2} + \Z &= \biggl\{\frac{1}{2}; \frac{1}{2} \pm 1; \frac{1}{2} \pm 2; \dots \biggr\} \\
		\sqrt{2} + \Z &= \{\sqrt{2}; \sqrt{2} \pm 1; \sqrt{2} \pm 2; \dots \}
	\end{align}

	Vogliamo cercare quali laterali $x + \Z$ appartengono alla torsione del gruppo quoziente, ovvero quei laterali tali che:
	
	\begin{equation}
		\exists m \in \Z, m \ne 0 \taleche m(x + \Z) = \Z
	\end{equation}

	Attenzione che stiamo usando la notazione additiva, e che l'identità del gruppo quoziente è il sottogruppo normale $\Z$.
	
	Eseguiamo il prodotto e teniamo presente che un qualunque intero moltiplicato per un intero appartiene agli interi ($m\Z = \Z$), quindi:
	
	\begin{equation}
		mx + \Z = \Z
	\end{equation}

	E, similmente: per ottenere un intero da un intero è necessario sommare un intero, quindi:
	
	\begin{equation}
		mx \in \Z
	\end{equation}

	Se $mx$ è un intero, significa che $x$ è il quoziente tra un intero e $m$ (che è a sua volta un intero), quindi $x$ è un razionale:
	
	\begin{equation}
		x \in \Q
	\end{equation} 

	Se quindi $x$ è un razionale, allora i laterali che appartengono alla torsione cercata appartengono a $\Q/\Z$:
	
	\begin{equation}
		T(\R/\Z) = \Q/\Z
	\end{equation}

	E con questa la prima parte è dimostrata. Per la seconda parte consideriamo che:
	
	\begin{equation}
		\label{eq:torsioni_reali}
		\dfrac{\R/\Z}{T(\R/\Z)} = \dfrac{\R/\Z}{\Q/\Z} \isomorfo \R/\Q
	\end{equation}

	Ora, per il risultato dell'esercizio~\ref{ex:torsione_del_quoziente}, si ha:
	
	\begin{equation}
		T\biggl(\dfrac{\R/\Z}{T(\R/\Z)}\biggr) = \{T(\R/\Z)\}
	\end{equation}

	Ovvero il gruppo quoziente $\dfrac{\R/\Z}{T(\R/\Z)}$ non ha torsione se non l'identità $T(\R/\Z)$.

	Quindi per l'isomorfismo della formula~\eqref{eq:torsioni_reali}, anche il gruppo quoziente $\R/\Q$ non ha torsione se non l'identità $\Q$:
	
	\begin{equation}
		T(\R/\Q) = \{\Q\}
	\end{equation}
	
\end{soluzione}

Possiamo interpretare il risultato di questo esercizio osservando che prendendo la potenza di un numero razionale ottengo un numero razionale, mentre se prendo la potenza di un numero irrazionale non ottengo mai un numero razionale.

\begin{esercizio}
	Se $G$ è un gruppo abeliano finitamente generato e tutti i suoi generatori hanno ordine finito, allora $G$ è finito.
\end{esercizio}
\begin{soluzione}
	Sia:
	
	\begin{equation}
		G = \gen{g_1, g_2, \dots, g_t}
	\end{equation}

	Sappiamo inoltre che:
	
	\begin{equation}
		\forall i \in [1, t] \quad \ordine{g_i} = n_i \text{ finito}
	\end{equation}

	Ogni elemento di $G$ è generato da un prodotto dei generatori. Se il gruppo non è abeliano, non posso riordinare gli elementi, mentre se è abeliano posso farlo, quindi ogni elementi $g \in G$ lo posso scrivere come:
	
	\begin{equation}
		g = g_1^{a_1} \cdot g_2^{a_2} \cdot \dots \cdot g_t^{a_t} \text{ con $a_i < n_i$}
	\end{equation}

	Quindi per il principio di moltiplicazione ho al più $n_1 \cdot n_2 \cdot \dots \cdot n_t$ scelte e:
	
	\begin{equation}
		\ordine{G} \le n_1 \cdot n_2 \cdot \dots \cdot n_t
	\end{equation}
\end{soluzione}

\begin{esercizio}
	Dimostrare che $\Q/\Z$ non è finitamente generato.
\end{esercizio}
\begin{soluzione}
	Ogni elemento di $\Q/\Z$ ha ordine finito, infatti ogni suo elemento ha forma:
	
	\begin{equation}
		\frac{a}{b} + \Z \in \dfrac{\Q}{\Z} 
	\end{equation}

	tali elementi sono finiti se:
	
	\begin{equation}
		\exists m > 0 \taleche m\biggl( \frac{a}{b} + \Z \biggr) = \Z
	\end{equation}

	E' sufficiente prendere un $m$ che sia multiplo di $b$.
	
	Poniamo per assurdo che $\Q/\Z$ sia finitamente generato. Allora per l'esercizio precedente, visto che tutti gli elementi (e quindi anche gli eventuali generatori) sono finiti dovrebbe essere finito. Ma non lo è. Quindi non è finitamente generato.
\end{soluzione}

Possiamo anche concludere che $\Q/\Z$ è un gruppo in cui tutti gli elementi hanno ordine finito, ma:

\begin{equation}
	\exp \dfrac{\Q}{\Z} = \infty
\end{equation}
	\chapter{Esercizi sugli omomorfismi}

\begin{esercizio}
	\label{ex:automorfismo_iota}
	Sia:
	
	\begin{align}
		\iota : G &\longrightarrow G \\
		x &\longrightarrow x^{-1}
	\end{align}

	$\iota$ è un automorfismo se e solo se $G$ è abeliano.
	\footnote{Cfr. \cite[pag. 63, es. n. 3]{jacobson}.}
\end{esercizio}
\begin{soluzione}
	$\iota$ è una funzione biiettiva perché è l'inversa di se stessa. Ma è un omomorfismo (e quindi un automorfismo)?
	
	E' un omomorfismo se, $\forall x_1, x_2 \in G$:
	
	\begin{align}
		\iota(x_1x_2) &= \iota(x_1)\iota(x_2) &&\text{per la definizione di omomorfismo} \\
		(x_1x_2)^{-1} &= x_1^{-1}x_2^{-1} &&\text{per la definizione di $\iota$} \\
		x_2^{-1}x_1^{-1} &= x_1^{-1}x_2^{-1} &&\text{per l'operazione di inversione} \\
		(x_2^{-1}x_1^{-1})^{-1} &= (x_1^{-1}x_2^{-1})^{-1} &&\text{invertendo i due membri} \\
		x_1x_2 &= x_2x_1 &&\text{per l'operazione di inversione}
	\end{align}

	Quindi $G$ è abeliano.
\end{soluzione}

\begin{teorema}[{\cite[Week 2, Point 3]{lucchini_week}; cfr~\cite[pag. 63, es. n. 4]{jacobson}}]
	\label{thr:automorfismi_gruppi_ciclici}
	Se $G$ è un gruppo ciclico infinito:
	\begin{equation*}
		\Aut G = \gen{\iota} \isomorfo C_2
	\end{equation*}

	Se $G$ è un gruppo ciclico finito:
	\begin{equation*}
		\ordine{\Aut G} = \varphi(m)
	\end{equation*}
\end{teorema}
\begin{dimostrazione}
	Dato che $G = \gen{g}$ è un gruppo ciclico, ogni endomorfismo $\alpha \in \End G$ trasforma il generatore $g$ in un qualche altro elemento di $G$, elemento che è una potenza del generatore:
	\begin{equation*}
		\alpha: g \longmapsto g^k
	\end{equation*}

	Affinché un tale endomorfismo sia un automorfismo è necessario che l'elemento immagine $g^k$ sia a sua volta generatore del gruppo, ovvero:
	\begin{equation*}
		\gen{g^k} = \gen{g}
	\end{equation*}

	Studiamo ora i due casi distinti.
	
	Nel caso di gruppo $G$ \textbf{ciclico e infinito}, sono solo due i valori di $k$ per cui $g^k$ è generatore di $G$, ovvero -1 e +1 (vedi teorema~\ref{thr:generatori_gruppi_ciclici_infiniti}).
	
	Quindi abbiamo due soli automorfismi per i gruppi ciclici infiniti: l'automorfismo identico e l'automorfismo $\iota$ (vedi esercizio~\ref{ex:automorfismo_iota}) che inverte tutti gli elementi di $G$.
	
	Quindi:
	\begin{equation*}
		\Aut G = \gen{\iota} \isomorfo C_2
	\end{equation*}

	Se invece $G$ è un gruppo \textbf{ciclico finito di ordine $m$}, posso considerare solo i valori di $k$ tali che $0 \le k < m$. Il numero di generatori del gruppo è pari a $\varphi(m)$ (vedi teorema~\ref{thr:generatori_gruppi_ciclici_finiti}). Quindi:
	\begin{equation*}
		\ordine{\Aut G} = \varphi(m)
	\end{equation*}
\end{dimostrazione}

\begin{esercizio}
	\label{ex:automorfismi_interni}
	Sia $G$ un gruppo e $a \in G$ un suo elemento. Definiamo l'\emph{automorfismo interno} (o \emph{coniugio}):
	
	\begin{align}
		I_a: G &\longrightarrow G \\
		 x &\longmapsto axa^{-1}
	\end{align}

	Verifica che $I_a$ è un automorfismo.
	
	Dimostra che:
	
	\begin{align}
		I: G &\longrightarrow \Aut G \\
		a &\longmapsto I_a
	\end{align}
	
	è un omomorfismo e che:
	
	\begin{equation}
		\ker I = Z(G)
	\end{equation}
	
	Quindi concludi che:
	
	\begin{equation}
		\Inn G := \{I_a \taleche a \in G\}
	\end{equation} 

	è un sottogruppo di $\Aut G$ con:
	
	\begin{equation}
		\Inn G \isomorfo G/Z(G)
	\end{equation}
	
	Verifica che $\Inn G$ è un sottogruppo normale di $\Aut G$. $\Aut G/\Inn G$ è chiamato \emph{gruppo degli automorfismi esterni}.
	\footnote{Cfr. \cite[pag. 63, es. n. 6]{jacobson}}
\end{esercizio}
\begin{soluzione}
	Per verificare che $I_a$ è un automorfismo devo:
	
	\begin{enumerate}
		\item verificare che è un omomorfismo, ovvero rispetta la proprietà~\eqref{eqn:omomorfismo_proprieta};
		\item verificare che è un endomorfismo, ovvero una funzione di un gruppo su se stesso; ma questo è già verificato per definizione di $I_a$;
		\item verificare che è un automorfismo, ovvero che la mappa è invertibile.
	\end{enumerate}

	Per verificare che $I_a$ è un omomorfismo basta osservare che:
	
	\begin{align}
		I_a(x_1x_2) &= ax_1x_2a^{-1} && \text{per definizione di $I_a$} \\
		&= ax_1a^{-1}ax_2a^{-1} && \text{perché $a^{-1}a = 1$} \\
		&= I_a(x_1)I_a(x_2) && \text{per definizione di $I_a$}
	\end{align}

	Per verificare che $I_a$ è invertibile (e quindi un automorfismo) basta osservare che:
	
	\begin{align}
		y = I_a(x) & \quad\text{chiamo $y$ l'immagine di $x$} \\
		y = axa^{-1} & \quad\text{per definizione di $I_a$} \\
		a^{-1}ya = a^{-1}(axa^{-1})a & \quad\text{per il secondo principio di equivalenza} \\
		a^{-1}ya = x & \quad\text{per la proprietà distributiva} \\
	\end{align}

	Quindi risulta che:
	
	\begin{align}
		x &= I_a^{-1}(y) && \text{la funzione inversa} \\
		&= a^{-1}ya && \text{per quanto visto sopra} \\
		&= a^{-1}y(a^{-1})^{-1} && \text{per definizione di elemento inverso} \\
		&= I_{a^{-1}}(y) && \text{per definizione di $I_a$}
	\end{align}

	Quindi la mappa $I_a$ è invertibile perché:
	
	\begin{equation}
		I_a^{-1} = I_{a^{-1}}
	\end{equation}

	Per dimostrare che la mappa $I$ è un omomorfismo basta verificare che soddisfa la proprietà~\eqref{eqn:omomorfismo_proprieta}. Ovvero, chiamata $\circ$ l'operazione di composizione di due automorfismi interni, devo dimostrare che:
	
	\begin{align}
		I(a_1a_2) &= I(a_1) \circ I(a_2) \\
		I_{a_1a_2} &= I_{a_1} \circ I_{a_2}
	\end{align}

	Verificare questa significa verificare che $\forall x \in G$:
	
	\begin{equation}
		I_{a_1a_2}(x) = (I_{a_1} \circ I_{a_2})(x)
	\end{equation}
	
	Si ha infatti:
	
	\begin{align}
		I_{a_1a_2}(x) &= (a_1a_2)x(a_1a_2)^{-1} && \text{per definizione di $I_a$} \\
		&= a_1a_2xa_2^{-1}a_1^{-1} && \text{per definizione di inverso} \\
		&= a_1 I_{a_2}(x) a_1^{-1} && \text{per definizione di $I_a$} \\
		&= I_{a_1}(I_{a_2}(x)) && \text{per definizione di $I_a$} \\
		&= (I_{a_1} \circ I_{a_2})(x) && \text{per composizione di mappe}
	\end{align}

	\emph{Nota bene: } l'omomorfismo $I$ in genere non è suriettivo. Se, per esempio, $G$ è abeliamo, allora anche la mappa $\iota$ dell'esercizio~\ref{ex:automorfismo_iota} è un automorfismo, ma $\iota$ non è un'immagine di $I$.
	
	Il nucleo di $I$ può essere determinato come:
	
	\begin{align}
		\ker I &= \{a \taleche I_a = 1\} && \text{per definizione di nucleo} \\
		&= \{a \taleche \forall x \in G\,\,  axa^{-1} = x \} && \text{per definizione di $I_a$} \\
		&= \{a \taleche \forall x \in G\,\, ax = xa \} && \text{per la proprietà degli elementi inversi} \\
		&= Z(G) && \text{per definizione di centro}
	\end{align}
	
	Per verificare che $\Inn G$ è un sottogruppo di $\Aut G$ basta osservare che è l'immagine di $G$ tramite la mappa $I$:
	
	\begin{equation}
		\Inn G = I(G)
	\end{equation}

	Quindi per il teorema fondamentale degli omomorfismi~\ref{thr:Omomorfismi_fondamentale}, $\Inn G$, che è l'immagine dell'omomorfismo $I$, è un sottogruppo di $\Aut G$. Inoltre, visto che $\ker I = Z(G)$, allora $G/Z(G)$ è isomorfo all'immagine $\Inn G$ (figura~\ref{fig:automorfismi_interni}).
	
	
	Per verificare che $Inn G$ è un sottogruppo normale di $Aut G$ dobbiamo verificare che, per ogni $A \in \Aut G$:
	
	\begin{equation}
		A \circ \Inn G \circ A^{-1} = \Inn G
	\end{equation}

	Ovvero:
	
	\begin{equation}
		\forall I_a \in \Inn G \quad A \circ I_a \circ A^{-1} \in \Inn G
	\end{equation}
	
	Ovvero:
	
	\begin{equation}
		\forall I_a \in \Inn G,\, \forall x \in G,\, \exists I_b \in \Inn G \quad (A \circ I_a \circ A^{-1})(x) = I_b(x)
	\end{equation}

	Infatti:
	
	\begin{align}
		(A \circ I_a \circ A^{-1})(x) &= A(I_a(A^{-1}(x))) &&\text{per composizione di automorfismi} \\
		&= A(a A^{-1}(x) a^{-1}) && \text{per definizione di $I_a$} \\
		&= A(a) A(A^{-1}(x))A(a^{-1}) && \text{perché A è un automorfismo} \\
		&= A(a) A(A^{-1}(x))A(a)^{-1} && \text{perché A è un automorfismo} \\
		&= A(a) x A(a)^{-1} && \text{perché $A$ e $A^{-1}$ sono mappe inverse} \\
		&= I_{A(a)}(x) && \text{per definizione di $I_a$}
	\end{align}
	
\end{soluzione}

\begin{figure}[tp]
	\centering
	\tikz {
		\node (a) at (0,3) {$G$};
		\node (b) at (3,3) {$\Aut G$};
		\node (c) at (0,0) {$G/Z(G)$};
		\draw (a) edge[->] node[above] {$I$} (b); 
		\draw (a) edge[->] (c);
		\draw (c) edge[->] (b);
	}
	\caption{Automorfismi interni}
	\label{fig:automorfismi_interni}
\end{figure}

\begin{esercizio}
	Se $\Aut G = 1$ allora $\ordine{G} \le 2$.
	\footnote{Cfr. \cite[pag. 63, es. n. 8]{jacobson}}
\end{esercizio}
\begin{soluzione}
	Se non ci sono automorfismi, non ci sono neanche automorfismi interni. Quindi deve risultare:
	
	\begin{align}
		\Inn G = 1 &\quad\text{perché l'automorfismo interno deve essere identico} \\
		G/Z(G) = 1 &\quad\text{perché gli automorfismi interni sono isomorfi a $G/Z(G)$} \\
		Z(G) = G &\quad\text{perché il gruppo quoziente contiene solo l'identità} \\
		G \text{ è abeliano} &\quad\text{perché il gruppo coincide con il suo centro}
	\end{align}

	Se il gruppo è abeliano, allora è presente anche l'automorfismo $\iota$ (vedi esercizio~\ref{ex:automorfismo_iota}), ma anche tale automorfismo deve essere identico, quindi:
	
	\begin{align}
		\iota = 1 &\quad\text{perché l'automorfismo deve essere identico} \\
		\forall g \in G\quad g = g^{-1} &\quad\text{perché ogni elemento deve essere inverso di se stesso} \\
		\forall g \in G\quad g^2 = 1 &\quad\text{per il secondo principio di equivalenza}
	\end{align}

	Quindi tutti gli elementi di $G$ hanno al più ordine 2.

	Passiamo ora alla notazione additiva. 
	
	Supponiamo di avere un gruppo $G$ abeliano in cui, $\forall g \in G$, $pg = 0$, con $p$ primo.
	
	Prendiamo inoltre il gruppo $F = \Z/p\Z$. Gli elementi di questo gruppo hanno la forma $z + p\Z$.
	
	Definiamo inoltre l'operazione:
	
	\begin{equation}
		(z + p\Z) \circ g := zg
	\end{equation}

	Verifichiamo innanzitutto che questa operazione sia ben definita, ovvero se prendiamo due elementi equivalenti di $F$ otteniamo lo stesso elemento di $G$. Se ho due elementi equivalenti risulta:
	
	\begin{equation}
		z_1 + p\Z = z_2 + p\Z
	\end{equation}

	Risulta ora:
	
	\begin{equation}
		z_1g - z_2g = (z_1 - z_2)g
	\end{equation}

	Ma $z_1 \cong z_2 \mod p$, quindi $z_1 - z_2$ è un multiplo di $p$:
	
	\begin{equation}
		z_1g - z_2g = (z_1 - z_2)g = tpg = t0 = 0
	\end{equation}

	Quindi:
	
	\begin{equation}
		z_1g - z_2g = 0 \quad\Longrightarrow\quad z_1g = z_2g
	\end{equation}

	Quindi l'operazione è ben definita.
	
	L'operazione esterna è un prodotto esterno di un vettore ($g$) per uno scalare ($z$): $G$ è uno spazio vettoriale su $F$.
	
	Quindi \emph{se $G$ è un gruppo abeliano con tutti gli elementi di ordine $p$, $G$ può essere considerato uno spazio vettoriale su $F = \Z/p\Z$}.
	
	Inoltre un automorfismo di $G$ è un'applicazione lineare invertibile nello spazio vettoriale $G$ e \emph{$\Aut G$ coincide con il gruppo delle applicazioni lineari invertibili}.
	
	Torniamo all'esercizio in cui abbiamo capito che abbiamo un gruppo $G$ abeliano i cui elementi hanno tutti ordine al più 2. Quindi $G$ è uno spazio vettoriale su $F = \Z/2\Z$.
	
	Se $G$ è uno spazio vettoriale, ha anche una \emph{base} e possiamo domandarci qual è la dimensione dello spazio.
	
	Ipotizziamo (per assurdo) che la dimensione sia maggiore o uguale a 2. Esistono quindi in $G$ almeno due vettori $g_1$ e $g_2$ linearmente indipendenti che appartengono ad una base dello spazio vettoriale.
	
	Esiste quindi un'applicazione lineare che scambia $g_1$ con $g_2$ lasciando fissi tutti gli altri vettori di base. Tale applicazione lineare è un automorfismo non identico, quindi non è ammesso per il gruppo $G$ che stiamo considerando.
	
	Quindi il gruppo $G$ deve avere dimensione al massimo 1.
	
	Quindi il gruppo $G$ può avere dimensione 1 e avere due elementi (la base e il vettore nullo), oppure avere dimensione 0 e avere un solo elemento (il vettore nullo).
	
	Quindi:
	
	\begin{equation}
		\ordine{G} \le 2
	\end{equation}
	
\end{soluzione}

\begin{esercizio}
	Dimostrare che:
	
	\begin{equation}
		\Aut(\Sym(3)) \isomorfo \Sym(3)
	\end{equation}
\end{esercizio}
\begin{soluzione}
	Per quanto visto nell'esercizio~\ref{ex:automorfismi_interni} sappiamo che:
	
	\begin{equation}
		\Sym(3) / Z(\Sym(3)) \isomorfo \Inn(Sym(3)) \normale \Aut(Sym(3))
	\end{equation}

	Ma nessun elemento di $Z(\Sym(3))$ (a parte l'identità) è commutativo, quindi:
	
	\begin{gather}
		Z(\Sym(3)) = 1 \quad\Longrightarrow\quad \Sym(3) / Z(\Sym(3)) = \Sym(3)
	\end{gather}

	Per cui possiamo concludere che:
	
	\begin{equation}
		\Sym(3) \isomorfo \Inn(Sym(3)) \normale \Aut(Sym(3))
	\end{equation}

	Ci resta ora da dimostrare che $\Aut(\Sym(3))$ è isomorfico ad un sottogruppo di $\Sym(3))$.
	\footnote{La soluzione di questo esercizio non è presente nelle registrazioni del corso, e la seconda parte della dimostrazione riportata su \cite{lucchini_week} è troppo lacunosa per poter essere compresa.}
\end{soluzione}
	\chapter{Azione di un gruppo su un insieme}

Quando si ha a che fare con i gruppi finiti è opportuno procurarsi degli intelligenti \emph{argomenti di conteggio}.

Conosciamo bene i gruppi di permutazione: è utile realizzare un gruppo come gruppo di permutazioni.

Lo sappiamo fare con il teorema di Cayley \textbf{RIF}. Ma se ho un gruppo di ordine $n$, lo devo mettere nel gruppo simmetrico di grado $n$, che ha ordine $n^2$, quindi mi trovo in un ambiente troppo grande.

Allora un trucco è quello di realizzare il mio gruppo come gruppo di permutazioni ma in modi diversi, sperando di trovarne uno semplice.

\section[Azione di un gruppo su un insieme]{Azione di un gruppo su un insieme\footnote{\cite[8 novembre 2021]{lucchini}}}

Prendiamo un gruppo $G$ e un insieme $\Omega$.
Consideriamo l'omomorfismo:

\begin{equation}
	\sigma: G \longrightarrow \Sym(\Omega)
\end{equation}

Non possiamo pretendere che $\sigma$ sia biiettiva.

Avremo quindi un sottogruppo normale rappresentato da $\ker \sigma$ tale che:

\begin{equation}
	\dfrac{G}{\ker \sigma} \isomorfo \Sym(\Omega)
\end{equation} 

Più omomorfismo ho di questo tipo più probabilità ho di trovare un buon elemento di conteggio.

Consideriamo $g \in G$ e $\omega \in \Omega$. $\sigma(g)$ è una permutazione di $\Omega$. Quindi posso applicare la permutazione $\sigma(g)$ a $\omega$, e trovo che questa sta in $\Omega$:

\begin{equation}
	\sigma(g)(\omega) \in \Omega
\end{equation}

Definiamo una notazione più rapida:

\begin{equation}
	g \circ \omega := \sigma(g)(\omega)
\end{equation}

Un'azione è una mappa tale che:

\begin{gather}
	G \times \Omega \longrightarrow \Omega \\
	(g, \omega) \longmapsto g \circ \omega
\end{gather}

Proprietà di $\circ$:

\begin{enumerate}
	\item $1 \circ \omega = \omega \quad \forall \omega \in \Omega$ \\
	perché $\sigma$ è un omomorfismo, quindi $1_G \longmapsto 1_{\Sym(\Omega)}$;
	\item $g_1 \circ (g_2 \circ \omega) = (g_1 g_2) \circ \omega \quad \forall g_1, g_2 \in G; \forall \omega \in \Omega$
\end{enumerate}

Se conosco la mappa, come trovo $\sigma$?

Per ogni elemento $g \in G$ definisco:

\begin{gather}
	\sigma_g: \Omega \longrightarrow \Omega \\
	\omega \longmapsto g \circ \omega
\end{gather}

Questa $\sigma_g$ è una biiezione, perché ha come inverso la $\sigma_{g^{-1}}$.

Ora posso costruire la mappa:

\begin{gather}
	\sigma: G \longrightarrow \Sym(\Omega) \\
	g \longmapsto \sigma_g
\end{gather}

E questa mappa è un omomorfismo.

Non dimenticare:

\begin{align}
	\ker \sigma &= \{g \in G \taleche g \circ \omega = \omega, \forall \omega \in \Omega\} \\
	&= \{g \in G \taleche \sigma(g) = 1_{\Sym(\Omega)}\}
\end{align}

\section[Stabilizzatore]{Stabilizzatore\footnote{\cite[8 novembre 2021]{lucchini}}}

Sia dato un gruppo $G$ che agisce su un insieme $\Omega$, e un elemento $\omega \in \Omega$. Lo \textbf{stabilizzatore} di $\omega$ è:

\begin{equation}
	\Stab_G(\omega) = G_\omega = \{g \in G \taleche g \circ \omega = \omega\}
\end{equation}

\begin{teorema}
	Gli stabilizzatori sono sottogruppi di $G$:
	
	\begin{equation}
		\Stab_g(\omega) \le G
	\end{equation}
\end{teorema}
\begin{dimostrazione}
	Per quanto riguarda la composizione, consideriamo due elementi $g_1$ e $g_2$ appartenenti allo $Stab_G(\omega)$. Si ha:
	
	\begin{align}
		g_1g_2 \circ \omega &= g_1 \circ (g_2 \circ \omega) &\text{ per la proprietà 2} \\
		&= g_1 \circ \omega &\text{ perché } g_2 \in \Stab_G(\omega) \\
		&= \omega &\text{ perché } g_1 \in \Stab_G(\omega) \\
		& \Longrightarrow g_1g_2 \in \Stab_G(\omega)
	\end{align}

	Per quanto riguarda l'inverso, consideriamo un generico elemento $g \in G$. Si ha:
	
	\begin{align}
		g \circ \omega = \omega &\Longrightarrow g^{-1} \circ (g \circ \omega) = g^{-1} \circ \omega \\
		&\Longrightarrow g^{-1}g \circ \omega = g^{-1} \circ \omega \\
		&\Longrightarrow 1 \circ \omega = g^{-1} \circ \omega \\
		&\Longrightarrow \omega = g^{-1} \circ \omega
	\end{align}
\end{dimostrazione}

\begin{teorema}
	L'intersezione di tutti gli stabilizzatori è il nucleo:
	
	\begin{equation}
		\bigcap_{\omega \in \Omega}\Stab_G(\omega) = \ker \sigma
	\end{equation}
\end{teorema}

\section[Esempi di azioni]{Esempi di azioni\footnote{\cite[8 novembre 2021]{lucchini}}}

\subsection{Azione di Cayley}

\begin{gather}
	G = \Omega \\
	g \circ x := gx \\
	\Stab_G(x) = 1 \\
	\ker \sigma = 1
\end{gather}

Ritroviamo il teorema di Cayley \textbf{RIF}: $G \le \Sym(G)$.

\subsection{Coniugato di un elemento}

\begin{gather}
	G = \Omega \\
	g \circ x := gxg^{-1}
\end{gather}

Quindi l'azione dell'elemento $g$ sull'elemento $x$ è il coniugato di $x$ tramite $g$.

Abbiamo quindi che $\sigma_g$ è l'automorfismo interno indotto da $g$ per coniugazione \textbf{RIF}.

\begin{align}
	\Stab_G(x) &= \{g \taleche gxg^{-1} = x \} = \\\
	&= \{g \taleche gx = xg \} = \\
	&= C_G(x)
\end{align}

Lo stabilizzante dell'elemento $x$ è il centralizzatore di $x$ in $G$.

\begin{align}
	\ker \sigma &= \bigcap_{x \in G} C_G(x) = \\
	&= Z(G)
\end{align}

Il nucleo è il centro di $G$.

Ritroviamo che $Z(G)$ è un sottogruppo normale \textbf{RIF} in quanto nucleo di un omomorfismo.

\subsection{Coniugato di un sottogruppo}

\begin{gather}
	\Omega = \{ H \taleche H \le G \} \\
	g \circ H := \{ghg^{-1} \taleche h \in H\} = gHg^{-1}
\end{gather}

L'azione dell'elemento $g$ sul sottogruppo $H$ è il coniugato di $H$ tramite $g$.

\begin{align}
	\Stab_G(x) &= \{g \taleche gHg^{-1} = H \} = \\\
	&= N_G(x)
\end{align}

Lo stabilizzatore l'insieme degli elementi $g$ che normalizzano $H$: è il \textbf{normalizzante} di $H$ in $G$.

\subsection{Azione sui laterali sinistri}

Dato un sottogruppo $H \le G$, consideriamo l'insieme dei laterali sinistri:

\begin{equation}
	\Omega = \{ xH \taleche x \in G \}
\end{equation}
	
Definiamo l'azione sui laterali sinistri:

\begin{equation}
	g \circ xH := gxH
\end{equation}

Stabilizzatore:

\begin{align}
	g \in \Stab_G(xH) &\Longleftrightarrow g \circ xH = xH \\
	&\Longleftrightarrow gxH = xH \\
	&\Longleftrightarrow gx \in xH \\
	&\Longleftrightarrow g \in xHx^{-1} \\
	\Stab_G(xH) &= xHx^{-1}
\end{align}

Definiamo il \textbf{cuore normale} di $H$ in $G$ come:

\begin{equation}
	H_G := \ker \sigma = \bigcap_{x \in G} \Stab_G(xH)
\end{equation}

Viene definito \emph{normale} perché è il nucleo di un'azione, quindi è un sottogruppo normale di $G$:

\begin{equation}
	H_G \normale G
\end{equation}

Viene definito \emph{cuore} perché è contenuto in $H$, infatti posso prendere $x = 1$ quando faccio i coniugati, per cui ottengo che:

\begin{equation}
	H_G \subseteq H
\end{equation}

\begin{teorema}
	$H_G$ è il più grande sottogruppo normale di G contenuto in H.
\end{teorema}
\begin{dimostrazione}
	Consideriamo un generico sottogruppo normale $N$ di $G$ contenuto in $H$:
	
	\begin{gather}
		N \subseteq H \\
		N \normale G
	\end{gather}

	Faccio il coniugio dei due sottogruppi:
	
	\begin{equation}
		gNg^{-1} \subseteq gHg^{-1}
	\end{equation}

	$N$ è normale quindi $gNg^{-1} = N$, quindi:
	
	\begin{equation}
		N \subseteq gHg^{-1}
	\end{equation}

	Questa proposizione è vera $\forall g$, quindi:
	
	\begin{equation}
		N \subseteq \bigcap_{g \in G} gHg^{-1} = H_G
	\end{equation}

	Quindi $H_G$ è il più grande sottogruppo normale di $G$ contenuto in $H$.
\end{dimostrazione}

\begin{teorema}
	\label{thr:Azioni_laterali_indice_finito}
	Dato $H \le G$ con $\indice{G}{H}= m$ finito, se $G$ agisce sull'insieme $\Omega = \{xH \taleche x \in G\}$, allora $\indice{G}{H_G}$ divide $m!$ (figura~\ref{fig:Azioni_laterali_indice_finito}).
\end{teorema}
\begin{dimostrazione}
	Ricordo che l'indice $\indice{G}{H}$ è il numero di laterali (sinistri o destri) di $H$ in $G$. Quindi la cardinalità di $\Omega$ è $m$ e l'omomorfismo $\sigma$ è:
	
	\begin{equation}
		\sigma: G \longrightarrow \Sym(\Omega) = S_m
	\end{equation}

	Inoltre il nucleo $\ker \sigma$ è il cuore normale $H_G$, che è un sottogruppo normale di $H$. Quindi per il secondo corollario~\ref{crl:Omomorfismi_fondamentale_2} del teorema fondamentale degli omomorfismi si ha:
	
	\begin{equation}
		\dfrac{G}{\ker \sigma} \isomorfo \sigma(G) \le S_m
	\end{equation}

	Per il teorema~\ref{thr:Laterali_Lagrange} di Lagrange, l'ordine di un sottogruppo divide l'ordine del gruppo, quindi:
	
	\begin{equation}
		\indice{G}{H_G} = \ordine{\dfrac{G}{\ker \sigma}} = \ordine{\sigma(G)} \text{ divide } \ordine{S_m} = m!
	\end{equation}
	
	
\end{dimostrazione}

\begin{figure}[tp]
	\centering
	\tikz {
		\node (a) at (0,4) {$G$};
		\node (b) at (0,2) {$H$};
		\node (c) at (0,0) {$\ker \sigma = H_G$};
		\draw (a) edge[-] (b);
		\draw (b) edge[-] (c);
		\draw[decorate,decoration={brace,raise=20pt}] (a) -- (b) node[pos=.5,right=23pt,black]{$m$};
		\draw[decorate,decoration={brace,raise=50pt}] (a) -- (c) node[pos=.5,right=53pt,black]{divide $m!$};
	}
	\caption{Rappresentazione del teorema \ref{thr:Azioni_laterali_indice_finito}.}
	\label{fig:Azioni_laterali_indice_finito}
\end{figure}

\begin{teorema}
	Se un gruppo $G$ infinito contiene un sottogruppo $H$ di indice finito, allora contiene anche un sottogruppo normale di indice finito.
\end{teorema}
\begin{dimostrazione}
	Per il teorema~\ref{thr:Azioni_laterali_indice_finito}, chiamato $m = \indice{G}{H}$, $G$ contiene il sottogruppo normale $H_G$ il cui indice divide $m!$. Quindi $\indice{G}{H_G}$ è finito.
\end{dimostrazione}

Dimostriamo ora un'estensione del teorema \textbf{RIFERIMENTO}.

\begin{esercizio}
	Sia dato un gruppo finito $G$ con $p$ il più piccolo divisore dell'ordine $\ordine{G}$. Se $H \le G$ e $\indice{G}{H} = p$, allora $H$ è normale.
	\footnote{\cite[Week 2, Exercise 8]{lucchini_week}}
\end{esercizio}
\begin{soluzione}
	Per il teorema~\ref{thr:Azioni_laterali_indice_finito} sappiamo che:
	
	\begin{equation}
		\indice{G}{H_G} \text{ divide } p!
	\end{equation}

	Quindi:
	\footnote{
		Se $\indice{G}{H_G}$ divide $p!$, allora esiste un intero positivo $a$ per il quale:
		
		\begin{equation}
			a \cdot \indice{G}{H_G} = p! \Longrightarrow \indice{G}{H_G} = \dfrac{p!}{a}
		\end{equation} 
	
		Per il teorema \textbf{RIFERIMENTO} abbiamo:
		
		\begin{equation}
			\indice{G}{H_G} = \indice{G}{H} \cdot \indice{H}{H_G} \Longrightarrow
			\dfrac{p!}{a} = p \cdot \indice{H}{H_G} \Longrightarrow
			\indice{H}{H_G} = \dfrac{(p-1)!}{a}
		\end{equation}
	
		Quindi $\indice{H}{H_G}$ divide $(p-1)!$.
	}
	
	\begin{equation}
		\indice{H}{H_G} \text{ divide } \dfrac{p!}{p} \Longrightarrow \indice{H}{H_G} \text{ divide } (p-1)!
	\end{equation}

	Per il teorema~\ref{thr:Laterali_Lagrange} di Lagrange:
	
	\begin{equation}
		\indice{H}{H_G} \dividetxt \ordine{G}
	\end{equation}

	In $(p-1)!$ tutti i divisori sono più piccoli di $p$, il quale è il più piccolo divisore primo di $\ordine{G}$. Quindi $(p-1)!$ e $\ordine{G}$ sono coprimi, quindi il loro MCD è 1. Quindi:
	
	\begin{equation}
		\indice{H}{H_G} = 1 \Longrightarrow H = H_G
	\end{equation} 

	Quindi $H$ è normale.

\end{soluzione}

\section[Orbita di un elemento di un insieme]{Orbita di un elemento di un insieme\footnote{\cite[8 novembre 2021]{lucchini}}}

Considero un gruppo $G$ che agisce su un insieme $\Omega$. Definisco la seguente relazione tra gli elementi di $\Omega$:

\begin{equation}
	\omega_1 \eq \omega_2 \sse \exists g \in G \taleche \omega_2 = g \circ \omega_1
\end{equation}

\begin{teorema}
	$\eq$ è una relazione di equivalenza.
\end{teorema}
\begin{dimostrazione}
	La relazione $\eq$ è riflessiva, infatti:
	
	\begin{equation}
		\omega = 1 \circ \omega
	\end{equation}

	La relazione $\eq$ è simmetrica, infatti:
	
	\begin{equation}
		\omega_2 = g \circ \omega_1 \Longrightarrow g^{-1} \circ \omega_2 = g^{-1} \circ g \circ \omega_1 \Longrightarrow \omega_1 = g^{-1} \circ \omega_2
	\end{equation}

	La relazione $\eq$ è transitiva, infatti:
	
	\begin{equation}
		\omega_2 = g_1 \circ \omega_1 \land \omega_3 = g_2 \circ \omega_2 \Longrightarrow \omega_1 = g_2g_1 \circ \omega_1
	\end{equation}
\end{dimostrazione}

Possiamo allora passare alle classi di equivalenza che sono le \textbf{orbite} di $G$ su $\Omega$.

Posso indicare con $G \circ \omega$ l'\textbf{orbita} di $\omega$.

\begin{teorema}
	\label{thr:orbite}
	Dati:
	
	\begin{itemize}
		\item $G$ gruppo che agisce sull'insieme $\Omega$;
		\item $\omega \in \Omega$;
		\item $O = G \circ \omega$;
		\item $H = \Stab_G(\omega)$;
		\item $\Lambda = \{xH \taleche x \in G\}$
	\end{itemize}

	La funzione:
	
	\begin{align}
		\gamma :\quad \Lambda &\longrightarrow O \\
		xH &\longmapsto x \circ \omega
	\end{align}

	è una funzione ben definita e biiettiva.
\end{teorema}
\begin{dimostrazione}
	La funzione $\gamma$ è ben definita, infatti è indipendente dal rappresentante del laterale sinistro che utilizzo. Se infatti $x_1H = x_2H$, allora:
	\begin{equation*}
		\exists h \in H \taleche x_2 = x_1 h
	\end{equation*}
	
	Quindi accade che:
	
	\begin{align}
		x_2 \circ \omega &= x_1 h \circ \omega \\
		&= x_1 \circ (h \circ \omega) \\
		&= x_1 \circ \omega
	\end{align}

	La funzione $\gamma$ è suriettiva per definizione di orbita. 
	
	La funzione $\gamma$ è iniettiva perché:
	
	\begin{align}
		\gamma(x_1H) = \gamma(x_2H) 
		&\Longrightarrow x_1 \circ \omega = x_2 \circ \omega \\
		&\Longrightarrow x_2^{-1} \circ x_1 \circ \omega = \omega \\
		&\Longrightarrow x_2^{-1} x_1 \in H \quad \text{per definizione di stabilizzatore} \\
		&\Longrightarrow x_1H = x_2H
	\end{align}
\end{dimostrazione}

Se esiste una corrispondenza biiettiva significa che $\Lambda$ e $O$ hanno la stessa cardinalità:

\begin{equation}
	\label{eq:ordine_orbita} \ordine{O} = \ordine{\Lambda} = \indice{G}{\Stab_G(\omega)}
\end{equation}

Per calcolare quanti elementi ci sono nell'orbita di un elemento $\omega$ basta calcolare lo sbabilizzatore e l'indice.

Se $G$ agisce su $G$ \textbf{per coniugio}, come sono fatte le orbite?

Prendiamo un generico elemento $x \in G$. L'orbita è:

\begin{equation}
	G \circ x = \{gxg^{-1} \taleche g \in G\}
\end{equation}

Ovvero è la classe di coniugio di $x$ in $G$. Per calcolare quanti elementi ha questa classe basta:

\begin{equation}
	\ordine{G \circ x} = \indice{G}{\Stab_G(x)} = \indice{G}{C_G(x)}
\end{equation}

\textbf{L'ordine della classe di coniugio di un elemento è uguale all'indice del centralizzante dello stesso elemento:}

\begin{equation}
	\label{eq:ordine_classe_coniugio}
	\ordine{G \circ x} = \indice{G}{C_G(x)}
\end{equation}

Se invece G agisce \textbf{per coniugio sui sottogruppi}, dato un sottogruppo $H \le G$, quanti sono i coniugati di $H$ in $G$?

\begin{equation}
	\ordine{G \circ H} = \indice{G}{\Stab_G(H)} = \indice{G}{N_G(x)}
\end{equation}


	\chapter{Equazione delle classi}

\section[Equazione delle classi]{Equazione delle classi\footnote{\cite[8 novembre 2021]{lucchini}; \cite[Week 2, 5]{lucchini_week}}}

Consideriamo un insieme $G$ che agisce per coniugio sui propri elementi:

\begin{equation}
	g \circ x = gxg^{-1}
\end{equation}

Consideriamo le diverse orbite dell'azione, che sono anche le classi di coniugio:

\begin{equation}
	C_1, C_2, \dots, C_T
\end{equation}

Le classi formano una partizione dell'insieme $G$, quindi la cardinalità dell'insieme è pari alla somma delle cardinalità delle classi:

\begin{equation}
	\ordine{G} = \sum_i \ordine{C_i}
\end{equation}

Per ogni classe di coniugio posso scegliere un elemento $g_i \in C_i$:

\begin{equation}
	\ordine{G} = \sum_i \indice{G}{C_G(g_i)}
\end{equation}

Quest'ultima è detta \textbf{equazione delle classi}.

	Se prendo un elemento $g \in Z(G)$ allora la sua classe di coniugio $\{g\}$ è formata dal solo elemento $g$, quindi è formata da un solo elemento.

Quindi abbiamo $r$ classi di coniugio di cardinalità 1, mentre le altre avranno cardinalità maggiore di 1:

\begin{gather}
	\ordine{C_1} = \dots = \ordine{C_r} = 1 \\
	\ordine{C_k} > 0 \quad \forall k > r
\end{gather}

Quindi:

\begin{equation}
	r = \ordine{Z(G)}
\end{equation}

Passiamo all'equazione delle classi:

\begin{align}
	\ordine{G} &= \sum_i \ordine{C_i} = \\
	&= \ordine{Z(G)} + \sum_{k> r}\ordine{C_k}
\end{align}

\begin{teorema}
	Se $\ordine{G} = p^n$ con $p$ primo e $n > 0$, allora $Z(G) \ne 1$.
	\footnote{\cite[Teorema 1.11, pag. 76]{jacobson}}
\end{teorema}
\begin{dimostrazione}	
	Per il teorema~\ref{thr:Laterali_Lagrange} di Lagrange:
	
	\begin{equation}
		\ordine{C_k} = \indice{G}{C_G(g_k)} \text{ divide } \ordine{G} = p^n
	\end{equation} 
	
	Quindi $\ordine{C_k}$ è divisibile per $p$, ovvero:
	
	\begin{equation}
		\ordine{C_k} \equiv 0 \mod p
	\end{equation}
	
	Ma anche:
	
	\begin{equation}
		\ordine{G} \equiv 0 \mod p
	\end{equation}
	
	Quindi $\ordine{Z(G)}$ è differenza di due numeri congrui a 0 modulo $p$:
	
	\begin{equation}
		\ordine{Z(G)} \equiv 0 \mod p
	\end{equation}
	
	Ma $Z(G)$ contiene almeno l'unità 1, quindi il suo ordine è diverso da 0. Quindi:
	
	\begin{equation}
		\ordine{Z(G))} \ge p
	\end{equation}
\end{dimostrazione}

	\chapter{Esercizi sulle classi di coniugio}

\begin{corollario}
	Due elementi di $S_n$ sono coniugati se hanno la stessa struttura ciclica.
	\footnote{Nella registrazione \cite[13 ottobre 2021]{lucchini} manca il primo pezzo, dove immagino abbia discusso come due elementi coniugati di $S_n$ hanno la stessa struttura ciclica. Per esempio, se prendiamo:
	
		\begin{equation}
			\alpha = (124) \in S_6
		\end{equation}

	e lo coniughiamo con:
	
	\begin{equation}
		\beta = (13)(25)(46)
	\end{equation} 

	Mi basta sostituire gli elementi di $\alpha$ usando i cicli di $\beta$, quindi:
	
	\begin{itemize}
		\item mando l'1 nel 3 perché in $\beta$ c'è (13);
		\item mando il 2 nel 5 perché in $\beta$ c'è (25);
		\item mando il 4 nel 6 perché in $\beta$ c'è (46).
	\end{itemize}

	Quindi:
	
	\begin{equation}
		\beta \alpha \beta^{-1} = (356)
	\end{equation}
	}
\end{corollario}

Per esempio:

\begin{gather}
	\sigma_1 = (149)(25) \in S_{10} \\
	\sigma_2 = (273)(14) \in S_{10}
\end{gather}

Cerco un $\beta$ tale che:

\begin{equation}
	\beta \sigma_1 \beta^{-1} = \sigma_2
\end{equation}

$\beta$ deve:
\begin{itemize}
	\item mandare l'1 nel 2;
	\item mandare il 4 nel 7;
	\item mandare il 9 nel 3;
	\item mandare il 2 nell'1;
	\item mandare il 5 nel 4.
\end{itemize}

Ci sono ancora diversi gradi di libertà perché ci sono diversi elementi non nominati. Una possibile soluzione potrebbe essere:

\begin{equation}
	\beta = (12)(547)(39)
\end{equation}

\begin{esercizio}
	Descrivi le classi di coniugio e i centralizzatori degli elementi nel gruppo simmetrico $S_4$.
\end{esercizio}
\begin{soluzione}
		
	Per ogni elemento voglio contare i coniugati e vedere chi è il centralizzante.

	Per il teorema precedente\footnote{mancante}, le classi di coniugio corrispondono alle possibili strutture cicliche. In particolare \textbf{il numero di coniugati corrisponde al numero di elementi della struttura ciclica}.
	
	Di identità ce n'è una sola.
	
	I 2-cicli sono pari al numero di combinazioni di 4 elementi in gruppi da 2. Uso le combinazioni perché i cicli $(12)$ e $(21)$ sono in verità lo stesso ciclo, quindi non devo considerare gruppi con elementi ordinati. Quindi:
	
	\begin{equation}
		\text{numero di 2-cicli } = \binom{4}{2} = \dfrac{4!}{2! \cdot 2!} = \dfrac{4 \cdot 3}{2} = 6
	\end{equation}

	Per contare i 3-cicli devo:
	
	\begin{itemize}
		\item decidere quale elemento deve stare fermo, e ho 4 scelte (per esempio tengo fermo l'elemento 4);
		\item per ciascun elemento fermo ho un 3-ciclo e il suo inverso (quindi ho i cicli $(123)$ e $(132)$).
	\end{itemize}

	Quindi in totale ho:
	
	\begin{equation}
		\text{numero di 3-cicli } = 4 \cdot 2 = 8
	\end{equation}

	Per contare i 4-cicli considero che il primo elemento è indifferente, mentre gli altri possono essere in qualunque ordine. Quindi:
	
	\begin{equation}
		\text{numero di 4-cicli } = 3! = 6
	\end{equation}

	Per contare i doppi 2-cicli basta accoppiare i 2-cicli, quindi:
	
	\begin{equation}
		\text{numero di doppi 2-cicli } = 6 : 2 = 3
	\end{equation}
	
	Dal momento che ho fatto una partizione, la somma deve fare l'ordine di $S_4$, che è 24, infatti:
	
	\begin{equation}
		1 + 6 + 8 + 6 + 3 = 24
	\end{equation}
	
	Per la formula~\eqref{eq:ordine_classe_coniugio} posso calcolare l'ordine dei centralizzanti come:
	
	\begin{equation}
		\ordine{C_G(g)} = \dfrac{\ordine{G \circ x}}{\ordine{G}}
	\end{equation}

	Quindi:
	
	\begin{itemize}
		\item l'identità ha la classe di coniugio di ordine 1, quindi il centralizzante di ordine 24 (infatti tutti gli elementi di un qualunque gruppo commutano con l'identità);
		\item il 2-ciclo $(12)$ ha la classe di coniugio di ordine 6, quindi il centralizzante di ordine 4;
		\item e così via...
	\end{itemize}
	
	Rimane da determinare i centralizzanti.
	
	Il centralizzante dell'identità è l'intero insieme $S_4$, visto che tutti gli elementi del gruppo commutano con l'identità e visto che l'ordine del centralizzante corrisponde all'ordine del gruppo.
	
	Chi commuta con $(12)$? Devo trovare 4 elementi. $(12)$ commuta con se stesso. Inoltre $(34)$ commuta con $(12)$ perché è un ciclo disgiunto. Il gruppo $\gen{(12)(34)}$ ha ordine 4, quindi è questo il centralizzante che sto cercando:
	
	\begin{equation}
		C_G((12)) = \gen{(12),(34)} \,\isomorfo C_2 \times C_2
	\end{equation}

	Chi commuta con $(123)$? Devo trovare 3 elementi. $(123)$ commuta con se stesso. Il sottogruppo $\gen{(123)}$ ha 3 elementi: 1, $(123)$ e $(132)$. Quindi è questo il centralizzante che sto cercando:
	
	\begin{equation}
		C_G((123)) = \gen{(123)} \,\isomorfo C_3
	\end{equation}
	
	Chi commuta con $(1234)$? Devo trovare 4 elementi. $(1234)$ commuta con se stesso e il sottogruppo $\gen{(1234)}$ ha 4 elementi: 1, $(1234)$, $(13)(24)$ e $(1432)$. Quindi è questo il centralizzante che sto cercando:
	
	\begin{equation}
		C_G((1234)) = \gen{(1234)} \,\isomorfo C_4
	\end{equation}
	
	Chi commuta con $(12)(34)$? Devo trovare 8 elementi.
	
	Osservo che:
	
	\begin{equation}
		(12)(34) = (1324)^2
	\end{equation}

	perché l'1 va nel 2 e viceversa, e il 3 va nel 4 e viceversa. Ma dal momento che $(1324)$ è quadrato di $(12)(34)$, i due termini commutano.
	
	Ho trovato i primi 4 elementi:
	
	\begin{equation}
		\gen{(1324)} = \{1, (1324), (12)(34), (1423)\}
	\end{equation}
	
	Me ne servono altri 4.
	
	Facciamo riferimento a $D_4$. $(1324)$ è una rotazione e il gruppo generato da questo elemento corrisponde a tutte le rotazioni di un quadrato (figura~\ref{fig:Isometrie_D4_da_1324}). Inoltre l'elemento $(12)(34)$ corrisponde alla rotazione di $\pi$. Il nostro elemento, quindi, è centro di $D_4$. Ci basta quindi prendere una riflessione di $D_4$ e trovare così tutti gli elementi che commutano con $(12)(34)$. Una tale riflessione potrebbe essere $(14)(23)$. Quindi il centro cercato è:
	
	\begin{equation}
		C_G((12)(34)) = \gen{(1324), (14)(23)} \isomorfo C_4 \times C_2 \isomorfo D_4
	\end{equation}
	
	
\end{soluzione}

	\begin{sidewaystable}
	\centering
	\begin{tabular}{ccccc}
		\toprule
		Struttura ciclica & Elemento & Numero coniugati & Ordine centralizzante & Centralizzante \\
		\midrule
		identità & 1 & 1 & 24 & $S_4$ \\
		2-cicli & (12) & $\binom{4}{2} = 6$ & 4 & $\gen{(12), (34)} \,\isomorfo C_2 \times C_2$ \\
		3-cicli & (123) & $4 \cdot 2 = 8$ & 3 & $\gen{(123)} \,\isomorfo C_3$ \\
		4-cicli & (1234) & $3! = 6$ & 4 & $\gen{(12345)} \,\isomorfo C_4$ \\
		doppi 2-cicli & (12)(34) & 6:2 = 3 & 8 & $\gen{(1324),(14)(23)} \,\isomorfo C_4 \times C_2 \isomorfo D_4$ \\
		\midrule
		& & $24 = \ordine{S_4}$ & & \\ 
		\bottomrule
	\end{tabular}
	\caption{Classi di coniugio e centralizzanti di $S_4$}
	\label{fig:classi_coniugio_s4}
\end{sidewaystable}

	\begin{figure}[tp]
	\centering
	\begin{tikzpicture}[line cap=round,line join=round,>=triangle 45,x=1cm,y=1cm]
		\begin{axis}[
			x=2cm,y=2cm,
			axis lines=middle,
			xmin=-2.5,
			xmax=2.5,
			ymin=-1.5,
			ymax=1.5,
			xtick={-3,3},
			ytick={-2,2},]
			\clip(-2.5,-1.5) rectangle (2.5,1.5);
			\fill[line width=2pt,color=figura,fill=figura,fill opacity=0.1] (1,-1) -- (1,1) -- (-1,1) -- (-1,-1) -- cycle;
			%\draw [line width=2pt] (0,0) circle (2cm);
			\draw [line width=2pt,color=figura] (1,-1)-- (1,1);
			\draw [line width=2pt,color=figura] (1,1)-- (-1,1);
			\draw [line width=2pt,color=figura] (-1,1)-- (-1,-1);
			\draw [line width=2pt,color=figura] (-1,-1)-- (1,-1);
			\begin{scriptsize}
				\draw (1.15,-1.05) node {1};
				\draw (1.15,1.05) node {3};
				\draw (-1.15,1.05) node {2};
				\draw (-1.15,-1.05) node {4};
				\draw (2.4,-.1) node {$x$};
				\draw (-0.1,1.4) node {$y$};
			\end{scriptsize}
		\end{axis}
	\end{tikzpicture}
	\caption{Gruppo $D_4$ per le rotazioni $(1324)^n$}
	\label{fig:Isometrie_D4_da_1324}
\end{figure}

\begin{esercizio}
	$\sigma \in S_n$ sia un $n$-ciclo. Determinare il suo centralizzante $C_{S_n}(\sigma)$.
\end{esercizio}
\begin{soluzione}
	Innanzitutto determiniamo quanti sono gli $n$-cicli. Al primo posto posso mettere un elemento qualunque, per semplicità poniamo di porre un 1. Devo poi combinare gli altri $n-1$ elementi in tutti i possibili ordini, quindi avrò:
	
	\begin{equation}
		\text{numero di $n$-cicli } = \ordine{^{S_n}\sigma} = P_n = n! 
	\end{equation}  

	Quindi per la formula~\eqref{eq:ordine_classe_coniugio} il centralizzante ha ordine:
	
	\begin{equation}
		\ordine{C_{S_n}(\sigma)} = \dfrac{\ordine{S_n}}{\ordine{^{S_n}\sigma}} = \dfrac{n!}{(n-1)!} = n
	\end{equation}

	Osserviamo ora che $\gen{\sigma}$, ovvero il più piccolo sottogruppo che contiene $\sigma$ ha ordine $n$ quindi:
	
	\begin{equation}
		C_{S_n}(\sigma) = \gen{\sigma} \isomorfo C_n
	\end{equation}

	Le uniche permutazioni di grado $n$ che commutano con un $n$-ciclo sono le potenze di quel ciclo.
\end{soluzione}

\begin{esercizio}
	Trovare tutti i sottogruppi normali di $S_4$.
\end{esercizio}
\begin{soluzione}
	Alcuni sottogruppi normali di $S_4$ sono di immediata individuazione:
	
	\begin{itemize}
		\item $S_4$: ogni gruppo è normale di se stesso;
		\item 1: il sottogruppo dell'identità è normale in quanto, per definizione, ogni elemento commuta con l'identità;
		\item $A_4$: il gruppo alterno ha indice 2, quindi è normale (per il teorema~\ref{thr:sottogruppi_di_indice_2})
	\end{itemize}

	Inoltre se consideriamo il gruppo diedrale $D_4$ abbiamo il sottogruppo:
	
	\begin{equation}
		V = \gen{(12)(34), (14)(23)} = \{1, (12)(34), (14)(23), (13)(24)\}
	\end{equation}

	$V$ è il \textbf{gruppo di Klein}. E' un sottogruppo normale in quanto composto dall'identità e dalla classe di coniugio dei doppi scambi.
	
	Abbiamo quindi trovato finora 4 sottogruppi normali di $S_4$:
	
	\begin{equation}
		1, V, A_4, S_4 \normale S_4
	\end{equation}

	Vogliamo ora dimostrare che non ce ne possono essere altri.
	
	Ipotizziamo per assurdo che esista un sottogruppo $N$ normale di $S_4$. Dobbiamo analizzare due situazioni.
	
	\textbf{Primo caso:} $N$ contiene $V$.
	
	Per il teorema di corrispondenza~\ref{thr:corrispondenza} i sottogruppi normali di $S_4$ che contengono $V$ sono in biiezione con i sottogruppi normali di $S_4/V$.
	
	Studiamo $S_4/V$.
	
	Consideriamo il sottogruppo di $S_4$ delle permutazioni che tengono ferme l'elemento 4 e permutano gli altri 3:
	
	\begin{equation}
		H = \{\sigma \taleche \sigma(4) = 4\} \le S_4
	\end{equation}

	Se tale gruppo agisce solo sugli elementi 1, 2 e 3, esso è isomorfo ad $S_3$:
	
	\begin{equation}
		H \isomorfo S_3
	\end{equation}

	Consideriamo ora l'insieme $HV$: visto che $V$ è un sottogruppo normale, allora per il teorema~\ref{thr:prodotto_sottogruppi} $HV$ è un sottogruppo di $S_4$.
	
	L'intersezione $H \cap V$ contiene solo l'identità in quanto $V$ contiene i doppi scambi e nessun doppio scambio tiene fermo 4.
	
	Quindi, per il teorema~\ref{thr:ordine_prodotto_sottogruppi}:
	
	\begin{equation}
		\ordine{HV} = \dfrac{\ordine{H}\ordine{V}}{\ordine{H \cap V}} = \dfrac{6 \cdot 4}{1} = 24
	\end{equation}

	Quindi:
	
	\begin{equation}
		\ordine{HV} = \ordine{S_4} \quad\Longrightarrow\quad HV = S_4
	\end{equation}

	Ora, per il teorema dei due sottogruppi~\ref{thr:due_sottogruppi} abbiamo:
	
	\begin{equation}
		\dfrac{S_4}{V} = \dfrac{HV}{V} \isomorfo \dfrac{H}{H \cap V} \isomorfo H \isomorfo S_3
	\end{equation}

	Quindi per scoprire i sottogruppi normali di $S_4/V$ ci basta conoscere i sottogruppi normali di $S_3$.
	
	I sottogruppi di $S_3$ sono:
	
	\begin{itemize}
		\item $\gen{(12)}$
		\item $\gen{(13)}$
		\item $\gen{(23)}$
		\item $\gen{(123)}$
	\end{itemize}

	I primi tre sottogruppi non sono normali, in quanto i loro generatori sono tra loro coniugati. Il quarto ha indice 2 quindi è normale.
	
	Posso infine usare il teorema di corrispondenza~\ref{thr:corrispondenza} e trovare i sottogruppi normali che contengono $V$: mi basta trovare le anti-immagini dei sottogruppi normali di $S_3$. $S_3$ è in corrispondenza con $S_4$, mentre l'identità di $S_3$ è in corrispondenza con il sottogruppo $V$ di $S_4$. Dobbiamo determinare chi è in corrispondenza con il sottogruppo $\gen{(123)}$ di $S_3$, ma tale sottogruppo ha indice 2, quindi è in corrispondenza con il sottogruppo normale di indice 2 di $S_4$, che è $A_4$ (figura~\ref{fig:corrispondenza_per_S4_su_V}).
	
	
	Non ci sono quindi altri sottogruppi normali di $S_4$ che contengono $V$.
	
	\textbf{Secondo caso:} $N$ non contiene $V$.
	
	Se $N$ non contiene $V$, significa che $N$ non contiene nessun doppio scambio.
	
	Il quadrato di un 4-ciclo è un doppio scambio:
	
	\begin{equation}
		(1234)^4 = (13)(24)
	\end{equation}
	
	Quindi se $N$ contenesse i 4-cicli allora conterrebbe anche i doppi scambi. Visto che non può contenere doppi scambi, $N$ non contiene i 4-cicli.
	
	Il prodotto di due 3-cicli è un doppio scambio:
	
	\begin{equation}
		(123) \cdot (124) = (13)(24)
	\end{equation}

	Quindi se $N$ contenesse i 3-cicli allora conterrebbe anche i doppi scambi. Visto che non può contenere doppi scambi, $N$ non contiene i 3-cicli.
	
	Il prodotto di due scambi disgiunti è un doppio scambio:
	
	\begin{equation}
		(12)\cdot (34) = (12)(34)
	\end{equation}

	Quindi se $N$ contenesse gli scambi allora conterrebbe anche i doppi scambi. Visto che non può contenere doppi scambi, $N$ non contiene gli scambi.

	Quindi solo l'identità può stare in $N$, ovvero se $N$ non contiene $V$, allora $N = 1$.
	
	Quindi non ho altri sottogruppi normali di $S_4$ al di fuori di $S_4$, $A_4$, $V$ e $1$.
\end{soluzione}
	\begin{figure}[tp]
	\centering
	\tikz {
		\node (a) at (0,4) {$S_4$};
		\node (b) at (0,2) {$N = A_4$};
		\node (c) at (0,0) {$V$};
		\node (x1) at (2,4) {$\Longleftrightarrow$};
		\node (x2) at (2,2) {$\Longleftrightarrow$};
		\node (x3) at (2,0) {$\Longleftrightarrow$};
		\node (d) at (4,4) {$S_4/V$};
		\node (e) at (4,2) {$N/V = A_4/V$};
		\node (f) at (4,0) {$V/V = 1$};
		\node (y1) at (6,4) {$\isomorfo$};
		\node (y2) at (6,2) {$\isomorfo$};
		\node (y3) at (6,0) {$\isomorfo$};
		\node (g) at (8,4) {$S_3$};
		\node (h) at (8,2) {$\gen{(123)}$};
		\node (i) at (8,0) {$1$};
		\draw (a) edge[->] (b) 
		(b) edge[->] (c);
		\draw (d) edge[->] (e) 
		(e) edge[->] (f);
		\draw (g) edge[->] (h) 
		(h) edge[->] (i);
	}
	\caption{Teorema di corrispondenza per $S_4/V$.}
	\label{fig:corrispondenza_per_S4_su_V}
\end{figure}

Osserviamo che in generale \textbf{la relazione di normalità non è transitiva}.

Prendiamo, per esempio:
\begin{equation}
	H = \gen{(12)(34)} \congruente C_2
\end{equation}

Prendiamo inoltre il gruppo di Klein $V$ che è abeliano, quindi tutti i suoi sottogruppi sono normali:
\begin{equation}
	H \normale V
\end{equation}

Inoltre abbiamo visto prima che:
\begin{equation}
	V \normale S_4
\end{equation}

Però $H$ non è un sottogruppo normale di $S_4$.

\begin{teorema}
	\label{th:uguaglianza_di_coniugi}
	Supponi che $xg_1 x^{-1} = g_2$.
	Dimostra che $y g_1 y^{-1} = g_2$ se e solo se $y \in x C_G(g_1)$.
\end{teorema}
\begin{dimostrazione}
	Vogliamo che i due coniugi siano uguali:
	\begin{equation}
		y g_1 y^{-1} = x g_1 x^{-1}
	\end{equation}
	Moltiplico per gli inversi per portare le $x$ al primo membro:
	\begin{gather}
		x^{-1} y g_1 y^{-1} x = g_1 \\
		x^{-1} y \in C_G(g_1) \\
		y \in x C_G(g_1)
	\end{gather}
\end{dimostrazione}

\begin{esercizio}
	\label{th:non_coniugati_in_a4}
	Dimostrare che $(123)$ e $(132)$ sono coniugati in $S_4$ ma non sono coniugati in $A_4$.
\end{esercizio}
\begin{soluzione}
	I 3-cicli $(123)$ e $(132)$ sono coniugati in $S_4$ perché hanno la stessa struttura ciclica.

	Determiniamo allora tutti i $\tau \in S_4$ tali che $\tau (123) \tau^{-1} = (132)$ e selezioniamo poi quelli
	presenti anche in $A_4$.

	Lo scambio $(23)$ ha le caratteristiche desiderate.

	Per il teorema~\ref{th:uguaglianza_di_coniugi} posso trovare tutti gli altri valori $\tau$:
	\begin{align}
		\tau \in (23) C_{S_4}(123) &= (23) \gen{(123)} = \\
		&= (23)\{1, (123), (132)\} = \\
		&= \{(23), (12), (13)\}
	\end{align}
	Abbiamo trovato solo permutazioni dispari.
	Quindi non esiste una permutazione $\tau$ pari tale che $\tau (123) \tau^{-1} = (132)$.

	Quindi $(123)$ e $(132)$ non sono coniugati in $A_4$: appartengono a classi di coniugio diverse.
\end{soluzione}

\begin{esercizio}
	Determinare le classi di coniugio di $A_4$.
\end{esercizio}

\begin{soluzione}
	Il gruppo $A_4$ contiene l'identità, i doppi scambi e i 3-cicli:
	\begin{equation}
		1 \quad (12)(34) \quad (123)
	\end{equation}

	In $A_4$ non ho più la proprietà che i coniugati di una permutazioni coincidono con tutte e sole le permutazioni
	con la stessa struttura ciclica.
	Quindi non ho uno strumento per contare facilmente i coniugati.
	Tuttavia conoscendo i centralizzanti di $S_4$ diventa facile trovare i centralizzanti di $A_4$ infatti:
	\begin{equation}
		C_{A_4}(\sigma) = C_{S_4}(\sigma) \cap A_4
	\end{equation}

	Determinati i centralizzanti e i loro ordini posso determinare le cardinalità delle classi di coniugio.

	\textbf{Centralizzante di $(12)(34)$.}
	
	Il centralizzante in $S_4$ dei doppi scambi è il gruppo diedrale $D_4$, quindi:
	\begin{equation}
		C_{A_4}((12)(34)) = A_4 \cap D_4
	\end{equation}
	
	$D_4$ ha ordine 8 e contiene delle permutazioni dispari.
	Se seleziono solo le permutazioni pari ottengo un gruppo di Klein $V$ che ha ordine 4.
	Quindi:
	\begin{gather}
		\ordine{C_{A_4}((12)(34))} = \ordine{V} = 4
	\end{gather}

	Il numero di coniugati di $(12)(34)$ è $12:4 = 3$, che corrisponde al numero di doppi scambi.

	\textbf{Centralizzante di $(123)$.}

	Il centralizzante in $S_4$ di un 3-ciclo è il sottogruppo generato dal 3-ciclo stesso, quindi:
	\begin{equation}
		C_{A_4}((123)) = A_4 \cap \gen{(123)} = \gen{(123)}
	\end{equation}

	Quindi il numero di coniugati è $(123)$ è $12:3 = 4$.
	Ma i 3-cicli in $A_4$ sono 8.
	Quindi non è più vero che tutti le permutazioni con la stessa struttura ciclica sono coniugate.

	Infatti nel teorema~\ref{th:non_coniugati_in_a4} abbiamo dimostrato che $(123)$ e $(132)$ non sono coniugate.

	Avremo quindi 4 coniugati per $(123)$ e 4 coniugati per $(132)$ (figura~\ref{fig:classi_coniugio_a4}).

\end{soluzione}

\begin{sidewaystable}
	\centering
	\begin{tabular}{ccccc}
		\toprule
		Struttura ciclica & Elemento & Numero coniugati & Ordine centralizzante & Centralizzante \\
		\midrule
		identità & 1 & 1 & 12 & $A_4$ \\
		3-cicli & (123) & 4 & 3 & $\gen{(123)} \,\isomorfo C_3$ \\
		3-cicli & (132) & 4 & 3 & $\gen{(123)} \,\isomorfo C_3$ \\
		doppi 2-cicli & (12)(34) & 3 & 4 & $\gen{(12)(34),(14)(23)} \,\isomorfo C_2 \times C_2 \isomorfo V$ \\
		\midrule
		& & $12 = \ordine{A_4}$ & & \\
		\bottomrule
	\end{tabular}
	\caption{Classi di coniugio e centralizzanti di $A_4$}
	\label{fig:classi_coniugio_a4}
\end{sidewaystable}

	\chapter{Sintesi sui prodotti}

Sia $G$ un gruppo; $H$ e $K$ due sottogruppi di $G$.

\section{Sottogruppo generato da $H$ e $K$}

$\gen{H, K}$ è il sottogruppo generato da $H$ e $K$, ovvero il più piccolo sottogruppo che contiene $H$ e $K$ (figura~\ref{fig:sottogruppo_generato_da_h_e_k}).

\begin{figure}[tp]
	\centering
	\tikz {
		\node (a) at (2,6) {$G$};
		\node (f) at (2,4) {$\gen{H, K}$};
		\node (b) at (0,2) {$H$};
		\node (d) at (4,2) {$K$};
		\draw (a) edge[->] (f);
		\draw (f) edge[->] (b);
		\draw (f) edge[->] (d);
	}
	\caption{Sottogruppo generato da $H$ e $K$.}
	\label{fig:sottogruppo_generato_da_h_e_k}
\end{figure} 

\section{Prodotto di sottogruppi}

\begin{equation}
	HK = \{hk \taleche h \in H \land k \in K\}
\end{equation}

$HK$ è l'insieme degli elementi di $G$ che sono esprimibili come prodotto di un elemento di $H$ per un elemento di $K$.

\begin{teorema}
	Se $H$ e $K$ sono finiti, allora:
	
	\begin{equation}
		\ordine{HK} = \dfrac{\ordine{H}\ordine{K}}{\ordine{H \cap K}}
	\end{equation}
\end{teorema}

$HK$ non è sempre un sottogruppo.

Per esempio:

\begin{gather}
	G = S_3 \\
	H = \gen{(12)} = \{1, (12)\}
	K = \gen{(13)} = \{1, (13)\} 
\end{gather}

L'insieme prodotto contiene:

\begin{gather}
	1 \cdot 1 = 1 \\
	1 \cdot (13) = (13) \\
	(12) \cdot 1 = (12) \\
	(12) \cdot (13) = (132)
\end{gather}

Quindi $\ordine{HK} = 4$, ma 4 non divide $6 = \ordine{S_3}$ quindi per il teorema di Lagrange~\ref{thr:Laterali_Lagrange} $HK$ non può essere un sottogruppo di $S_3$. Infatti mancano:

\begin{gather}
	(132)^{-1} = (123) \\
	(132)\cdot(12) = (23)
\end{gather}

\begin{teorema}
	$HK$ è un sottogruppo se e solo se $HK = KH$.
	
	In particolare questo si verifica quando uno dei due sottogruppi è normale.
\end{teorema}

In generale vale:

\begin{equation}
	HK \subseteq \gen{H, K}
\end{equation}

Infatti:

\begin{gather}
	H = \gen{(12)} \\
	K = \gen{(13)} \\
	HK \subseteq\, \gen{H, K} \,= S_3
\end{gather}

\begin{teorema}
	Se $HK$ è un sottogruppo, allora $HK = \gen{H, K}$.
\end{teorema}

\section{Prodotto diretto}

\begin{equation}
	H \times K = \{(h, k) \taleche h \in H \land k \in K\}
\end{equation}

Il prodotto tra due elementi del prodotto diretto viene effettuata componenti per componente:

\begin{equation}
	(h, k)(h', k') = (hh', kk')
\end{equation}

\begin{teorema}
	G \isomorfo H $\times$ K se e solo se $\begin{cases}
		G=HK \\
		H, K \normale G \\
		H \cap K = 1
	\end{cases}$.
\end{teorema}

Per comodità non si utilizza la notazione $(h, k)$ ma si scrivere $hk$.

\section{Ricapitolando}

Il prodotto di gruppi ci dà un'informazione solo sugli elementi, che possono essere scritti come prodotto di un elemento per ciascun fattore.

Il prodotto diretto è un caso particolare del precedente, in cui abbiamo anche un'informazione sull'operazione, che deve essere svolta componente per componente.

Consideriamo, per esempio, il gruppo $D_n$ in cui abbiamo:

\begin{itemize}
	\item $\gen{\rho}$: il sottogruppo delle rotazioni;
	\item $\gen{\iota}$: il sottogruppo generato da una riflessione.
\end{itemize}

Si ha:

\begin{equation}
	D_n = \gen{\rho}\gen{\iota}
\end{equation}

Quindi ogni elemento di $D_n$ è prodotto di un elemento di $\gen{\rho}$ e di un elemento di $\gen{\iota}$.

Ma:

\begin{equation}
	D_n \not\isomorfo \gen{\rho} \times \gen{\iota}
\end{equation}

Infatti le coppie $(\rho, 1)$ e $(1, \iota)$ commutano:

\begin{gather}
	(\rho, 1)(1, \iota) = (\rho,\iota) \\
	(1, \iota)(\rho, 1) = (\rho,\iota) \\
\end{gather}

Invece gli elementi $\rho$ e $\iota$ non commutano perché:

\begin{equation}
	\iota\rho\iota = \rho^{-1} \quad\Longleftrightarrow\quad \rho\iota = \iota\rho^{-1}
\end{equation}

	\part{Appendici}
	\chapter{$S_3$}
\label{ch:S3}

\section{Caratteristiche del gruppo}
\label{sec:s3_caratteristiche}

\begin{center}
	\begin{tabular}{lll}
		Gruppo abeliano & No & Per esempio: $(12)(13) \neq (13)(12)$ \\
		Gruppo ciclico & No & Non contiene elementi di ordine 6. \\
		Ordine & 6 & $P_3 = 3! = 6$\\
		Esponente & 6 & Il gruppo contiene elementi di ordine 1, 2 e 3. \\
		Isomorfismi & $C_2 \times C_3$ & $S_3 = \gen{(12), (123)}$ \\
 		&  $D_3$ & 
	\end{tabular}
\end{center}

\section{Elementi}
\label{sec:s3_elementi}

\begin{center}
	\[
	\begin{array}{cccc}
		\toprule
		\text{Elemento} & \text{Ciclo} & \text{Inverso} & \text{Ordine} \\
		\midrule
		\begin{pmatrix}
			123 \\ 123
		\end{pmatrix}
		& (1)	& (1) & 1 \\
		\begin{pmatrix}
			123 \\ 132
		\end{pmatrix}
		& (23) & (23) & 2 \\
		\begin{pmatrix}
			123 \\ 213
		\end{pmatrix}
		& (12) & (12) & 2 \\
		\begin{pmatrix}
			123 \\ 231
		\end{pmatrix}
		& (123) & (132) & 3 \\
		\begin{pmatrix}
			123 \\ 312
		\end{pmatrix}
		& (132) & (123) & 3 \\
		\begin{pmatrix}
			123 \\ 321
		\end{pmatrix}
		& (13) & (13) & 2 \\
		\bottomrule
	\end{array}
	\]
\end{center}

\section{Tavola di Cayley}
\label{sec:s3_caylay}

\begin{center}
	\[
	\begin{array}{cccccc}
		\midrule
		(1) & (12) & (13) & (23) & (123) & (132) \\
		(12) & (1) & (132) & (123) & (23) & (13) \\
		(13) & (123) & (1) & (132) & (12) & (23) \\
		(23) & (132) & (123) & (1) & (13) & (12) \\
		(123) & (13) & (23) & (12) & (132) & (1) \\
		(132) & (23) & (12) & (13) & (1) & (123) \\
		\bottomrule
	\end{array}
	\]
\end{center}

\section{Sottogruppi}
\label{sec:s3_sottogruppi}

\begin{center}
	\begin{tabular}{ccccccc}
		\toprule
		Generatori & Sottogruppo & Ordine & Indice & Ciclico & Normale & p-Sylow \\
		\midrule
		 & 1 & 1 & 6 & Sì & Sì & No \\
		$\gen{(12)}$ & $\{(1), (12)\}$ & 2 & 3 & Sì & No & Sì \\
		$\gen{(13)}$ & $\{(1), (13)\}$ & 2 & 3 & Sì & No & Sì \\
		$\gen{(23)}$ & $\{(1), (23)\}$ & 2 & 3 & Sì & No & Sì \\
		$\gen{(123)}$ & $\{(1), (123), (132)\}$ & 3 & 2 & Sì & Sì & Sì \\
		$\gen{(12)(123)}$ & $G$ & 6 & 1 & No & Sì & No \\
		\bottomrule
	\end{tabular}
\end{center}

\section{Classi di coniugio e centralizzanti}
\label{sec:s3_classi_coniugio}

\begin{center}
	\begin{tabular}{ccc}
		\toprule
		Classe & N. elementi & Ordine centralizzante \\
		\midrule
		\{1\} & 1 & 6 \\
		\{(12), (13), (23)\} & 3 & 2 \\
		\{(123), (132)\} & 2 & 3 \\
		\bottomrule
	\end{tabular}
\end{center}

\begin{gather*}
	C_{S_3}(1) = G \\
	C_{S_3}((12)) = \gen{(12)} \\
	C_{S_3}((123)) = \gen{(123)} \\
	Z(S_3) = 1
\end{gather*}

	\chapter{$D_3$}
\label{ch:d3}

Gruppo delle isometrie che mandano un triangolo equilatero in se stesso (figura~\ref{fig:D_3_Triangolo_con_numeri}).

\begin{figure}[tp]
	\centering
	\begin{tikzpicture}[line cap=round,line join=round,>=triangle 45,x=1cm,y=1cm]
		\begin{axis}[
			x=2cm,y=2cm,
			axis lines=middle,
			xmin=-2.5,
			xmax=2.5,
			ymin=-1.5,
			ymax=1.5,
			xtick={-3,3},
			ytick={-2,2},]
			\clip(-2.5,-1.5) rectangle (2.5,1.5);
			\fill[line width=2pt,color=figura,fill=figura,fill opacity=0.10000000149011612] (1,0) -- (-0.5,0.8660254037844387) -- (-0.5,-0.8660254037844384) -- cycle;
			%\draw [line width=2pt] (0,0) circle (2cm);
			\draw [line width=2pt,color=figura] (1,0)-- (-0.5,0.8660254037844387);
			\draw [line width=2pt,color=figura] (-0.5,0.8660254037844387)-- (-0.5,-0.8660254037844384);
			\draw [line width=2pt,color=figura] (-0.5,-0.8660254037844384)-- (1,0);
			\begin{scriptsize}
				% \draw [fill=uuuuuu] (1,0) circle (2pt);
				\draw (1.05,0.15) node {1};
				% \draw [fill=uuuuuu] (-0.5,0.8660254037844387) circle (2pt);
				\draw (-0.55,1) node {2};
				% \draw [fill=uuuuuu] (-0.5,-0.8660254037844384) circle (2pt);
				\draw (-0.55,-1) node {3};
				\draw (2.4,-.1) node {$x$};
				\draw (-0.1,1.4) node {$y$};
			\end{scriptsize}
		\end{axis}
	\end{tikzpicture}
	\caption{Triangolo equilatero di riferimento}
	\label{fig:D_3_Triangolo_con_numeri}
\end{figure}

Chiamiamo $\rho$ la rotazione di $\frac{2\pi}{3}$ e $\iota$ la riflessione lungo l'asse delle ascisse (vedi il capitolo~\ref{cpt:Isometrie}):

\begin{gather*}
	\rho = \rho_{\frac{2\pi}{3}} \\
	\iota = \iota_0 \\
	D_3 = \gen{\rho, \iota}
\end{gather*}

\section{Caratteristiche del gruppo}
\label{sec:d3_caratteristiche_gruppo}

\begin{center}
	\begin{tabular}{lll}
		Gruppo abeliano & No & Per esempio: $\iota \cdot \rho\iota \neq \rho\iota \cdot \iota$ \\
		Gruppo ciclico & No & Non contiene elementi di ordine 6. \\
		Ordine & 6 & $P_3 = 3! = 6$\\
		Esponente & 6 & Il gruppo contiene elementi di ordine 1, 2 e 3. \\
		Isomorfismi & $C_2 \times C_3$ & $D_3 = \gen{\rho, \iota}$  \\
		& $S_3$ & 
	\end{tabular}
\end{center}

\section{Elementi}
\label{sec:d3_elementi}

\begin{center}
	\[
	\begin{array}{cccc}
		\toprule
		\text{Elemento} & \text{Ciclo} & \text{Inverso} & \text{Ordine} \\
		\midrule
		1 & (1)	& 1 & 1 \\
		\iota & (23) & \iota & 2 \\
		\rho\iota & (12) & \rho\iota & 2 \\
		\rho^2\iota & (13) & \rho^2\iota & 2 \\
		\rho & (123) & \rho^2 & 3 \\
		\rho^2 & (132) & \rho & 3 \\
		\bottomrule
	\end{array}
	\]
\end{center}

\section{Tavola di Cayley}
\label{sec:d3_tavola_caylay}

\begin{center}
	\[
	\begin{array}{cccccc}
		\midrule
		1 & \rho\iota & \rho^2\iota & \iota & \rho & \rho^2 \\
		\rho\iota & 1 & \rho^2 & \rho & \iota & \rho^2\iota \\
		\rho^2\iota & \rho & 1 & \rho^2 & \rho\iota & \iota \\
		\iota & \rho^2 & \rho & 1 & \rho^2\iota & \rho\iota \\
		\rho & \rho^2\iota & \iota & \rho\iota & \rho^2 & 1 \\
		\rho^2 & \iota & \rho\iota & \rho^2\iota & 1 & \rho \\
		\bottomrule
	\end{array}
	\]
\end{center}

\section{Sottogruppi non banali}
\label{sec:d3_sottogruppi_non_banali}

\begin{center}
	\begin{tabular}{ccccccc}
		\toprule
		Generatori & Elementi & Ordine & Indice & Ciclico & Normale & p-Sylow \\
		\midrule
		$\gen{\rho\iota}$ & $\{1, \rho\iota\}$ & 2 & 3 & Sì & No & Sì \\
		$\gen{\rho^2\iota}$ & $\{1, \rho^2\iota\}$ & 2 & 3 & Sì & No & Sì \\
		$\gen{\iota}$ & $\{1, \iota\}$ & 2 & 3 & Sì & No & Sì \\
		$\gen{\rho} = \gen{\rho^2}$ & $\{1, \rho, \rho^2\}$ & 3 & 2 & Sì & Sì & Sì \\
		\bottomrule
	\end{tabular}
\end{center}

% centro


	\printbibliography[heading=bibintoc,title={Bibliografia}]

	\listoffigures

	\tableofcontents

\end{document}