\chapter{Lista di esercizi}

\begin{esercizio}
	\label{ex:ordine_HK}
	Se $H$ e $K$ sono sottogruppi finiti di un gruppo $G$, allora
	
	\begin{equation}
		\ordine{HK} = \dfrac{\ordine{H}\ordine{K}}{\ordine{H \cap K}}
	\end{equation}
\end{esercizio}
\begin{soluzione}
	Consideriamo la seguente funzione:
	
	\begin{align}
		\gamma : H \times K &\longrightarrow HK \\
		(h, k) &\longmapsto hk
	\end{align}

	La funzione $\gamma$ è suriettiva per definizione. Raramente è iniettiva.
	
	Cosa succede quando abbiamo immagini uguali di coppie diverse?
	
	\begin{gather}
		\gamma(h_1, k_1) = \gamma(h_2, k_2) \\
		h_1k_1 = h_2k_2 \\
		h_2^{-1}h_1 = k_2k_1^{-1}
	\end{gather}

	L'elemento $h_2^{-1}h_1$ appartiene ad $H$, come l'elemento $k_2k_1^{-1}$ appartiene a $K$. Ma abbiamo verificato che questi elementi in verità sono uno stesso elemento che quindi sta sia in $H$ sia in $K$. Quindi sta anche in $H \cap K$. Chiamiamo $x$ tale elemento. Risulta quindi che:
	
	\begin{equation}
		\begin{cases}
			h_1 = h_2x \\
			k_1 = x^{-1}k_2
		\end{cases}
	\end{equation}

	Viceversa, prendendo un elemento $x \in H \cap K$ abbiamo che:
	
	\begin{equation}
		hk = hx \, x^{-1}k
	\end{equation}

	Quindi:
	
	\begin{equation}
		\gamma^{-1}(hk) = \{(hx, x^{-1}k) \taleche x \in H \cap K\}
	\end{equation}

	Quindi la controimmagine $\gamma^{-1}(hk)$ ha cardinalità $\ordine{H \cap K}$, e questo vale per ogni coppia di elementi $h$ e $k$. Ma possiamo scegliere l'elemento $h$ tra $\ordine{H}$ elementi, e possiamo scegliere l'elemento $k$ tra $\ordine{K}$ elementi. Quindi per il principio di moltiplicazione:
	
	\begin{equation}
		\ordine{HK} = \dfrac{\ordine{H \times K}}{\ordine{H \cap K}} = \dfrac{\ordine{H}\ordine{K}}{\ordine{H \cap K}}
	\end{equation}

\end{soluzione}

\begin{esercizio}
	Sia $G$ un gruppo finito. Se $H$ e $K$ sono sottogruppi di $G$ con indici coprimi, allora:
	
	\begin{equation}
		HK = G
	\end{equation}
\end{esercizio}
\begin{soluzione}
	Per quanto visto nell'esercizio~\ref{ex:ordine_HK}, si ha:
	
	\begin{align}
		\ordine{HK} 
		&= \dfrac{\ordine{H}\ordine{K}}{\ordine{H \cap K}} = \\
		&= \dfrac{\ordine{H}\ordine{K}}{\ordine{H \cap K}} \cdot \dfrac{\ordine{G}\ordine{G}}{\ordine{G}\ordine{G}} = \\
		&= \dfrac{\ordine{H}}{\ordine{G}} \cdot \dfrac{\ordine{K}}{\ordine{G}}\cdot\dfrac{\ordine{G}}{\ordine{H \cap K}} \cdot \ordine{G} = \\
		&= \dfrac{\indice{G}{H \cap K}}{\indice{G}{H}\indice{G}{k}}\ordine{G} = \\
		&= \ordine{G}
	\end{align}

	Sappiamo che $HK \subseteq G$, quindi se hanno la stessa cardinalità:
	
	\begin{equation}
		HK = G
	\end{equation}
\end{soluzione}