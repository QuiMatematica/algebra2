\chapter{$D_3$}

Gruppo delle isometrie che mandano un triangolo equilatero in se stesso (figura~\ref{fig:D_3_Triangolo_con_numeri}).

\begin{figure}[tp]
	\centering
	\begin{tikzpicture}[line cap=round,line join=round,>=triangle 45,x=1cm,y=1cm]
		\begin{axis}[
			x=2cm,y=2cm,
			axis lines=middle,
			xmin=-2.5,
			xmax=2.5,
			ymin=-1.5,
			ymax=1.5,
			xtick={-3,3},
			ytick={-2,2},]
			\clip(-2.5,-1.5) rectangle (2.5,1.5);
			\fill[line width=2pt,color=figura,fill=figura,fill opacity=0.10000000149011612] (1,0) -- (-0.5,0.8660254037844387) -- (-0.5,-0.8660254037844384) -- cycle;
			%\draw [line width=2pt] (0,0) circle (2cm);
			\draw [line width=2pt,color=figura] (1,0)-- (-0.5,0.8660254037844387);
			\draw [line width=2pt,color=figura] (-0.5,0.8660254037844387)-- (-0.5,-0.8660254037844384);
			\draw [line width=2pt,color=figura] (-0.5,-0.8660254037844384)-- (1,0);
			\begin{scriptsize}
				% \draw [fill=uuuuuu] (1,0) circle (2pt);
				\draw (1.05,0.15) node {1};
				% \draw [fill=uuuuuu] (-0.5,0.8660254037844387) circle (2pt);
				\draw (-0.55,1) node {2};
				% \draw [fill=uuuuuu] (-0.5,-0.8660254037844384) circle (2pt);
				\draw (-0.55,-1) node {3};
				\draw (2.4,-.1) node {$x$};
				\draw (-0.1,1.4) node {$y$};
			\end{scriptsize}
		\end{axis}
	\end{tikzpicture}
	\caption{Triangolo equilatero di riferimento}
	\label{fig:D_3_Triangolo_con_numeri}
\end{figure}

Chiamiamo $\rho$ la rotazione di $\frac{2\pi}{3}$ e $\iota$ la riflessione lungo l'asse delle ascisse (vedi il capitolo~\ref{cpt:Isometrie}):

\begin{gather*}
	\rho = \rho_{\frac{2\pi}{3}} \\
	\iota = \iota_0 \\
	D_3 = \gen{\rho, \iota}
\end{gather*}

\section{Caratteristiche del gruppo}

\begin{center}
	\begin{tabular}{lll}
		Gruppo abeliano & No & Per esempio: $\iota\,\rho\iota \neq \rho\iota\,\iota$ \\
		Gruppo ciclico & No & Non contiene elementi di ordine 6. \\
		Ordine & 6 & $P_3 = 3! = 6$\\
		Esponente & 6 & Il gruppo contiene elementi di ordine 1, 2 e 3. \\
		Isomorfismi & $C_2 \times C_3$ &  \\
		& $S_3$ & 
	\end{tabular}
\end{center}

\section{Elementi}

\begin{center}
	\[
	\begin{array}{cccc}
		\toprule
		\text{Elemento} & \text{Ciclo} & \text{Inverso} & \text{Ordine} \\
		\midrule
		1 & (1)	& 1 & 1 \\
		\iota & (23) & \iota & 2 \\
		\rho\iota & (12) & \rho\iota & 2 \\
		\rho^2\iota & (13) & \rho^2\iota & 2 \\
		\rho & (123) & \rho^2 & 3 \\
		\rho^2 & (132) & \rho & 3 \\
		\bottomrule
	\end{array}
	\]
\end{center}

\section{Tavola di Cayley}

\begin{center}
	\[
	\begin{array}{cccccc}
		\midrule
		1 & \rho\iota & \rho^2\iota & \iota & \rho & \rho^2 \\
		\rho\iota & 1 & \rho^2 & \rho & \iota & \rho^2\iota \\
		\rho^2\iota & \rho & 1 & \rho^2 & \rho\iota & \iota \\
		\iota & \rho^2 & \rho & 1 & \rho^2\iota & \rho\iota \\
		\rho & \rho^2\iota & \iota & \rho\iota & \rho^2 & 1 \\
		\rho^2 & \iota & \rho\iota & \rho^2\iota & 1 & \rho \\
		\bottomrule
	\end{array}
	\]
\end{center}

\section{Sottogruppi non banali}

\begin{center}
	\begin{tabular}{ccccccc}
		\toprule
		Generatori & Elementi & Ordine & Indice & Ciclico & Normale & p-Sylow \\
		\midrule
		$\gen{\rho\iota}$ & $\{1, \rho\iota\}$ & 2 & 3 & Sì & No & Sì \\
		$\gen{\rho^2\iota}$ & $\{1, \rho^2\iota\}$ & 2 & 3 & Sì & No & Sì \\
		$\gen{\iota}$ & $\{1, \iota\}$ & 2 & 3 & Sì & No & Sì \\
		$\gen{\rho} = \gen{\rho^2}$ & $\{1, \rho, \rho^2\}$ & 3 & 2 & Sì & Sì & Sì \\
		\bottomrule
	\end{tabular}
\end{center}

% sottogruppi
% centro
