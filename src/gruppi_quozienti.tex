\chapter{Gruppi quozienti}

Si $G$ un gruppo e $N$ un suo sottogruppo normale:

\begin{equation}
	N \normale G
\end{equation}

Definiamo una relazione di congruenza:

\begin{equation}
	a \equiv b \,(\text{mod } N) \quad \text{ se } a^{-1}b \in N
\end{equation}

ovvero se:

\begin{equation}
	b = aN
\end{equation}

ovvero se:

\begin{equation}
	\exists n \in N \taleche b = an
\end{equation}

Vedi \cite[pag. 56]{jacobson} per la dimostrazione che questa relazione è di equivalenza.

Le classi di equivalenza corrispondono ai laterali del sottogruppo $N$, e formano un gruppo in quanto:

\begin{gather}
	(gN)(hN) = ghN \\
	N \text{ è l'unità} \\
	(gN)^{-1} = g^{-1}N
\end{gather}

Definiamo \textbf{gruppo quoziente} di $G$ relativo a $N$ il gruppo formato dalle classi di equivalenza $G/\equiv$. Indichiamo questo gruppo con $G/N$.

Le classi di equivalenza formano una partizione del gruppo $G$, pertanto gli elementi del gruppo quoziente $G/N$ rappresentano a loro volta una partizione di $G$. Uno degli elementi è rappresentato dal sottogruppo normale $N$ (figura~\ref{fig:Gruppi_quozienti_partizione}).

\begin{figure}[tp]
	\centering
	\begin{tikzpicture}[line cap=round,line join=round,>=triangle 45,x=1cm,y=1cm]
		\clip(-4.5,-3.4) rectangle (4.5,3.4);
		\draw [rotate around={0:(0,0)},line width=1pt] (0,0) ellipse (4cm and 3cm);
		\draw [shift={(-11,0)},line width=1pt] plot[domain=-0.27:0.27,variable=\t]({9*cos(\t r)},{9*sin(\t r)});
		\draw [shift={(-1,12)},line width=1pt] plot[domain=4.63:5.12,variable=\t]({12*cos(\t r)},{12*sin(\t r)});
		\draw [shift={(12,3)},line width=1pt]  plot[domain=3.4:3.65,variable=\t]({11*cos(\t r)},{11*sin(\t r)});
		\draw (3.2,2.8) node[anchor=north west] {$G$};
		\draw (-0.5,-1) node[anchor=north west] {$N$};
		\draw (-3.5,0.5) node[anchor=north west] {$g_1N$};
		\draw (-0.5,2) node[anchor=north west] {$g_2N$};
		\draw (2.4,-0.2) node[anchor=north west] {$g_3N$};
	\end{tikzpicture}
	\caption{Gruppo quoziente $G/N$ come partizione del gruppo $G$.}
	\label{fig:Gruppi_quozienti_partizione}
\end{figure}

Ogni gruppo $G \ne 1$ ha due sottogruppi normali: $G$ e $1$. $G$ è detto \textbf{semplice} se ha solo questi sottogruppi normali.