\chapter{$S_3$}
\label{ch:S3}

\section{Caratteristiche del gruppo}
\label{sec:s3_caratteristiche}

\begin{center}
	\begin{tabular}{lll}
		Gruppo abeliano & No & Per esempio: $(12)(13) \neq (13)(12)$ \\
		Gruppo ciclico & No & Non contiene elementi di ordine 6. \\
		Ordine & 6 & $P_3 = 3! = 6$\\
		Esponente & 6 & Il gruppo contiene elementi di ordine 1, 2 e 3. \\
		Isomorfismi & $C_2 \times C_3$ & $S_3 = \gen{(12), (123)}$ \\
 		&  $D_3$ & 
	\end{tabular}
\end{center}

\section{Elementi}
\label{sec:s3_elementi}

\begin{center}
	\[
	\begin{array}{cccc}
		\toprule
		\text{Elemento} & \text{Ciclo} & \text{Inverso} & \text{Ordine} \\
		\midrule
		\begin{pmatrix}
			123 \\ 123
		\end{pmatrix}
		& (1)	& (1) & 1 \\
		\begin{pmatrix}
			123 \\ 132
		\end{pmatrix}
		& (23) & (23) & 2 \\
		\begin{pmatrix}
			123 \\ 213
		\end{pmatrix}
		& (12) & (12) & 2 \\
		\begin{pmatrix}
			123 \\ 231
		\end{pmatrix}
		& (123) & (132) & 3 \\
		\begin{pmatrix}
			123 \\ 312
		\end{pmatrix}
		& (132) & (123) & 3 \\
		\begin{pmatrix}
			123 \\ 321
		\end{pmatrix}
		& (13) & (13) & 2 \\
		\bottomrule
	\end{array}
	\]
\end{center}

\section{Tavola di Cayley}
\label{sec:s3_caylay}

\begin{center}
	\[
	\begin{array}{cccccc}
		\midrule
		(1) & (12) & (13) & (23) & (123) & (132) \\
		(12) & (1) & (132) & (123) & (23) & (13) \\
		(13) & (123) & (1) & (132) & (12) & (23) \\
		(23) & (132) & (123) & (1) & (13) & (12) \\
		(123) & (13) & (23) & (12) & (132) & (1) \\
		(132) & (23) & (12) & (13) & (1) & (123) \\
		\bottomrule
	\end{array}
	\]
\end{center}

\section{Sottogruppi}
\label{sec:s3_sottogruppi}

\begin{center}
	\begin{tabular}{ccccccc}
		\toprule
		Generatori & Sottogruppo & Ordine & Indice & Ciclico & Normale & p-Sylow \\
		\midrule
		 & 1 & 1 & 6 & Sì & Sì & No \\
		$\gen{(12)}$ & $\{(1), (12)\}$ & 2 & 3 & Sì & No & Sì \\
		$\gen{(13)}$ & $\{(1), (13)\}$ & 2 & 3 & Sì & No & Sì \\
		$\gen{(23)}$ & $\{(1), (23)\}$ & 2 & 3 & Sì & No & Sì \\
		$\gen{(123)}$ & $\{(1), (123), (132)\}$ & 3 & 2 & Sì & Sì & Sì \\
		$\gen{(12)(123)}$ & $G$ & 6 & 1 & No & Sì & No \\
		\bottomrule
	\end{tabular}
\end{center}

\section{Classi di coniugio e centralizzanti}
\label{sec:s3_classi_coniugio}

\begin{center}
	\begin{tabular}{ccc}
		\toprule
		Classe & N. elementi & Ordine centralizzante \\
		\midrule
		\{1\} & 1 & 6 \\
		\{(12), (13), (23)\} & 3 & 2 \\
		\{(123), (132)\} & 2 & 3 \\
		\bottomrule
	\end{tabular}
\end{center}

\begin{gather*}
	C_{S_3}(1) = G \\
	C_{S_3}((12)) = \gen{(12)} \\
	C_{S_3}((123)) = \gen{(123)} \\
	Z(S_3) = 1
\end{gather*}
