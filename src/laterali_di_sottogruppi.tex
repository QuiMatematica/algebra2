\chapter{Laterali di un sottogruppo}
\label{ch:laterali_sottogruppo}

\begin{teorema}[Teorema di Lagrange]
	\label{thr:Laterali_Lagrange}
	L'ordine di un sottogruppo $H$ di un gruppo finito $G$ è un fattore dell'ordine di $G$.
	Più precisamente:
	\begin{equation*}
		\ordine{G} = \ordine{H} \cdot \indice{G}{H}
	\end{equation*}
\end{teorema}
\begin{dimostrazione}
	Vedi \cite[pag. 52]{jacobson}.
\end{dimostrazione}

\begin{corollario}
	\label{cor:sottogruppi_ordini_coprimi}
	Siano $H$ e $K$ due sottogruppi di un gruppo $G$, i cui ordini sono coprimi.
	Allora l'intersezione dei due sottogruppi contiene solo l'identità:
	\begin{equation*}
		\MCD{\ordine{H}}{\ordine{K}} = 1 \allora \ordine{H \cap K} = 1
	\end{equation*}
\end{corollario}
\begin{dimostrazione}
	L'intersezione di due sottogruppi è un sottogruppo di entrambi i sottogruppi.
	Quindi, per il teorema di Lagrange:
	\begin{gather*}
		\ordine{H \cap K} \dividetxt \ordine{H} \\
		\ordine{H \cap K} \dividetxt \ordine{K}
	\end{gather*}
	Ma gli ordini dei sottogruppi $H$ e $K$ sono coprimi, quindi l'unico divisore in comune è l'1.
\end{dimostrazione}