\chapter{Equazione delle classi}
\label{ch:equazione_delle_classi}

Consideriamo un insieme $G$ che agisce per coniugio sui propri elementi:
\begin{equation*}
	g \circ x = gxg^{-1}
\end{equation*}

Consideriamo le diverse orbite dell'azione, che sono anche le classi di coniugio:
\begin{equation*}
	C_1, C_2, \dots, C_T
\end{equation*}

Le classi formano una partizione dell'insieme $G$, quindi la cardinalità dell'insieme è pari alla somma delle cardinalità delle classi:
\begin{equation*}
	\ordine{G} = \sum_i \ordine{C_i}
\end{equation*}

Per ogni classe di coniugio posso scegliere un elemento $g_i \in C_i$:
\begin{equation*}
	\ordine{G} = \sum_i \indice{G}{C_G(g_i)}
\end{equation*}

Quest'ultima è detta \textbf{equazione delle classi}.

Se prendo un elemento $g \in Z(G)$ allora la sua classe di coniugio $\{g\}$ è formata dal solo elemento $g$, quindi è formata da un solo elemento.

Quindi abbiamo $r$ classi di coniugio di cardinalità 1, mentre le altre avranno cardinalità maggiore di 1:
\begin{gather*}
	\ordine{C_1} = \dots = \ordine{C_r} = 1 \\
	\ordine{C_k} > 0 \quad \forall k > r
\end{gather*}

Quindi:
\begin{equation*}
	r = \ordine{Z(G)}
\end{equation*}

Passiamo all'equazione delle classi:
\begin{gather}
	\ordine{G} = \sum_i \ordine{C_i} \notag \\
	\label{eq:equazione_delle_classi} \ordine{G} = \ordine{Z(G)} + \sum_{k> r}\ordine{C_k}
\end{gather}

\begin{teorema}[{\cite[teorema 1.11, pag. 76]{jacobson}}]
	\label{thr:ordine_potenza_di_primo}
	Se $\ordine{G} = p^n$ con $p$ primo e $n > 0$, allora $Z(G) \ne 1$.
\end{teorema}
\begin{dimostrazione}	
	Per il teorema~\ref{thr:Laterali_Lagrange} di Lagrange:
	\begin{equation*}
		\ordine{C_k} = \indice{G}{C_G(g_k)} \text{ divide } \ordine{G} = p^n
	\end{equation*}
	
	Quindi $\ordine{C_k}$ è divisibile per $p$, ovvero:
	\begin{equation*}
		\ordine{C_k} \equiv 0 \mod p
	\end{equation*}
	
	Ma anche:
	\begin{equation*}
		\ordine{G} \equiv 0 \mod p
	\end{equation*}
	
	Quindi $\ordine{Z(G)}$ è differenza di due numeri congrui a 0 modulo $p$:
	\begin{equation*}
		\ordine{Z(G)} \equiv 0 \mod p
	\end{equation*}
	
	Ma $Z(G)$ contiene almeno l'unità 1, quindi il suo ordine è diverso da 0. Quindi:
	\begin{equation*}
		\ordine{Z(G))} \ge p
	\end{equation*}
\end{dimostrazione}

\begin{esercizio}[{\cite[Week 2, Exercise 8]{lucchini_week}}]
	Dimostrare che ogni gruppo di ordine $p^2$, con $p$ primo, è abeliano
\end{esercizio}
\begin{dimostrazione}
	Per il teorema~\ref{thr:ordine_potenza_di_primo}, $Z(G) \ne 1$.
	Se $Z(G) = G$, allora $G$ è abeliano.
	Quindi dobbiamo escludere che $\ordine{Z(G)} = p$.

	Ipotizziamo per assurdo che $\ordine{Z(G)} = p$.
	Il gruppo quoziente $G/Z(G)$ è ciclico, quindi può essere definito come:
	\begin{equation*}
		G/Z(G) = \gen{a Z(G)}
	\end{equation*}

	Questo significa che ogni elemento di $G$ può essere scritto come:
	\begin{equation*}
		g = a^i z \quad \text{con } i \in \Z, z \in Z(G).
	\end{equation*}

	Prendiamo ora due qualunque elementi di $G$:
	\begin{gather*}
		g_1 = a^{i_i} z_1 \\
		g_2 = a^{i_2} z_2
	\end{gather*}

	Si ha:
	\begin{align*}
		g_1 g_2 &= a^{i_1} z_1 a^{i_2} z_2 = && \text{sostituzione} \\
		&= a^{i_1} a^{i_2} z_1 z_2 = && \text{perché $z_1, z_2 \in Z(G)$} \\
		&= a^{i_1 + i_2} z_1 z_2 = && \text{proprietà delle potenze} \\
		&= a^{i_2} a^{i_1} z_1 z_2 = && \text{proprietà delle potenze} \\
		&= a^{i_2} z_2 a^{i_1} z_1 = && \text{perché $z_1, z_2 \in Z(G)$} \\
		&= g_2 g_1 && \text{sostituzione}
	\end{align*}

	Quindi tutti gli elementi di $G$ commutano, $G = Z(G)$ e $G$ è abeliano.
\end{dimostrazione}

