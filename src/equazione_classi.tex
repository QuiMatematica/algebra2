\chapter{Equazione delle classi}
\label{ch:equazione_delle_classi}

Consideriamo un insieme $G$ che agisce per coniugio sui propri elementi:
\begin{equation*}
	g \circ x = gxg^{-1}
\end{equation*}

Consideriamo le diverse orbite dell'azione, che sono anche le classi di coniugio:
\begin{equation*}
	C_1, C_2, \dots, C_T
\end{equation*}

Le classi formano una partizione dell'insieme $G$, quindi la cardinalità dell'insieme è pari alla somma delle cardinalità delle classi:
\begin{equation*}
	\ordine{G} = \sum_i \ordine{C_i}
\end{equation*}

Per ogni classe di coniugio posso scegliere un elemento $g_i \in C_i$:
\begin{equation*}
	\ordine{G} = \sum_i \indice{G}{C_G(g_i)}
\end{equation*}

Quest'ultima è detta \textbf{equazione delle classi}.

Se prendo un elemento $g \in Z(G)$ allora la sua classe di coniugio $\{g\}$ è formata dal solo elemento $g$, quindi è formata da un solo elemento.

Quindi abbiamo $r$ classi di coniugio di cardinalità 1, mentre le altre avranno cardinalità maggiore di 1:
\begin{gather*}
	\ordine{C_1} = \dots = \ordine{C_r} = 1 \\
	\ordine{C_k} > 0 \quad \forall k > r
\end{gather*}

Quindi:
\begin{equation*}
	r = \ordine{Z(G)}
\end{equation*}

Passiamo all'equazione delle classi:
\begin{gather}
	\ordine{G} = \sum_i \ordine{C_i} \notag \\
	\label{eq:equazione_delle_classi} \ordine{G} = \ordine{Z(G)} + \sum_{k> r}\ordine{C_k}
\end{gather}

\begin{teorema}[{\cite[teorema 1.11, pag. 76]{jacobson}}]
	\label{thr:ordine_potenza_di_primo}
	Se $\ordine{G} = p^n$ con $p$ primo e $n > 0$, allora $Z(G) \ne 1$.
\end{teorema}
\begin{dimostrazione}	
	Per il teorema~\ref{thr:Laterali_Lagrange} di Lagrange:
	\begin{equation*}
		\ordine{C_k} = \indice{G}{C_G(g_k)} \text{ divide } \ordine{G} = p^n
	\end{equation*}
	
	Quindi $\ordine{C_k}$ è divisibile per $p$, ovvero:
	\begin{equation*}
		\ordine{C_k} \equiv 0 \mod p
	\end{equation*}
	
	Ma anche:
	\begin{equation*}
		\ordine{G} \equiv 0 \mod p
	\end{equation*}
	
	Quindi $\ordine{Z(G)}$ è differenza di due numeri congrui a 0 modulo $p$:
	\begin{equation*}
		\ordine{Z(G)} \equiv 0 \mod p
	\end{equation*}
	
	Ma $Z(G)$ contiene almeno l'unità 1, quindi il suo ordine è diverso da 0. Quindi:
	\begin{equation*}
		\ordine{Z(G))} \ge p
	\end{equation*}
\end{dimostrazione}
