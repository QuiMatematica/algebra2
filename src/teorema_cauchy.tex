\chapter{Teorema di Cauchy}
\label{ch:teorema_cauchy}

Riprendiamo una domanda ricorrente.

\emph{Dato un divisore dell'ordine del gruppo, esiste un sottogruppo con quell'ordine?}

La risposta è: generalmente no.

Prendiamo per esempio il gruppo $A_5$, di ordine 60.

Abbiamo visto con il teorema~\ref{th:a5_semplice} che $A_5$ è un gruppo semplice.
Quindi non può contenere un sottogruppo di ordine 30, altrimenti avrebbe indice 2 e dovrebbe essere normale.

Inoltre verifichiamo che $A_5$ non contiene sottogruppi di ordine 15, ovvero d'indice 4.
Ipotizziamo per assurdo che esista un tale gruppo, che chiamiamo $H$:
\begin{equation*}
    \indice{A_5}{H} = 4
\end{equation*}

Se c'è un sottogruppo di indice 4, per il teorema~\ref{thr:Azioni_laterali_indice_finito}, il quoziente sul
cuore normale $A_5/H_{A_5}$ è isomorfo ad un sottogruppo del gruppo simmetrico di grado l'indice:
\begin{equation*}
    \dfrac{A_5}{H_{A_5}} \isomorfo \sigma(A_5) \le S_4
\end{equation*}

Il gruppo $A_5$ è semplice, quindi il cuore normale $H_{A_5}$ deve essere il sottogruppo identico:
\begin{equation*}
    A_5 \isomorfo \sigma(A_5) \le S_4
\end{equation*}

Questo non può essere perché $\ordine{A_5} = 60$ e $\ordine{S_4} = 24$.

\begin{teorema}[Teorema di Cauchy]
    \label{thr:cauchy}
    Se $G$ è un gruppo finito e un primo $p$ divide $\ordine{G}$, allora $G$ contiene un elemento di ordine $p$.
\end{teorema}

\begin{dimostrazione}
    Definiamo:
    \begin{gather*}
        X = \gen{\sigma} \quad \text{con } \sigma = (12 \dots p) \\
        \Omega = \{(g_1, g_2, \dots, g_p) \in G^p \taleche g_1 \cdot g_2 \cdot \dots \cdot g_p = 1\}
    \end{gather*}

    Il prodotto degli elementi delle $p$-tuple di $\Omega$ deve essere considerato ordinato, perché non è detto che $G$
    sia abeliano.

    Definiamo l'azione di $X$ su $\Omega$;
    la definiamo solo per il generatore, gli altri elementi ereditano la definizione:
    \begin{equation*}
        \sigma \circ (g_1, g_2, \dots, g_p) = (g_p, g_1, \dots, g_{p-1})
    \end{equation*}

    Dobbiamo assicurarci che il nuovo elemento ottenuto dall'azione di $\sigma$ appartenga comunque a $\Omega$:
    \begin{align*}
        g_p \cdot g_1 \cdot \dots \cdot g_{p-1} &= (g_p \cdot g_1 \cdot \dots \cdot g_{p-1}) \cdot (g_p \cdot g_p^{-1}) = \\
        &= g_p \cdot (g_1 \cdot \dots \cdot g_{p-1} \cdot g_p) \cdot g_p^{-1} = \\
        &= g_p \cdot 1 \cdot g_p^{-1} = \\
        &= 1
    \end{align*}

    Determiniamo quanti elementi ha $\Omega$.
    Per ottenere una $p$-tupla opportuna posso scegliere come voglio il primo elemento, come pure il secondo, come pure
    il terzo, \dots, come pure il penultimo, purché l'ultimo sia l'inverso di del prodotto dei precedenti. Quindi:
    \begin{equation*}
        \ordine{\Omega} = \ordine{G}^{p-1}
    \end{equation*}

    $p$ divide $\ordine{G}$, quindi:
    \begin{equation*}
        \ordine{G}^{p-1} \equiv 0 \mod p
    \end{equation*}
    
    Consideriamo le orbite:
    \begin{gather*}
        \Omega = O_1 \overset{\circ}{\cup} O_2 \overset{\circ}{\cup} \dots \overset{\circ}{\cup} O_t \quad \text{con } O_i \text{ orbita} \\
        \ordine{\Omega} = \sum_{i} \ordine{O_i} \\
        \sum_i \ordine{O_i} \equiv 0 \mod p
    \end{gather*}

    Determiniamo quanti elementi hanno le orbite.
    Scelgo un generico elemento $\omega_i \in O_i$.
    Per la formula~\eqref{eq:ordine_orbita}, conseguenza del teorema~\ref{thr:orbite}, sappiamo che:
    \begin{equation*}
        \ordine{O_i} = \indice{X}{\Stab_X(\omega_i)}
    \end{equation*}

    $X$ è di ordine $p$, l'indice deve essere un divisore dell'ordine del gruppo, quindi:
    \begin{equation*}
        \ordine{O_i} = \indice{X}{\Stab_X(\omega_i)} \in \{1, p\}
    \end{equation*}

    Gli addendi di $\sum_i \ordine{O_i}$ hanno ordine 1 o $p$.

    Gli addendi di ordine $p$ sono $\equiv 0 \mod p$.

    Quindi gli addendi che danno un contributo non banale sono quelli di ordine 1.

    Se $\ordine{O_i} = 1$, allora $\omega_i$ è fissato da $X$:
    \begin{equation*}
        \ordine{O_i} = 1 \allora \Stab_{X}(\omega_i) = X \allora \sigma \circ \omega_i = \omega_i
    \end{equation*}

    Le orbite di cardinalità 1 sono tante quante gli elementi di:
    \begin{gather*}
        \Delta = \{\omega \in \Sigma \taleche \sigma \circ \omega = \omega\} \\
        \ordine{\Delta} \equiv 0 \mod p
    \end{gather*}

    Determiniamo chi sono gli elementi di $\Delta$.
    Prendiamo un $\omega \in \Delta$:
    \begin{gather*}
        \sigma \circ \omega = \omega \quad \text{con } \omega = (g_1, g_2, \dots, g_p) \\
        \allora \sigma \circ \omega = (g_p, g_1, \dots, g_{p-1}) \\
        \allora (g_1, g_2, \dots, g_p) = (g_p, g_1, \dots, g_{p-1}) \\
        \allora g_1 = g_2 = \dots = g_p = g \quad \text{con } g \in G
    \end{gather*}

    Inoltre il prodotto delle $p$-tuple deve essere 1:
    \begin{equation*}
        \omega \in \Sigma \allora g_1 \cdot \dots \cdot g_p = g^p = 1
    \end{equation*}

    Quindi:
    \begin{equation*}
        \Delta = \{(g, \dots, g) \in G^p \taleche g^p = 1\}
    \end{equation*}

    Posso allora definire un nuovo insieme:
    \begin{gather*}
        \Phi = \{g \in G \taleche g^p = 1\} \\
        \ordine{\Phi} \equiv 0 \mod p
    \end{gather*}

    Se $g^p = 1$ allora o $g = 1$ o $\ordine{g} = p$.

    L'identità appartiene a $\Phi$, quindi $\Phi$ non è vuoto.
    Quindi:
    \begin{equation*}
        \exists g \in \Phi \text{ con } g \ne 1 \allora \ordine{g} = p
    \end{equation*}

\end{dimostrazione}

\begin{esercizio}[{\cite[Es. 4 pag. 82]{jacobson}}]
    Determinare tutti i gruppi $G$ con $\ordine{G} = 6$.
\end{esercizio}

\begin{soluzione}
    Per il teorema di Cauchy~\ref{thr:cauchy}, il gruppo $G$ contiene un elemento $a$ di ordine 3 e un elemento $b$ di
    ordine 2.

    Quindi $G$ ha almeno due sottogruppi: $A = \gen{a}$ normale (perché di indice 2) e $B = \gen{b}$.

    Inoltre $AB \le G$ visto che $A$ è un sottogruppo normale (teorema~\ref{thr:prodotto_sottogruppi}), e poiché
    hanno indici coprimi $AB = G$ (esercizio~\ref{ex:indici_coprimi}).

    Quindi:
    \begin{equation*}
        \forall g \in G : g = a^r b^s
    \end{equation*}

    con $r$ ridotto modulo 3 e $s$ ridotto modulo 2.

    $A$ è normale, quindi $b a b \in A$, quindi $b a b = a^i$ con $i \in \{0,1,-1\}$.

    $i$ non può essere 0 perché vale solo con $a = 1$.
    Quindi $i \in {1, -1}$.

    Capiamo allora che abbiamo \textbf{solo due gruppi di ordine 6}.

    \textbf{Caso 1: $i = 1$}
    \begin{equation*}
        ab = ba
    \end{equation*}

    Abbiamo un elemento di ordine 3 che commuta con un elemento di ordine 2, quindi:
    \begin{equation*}
        \ordine{ab} = 6
    \end{equation*}

    Il gruppo $G$ contiene un elemento di ordine 6, quindi:
    \begin{equation*}
        G = \gen{ab} \isomorfo 6
    \end{equation*}

    \textbf{Caso 2: $i = -1$}
    \begin{equation*}
        bab = a^{-1}
    \end{equation*}

    Abbiamo una rotazione di ordine 3 e una riflessione di ordine 2:
    \begin{equation*}
        G = \gen{a, b} \isomorfo D_3 \isomorfo S_3
    \end{equation*}

\end{soluzione}