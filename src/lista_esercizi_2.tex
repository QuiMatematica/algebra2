\chapter{Esercizi}
\label{ch:esercizi_settimana_2}

\begin{esercizio}
    Trovare i gruppi finiti che hanno esattamente due classi di coniugio.

    Ovvero:

    Sia $G$ un gruppo finito e supponiamo che ogni due elementi non identici di $G$ sono coniugati.
    Allora $\ordine{G} \le 2$.
\end{esercizio}
\begin{soluzione}
    Se il gruppo ha esattamente due classi di coniugio, allora:
    \begin{equation*}
        G = C_1 \overset{\circ}{\cup} C_2
    \end{equation*}

    La prima classe di coniugio la conosco: è quella dell'identità.
    \begin{gather*}
        C_1 = \{1\} \\
        \ordine{G} = 0 \allora \ordine{C_2} = n-1
    \end{gather*}

    Prendiamo un elemento $x \in C_2$:
    \begin{gather*}
        \indice{G}{C_G(x)} = \ordine{C_2} = n-1 \\
        \indice{G}{C_G(x)} \text{ divide } \ordine{G} = n \\
        n-1 \divisore n
    \end{gather*}

    Ma questo succede solo se $n = 2$.
    \begin{equation*}
        G \isomorfo C_2
    \end{equation*}
\end{soluzione}