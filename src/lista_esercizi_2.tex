\chapter{Esercizi}
\label{ch:esercizi_settimana_2}

\begin{esercizio}
    Trovare i gruppi finiti che hanno esattamente due classi di coniugio.

    Ovvero:

    Sia $G$ un gruppo finito e supponiamo che ogni due elementi non identici di $G$ sono coniugati.
    Allora $\ordine{G} \le 2$.
\end{esercizio}
\begin{soluzione}
    Se il gruppo ha esattamente due classi di coniugio, allora:
    \begin{equation*}
        G = C_1 \overset{\circ}{\cup} C_2
    \end{equation*}

    La prima classe di coniugio la conosco: è quella dell'identità.
    \begin{gather*}
        C_1 = \{1\} \\
        \ordine{G} = 0 \allora \ordine{C_2} = n-1
    \end{gather*}

    Prendiamo un elemento $x \in C_2$:
    \begin{gather*}
        \indice{G}{C_G(x)} = \ordine{C_2} = n-1 \\
        \indice{G}{C_G(x)} \text{ divide } \ordine{G} = n \\
        n-1 \divisore n
    \end{gather*}

    Ma questo succede solo se $n = 2$.
    \begin{equation*}
        G \isomorfo C_2
    \end{equation*}
\end{soluzione}

\begin{esercizio}
    Sia $G$ un gruppo finito.
    Determinare la probabilità che due elementi (anche uguali) commutino.
\end{esercizio}
\begin{soluzione}
    Visto che $G$ è finito possiamo applicare la definizione classica di probabilità.

    Il casi favorevoli sono dati dall'insieme $\{(g_1, g_2) \taleche g_1 g_2 = g_2 g_1\}$.

    Il numero di casi totali è dato da $\ordine{G}^2$.

    Determiniamo il numero di coppie che commutano.

    Il primo elemento lo posso scegliere come voglio.
    Il secondo elemento deve coniugare con il primo, quindi deve appartenere al centralizzante del primo elemento.
    \begin{equation*}
        \text{casi favorevoli } = \sum_{g \in G} \ordine{C_G(g)}
    \end{equation*}

    Raggruppiamo gli elementi nelle classi di coniugio, perché hanno i centralizzanti della stessa cardinalità.

    Chiamo $C_1, \dots, C_k$ le classi di coniugio, e per ciascuna prendo un rappresentante $g_i = C_i$:
    \begin{equation*}
        \ordine{C_1} = \indice{G}{C_G(g_i)} = \indice{G}{C_G(x)} \,\,\forall x \in C_1
    \end{equation*}

    Quindi:
    \begin{equation*}
        \text{casi favorevoli } = \sum_{1 \le i \le k} \ordine{C_G(g)}\ordine{C_i}
    \end{equation*}

    Possiamo ora calcolare la probabilità:
    \begin{align*}
        P(G) &= \dfrac{ \sum_{1 \le i \le k} \ordine{C_G(g)}\ordine{C_i} }{ \ordine{G}^2 } = \\
        &= \dfrac{ \sum_i \ordine{C_G(g_i)} \cdot \dfrac{\ordine{G}}{\ordine{C_G(g_i)}}}{\ordine{G}^2 } = \\
        &= \dfrac{k}{\ordine{G}}
    \end{align*}

    Per esempio la probabilità è 1 quando il gruppo è commutativo, ovvero tutte le classi di coniugio sono fatte da
    un unico elemento e il numero delle classi è uguale all'ordine del gruppo.
\end{soluzione}