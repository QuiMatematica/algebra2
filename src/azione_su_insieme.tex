\chapter{Azione di un gruppo su un insieme}

Quando si ha a che fare con i gruppi finiti è opportuno procurarsi degli intelligenti \emph{argomenti di conteggio}.

Conosciamo bene i gruppi di permutazione: è utile realizzare un gruppo come gruppo di permutazioni.

Lo sappiamo fare con il teorema di Cayley \textbf{RIF}. Ma se ho un gruppo di ordine $n$, lo devo mettere nel gruppo simmetrico di grado $n$, che ha ordine $n^2$, quindi mi trovo in un ambiente troppo grande.

Allora un trucco è quello di realizzare il mio gruppo come gruppo di permutazioni ma in modi diversi, sperando di trovarne uno semplice.

\section[Azione di un gruppo su un insieme]{Azione di un gruppo su un insieme\footnote{\cite[8 novembre 2021]{lucchini}}}

Prendiamo un gruppo $G$ e un insieme $\Omega$.
Consideriamo l'omomorfismo:

\begin{equation}
	\sigma: G \longrightarrow \Sym(\Omega)
\end{equation}

Non possiamo pretendere che $\sigma$ sia biiettiva.

Avremo quindi un sottogruppo normale rappresentato da $\ker \sigma$ tale che:

\begin{equation}
	\dfrac{G}{\ker \sigma} \isomorfo \Sym(\Omega)
\end{equation} 

Più omomorfismo ho di questo tipo più probabilità ho di trovare un buon elemento di conteggio.

Consideriamo $g \in G$ e $\omega \in \Omega$. $\sigma(g)$ è una permutazione di $\Omega$. Quindi posso applicare la permutazione $\sigma(g)$ a $\omega$, e trovo che questa sta in $\Omega$:

\begin{equation}
	\sigma(g)(\omega) \in \Omega
\end{equation}

Definiamo una notazione più rapida:

\begin{equation}
	g \circ \omega := \sigma(g)(\omega)
\end{equation}

Un'azione è una mappa tale che:

\begin{gather}
	G \times \Omega \longrightarrow \Omega \\
	(g, \omega) \longmapsto g \circ \omega
\end{gather}

Proprietà di $\circ$:

\begin{enumerate}
	\item $1 \circ \omega = \omega \quad \forall \omega \in \Omega$ \\
	perché $\sigma$ è un omomorfismo, quindi $1_G \longmapsto 1_{\Sym(\Omega)}$;
	\item $g_1 \circ (g_2 \circ \omega) = (g_1 g_2) \circ \omega \quad \forall g_1, g_2 \in G; \forall \omega \in \Omega$
\end{enumerate}

Se conosco la mappa, come trovo $\sigma$?

Per ogni elemento $g \in G$ definisco:

\begin{gather}
	\sigma_g: \Omega \longrightarrow \Omega \\
	\omega \longmapsto g \circ \omega
\end{gather}

Questa $\sigma_g$ è una biiezione, perché ha come inverso la $\sigma_{g^{-1}}$.

Ora posso costruire la mappa:

\begin{gather}
	\sigma: G \longrightarrow \Sym(\Omega) \\
	g \longmapsto \sigma_g
\end{gather}

E questa mappa è un omomorfismo.

Non dimenticare:

\begin{align}
	\ker \sigma &= \{g \in G \taleche g \circ \omega = \omega, \forall \omega \in \Omega\} \\
	&= \{g \in G \taleche \sigma(g) = 1_{\Sym(\Omega)}\}
\end{align}

\section[Stabilizzatore]{Stabilizzatore\footnote{\cite[8 novembre 2021]{lucchini}}}

Sia dato un gruppo $G$ che agisce su un insieme $\Omega$, e un elemento $\omega \in \Omega$. Lo \textbf{stabilizzatore} di $\omega$ è:

\begin{equation}
	\Stab_G(\omega) = G_\omega = \{g \in G \taleche g \circ \omega = \omega\}
\end{equation}

\begin{teorema}
	Gli stabilizzatori sono sottogruppi di $G$:
	
	\begin{equation}
		\Stab_g(\omega) \le G
	\end{equation}
\end{teorema}
\begin{dimostrazione}
	Per quanto riguarda la composizione, consideriamo due elementi $g_1$ e $g_2$ appartenenti allo $Stab_G(\omega)$. Si ha:
	
	\begin{align}
		g_1g_2 \circ \omega &= g_1 \circ (g_2 \circ \omega) &\text{ per la proprietà 2} \\
		&= g_1 \circ \omega &\text{ perché } g_2 \in \Stab_G(\omega) \\
		&= \omega &\text{ perché } g_1 \in \Stab_G(\omega) \\
		& \Longrightarrow g_1g_2 \in \Stab_G(\omega)
	\end{align}

	Per quanto riguarda l'inverso, consideriamo un generico elemento $g \in G$. Si ha:
	
	\begin{align}
		g \circ \omega = \omega &\Longrightarrow g^{-1} \circ (g \circ \omega) = g^{-1} \circ \omega \\
		&\Longrightarrow g^{-1}g \circ \omega = g^{-1} \circ \omega \\
		&\Longrightarrow 1 \circ \omega = g^{-1} \circ \omega \\
		&\Longrightarrow \omega = g^{-1} \circ \omega
	\end{align}
\end{dimostrazione}

\begin{teorema}
	L'intersezione di tutti gli stabilizzatori è il nucleo:
	
	\begin{equation}
		\bigcap_{\omega \in \Omega}\Stab_G(\omega) = \ker \sigma
	\end{equation}
\end{teorema}

\section[Esempi di azioni]{Esempi di azioni\footnote{\cite[8 novembre 2021]{lucchini}}}

\subsection{Azione di Cayley}

\begin{gather}
	G = \Omega \\
	g \circ x := gx \\
	\Stab_G(x) = 1 \\
	\ker \sigma = 1
\end{gather}

Ritroviamo il teorema di Cayley \textbf{RIF}: $G \le \Sym(G)$.

\subsection{Coniugato di un elemento}

\begin{gather}
	G = \Omega \\
	g \circ x := gxg^{-1}
\end{gather}

Quindi l'azione dell'elemento $g$ sull'elemento $x$ è il coniugato di $x$ tramite $g$.

Abbiamo quindi che $\sigma_g$ è l'automorfismo interno indotto da $g$ per coniugazione \textbf{RIF}.

\begin{align}
	\Stab_G(x) &= \{g \taleche gxg^{-1} = x \} = \\\
	&= \{g \taleche gx = xg \} = \\
	&= C_G(x)
\end{align}

Lo stabilizzante dell'elemento $x$ è il centralizzatore di $x$ in $G$.

\begin{align}
	\ker \sigma &= \bigcap_{x \in G} C_G(x) = \\
	&= Z(G)
\end{align}

Il nucleo è il centro di $G$.

Ritroviamo che $Z(G)$ è un sottogruppo normale \textbf{RIF} in quanto nucleo di un omomorfismo.

\subsection{Coniugato di un sottogruppo}

\begin{gather}
	\Omega = \{ H \taleche H \le G \} \\
	g \circ H := \{ghg^{-1} \taleche h \in H\} = gHg^{-1}
\end{gather}

L'azione dell'elemento $g$ sul sottogruppo $H$ è il coniugato di $H$ tramite $g$.

\begin{align}
	\Stab_G(x) &= \{g \taleche gHg^{-1} = H \} = \\\
	&= N_G(x)
\end{align}

Lo stabilizzatore l'insieme degli elementi $g$ che normalizzano $H$: è il \textbf{normalizzante} di $H$ in $G$.

\subsection{Azione sui laterali sinistri}

Dato un sottogruppo $H \le G$, consideriamo l'insieme dei laterali sinistri:

\begin{equation}
	\Omega = \{ xH \taleche x \in G \}
\end{equation}
	
Definiamo l'azione sui laterali sinistri:

\begin{equation}
	g \circ xH := gxH
\end{equation}

Stabilizzatore:

\begin{align}
	g \in \Stab_G(xH) &\Longleftrightarrow g \circ xH = xH \\
	&\Longleftrightarrow gxH = xH \\
	&\Longleftrightarrow gx \in xH \\
	&\Longleftrightarrow g \in xHx^{-1} \\
	\Stab_G(xH) &= xHx^{-1}
\end{align}

Definiamo il \textbf{cuore normale} di $H$ in $G$ come:

\begin{equation}
	H_G := \ker \sigma = \bigcap_{x \in G} \Stab_G(xH)
\end{equation}

Viene definito \emph{normale} perché è il nucleo di un'azione, quindi è un sottogruppo normale di $G$:

\begin{equation}
	H_G \normale G
\end{equation}

Viene definito \emph{cuore} perché è contenuto in $H$, infatti posso prendere $x = 1$ quando faccio i coniugati, per cui ottengo che:

\begin{equation}
	H_G \subseteq H
\end{equation}

\begin{teorema}
	$H_G$ è il più grande sottogruppo normale di G contenuto in H.
\end{teorema}
\begin{dimostrazione}
	Consideriamo un generico sottogruppo normale $N$ di $G$ contenuto in $H$:
	
	\begin{gather}
		N \subseteq H \\
		N \normale G
	\end{gather}

	Faccio il coniugio dei due sottogruppi:
	
	\begin{equation}
		gNg^{-1} \subseteq gHg^{-1}
	\end{equation}

	$N$ è normale quindi $gNg^{-1} = N$, quindi:
	
	\begin{equation}
		N \subseteq gHg^{-1}
	\end{equation}

	Questa proposizione è vera $\forall g$, quindi:
	
	\begin{equation}
		N \subseteq \bigcap_{g \in G} gHg^{-1} = H_G
	\end{equation}

	Quindi $H_G$ è il più grande sottogruppo normale di $G$ contenuto in $H$.
\end{dimostrazione}

\begin{teorema}
	\label{thr:Azioni_laterali_indice_finito}
	Dato $H \le G$ con $\indice{G}{H}= m$ finito, se $G$ agisce sull'insieme $\Omega = \{xH \taleche x \in G\}$, allora $\indice{G}{H_G}$ divide $m!$ (figura~\ref{fig:Azioni_laterali_indice_finito}).
\end{teorema}
\begin{dimostrazione}
	Ricordo che l'indice $\indice{G}{H}$ è il numero di laterali (sinistri o destri) di $H$ in $G$. Quindi la cardinalità di $\Omega$ è $m$ e l'omomorfismo $\sigma$ è:
	
	\begin{equation}
		\sigma: G \longrightarrow \Sym(\Omega) = S_m
	\end{equation}

	Inoltre il nucleo $\ker \sigma$ è il cuore normale $H_G$, che è un sottogruppo normale di $H$. Quindi per il secondo corollario~\ref{crl:Omomorfismi_fondamentale_2} del teorema fondamentale degli omomorfismi si ha:
	
	\begin{equation}
		\dfrac{G}{\ker \sigma} \isomorfo \sigma(G) \le S_m
	\end{equation}

	Per il teorema~\ref{thr:Laterali_Lagrange} di Lagrange, l'ordine di un sottogruppo divide l'ordine del gruppo, quindi:
	
	\begin{equation}
		\indice{G}{H_G} = \ordine{\dfrac{G}{\ker \sigma}} = \ordine{\sigma(G)} \text{ divide } \ordine{S_m} = m!
	\end{equation}
	
	
\end{dimostrazione}

\begin{figure}[tp]
	\centering
	\tikz {
		\node (a) at (0,4) {$G$};
		\node (b) at (0,2) {$H$};
		\node (c) at (0,0) {$\ker \sigma = H_G$};
		\draw (a) edge[-] (b);
		\draw (b) edge[-] (c);
		\draw[decorate,decoration={brace,raise=20pt}] (a) -- (b) node[pos=.5,right=23pt,black]{$m$};
		\draw[decorate,decoration={brace,raise=50pt}] (a) -- (c) node[pos=.5,right=53pt,black]{divide $m!$};
	}
	\caption{Rappresentazione del teorema \ref{thr:Azioni_laterali_indice_finito}.}
	\label{fig:Azioni_laterali_indice_finito}
\end{figure}

\begin{teorema}
	Se un gruppo $G$ infinito contiene un sottogruppo $H$ di indice finito, allora contiene anche un sottogruppo normale di indice finito.
\end{teorema}
\begin{dimostrazione}
	Per il teorema~\ref{thr:Azioni_laterali_indice_finito}, chiamato $m = \indice{G}{H}$, $G$ contiene il sottogruppo normale $H_G$ il cui indice divide $m!$. Quindi $\indice{G}{H_G}$ è finito.
\end{dimostrazione}

Dimostriamo ora un'estensione del teorema \textbf{RIFERIMENTO}.

\begin{esercizio}
	Sia dato un gruppo finito $G$ con $p$ il più piccolo divisore dell'ordine $\ordine{G}$. Se $H \le G$ e $\indice{G}{H} = p$, allora $H$ è normale.
	\footnote{\cite[Week 2, Exercise 8]{lucchini_week}}
\end{esercizio}
\begin{soluzione}
	Per il teorema~\ref{thr:Azioni_laterali_indice_finito} sappiamo che:
	
	\begin{equation}
		\indice{G}{H_G} \text{ divide } p!
	\end{equation}

	Quindi:
	\footnote{
		Se $\indice{G}{H_G}$ divide $p!$, allora esiste un intero positivo $a$ per il quale:
		
		\begin{equation}
			a \cdot \indice{G}{H_G} = p! \Longrightarrow \indice{G}{H_G} = \dfrac{p!}{a}
		\end{equation} 
	
		Per il teorema \textbf{RIFERIMENTO} abbiamo:
		
		\begin{equation}
			\indice{G}{H_G} = \indice{G}{H} \cdot \indice{H}{H_G} \Longrightarrow
			\dfrac{p!}{a} = p \cdot \indice{H}{H_G} \Longrightarrow
			\indice{H}{H_G} = \dfrac{(p-1)!}{a}
		\end{equation}
	
		Quindi $\indice{H}{H_G}$ divide $(p-1)!$.
	}
	
	\begin{equation}
		\indice{H}{H_G} \text{ divide } \dfrac{p!}{p} \Longrightarrow \indice{H}{H_G} \text{ divide } (p-1)!
	\end{equation}

	Per il teorema~\ref{thr:Laterali_Lagrange} di Lagrange:
	
	\begin{equation}
		\indice{H}{H_G} \dividetxt \ordine{G}
	\end{equation}

	In $(p-1)!$ tutti i divisori sono più piccoli di $p$, il quale è il più piccolo divisore primo di $\ordine{G}$. Quindi $(p-1)!$ e $\ordine{G}$ sono coprimi, quindi il loro MCD è 1. Quindi:
	
	\begin{equation}
		\indice{H}{H_G} = 1 \Longrightarrow H = H_G
	\end{equation} 

	Quindi $H$ è normale.

\end{soluzione}

\section[Orbita di un elemento di un insieme]{Orbita di un elemento di un insieme\footnote{\cite[8 novembre 2021]{lucchini}}}

Considero un gruppo $G$ che agisce su un insieme $\Omega$. Definisco la seguente relazione tra gli elementi di $\Omega$:

\begin{equation}
	\omega_1 \eq \omega_2 \sse \exists g \in G \taleche \omega_2 = g \circ \omega_1
\end{equation}

\begin{teorema}
	$\eq$ è una relazione di equivalenza.
\end{teorema}
\begin{dimostrazione}
	La relazione $\eq$ è riflessiva, infatti:
	
	\begin{equation}
		\omega = 1 \circ \omega
	\end{equation}

	La relazione $\eq$ è simmetrica, infatti:
	
	\begin{equation}
		\omega_2 = g \circ \omega_1 \Longrightarrow g^{-1} \circ \omega_2 = g^{-1} \circ g \circ \omega_1 \Longrightarrow \omega_1 = g^{-1} \circ \omega_2
	\end{equation}

	La relazione $\eq$ è transitiva, infatti:
	
	\begin{equation}
		\omega_2 = g_1 \circ \omega_1 \land \omega_3 = g_2 \circ \omega_2 \Longrightarrow \omega_1 = g_2g_1 \circ \omega_1
	\end{equation}
\end{dimostrazione}

Possiamo allora passare alle classi di equivalenza che sono le \textbf{orbite} di $G$ su $\Omega$.

Posso indicare con $G \circ \omega$ l'\textbf{orbita} di $\omega$.

\begin{teorema}
	\label{thr:orbite}
	Dati:
	
	\begin{itemize}
		\item $G$ gruppo che agisce sull'insieme $\Omega$;
		\item $\omega \in \Omega$;
		\item $O = G \circ \omega$;
		\item $H = \Stab_G(\omega)$;
		\item $\Lambda = \{xH \taleche x \in G\}$
	\end{itemize}

	La funzione:
	
	\begin{align}
		\gamma :\quad \Lambda &\longrightarrow O \\
		xH &\longmapsto x \circ \omega
	\end{align}

	è una funzione ben definita e biiettiva.
\end{teorema}
\begin{dimostrazione}
	La funzione $\gamma$ è ben definita, infatti è indipendente dal rappresentante del laterale sinistro che utilizzo. Se infatti $x_1H = x_2H$, allora:
	\begin{equation*}
		\exists h \in H \taleche x_2 = x_1 h
	\end{equation*}
	
	Quindi accade che:
	
	\begin{align}
		x_2 \circ \omega &= x_1 h \circ \omega \\
		&= x_1 \circ (h \circ \omega) \\
		&= x_1 \circ \omega
	\end{align}

	La funzione $\gamma$ è suriettiva per definizione di orbita. 
	
	La funzione $\gamma$ è iniettiva perché:
	
	\begin{align}
		\gamma(x_1H) = \gamma(x_2H) 
		&\Longrightarrow x_1 \circ \omega = x_2 \circ \omega \\
		&\Longrightarrow x_2^{-1} \circ x_1 \circ \omega = \omega \\
		&\Longrightarrow x_2^{-1} x_1 \in H \quad \text{per definizione di stabilizzatore} \\
		&\Longrightarrow x_1H = x_2H
	\end{align}
\end{dimostrazione}

Se esiste una corrispondenza biiettiva significa che $\Lambda$ e $O$ hanno la stessa cardinalità:

\begin{equation}
	\label{eq:ordine_orbita} \ordine{O} = \ordine{\Lambda} = \indice{G}{\Stab_G(\omega)}
\end{equation}

Per calcolare quanti elementi ci sono nell'orbita di un elemento $\omega$ basta calcolare lo sbabilizzatore e l'indice.

Se $G$ agisce su $G$ \textbf{per coniugio}, come sono fatte le orbite?

Prendiamo un generico elemento $x \in G$. L'orbita è:

\begin{equation}
	G \circ x = \{gxg^{-1} \taleche g \in G\}
\end{equation}

Ovvero è la classe di coniugio di $x$ in $G$. Per calcolare quanti elementi ha questa classe basta:

\begin{equation}
	\ordine{G \circ x} = \indice{G}{\Stab_G(x)} = \indice{G}{C_G(x)}
\end{equation}

\textbf{L'ordine della classe di coniugio di un elemento è uguale all'indice del centralizzante dello stesso elemento:}

\begin{equation}
	\label{eq:ordine_classe_coniugio}
	\ordine{G \circ x} = \indice{G}{C_G(x)}
\end{equation}

Se invece G agisce \textbf{per coniugio sui sottogruppi}, dato un sottogruppo $H \le G$, quanti sono i coniugati di $H$ in $G$?

\begin{equation}
	\ordine{G \circ H} = \indice{G}{\Stab_G(H)} = \indice{G}{N_G(x)}
\end{equation}

