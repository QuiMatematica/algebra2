\chapter{Esercizi sulle classi di coniugio}

\begin{corollario}
	Due elementi di $S_n$ sono coniugati se hanno la stessa struttura ciclica.
	\footnote{Nella registrazione \cite[13 ottobre 2021]{lucchini} manca il primo pezzo, dove immagino abbia discusso come due elementi coniugati di $S_n$ hanno la stessa struttura ciclica. Per esempio, se prendiamo:
	
		\begin{equation}
			\alpha = (124) \in S_6
		\end{equation}

	e lo coniughiamo con:
	
	\begin{equation}
		\beta = (13)(25)(46)
	\end{equation} 

	Mi basta sostituire gli elementi di $\alpha$ usando i cicli di $\beta$, quindi:
	
	\begin{itemize}
		\item mando l'1 nel 3 perché in $\beta$ c'è (13);
		\item mando il 2 nel 5 perché in $\beta$ c'è (25);
		\item mando il 4 nel 6 perché in $\beta$ c'è (46).
	\end{itemize}

	Quindi:
	
	\begin{equation}
		\beta \alpha \beta^{-1} = (356)
	\end{equation}
	}
\end{corollario}

Per esempio:

\begin{gather}
	\sigma_1 = (149)(25) \in S_{10} \\
	\sigma_2 = (273)(14) \in S_{10}
\end{gather}

Cerco un $\beta$ tale che:

\begin{equation}
	\beta \sigma_1 \beta^{-1} = \sigma_2
\end{equation}

$\beta$ deve:
\begin{itemize}
	\item mandare l'1 nel 2;
	\item mandare il 4 nel 7;
	\item mandare il 9 nel 3;
	\item mandare il 2 nell'1;
	\item mandare il 5 nel 4.
\end{itemize}

Ci sono ancora diversi gradi di libertà perché ci sono diversi elementi non nominati. Una possibile soluzione potrebbe essere:

\begin{equation}
	\beta = (12)(547)(39)
\end{equation}

\begin{esercizio}
	Descrivi le classi di coniugio e i centralizzatori degli elementi nel gruppo simmetrico $S_4$.
\end{esercizio}
\begin{soluzione}
		
	Per ogni elemento voglio contare i coniugati e vedere chi è il centralizzante.

	Per il teorema precedente\footnote{mancante}, le classi di coniugio corrispondono alle possibili strutture cicliche. In particolare \textbf{il numero di coniugati corrisponde al numero di elementi della struttura ciclica}.
	
	Di identità ce n'è una sola.
	
	I 2-cicli sono pari al numero di combinazioni di 4 elementi in gruppi da 2. Uso le combinazioni perché i cicli $(12)$ e $(21)$ sono in verità lo stesso ciclo, quindi non devo considerare gruppi con elementi ordinati. Quindi:
	
	\begin{equation}
		\text{numero di 2-cicli } = \binom{4}{2} = \dfrac{4!}{2! \cdot 2!} = \dfrac{4 \cdot 3}{2} = 6
	\end{equation}

	Per contare i 3-cicli devo:
	
	\begin{itemize}
		\item decidere quale elemento deve stare fermo, e ho 4 scelte (per esempio tengo fermo l'elemento 4);
		\item per ciascun elemento fermo ho un 3-ciclo e il suo inverso (quindi ho i cicli $(123)$ e $(132)$).
	\end{itemize}

	Quindi in totale ho:
	
	\begin{equation}
		\text{numero di 3-cicli } = 4 \cdot 2 = 8
	\end{equation}

	Per contare i 4-cicli considero che il primo elemento è indifferente, mentre gli altri possono essere in qualunque ordine. Quindi:
	
	\begin{equation}
		\text{numero di 4-cicli } = 3! = 6
	\end{equation}

	Per contare i doppi 2-cicli basta accoppiare i 2-cicli, quindi:
	
	\begin{equation}
		\text{numero di doppi 2-cicli } = 6 : 2 = 3
	\end{equation}
	
	Dal momento che ho fatto una partizione, la somma deve fare l'ordine di $S_4$, che è 24, infatti:
	
	\begin{equation}
		1 + 6 + 8 + 6 + 3 = 24
	\end{equation}
	
	Per la formula~\eqref{eq:ordine_classe_coniugio} posso calcolare l'ordine dei centralizzanti come:
	
	\begin{equation}
		\ordine{C_G(g)} = \dfrac{\ordine{G \circ x}}{\ordine{G}}
	\end{equation}

	Quindi:
	
	\begin{itemize}
		\item l'identità ha la classe di coniugio di ordine 1, quindi il centralizzante di ordine 24 (infatti tutti gli elementi di un qualunque gruppo commutano con l'identità);
		\item il 2-ciclo $(12)$ ha la classe di coniugio di ordine 6, quindi il centralizzante di ordine 4;
		\item e così via...
	\end{itemize}
	
	Rimane da determinare i centralizzanti.
	
	Il centralizzante dell'identità è l'intero insieme $S_4$, visto che tutti gli elementi del gruppo commutano con l'identità e visto che l'ordine del centralizzante corrisponde all'ordine del gruppo.
	
	Chi commuta con $(12)$? Devo trovare 4 elementi. $(12)$ commuta con se stesso. Inoltre $(34)$ commuta con $(12)$ perché è un ciclo disgiunto. Il gruppo $\gen{(12)(34)}$ ha ordine 4, quindi è questo il centralizzante che sto cercando:
	
	\begin{equation}
		C_G((12)) = \gen{(12),(34)} \,\isomorfo C_2 \times C_2
	\end{equation}

	Chi commuta con $(123)$? Devo trovare 3 elementi. $(123)$ commuta con se stesso. Il sottogruppo $\gen{(123)}$ ha 3 elementi: 1, $(123)$ e $(132)$. Quindi è questo il centralizzante che sto cercando:
	
	\begin{equation}
		C_G((123)) = \gen{(123)} \,\isomorfo C_3
	\end{equation}
	
	Chi commuta con $(1234)$? Devo trovare 4 elementi. $(1234)$ commuta con se stesso e il sottogruppo $\gen{(1234)}$ ha 4 elementi: 1, $(1234)$, $(13)(24)$ e $(1432)$. Quindi è questo il centralizzante che sto cercando:
	
	\begin{equation}
		C_G((1234)) = \gen{(1234)} \,\isomorfo C_4
	\end{equation}
	
	Chi commuta con $(12)(34)$? Devo trovare 8 elementi.
	
	Osservo che:
	
	\begin{equation}
		(12)(34) = (1324)^2
	\end{equation}

	perché l'1 va nel 2 e viceversa, e il 3 va nel 4 e viceversa. Ma dal momento che $(1324)$ è quadrato di $(12)(34)$, i due termini commutano.
	
	Ho trovato i primi 4 elementi:
	
	\begin{equation}
		\gen{(1324)} = \{1, (1324), (12)(34), (1423)\}
	\end{equation}
	
	Me ne servono altri 4.
	
	Facciamo riferimento a $D_4$. $(1324)$ è una rotazione e il gruppo generato da questo elemento corrisponde a tutte le rotazioni di un quadrato (figura~\ref{fig:Isometrie_D4_da_1324}). Inoltre l'elemento $(12)(34)$ corrisponde alla rotazione di $\pi$. Il nostro elemento, quindi, è centro di $D_4$. Ci basta quindi prendere una riflessione di $D_4$ e trovare così tutti gli elementi che commutano con $(12)(34)$. Una tale riflessione potrebbe essere $(14)(23)$. Quindi il centro cercato è:
	
	\begin{equation}
		C_G((12)(34)) = \gen{(1324), (14)(23)} \isomorfo C_4 \times C_2 \isomorfo D_4
	\end{equation}
	
	
\end{soluzione}

	\begin{sidewaystable}
	\centering
	\begin{tabular}{ccccc}
		\toprule
		Struttura ciclica & Elemento & Numero coniugati & Ordine centralizzante & Centralizzante \\
		\midrule
		identità & 1 & 1 & 24 & $S_4$ \\
		2-cicli & (12) & $\binom{4}{2} = 6$ & 4 & $\gen{(12), (34)} \,\isomorfo C_2 \times C_2$ \\
		3-cicli & (123) & $4 \cdot 2 = 8$ & 3 & $\gen{(123)} \,\isomorfo C_3$ \\
		4-cicli & (1234) & $3! = 6$ & 4 & $\gen{(12345)} \,\isomorfo C_4$ \\
		doppi 2-cicli & (12)(34) & 6:2 = 3 & 8 & $\gen{(1324),(14)(23)} \,\isomorfo C_4 \times C_2 \isomorfo D_4$ \\
		\midrule
		& & $24 = \ordine{S_4}$ & & \\ 
		\bottomrule
	\end{tabular}
	\caption{Classi di coniugio e centralizzanti di $S_4$}
	\label{fig:classi_coniugio_s4}
\end{sidewaystable}

	\begin{figure}[tp]
	\centering
	\begin{tikzpicture}[line cap=round,line join=round,>=triangle 45,x=1cm,y=1cm]
		\begin{axis}[
			x=2cm,y=2cm,
			axis lines=middle,
			xmin=-2.5,
			xmax=2.5,
			ymin=-1.5,
			ymax=1.5,
			xtick={-3,3},
			ytick={-2,2},]
			\clip(-2.5,-1.5) rectangle (2.5,1.5);
			\fill[line width=2pt,color=figura,fill=figura,fill opacity=0.1] (1,-1) -- (1,1) -- (-1,1) -- (-1,-1) -- cycle;
			%\draw [line width=2pt] (0,0) circle (2cm);
			\draw [line width=2pt,color=figura] (1,-1)-- (1,1);
			\draw [line width=2pt,color=figura] (1,1)-- (-1,1);
			\draw [line width=2pt,color=figura] (-1,1)-- (-1,-1);
			\draw [line width=2pt,color=figura] (-1,-1)-- (1,-1);
			\begin{scriptsize}
				\draw (1.15,-1.05) node {1};
				\draw (1.15,1.05) node {3};
				\draw (-1.15,1.05) node {2};
				\draw (-1.15,-1.05) node {4};
				\draw (2.4,-.1) node {$x$};
				\draw (-0.1,1.4) node {$y$};
			\end{scriptsize}
		\end{axis}
	\end{tikzpicture}
	\caption{Gruppo $D_4$ per le rotazioni $(1324)^n$}
	\label{fig:Isometrie_D4_da_1324}
\end{figure}

\begin{esercizio}
	$\sigma \in S_n$ sia un $n$-ciclo. Determinare il suo centralizzante $C_{S_n}(\sigma)$.
\end{esercizio}
\begin{soluzione}
	Innanzitutto determiniamo quanti sono gli $n$-cicli. Al primo posto posso mettere un elemento qualunque, per semplicità poniamo di porre un 1. Devo poi combinare gli altri $n-1$ elementi in tutti i possibili ordini, quindi avrò:
	
	\begin{equation}
		\text{numero di $n$-cicli } = \ordine{^{S_n}\sigma} = P_n = n! 
	\end{equation}  

	Quindi per la formula~\eqref{eq:ordine_classe_coniugio} il centralizzante ha ordine:
	
	\begin{equation}
		\ordine{C_{S_n}(\sigma)} = \dfrac{\ordine{S_n}}{\ordine{^{S_n}\sigma}} = \dfrac{n!}{(n-1)!} = n
	\end{equation}

	Osserviamo ora che $\gen{\sigma}$, ovvero il più piccolo sottogruppo che contiene $\sigma$ ha ordine $n$ quindi:
	
	\begin{equation}
		C_{S_n}(\sigma) = \gen{\sigma} \isomorfo C_n
	\end{equation}

	Le uniche permutazioni di grado $n$ che commutano con un $n$-ciclo sono le potenze di quel ciclo.
\end{soluzione}

\begin{esercizio}
	Trovare tutti i sottogruppi normali di $S_4$.
\end{esercizio}
\begin{soluzione}
	Alcuni sottogruppi normali di $S_4$ sono di immediata individuazione:
	
	\begin{itemize}
		\item $S_4$: ogni gruppo è normale di se stesso;
		\item 1: il sottogruppo dell'identità è normale in quanto, per definizione, ogni elemento commuta con l'identità;
		\item $A_4$: il gruppo alterno ha indice 2, quindi è normale (per il teorema~\ref{thr:sottogruppi_di_indice_2})
	\end{itemize}

	Inoltre se consideriamo il gruppo diedrale $D_4$ abbiamo il sottogruppo:
	
	\begin{equation}
		V = \gen{(12)(34), (14)(23)} = \{1, (12)(34), (14)(23), (13)(24)\}
	\end{equation}

	$V$ è il \textbf{gruppo di Klein}. E' un sottogruppo normale in quanto composto dall'identità e dalla classe di coniugio dei doppi scambi.
	
	Abbiamo quindi trovato finora 4 sottogruppi normali di $S_4$:
	
	\begin{equation}
		1, V, A_4, S_4 \normale S_4
	\end{equation}

	Vogliamo ora dimostrare che non ce ne possono essere altri.
	
	Ipotizziamo per assurdo che esista un sottogruppo $N$ normale di $S_4$. Dobbiamo analizzare due situazioni.
	
	\textbf{Primo caso:} $N$ contiene $V$.
	
	Per il teorema di corrispondenza~\ref{thr:corrispondenza} i sottogruppi normali di $S_4$ che contengono $V$ sono in biiezione con i sottogruppi normali di $S_4/V$.
	
	Studiamo $S_4/V$.
	
	Consideriamo il sottogruppo di $S_4$ delle permutazioni che tengono ferme l'elemento 4 e permutano gli altri 3:
	
	\begin{equation}
		H = \{\sigma \taleche \sigma(4) = 4\} \le S_4
	\end{equation}

	Se tale gruppo agisce solo sugli elementi 1, 2 e 3, esso è isomorfo ad $S_3$:
	
	\begin{equation}
		H \isomorfo S_3
	\end{equation}

	Consideriamo ora l'insieme $HV$: visto che $V$ è un sottogruppo normale, allora per il teorema~\ref{thr:prodotto_sottogruppi} $HV$ è un sottogruppo di $S_4$.
	
	L'intersezione $H \cap V$ contiene solo l'identità in quanto $V$ contiene i doppi scambi e nessun doppio scambio tiene fermo 4.
	
	Quindi, per il teorema~\ref{thr:ordine_prodotto_sottogruppi}:
	
	\begin{equation}
		\ordine{HV} = \dfrac{\ordine{H}\ordine{V}}{\ordine{H \cap V}} = \dfrac{6 \cdot 4}{1} = 24
	\end{equation}

	Quindi:
	
	\begin{equation}
		\ordine{HV} = \ordine{S_4} \quad\Longrightarrow\quad HV = S_4
	\end{equation}

	Ora, per il teorema dei due sottogruppi~\ref{thr:due_sottogruppi} abbiamo:
	
	\begin{equation}
		\dfrac{S_4}{V} = \dfrac{HV}{V} \isomorfo \dfrac{H}{H \cap V} \isomorfo H \isomorfo S_3
	\end{equation}

	Quindi per scoprire i sottogruppi normali di $S_4/V$ ci basta conoscere i sottogruppi normali di $S_3$.
	
	I sottogruppi di $S_3$ sono:
	
	\begin{itemize}
		\item $\gen{(12)}$
		\item $\gen{(13)}$
		\item $\gen{(23)}$
		\item $\gen{(123)}$
	\end{itemize}

	I primi tre sottogruppi non sono normali, in quanto i loro generatori sono tra loro coniugati. Il quarto ha indice 2 quindi è normale.
	
	Posso infine usare il teorema di corrispondenza~\ref{thr:corrispondenza} e trovare i sottogruppi normali che contengono $V$: mi basta trovare le anti-immagini dei sottogruppi normali di $S_3$. $S_3$ è in corrispondenza con $S_4$, mentre l'identità di $S_3$ è in corrispondenza con il sottogruppo $V$ di $S_4$. Dobbiamo determinare chi è in corrispondenza con il sottogruppo $\gen{(123)}$ di $S_3$, ma tale sottogruppo ha indice 2, quindi è in corrispondenza con il sottogruppo normale di indice 2 di $S_4$, che è $A_4$ (figura~\ref{fig:corrispondenza_per_S4_su_V}).
	
	
	Non ci sono quindi altri sottogruppi normali di $S_4$ che contengono $V$.
	
	\textbf{Secondo caso:} $N$ non contiene $V$.
	
	Se $N$ non contiene $V$, significa che $N$ non contiene nessun doppio scambio.
	
	Il quadrato di un 4-ciclo è un doppio scambio:
	
	\begin{equation}
		(1234)^4 = (13)(24)
	\end{equation}
	
	Quindi se $N$ contenesse i 4-cicli allora conterrebbe anche i doppi scambi. Visto che non può contenere doppi scambi, $N$ non contiene i 4-cicli.
	
	Il prodotto di due 3-cicli è un doppio scambio:
	
	\begin{equation}
		(123) \cdot (124) = (13)(24)
	\end{equation}

	Quindi se $N$ contenesse i 3-cicli allora conterrebbe anche i doppi scambi. Visto che non può contenere doppi scambi, $N$ non contiene i 3-cicli.
	
	Il prodotto di due scambi disgiunti è un doppio scambio:
	
	\begin{equation}
		(12)\cdot (34) = (12)(34)
	\end{equation}

	Quindi se $N$ contenesse gli scambi allora conterrebbe anche i doppi scambi. Visto che non può contenere doppi scambi, $N$ non contiene gli scambi.

	Quindi solo l'identità può stare in $N$, ovvero se $N$ non contiene $V$, allora $N = 1$.
	
	Quindi non ho altri sottogruppi normali di $S_4$ al di fuori di $S_4$, $A_4$, $V$ e $1$.
\end{soluzione}
	\begin{figure}[tp]
	\centering
	\tikz {
		\node (a) at (0,4) {$S_4$};
		\node (b) at (0,2) {$N = A_4$};
		\node (c) at (0,0) {$V$};
		\node (x1) at (2,4) {$\Longleftrightarrow$};
		\node (x2) at (2,2) {$\Longleftrightarrow$};
		\node (x3) at (2,0) {$\Longleftrightarrow$};
		\node (d) at (4,4) {$S_4/V$};
		\node (e) at (4,2) {$N/V = A_4/V$};
		\node (f) at (4,0) {$V/V = 1$};
		\node (y1) at (6,4) {$\isomorfo$};
		\node (y2) at (6,2) {$\isomorfo$};
		\node (y3) at (6,0) {$\isomorfo$};
		\node (g) at (8,4) {$S_3$};
		\node (h) at (8,2) {$\gen{(123)}$};
		\node (i) at (8,0) {$1$};
		\draw (a) edge[->] (b) 
		(b) edge[->] (c);
		\draw (d) edge[->] (e) 
		(e) edge[->] (f);
		\draw (g) edge[->] (h) 
		(h) edge[->] (i);
	}
	\caption{Teorema di corrispondenza per $S_4/V$.}
	\label{fig:corrispondenza_per_S4_su_V}
\end{figure}

Osserviamo che in generale \textbf{la relazione di normalità non è transitiva}.

Prendiamo, per esempio:
\begin{equation*}
	H = \gen{(12)(34)} \isomorfo C_2
\end{equation*}

Prendiamo inoltre il gruppo di Klein $V$ che è abeliano, quindi tutti i suoi sottogruppi sono normali:
\begin{equation}
	H \normale V
\end{equation}

Inoltre abbiamo visto prima che:
\begin{equation}
	V \normale S_4
\end{equation}

Però $H$ non è un sottogruppo normale di $S_4$.

\begin{teorema}
	\label{th:uguaglianza_di_coniugi}
	Supponi che $xg_1 x^{-1} = g_2$.
	Dimostra che $y g_1 y^{-1} = g_2$ se e solo se $y \in x C_G(g_1)$.
\end{teorema}
\begin{dimostrazione}
	Vogliamo che i due coniugi siano uguali:
	\begin{equation}
		y g_1 y^{-1} = x g_1 x^{-1}
	\end{equation}
	Moltiplico per gli inversi per portare le $x$ al primo membro:
	\begin{gather}
		x^{-1} y g_1 y^{-1} x = g_1 \\
		x^{-1} y \in C_G(g_1) \\
		y \in x C_G(g_1)
	\end{gather}
\end{dimostrazione}

\begin{esercizio}
	\label{th:non_coniugati_in_a4}
	Dimostrare che $(123)$ e $(132)$ sono coniugati in $S_4$ ma non sono coniugati in $A_4$.
\end{esercizio}
\begin{soluzione}
	I 3-cicli $(123)$ e $(132)$ sono coniugati in $S_4$ perché hanno la stessa struttura ciclica.

	Determiniamo allora tutti i $\tau \in S_4$ tali che $\tau (123) \tau^{-1} = (132)$ e selezioniamo poi quelli
	presenti anche in $A_4$.

	Lo scambio $(23)$ ha le caratteristiche desiderate.

	Per il teorema~\ref{th:uguaglianza_di_coniugi} posso trovare tutti gli altri valori $\tau$:
	\begin{align}
		\tau \in (23) C_{S_4}(123) &= (23) \gen{(123)} = \\
		&= (23)\{1, (123), (132)\} = \\
		&= \{(23), (12), (13)\}
	\end{align}
	Abbiamo trovato solo permutazioni dispari.
	Quindi non esiste una permutazione $\tau$ pari tale che $\tau (123) \tau^{-1} = (132)$.

	Quindi $(123)$ e $(132)$ non sono coniugati in $A_4$: appartengono a classi di coniugio diverse.
\end{soluzione}

\begin{esercizio}
	Determinare le classi di coniugio di $A_4$.
\end{esercizio}

\begin{soluzione}
	Il gruppo $A_4$ contiene l'identità, i doppi scambi e i 3-cicli:
	\begin{equation}
		1 \quad (12)(34) \quad (123)
	\end{equation}

	In $A_4$ non ho più la proprietà che i coniugati di una permutazioni coincidono con tutte e sole le permutazioni
	con la stessa struttura ciclica.
	Quindi non ho uno strumento per contare facilmente i coniugati.
	Tuttavia conoscendo i centralizzanti di $S_4$ diventa facile trovare i centralizzanti di $A_4$ infatti:
	\begin{equation}
		C_{A_4}(\sigma) = C_{S_4}(\sigma) \cap A_4
	\end{equation}

	Determinati i centralizzanti e i loro ordini posso determinare le cardinalità delle classi di coniugio.

	\textbf{Centralizzante di $(12)(34)$.}
	
	Il centralizzante in $S_4$ dei doppi scambi è il gruppo diedrale $D_4$, quindi:
	\begin{equation}
		C_{A_4}((12)(34)) = A_4 \cap D_4
	\end{equation}
	
	$D_4$ ha ordine 8 e contiene delle permutazioni dispari.
	Se seleziono solo le permutazioni pari ottengo un gruppo di Klein $V$ che ha ordine 4.
	Quindi:
	\begin{gather}
		\ordine{C_{A_4}((12)(34))} = \ordine{V} = 4
	\end{gather}

	Il numero di coniugati di $(12)(34)$ è $12:4 = 3$, che corrisponde al numero di doppi scambi.

	\textbf{Centralizzante di $(123)$.}

	Il centralizzante in $S_4$ di un 3-ciclo è il sottogruppo generato dal 3-ciclo stesso, quindi:
	\begin{equation}
		C_{A_4}((123)) = A_4 \cap \gen{(123)} = \gen{(123)}
	\end{equation}

	Quindi il numero di coniugati è $(123)$ è $12:3 = 4$.
	Ma i 3-cicli in $A_4$ sono 8.
	Quindi non è più vero che tutti le permutazioni con la stessa struttura ciclica sono coniugate.

	Infatti nel teorema~\ref{th:non_coniugati_in_a4} abbiamo dimostrato che $(123)$ e $(132)$ non sono coniugate.

	Avremo quindi 4 coniugati per $(123)$ e 4 coniugati per $(132)$ (figura~\ref{fig:classi_coniugio_a4}).

\end{soluzione}

\begin{sidewaystable}
	\centering
	\begin{tabular}{ccccc}
		\toprule
		Struttura ciclica & Elemento & Numero coniugati & Ordine centralizzante & Centralizzante \\
		\midrule
		identità & 1 & 1 & 12 & $A_4$ \\
		3-cicli & (123) & 4 & 3 & $\gen{(123)} \,\isomorfo C_3$ \\
		3-cicli & (132) & 4 & 3 & $\gen{(123)} \,\isomorfo C_3$ \\
		doppi 2-cicli & (12)(34) & 3 & 4 & $\gen{(12)(34),(14)(23)} \,\isomorfo C_2 \times C_2 \isomorfo V$ \\
		\midrule
		& & $12 = \ordine{A_4}$ & & \\
		\bottomrule
	\end{tabular}
	\caption{Classi di coniugio e centralizzanti di $A_4$}
	\label{fig:classi_coniugio_a4}
\end{sidewaystable}

\section{Facciamo finta di essere Galois...}
\label{sec:coniugio_galois}

Facciamo finta di essere Galois e di voler trovare la formula risolutiva delle equazioni di quinto grado.

Mettiamo in corrispondenza la teoria dei gruppi e le formule risolutive.

\begin{quotation}
	La formula risolutiva delle equazioni di quinto grado esiste se e solo se $A_5$ non è un gruppo semplice.
\end{quotation}

C'è però il sospetto che $A_5$ sia un gruppo semplice.
Dimostriamolo.

Poi Galois dimostrerà che $A_n$ è semplice per ogni $n \ge 5$.

Questo apre la strada allo studio dei gruppi semplici.
Studio importante perché ogni volta che abbiamo un sottogruppo normale $N \normale G$ abbiamo altri due gruppi: $N$ e $G/N$.
Questi gruppi potrebbero avere altri sottogruppi \ldots
Affetto il gruppo fino ad avere solo gruppi semplici.

Se so tutto sui gruppi semplici, posso studiare tutti i gruppi finiti.

La lista dei gruppi semplici si è arricchita.
C'è anche il gruppo mostro scoperto nel 1984 grazie ai computer.
E c'è una dimostrazione mostruosa (15000 pagine per 600 articoli) che tutti i gruppi semplici sono \ldots (boh, non si
capisce dal video).

\begin{teorema}
	$A_5$ è un gruppo semplice.
\end{teorema}

\begin{dimostrazione}
	Ordine di $A_5$:
	\begin{equation*}
		\ordine{A_5} = \dfrac{5!}{2} = 60
	\end{equation*}

	Elenchiamo gli elementi di ordine pari che posso trovare:
	\begin{itemize}
		\item l'identità 1: ne abbiamo una sola;
		\item i doppi scambi del tipo $(12)(34)$: determino quale elemento sta fermo e per ciascuno di questi ho 3
		possibili doppi scambi;
		in totale: 15;
		\item i 3-cicli del tipo $(123)$: determino i due che stanno fermi e per questi ho 2 possibili 3-cicli;
		in totale: $\binom{5}{2} \cdot 2 = 20$
		\item i 5-cicli del tipo $(12345)$: $4! = 24$.
	\end{itemize}

	Se stessimo parlando di $S_5$ diremo che le classi di coniugio hanno tanti elementi tanti quanti le permutazioni
	di ciascuna struttura ciclica.
	Ma abbiamo lo stesso problema che abbiamo avuto nel passaggio da $S_4$ a $A_4$.

	Come si comportano le classi di coniugio in $A_5$?

	\textbf{Doppi cicli}
	\begin{gather*}
		\ordine{C_{S_5}((12)(34))} = \dfrac{\ordine{S_5}}{\text{num. coniugati}} = \dfrac{120}{15} = 0 \\
		C_{S_5}((12)(34)) = C_{S_4}((12)(34)) = D_4 \\
		C_{A_5}((12)(34)) = D_4 \cap A_5 = V \\
		\ordine{C_{A_5}((12)(34))} = 4
	\end{gather*}
	Ora, visto che $60 : 4 = 15$, deduco che tutti i doppi scambi sono coniugati in $A_5$.

	\textbf{3-cicli}
	\begin{gather*}
		\ordine{C_{S_5}((123))} = \dfrac{120}{20} = 6 \\
		C_{S_5}((123)) = \gen{(123)(45)} \isomorfo C_6 \\
		C_{A_5}((123)) = \gen{(123)(45)} \cap A_5 = \gen{(123)} \\
		\ordine{C_{A_5}((123))} = 3
	\end{gather*}
	Ora, visto che $60 : 3 = 20$, deduco che tutti i 3-cicli sono coniugati in $A_5$.

	\textbf{5-cicli}
	\begin{gather*}
		C_{S_5}((12345)) = \gen{(12345)} \\
		C_{A_5}((12345)) = \gen{(12345)} \cap A_5 = \gen{(12345)}
		\ordine{C_{A_5}} = 5
	\end{gather*}
	Numero di coniugati: $60:5 = 12$.

	Quindi i 24 5-cicli appartengono a 2 classi di coniugio, ciascuna con 12 elementi.

	Quindi le classi di coniugio di $A_5$ hanno cardinalità: 1, 15, 20, 12 e 12.

	Supponiamo ora di avere un $N \normale A_5$.

	\textbf{Un sottogruppo normale è unione di classi di coniugio.}

	Quindi $N$ deve essere unione di classi di coniugio.
	Ovvero l'ordine di $N$ deve essere pari ad 1 (ordine della
	classe di coniugio dell'unità) e di qualcuno tra 15, 20, 12 3 12.

	Ma devo anche trovare un divisore di 60.
	Divisore che non mi è possibile trovare.

	Quindi non ci sono sottogruppi normali se non 1 e $A_5$.
\end{dimostrazione}
