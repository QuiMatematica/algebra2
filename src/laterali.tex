\chapter{Laterali}

\section{Laterali destri e sinistri}

Dato un sottogruppo $H$ di un gruppo $G$, possiamo definire questa relazione d'equivalenza:

\begin{equation}
	g_1 \sim g_2 \Longleftrightarrow \exists h \in H \taleche g_2 = g_1 h
\end{equation}

Le classi di equivalenza di questa relazione si chiamano \textbf{laterali sinistri} e sono del tipo $xH$: prendo un elemento $x$ di $G$ e lo moltiplico per tutti gli elementi di $H$.

Il numero di classi di equivalenza si chiama \textbf{indice} e si indica con $\indice{G}{H}$.

L'indice dei laterali sinistri è uguale all'indice dei laterali destri.

\section{Trasversali}

Un \textbf{trasversale sinistro} è un insieme $T$ di rappresentanti per le classi laterali sinistre.

I laterali sinistri formano una partizione del gruppo, quindi possiamo scrivere:
\begin{equation*}
	G = \bigcup^\circ_{t \in T} tH
\end{equation*}

Un trasversale sinistro non è necessariamente un trasversale destro.

Consideriamo per esempio il gruppo $S_3$ (vedi capitolo~\ref{ch:S3}) e il suo sottogruppo:
\begin{equation*}
	H=\gen{(23)} = \{(1), (23)\}
\end{equation*}

I laterali sinistri di $H$ in $S_3$ sono:
\begin{gather*}
	1H = \{(1), (23)\} = (23)H \\
	(12)H = \{(12), (123)\} = (123)H \\
	(13)H = \{(13), (132)\} = (132)H
\end{gather*}

Mentre i laterali destri di $H$ in $S_3$ sono:
\begin{gather*}
	H1 = \{(1), (23)\} = H(23) \\
	H(12) = \{(12), (132)\} = H(132) \\
	H(13) = \{(13), (123)\} = H(123)
\end{gather*}

Quindi l'insieme

\begin{equation}
	T = \{(1), (13), (123)\}
\end{equation}

è un trasversale sinistro, ma non un trasversale destro.

\begin{teorema}
	Se $T$ è un trasversale sinistro, allora $T^{-1}$ è un trasversale destro, dove:
	
	\begin{equation}
		T^{-1} = \{t^{-1} \taleche t \in T\}
	\end{equation}
\end{teorema}
\begin{dimostrazione}
	Partiamo dall'unione disgiunta dei trasversali sinistri e applichiamo l'inverso:
	
	\begin{gather}
		G = \bigcup^\circ_{t \in T} tH \\
		G^{-1} = \bigcup^\circ_{t \in T} (tH)^{-1} \\
		G = \bigcup^\circ_{t \in T} H^{-1}t^{-1} \\
		G = \bigcup^\circ_{t \in T} Ht^{-1}
	\end{gather}

	Dove abbiamo potuto considerare che $G = G^{-1}$ e $H = G^{-1}$ in quanto gruppi.
\end{dimostrazione}

Abbiamo quindi dimostrato che:

\begin{teorema}
	L'indice, sia che lavori a destra sia che lavori a sinistra, è sempre lo stesso.
\end{teorema}

\section{Sottogruppo normale}

Un sottogruppo $N \le G$ è \textbf{normale} se si ha:

\begin{equation}
	gNg^{-1} = N \quad \forall g \in G
\end{equation}

oppure, che è equivalente:

\begin{equation}
	gN = Ng \quad \forall g \in G
\end{equation}

Si indica con:

\begin{equation}
	N \normale G
\end{equation}

Se $N$ è un sottogruppo normale, un laterale destro è anche laterale sinistro, e un trasversale destro è anche trasversale sinistro.

\section{Sottogruppo di sottogruppo}

\begin{teorema}
	\label{thr:fattorizzazione_indici}
	Se $K \le H \le G$, allora
	
	\begin{equation}
		\indice{G}{K} = \indice{G}{H} \cdot \indice{H}{K}
	\end{equation} 
\end{teorema}
\begin{dimostrazione}
	Chiamo $T$ un trasversale sinistro di $H$ in $G$.
	
	Chiamo $U$ un trasversale sinistro di $K$ in $H$.
	
	Vogliamo trovare un trasversale sinistro di $K$ in $G$:
	
	\begin{gather}
		G = \bigcup^\circ_{t \in T} tH \quad\land\quad H = \bigcup^\circ_{u \in U} uK \\
		G = \bigcup^\circ_{t \in T} t \biggl(\bigcup^\circ_{u \in U} uK\biggr) \\
		G = \bigcup^\circ_{t \in T, u \in U} tuK
	\end{gather}

	Quindi $TU$ è un trasversale sinistro di $K$ in $G$.
	
	Quindi:
	
	\begin{equation*}
		\indice{G}{K} = \ordine{TU} = \ordine{T} \cdot \ordine{U} = \indice{G}{H} \cdot \indice{H}{K}
	\end{equation*}

	Ovvero: \emph{il sottogruppo intermedio fattorizza l'indice}.
\end{dimostrazione}

\section{Intersezione di sottogruppi}
\label{sec:laterali_intersezione_sottogruppi}

\begin{esercizio}
	Dati due sottogruppi $H$ e $K$ di $G$ (figura~\ref{fig:Laterali_intersezione_di_sottogruppi}):
	\begin{equation*}
		\indice{G}{H \cap K} \le \indice{G}{H} \cdot \indice{G}{K}
	\end{equation*}
\end{esercizio}
\begin{figure}[tp]
	\centering
	\tikz {
		\node (a) at (2,4) {$G$};
		\node (b) at (0,2) {$H$};
		\node (d) at (4,2) {$K$};
		\node (e) at (2,0) {$H \cap K$};
		\draw (a) edge[->] (b) 
		(a) edge[->] (d) 
		(b) edge[->] (e)
		(d) edge[->] (e);
	}
	\caption{Intersezione di sottogruppi.}
	\label{fig:Laterali_intersezione_di_sottogruppi}
\end{figure}
\begin{soluzione}
	Siano $h_1, h_2 \in H$ tali che:
	
	\begin{equation}
		h_1(H \cap K) \ne h_2(H \cap K)
	\end{equation}

	Ovvero $h_1$ e $h_2$ stanno in laterali dell'intersezione diversi.
	
	\begin{gather}
		h_2^{-1}h_1 \not\in H \cap K \\
		h_2^{-1}h_1 \not\in K \quad \text{perché } h_1, h_2 \in H
		h_1 K \ne h_2 K
	\end{gather}

	Quindi $h_1$ e $h_2$ stanno in laterali di $K$ diversi. Quindi:
	
	\begin{equation}
		\indice{H}{H \cap K} \le \indice{G}{K}
	\end{equation}

	Quindi, per il teorema~\ref{thr:fattorizzazione_indici}, si ha:
	
	\begin{gather}
		\indice{G}{H \cap K} = \indice{G}{H} \cdot \indice{H}{H \cap K} \\
		\indice{G}{H \cap K} \le \indice{G}{H} \cdot \indice{G}{K}
	\end{gather}

	L'indice dell'intersezione è minore o uguale al prodotto degli indici.
\end{soluzione}

In generale l'uguaglianza non c'è, vedi per esempio la figura~\ref{fig:Laterali_intersezione_di_sottogruppi_esempio_s3}.

\begin{figure}[tp]
	\centering
	\tikz {
		\node (a) at (2,4) {$G = S_3$};
		\node (b) at (0,2) {$H = \gen{(12)}$};
		\node (d) at (4,2) {$K = \gen{(13)}$};
		\node (e) at (2,0) {$H \cap K = 1$};
		\draw (a) edge[->] (b); 
		\draw (a) edge[->] (d);
		\draw (b) edge[->] (e);
		\draw (d) edge[->] (e);
		\draw[decorate,decoration={brace,raise=40pt}] (0,2) -- (0,4) node[pos=.5,right=-55pt,black]{3};
		\draw[decorate,decoration={brace,raise=40pt}] (4,4) -- (4,2) node[pos=.5,right=43pt,black]{3};
		\draw[decorate,decoration={brace,raise=60pt}] (4,4) -- (4,0) node[pos=.5,right=63pt,black]{6};
	}
	\caption{Intersezione di sottogruppi: esempio con $S_3$.}
	\label{fig:Laterali_intersezione_di_sottogruppi_esempio_s3}
\end{figure}

\begin{teorema}
	Se $\indice{G}{H}$ e $\indice{G}{K}$ sono coprimi, allora
	
	\begin{equation}
		\indice{G}{H \cap K} = \indice{G}{H} \cdot \indice{G}{K}
	\end{equation}
\end{teorema}
\begin{dimostrazione}
	Per il teorema~\ref{thr:fattorizzazione_indici} si ha:
	
	\begin{gather}
		\indice{G}{H} \text{ divide } \indice{G}{H \cap K} \\
		\indice{G}{K} \text{ divide } \indice{G}{H \cap K}
	\end{gather}	

	Quindi:
	
	\begin{gather}
		\indice{G}{H} \cdot \indice{G}{K} \text{ divide } \indice{G}{H \cap K} \\
		\indice{G}{H} \cdot \indice{G}{K} \le \indice{G}{H \cap K}
	\end{gather}

	Ma nell'esercizio precedente abbiamo verificato che:
	
	\begin{equation}
		\indice{G}{H} \cdot \indice{G}{K} \ge \indice{G}{H \cap K}
	\end{equation}

	Quindi:
	
	\begin{equation}
		\indice{G}{H} \cdot \indice{G}{K} = \indice{G}{H \cap K}
	\end{equation}
\end{dimostrazione}

\section{Prodotto di sottogruppi}

\begin{teorema}
	\label{thr:prodotto_sottogruppi}
	Siano $H$ e $K$ due sottogruppi di $G$. Definiamo:
	
	\begin{equation}
		HK = \{hk \taleche h \in H, k \in K\}
	\end{equation}

	Si ha:

	\begin{equation}
		HK \le G \Longleftrightarrow HK = KH
	\end{equation}

	In particolare se $N$ è un sottogruppo normale di $G$, allora $NH = HN$ è un sottogruppo di $G$ per ogni $H \le G$.

\end{teorema}
\begin{dimostrazione}
	Partiamo da un controesempio. Se prendiamo i seguenti sottogruppi di $S_3$:
	
	\begin{equation}
		H = \{1, (12)\}, K = \{1, (13)\}
	\end{equation}
	
	otteniamo il prodotto di sottogruppi:
	
	\begin{equation*}
		HK = \{1, (12), (13), (123)\}
	\end{equation*}
	
	Questo insieme ha cardinalità 4, che non è un divisore dell'ordine di $S_3$ (che è 6), quindi,
	per il  teorema di Lagrange~\ref{thr:Laterali_Lagrange}, non è un sottogruppo di $S_3$.
	
	Supponiamo ora che $HK \le G$: allora si ha:
	
	\begin{equation*}
		HK = (HK)^{-1} = K^{-1}H^{-1} = KH
	\end{equation*}
	
	E questo dimostra la prima parte del teorema.
	
	Inoltre se $N$ è un sottogruppo normale si ha:
	
	\begin{equation*}
		nh = hh^{-1}nh = h(h^{-1}nh) = hn^* \quad \text{con } n^* \in N
	\end{equation*}
\end{dimostrazione}

\begin{teorema}
	\label{thr:ordine_prodotto_sottogruppi}
	Se $H$ e $K$ sono finiti, allora:

	\begin{equation*}
		\ordine{HK} = \dfrac{\ordine{H}\ordine{K}}{\ordine{H \cap K}}
	\end{equation*}
\end{teorema}

\begin{esercizio}
	Sia $G$ è un gruppo, $H \le G$, $N \normale G$ e:
	
	\begin{gather}
		HN = G \\
		H \cap N = 1
	\end{gather}

	Ovvero $H$ è il complementare di $N$.
	
	Esiste una funzione $\gamma: H \times N \longrightarrow G$ biiettiva, ma tale funzione non è un isomorfismo.
\end{esercizio}
\begin{soluzione}
	Se $HN = G$ significa che:
	
	\begin{equation*}
		\forall g \in G \, \exists h \in H, k \in K \taleche g = hn
	\end{equation*}

	Presi due elementi $h_1, h_2 \in H$ e due elementi $n_1, n_2 \in N$:
	
	\begin{equation*}
		h_1n_1 = h_2n_2 \Longrightarrow h_2^{-1}h_1 = n_2n_1^{-1}
	\end{equation*}

	Ma $h_2^{-1}h_1$ è un elemento di $H$ e $n_2n_1^{-1}$ è un elemento di $N$. E dal momento che questi due elementi sono in verità lo stesso elemento, allora tale valore sta anche in $H \cap N$. Ma nell'intersezione c'è solo l'unità, quindi:
	
	\begin{gather*}
		h_2^{-1}h_1 = 1 \Longrightarrow h_1 = h_2 \\
		n_2 n_1^{-1} = 1 \Longrightarrow n_1 = n_2
	\end{gather*}

	Questo significa che il prodotto $g = hn$ è unico per ogni elemento $g$.
	
	Quindi la funzione:
	
	\begin{align}
		\gamma: H \times N &\longrightarrow G \\
		(h, n) &\longmapsto hn
	\end{align}

	è suriettiva e iniettiva, quindi è biiettiva.
	
	Tuttavia non è garantito che $\gamma$ sia un isomorfismo.
	Infatti $G$ potrebbe non essere abeliano, $H$ e $K$ potrebbero essere abeliani quindi anche il prodotto $H \times K$ sarebbe abeliano.
	
	Per esempio se considero:
	\begin{gather*}
		G = S_3 \text{ non abeliano} \\
		H = \gen{(12)} \text{ ciclico quindi abeliano} \\
		N = \gen{(123)} \text{ ciclico quindi abeliano}
	\end{gather*}

	abbiamo che:
	\begin{gather*}
		HN = S_3 \\
		H \cap N = 1
	\end{gather*}

	Ma $H \times N$ è abeliano (perché prodotto diretto di gruppi abeliani), mentre $G$ non lo è.
	
	Questo significa che la funzione non rispetta l'operazione di composizione.
	Per esempio:
	
	\begin{gather}
		((12), (123)) \cdot ((12), 1) = (1, (123)) \\
		((12), 1) \cdot ((12), (123)) = (1, (123))
	\end{gather}

	Quindi la composizione di questi elementi del prodotto diretto è commutativa.
	In genere all'interno del prodotto diretto la composizione è commutativa perché ho sempre composizioni con l'unità
	o tra due elementi che sono inversi.
	
	Se passo alle loro immagini ho:
	
	\begin{gather}
		((12), (123)) \longmapsto (12) \cdot (123) = (23) \\
		((12), 1) \longmapsto (12) \cdot 1 = (12)
	\end{gather}

	E:
	
	\begin{gather}
		(12) \cdot (23) = (123) \\
		(23) \cdot (12) = (132)
	\end{gather}

	Quindi tra questi due elementi di $S_3$ il prodotto non è commutativo.
	
\end{soluzione}

\section{Sottogruppi di indice 2}

\begin{teorema}
	\label{thr:sottogruppi_di_indice_2}
	Si $H$ un sottogruppo di indice 2 di un gruppo $G$. Allora $H$ è normale.
\end{teorema}
\begin{dimostrazione}
	Dal momento che l'indice di $H$ è pari a 2, ogni trasversale sinistro contiene due elementi, quindi ci sono due laterali sinistri: $H$ e $xH$.
	
	Questi due insiemi sono disgiunti perché:
	
	\begin{equation}
		h_1x = h_2 \Longrightarrow x = h_1^{-1}h_2 \in H
	\end{equation}

	Quindi:
	
	\begin{equation}
		G = H \overset{\circ}{\cup} xH
	\end{equation}

	Ma lo stesso ragionamento lo possiamo fare anche con i laterali destri e ottenere:
	
	\begin{equation}
		G = H \overset{\circ}{\cup} Hx
	\end{equation}

	Da questo deduciamo che $xH$ e $Hx$ sono lo stesso insieme, quindi $H$ è normale.
\end{dimostrazione}