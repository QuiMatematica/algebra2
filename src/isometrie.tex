\chapter{Isometrie del piano}
\label{cpt:Isometrie}

\section[Isometrie del piano]{Isometrie del piano\footnote{\cite[7 novembre 2021]{lucchini}}}
Le isometrie sono le trasformazioni del piano che preservano le distanze. A noi interessano quelle che fissano l'origine.

E' immediato pensare che le isometrie rappresentano un sottogruppo delle trasformazioni del piano.

Le isometrie che fissano l'origine sono:

\begin{itemize}
	\item le rotazioni di un angolo $\alpha$, che indichiamo con $\rho_\alpha$ (figura~\ref{fig:Isometrie_Rotazione});
	\item le riflessioni attorno ad una retta che forma un angolo $\alpha$ con l'asse delle ascisse, che indichiamo con $\iota_\alpha$ (figura~\ref{fig:Isometrie_Riflessione}).
\end{itemize}

\section[Isometrie come matrici]{Isometrie come matrici\footnote{Approfondimento personale}}

Prendiamo sul piano cartesiano un generico punto $P(x; y)$ e facciamolo ruotare attorno all'origine del piano di un angolo $\alpha$. Otteniamo così un nuovo punto $P(x'; y')$ (figura~\ref{fig:Isometrie_Rotazione}).

\begin{figure}[tp]
	\centering
	\begin{tikzpicture}[line cap=round,line join=round,>=triangle 45,x=1cm,y=1cm]
		\begin{axis}[
			x=1cm,y=1cm,
			axis lines=middle,
			xmin=-1,
			xmax=5,
			ymin=-1,
			ymax=5,
			xtick={-10,11},
			ytick={-7,10},]
			\clip(-10.72,-7.2) rectangle (11.92,10.32);
			\draw [shift={(-1,-1)},line width=1pt,color=green,fill=green,fill opacity=0.1] (0,0) -- (0:1.3) arc (0:26.56505117707799:1.3) -- cycle;
			\draw [shift={(-1,-1)},line width=1pt,color=green,fill=green,fill opacity=0.1] (0,0) -- (26.56505117707799:1.5) arc (26.56505117707799:59.56505117707799:1.5) -- cycle;
			\draw [shift={(-1,-1)},line width=1pt,color=green] (0:1.3) arc (0:26.56505117707799:1.3);
			\draw [shift={(-1,-1)},line width=1pt,color=green] (0:1.17) arc (0:26.56505117707799:1.17);
			\draw [line width=1pt,color=gray] (0,0)-- (4,2);
			\draw [line width=1pt,color=gray] (0,0)-- (2.2654042017516423,3.8558972759509564);
			\begin{scriptsize}
				\draw [fill=blue] (4,2) circle (2.5pt);
				\draw[] (4.3,2.4) node {$P(x; y)$};
				\draw[] (-0.32,0.35) node {$O$};
				\draw[] (1.6,0.4) node {$\theta$};
				\draw [fill=blue] (2.2654042017516423,3.8558972759509564) circle (2.5pt);
				\draw[] (2.6,4.2) node {$P'(x'; y')$};
				\draw[] (1.35,1.2) node {$\alpha$};
			\end{scriptsize}
		\end{axis}
	\end{tikzpicture}
	\caption{Rotazione}
\label{fig:Isometrie_Rotazione}
\end{figure}

Utilizziamo le formule goniometriche che ci permettono di passare dalle coordinate cartesiane alle coordinate polari.

I \emph{moduli} dei punti $P$ e $P'$, ovvero le loro distanze dall'origine, sono uguali. Quindi possiamo scrivere $\overline{OP}$ per intendere indifferentemente i moduli dei due punti.

Chiamiamo $\theta$ l'\emph{argomento} del punto $P$. Se questo punto viene fatto ruotare di un angolo $\alpha$, l'argomento di $P'$ è $\theta + \alpha$.

Le coordinate del punto $P$ possono essere espresse come:

\begin{equation}
x = \overline{OP} \cos \theta
\end{equation}
\begin{equation}
y = \overline{OP} \sin \theta
\end{equation}

Calcoliamo l'ascissa del punto $P'$:

\begin{align}
	x' & = \overline{OP} \cos (\theta + \alpha) = \\
	& = \overline{OP} ( \cos \theta \cos \alpha - \sin \theta \sin \alpha) = \\
	& = \overline{OP} \cos \theta \cos \alpha - \overline{OP} \sin \theta \sin \alpha = \\
	& = x \cos \alpha - y \sin \alpha
\end{align}
 
Calcoliamo l'ordinata del punto $P'$:

\begin{align}
	y' & = \overline{OP} \sin (\theta + \alpha) = \\
	& = \overline{OP} ( \sin \theta \cos \alpha + \cos \theta \sin \alpha) = \\
	& = \overline{OP} \sin \theta \cos \alpha + \overline{OP} \cos \theta \sin \alpha = \\
	& = y \cos \alpha + x \sin \alpha
\end{align}

Quindi possiamo passare dalle coordinate di $P(x; y)$ alle coordinate $P'(x', y')$ con queste equazioni:

\begin{equation}
	\begin{cases}
		x' = x \cos \alpha - y \sin \alpha \\
		y' = y \cos \alpha + x \sin \alpha
	\end{cases}
\end{equation}

E queste equazioni possono essere sintetizzate come prodotto matriciale:

\begin{equation}
	\begin{bmatrix}
		x' \\
		y' 
	\end{bmatrix}
	=
	\begin{bmatrix}
		\cos \alpha & -\sin \alpha \\
		\sin \alpha & \cos \alpha
	\end{bmatrix}
	\begin{bmatrix}
		x \\
		y 
	\end{bmatrix}
\end{equation}

Chiamiamo $\rho_\alpha$ la matrice dell'isometria:

\begin{equation}
	\rho_\alpha =
	\begin{bmatrix}
		\cos \alpha & -\sin \alpha \\
		\sin \alpha & \cos \alpha
	\end{bmatrix}
\end{equation}

Consideriamo ora una riflessione attorno ad una retta $r$ che forma un angolo $\alpha$ con l'asse delle ascisse. Prendiamo sul piano cartesiano un generico punto $P(x; y)$ e facciamolo riflettere nel nuovo punto $P(x'; y')$ (figura~\ref{fig:Isometrie_Riflessione}).

\begin{figure}[tp]
	\centering
	\begin{tikzpicture}[line cap=round,line join=round,>=triangle 45,x=1cm,y=1cm]
		\begin{axis}[
			x=1cm,y=1cm,
			axis lines=middle,
			xmin=-1,
			xmax=5,
			ymin=-1,
			ymax=5,
			xtick={-10,11},
			ytick={-7,10},]
			\clip(-10.72,-7.2) rectangle (11.92,10.32);
			\draw [shift={(-1,-1)},line width=1pt,color=green,fill=green,fill opacity=0.1] (0,0) -- (0:1.3) arc (0:26.56505117707799:1.3) -- cycle;
			\draw [shift={(-1,-1)},line width=1pt,color=green,fill=green,fill opacity=0.1] (0,0) -- (0:2) arc (0:43:2) -- cycle;
			\draw [shift={(-1,-1)},line width=1pt,color=green] (0:1.3) arc (0:26.56505117707799:1.3);
			\draw [shift={(-1,-1)},line width=1pt,color=green] (0:1.17) arc (0:26.56505117707799:1.17);
			\draw [line width=1pt,color=gray] (0,0)-- (4,2);
			\draw [line width=1pt,color=gray,domain=-10.72:11.92] plot(\x,{(-0--3.4099918003124925*\x)/3.656768508095853});
			\draw [line width=1pt,color=gray] (0,0)-- (2.27415399549615,3.850743253551046);
			\draw [line width=1pt,color=gray,dash pattern=on 3pt off 3pt] (2.27415399549615,3.850743253551046)-- (4,2);
			\begin{scriptsize}
				\draw [fill=blue] (4,2) circle (2.5pt);
				\draw[] (4.3,2.4) node {$P(x; y)$};
				\draw [] (0,0) circle (2pt);
				\draw[] (-0.32,0.35) node {$O$};
				\draw[] (1.46,0.3) node {$\theta$};
				\draw[] (2.2,0.75) node {$\alpha$};
				\draw [fill=blue] (2.27415399549615,3.850743253551046) circle (2.5pt);
				\draw[] (2.3,4.29) node {$P'(x'; y')$};
			\end{scriptsize}
		\end{axis}
	\end{tikzpicture}
	\caption{Rotazione}
	\label{fig:Isometrie_Riflessione}
\end{figure}

Anche in questo caso utilizziamo le formule goniometriche che ci permettono di passare dalle coordinate cartesiane alle coordinate polari.

Indichiamo con $\overline{OP}$ i \emph{moduli} dei punti $P$ e $P'$.

Chiamiamo $\theta$ l'\emph{argomento} del punto $P$. Per calcolare l'argomento di $P'$ consideriamo che:

\begin{align}
	\widehat{P'Ox} & = \widehat{P'Or} + \alpha = \\
	& = \widehat{POr} + \alpha = \\
	& = (\alpha - \theta) + \alpha = \\
	& = 2\alpha - \theta
\end{align}

Calcoliamo l'ascissa del punto $P'$:

\begin{align}
	x' & = \overline{OP} \cos (2\alpha - \theta) = \\
	& = \overline{OP} ( \cos 2\alpha \cos \theta + \sin 2\alpha \sin \theta) = \\
	& = \overline{OP} \cos 2\alpha \cos \theta + \overline{OP} \sin 2\alpha \sin \theta = \\
	& = x \cos 2\alpha + y \sin 2\alpha
\end{align}

Calcoliamo l'ordinata del punto $P'$:

\begin{align}
	y' & = \overline{OP} \sin (2\alpha - \theta) = \\
	& = \overline{OP} ( \sin 2\alpha \cos \theta - \cos 2\alpha \sin \theta) = \\
	& = \overline{OP}  \sin 2\alpha \cos \theta + \overline{OP} \cos 2\alpha \sin \theta = \\
	& = x \sin 2\alpha - y \cos 2\alpha
\end{align}

Quindi possiamo passare dalle coordinate di $P(x; y)$ alle coordinate $P'(x', y')$ con queste equazioni:

\begin{equation}
	\begin{cases}
		x' = x \cos 2\alpha + y \sin 2\alpha \\
		y' = x \cos 2\alpha - y \sin 2\alpha
	\end{cases}
\end{equation}

E queste equazioni possono essere sintetizzate come prodotto matriciale:

\begin{equation}
	\begin{bmatrix}
		x' \\
		y' 
	\end{bmatrix}
	=
	\begin{bmatrix}
		\cos 2\alpha & \sin 2\alpha \\
		\sin 2\alpha & -\cos 2\alpha
	\end{bmatrix}
	\begin{bmatrix}
		x \\
		y 
	\end{bmatrix}
\end{equation}

Chiamiamo $\iota_\alpha$ la matrice dell'isometria:

\begin{equation}
	\iota_\alpha =
	\begin{bmatrix}
		\cos 2\alpha & \sin 2\alpha \\
		\sin 2\alpha & -\cos 2\alpha
	\end{bmatrix}
\end{equation}

\section[Il gruppo delle isometrie]{Il gruppo delle isometrie\footnote{\cite[7 novembre 2021]{lucchini}}}

Possiamo ora verifica che le isometrie formano un gruppo in quanto l'operazione di composizione è interna:

\begin{align}
	\rho_\alpha \rho_\beta & = 
	\begin{bmatrix}
		\cos \alpha & -\sin \alpha \\
		\sin \alpha & \cos \alpha
	\end{bmatrix}
	\begin{bmatrix}
		\cos \beta & -\sin \beta \\
		\sin \beta & \cos \beta
	\end{bmatrix} 
	= \notag \\
	& = 
	\begin{bmatrix}
		\cos \alpha \cos \beta -\sin \alpha \sin \beta & - \cos \alpha \sin \beta - \sin \alpha \cos \beta \\
		\sin \alpha \cos \beta + \cos \alpha \sin \beta & - \sin \alpha \sin \beta + \cos \alpha \cos \beta
	\end{bmatrix} 
	= \notag \\
	& = 
	\begin{bmatrix}
		\cos (\alpha + \beta) & - \sin (\alpha + \beta) \\
		\sin (\alpha + \beta) & \cos (\alpha + \beta)
	\end{bmatrix} 
	= \notag \\
	\label{eqn:Isometrie_R_R}
	& = \rho_{\alpha + \beta}
\end{align}

\begin{align}
	\iota_\alpha \iota_\beta & = 
	\begin{bmatrix}
		\cos 2\alpha & \sin 2\alpha \\
		\sin 2\alpha & -\cos 2\alpha
	\end{bmatrix}
	\begin{bmatrix}
		\cos 2\beta & \sin 2\beta \\
		\sin 2\beta & -\cos 2\beta
	\end{bmatrix} 
	= \notag \\
	& = 
	\begin{bmatrix}
		\cos 2\alpha \cos 2\beta +\sin 2\alpha \sin 2\beta & \cos 2\alpha \sin 2\beta - \sin 2\alpha \cos 2\beta \\
		\sin 2\alpha \cos 2\beta - \cos 2\alpha \sin 2\beta & \sin 2\alpha \sin 2\beta + \cos 2\alpha \cos 2\beta
	\end{bmatrix} 
	= \notag \\
	& = 
	\begin{bmatrix}
		\cos (2\alpha - 2\beta) & - \sin (2\alpha - 2\beta) \\
		\sin (2\alpha - 2\beta) & \cos (2\alpha - 2\beta)
	\end{bmatrix} 
	= \notag \\
	\label{eqn:Isometrie_I_I}
	& = \rho_{2\alpha - 2\beta}
\end{align}

\begin{align}
	\rho_\alpha \iota_\beta & = 
	\begin{bmatrix}
		\cos \alpha & -\sin \alpha \\
		\sin \alpha & \cos \alpha
	\end{bmatrix}
	\begin{bmatrix}
		\cos 2\beta & \sin 2\beta \\
		\sin 2\beta & -\cos 2\beta
	\end{bmatrix} 
	= \notag \\
	& = 
	\begin{bmatrix}
		\cos \alpha \cos 2\beta -\sin \alpha \sin 2\beta & \cos \alpha \sin 2\beta + \sin \alpha \cos 2\beta \\
		\sin \alpha \cos 2\beta + \cos \alpha \sin 2\beta & \sin \alpha \sin 2\beta - \cos \alpha \cos 2\beta
	\end{bmatrix} 
	= \notag \\
	& = 
	\begin{bmatrix}
		\cos (\alpha + 2\beta) & \sin (\alpha + 2\beta) \\
		\sin (\alpha + 2\beta) & - \cos (\alpha + 2\beta)
	\end{bmatrix} 
	= \notag \\
	\label{eqn:Isometrie_R_I}
	& = \iota_{\frac{\alpha}{2} + \beta}
\end{align}

\begin{align}
	\iota_\alpha \rho_\beta & = 
	\begin{bmatrix}
		\cos 2\alpha & \sin 2\alpha \\
		\sin 2\alpha & -\cos 2\alpha
	\end{bmatrix}
	\begin{bmatrix}
		\cos \beta & -\sin \beta \\
		\sin \beta & \cos \beta
	\end{bmatrix} 
	= \notag \\
	& = 
	\begin{bmatrix}
		\cos 2\alpha \cos \beta +\sin 2\alpha \sin \beta & - \cos 2\alpha \sin \beta + \sin 2\alpha \cos \beta \\
		\sin 2\alpha \cos \beta - \cos 2\alpha \sin \beta & - \sin 2\alpha \sin \beta - \cos 2\alpha \cos \beta
	\end{bmatrix} 
	= \notag \\
	& = 
	\begin{bmatrix}
		\cos (2\alpha - \beta) & \sin (2\alpha - \beta) \\
		\sin (2\alpha - \beta) & -\cos (2\alpha - \beta)
	\end{bmatrix} 
	= \notag \\
	\label{eqn:Isometrie_I_R}
	& = \iota_{\alpha-\frac{\beta}{2}}
\end{align}

Se quindi chiamo $R$ l'insieme di tutte le rotazioni e $I$ l'insieme di tutte le riflessioni ottengo:

\begin{equation}
	G = R \cup I
\end{equation}

E $G$ è un gruppo in quanto la composizioni delle isometrie è ancora un'isometria.

Inoltre dalla \eqref{eqn:Isometrie_R_R}, visto che la composizione di due rotazioni è ancora una rotazione, possiamo affermare che $R$ è un sottogruppo di $G$:

\begin{equation}
	R \le G
\end{equation}

Prendiamo ora una riflessione privilegiata, ovvero quella rispetto all'asse delle ascisse:

\begin{equation}
	\iota = \iota_0
\end{equation}

Dalla \eqref{eqn:Isometrie_R_I} ricaviamo che:

\begin{equation}
	\rho_\alpha \iota = \iota_{\frac{\alpha}{2}}
\end{equation}

Quindi attraverso la composizione con $\iota$, ad ogni rotazione corrisponde una riflessione con angolo dimezzato e, viceversa, ad ogni riflessione corrisponde una rotazione con angolo doppio. Quindi abbiamo tante riflessioni quante rotazioni, e non ci sono rotazioni che sono alche riflessioni nè viceversa.

Quindi possiamo affermare che possiamo ottenere l'insieme delle riflessioni come prodotto tra l'insieme delle rotazioni e $\iota$:

\begin{equation}
	I = R\iota
\end{equation}

Inoltre l'unione tra $R$ e $I$ che definisce il nostro gruppo $G$ è un'unione disgiunta a meno dell'unità:

\begin{equation}
	G = R \overset{\circ}{\cup} R\iota
\end{equation}

Inoltre avendo tante rotazioni quante riflessioni, risulta che $R$ è un sottogruppo di \emph{indice} 2.

\begin{equation}
	|G:R| = 2
\end{equation}

Per il teorema \textbf{TODO} i sottogruppi di indice 2 sono normali, quindi $R$ è un sottogruppo normale di $G$.

\begin{equation}
	R \unlhd G
\end{equation}

Dalle formule \eqref{eqn:Isometrie_I_R} e \eqref{eqn:Isometrie_R_I} capiamo che se compongo una riflessione con una rotazione ottengo una riflessione, e se compongo questa con un'altra riflessione ottengo una rotazione. Quindi se coniugo una rotazione con una riflessione ottengo uno rotazione.
Inoltre nel coniugare posso considerare che l'inversa di una riflessione è sempre la riflessione stessa, quindi posso coniugare semplicemente con la composizione:

\begin{equation}
	\iota_\beta \rho_\alpha \iota_\beta
\end{equation}

Andiamo a calcolare il risultato di questo coniugio:

\begin{align}
	\iota_\beta \rho_\alpha \iota_\beta 
	& = \iota_{\beta-\frac{\alpha}{2}} \iota_\beta = \notag \\
	& = \rho_{2(\beta-\frac{\alpha}{2}) - 2\beta} = \notag \\
	& = \rho_{2\beta-\alpha - 2\beta} = \notag \\
	\label{eqn:Isometrie_coniugio_rho}
	& = \rho_{-\alpha}
\end{align}

Ora posso dimenticare tutte le formule di prima e ricordare solo la \eqref{eqn:Isometrie_coniugio_rho}.

Il gruppo $G$ degli isomorfismi ha quindi queste caratteristiche:

\begin{equation}
	R \unlhd G
\end{equation}
\begin{equation}
	R\iota \le G
\end{equation}
\begin{equation}
	R \cap R\iota = 1
\end{equation}

Quindi per il teorema \textbf{TODO} il gruppo $G$ si può ottenere come prodotto dei sottogruppi $R$ e $R\iota$:

\begin{equation}
	G = R \, R\iota
\end{equation}

Quindi ogni elemento di $G$ si scrive in un solo modo: come prodotto di una rotazione che sta in $R$ con una riflessione che sta in $R\iota$.

\begin{equation}
	G = R\gen{\iota}
\end{equation}

Ogni elemento può essere scritto nella forma:

\begin{equation}
	g = \rho_\alpha \iota^\epsilon \quad\text{ con } \epsilon \in \{0,1\}
\end{equation}

Verifichiamo cosa succede se moltiplico due isometrie scritte in questo modo. Ne prendo 2 qualsiasi:

\begin{equation}
	g_1 = \rho_{\alpha_1} \iota^{\epsilon_1}
\end{equation}
\begin{equation}
	g_2 = \rho_{\alpha_2} \iota^{\epsilon_2}
\end{equation}

E li moltiplico:

\begin{align}
	g_1g_2 
	& = \rho_{\alpha_1} \iota^{\epsilon_1} \rho_{\alpha_2} \iota^{\epsilon_2} \notag \\
	& = \rho_{\alpha_1} \iota^{\epsilon_1} \rho_{\alpha_2}  \iota^{\epsilon_1}  \iota^{\epsilon_1} \iota^{\epsilon_2} \notag \\
	& = \rho_{\alpha_1} (\iota^{\epsilon_1} \rho_{\alpha_2}  \iota^{\epsilon_1})  \iota^{\epsilon_1} \iota^{\epsilon_2} \notag 
\end{align}

Si presentano quindi due casi

\begin{equation}
	g_1g_2 =
	\begin{cases}
		\rho_{\alpha_1} \rho_{\alpha_2} \iota^{\epsilon_2} = \rho_{\alpha_1 + \alpha_2} \iota^{\epsilon_2} & \text{se } \epsilon_1 = 0 \\
		\rho_{\alpha_1} \rho_{-\alpha_2} \iota \iota^{\epsilon_2} = \rho_{\alpha_1 - \alpha_2} \iota^{\epsilon_2 + 1} & \text{se } \epsilon_1 = 1 		
	\end{cases}
\end{equation}

\section[Ordine degli elementi]{Ordine degli elementi\protect\footnote{\cite[7 novembre 2021]{lucchini}}}

Per descrivere il sottogruppo $R$ utilizziamo il toro:

\begin{equation}
	T = \{c \in \mathbb{C} \taleche \modulo{c} = 1\}
\end{equation}

Consideriamo la mappa:

\begin{equation}
	R \longrightarrow T: \rho_\alpha \longmapsto e^{i\alpha}
\end{equation}

E' un isomorfismo perché:

\begin{equation}
	\rho_\alpha \rho_\beta = e^{i\alpha} e^{i\beta} = e^{i(\alpha + \beta)} = \rho_{\alpha + \beta}
\end{equation}

Quindi:

\begin{equation}
	R \isomorfo T
\end{equation}

Qual è l'ordine di una rotazione?

\begin{align}
	(\rho_\alpha)^n = 1 & \sse (e^{i\alpha})^n = 1 \\
	& \sse \alpha n = k\, 2 \pi \text{ con } k \in \Z \\
	& \sse \alpha = \dfrac{k}{n}\,2\pi \text{ con } k \in \Z \\
	& \sse \alpha \in \Q \, 2\pi
\end{align}

Per cui la torsione del sottogruppo $R$ è:

\begin{equation}
	\label{eqn:isometrie_T_di_R}
	T(R) = \{\rho_\alpha \taleche \alpha \in \Q\,2\pi\}
\end{equation}

Ovvero l'immagine tramite l'isomorfismo di $\rho_\alpha$ è una radice $n$-esima di 1 per un $n$ opportuno.

Al contrario:

\begin{equation}
	\ordine{\rho_\alpha} = \infty \sse \alpha \not\in \Q\,2\pi
\end{equation}

Quindi se voglio un elemento di ordine infinito, mi basta sceglierlo con $\alpha \not\in \Q\,2\pi$.

Dalla \eqref{eqn:isometrie_T_di_R} risulta che:

\begin{equation}
	\alpha = \dfrac{2\pi}{n} \Longrightarrow \ordine{\rho_\alpha} = n
\end{equation}

Quindi qualunque sia l'ordine che mi interessa, trovo sempre un elemento di quell'ordine che, nel toro, corrisponde ad una radice $n$-esima di 1.

Se ora prendiamo una rotazione $\rho_\alpha$ qualsiasi e la componiamo con $\iota$ otteniamo una riflessione, e tutte le riflessioni hanno ordine 2 in quanto sono l'inverso di se stesse. Ma se noi componiamo due riflessioni:

\begin{equation}
	\iota \cdot \iota\rho_\alpha = \rho_\alpha 
\end{equation}

L'ordine di $\rho_\alpha$ dipende dall'angolo $\alpha$. Quindi \emph{il prodotto di due elementi di ordine 2 può avere ordine infinito o un qualunque altro ordine si voglia}.

\section[Centro del gruppo delle isometrie]{Centro del gruppo delle isometrie\protect\footnote{\cite[7 novembre 2021]{lucchini}}}

Cerchiamo gli elementi $g \in Z(G)$: questi elementi possono essere riflessioni o rotazioni. Divido il problema in 2 parti.

\textbf{Caso 1: $g \in \iota R$}: è una riflessione

Se $g$ è una riflessione allora:

\begin{equation}
	g \rho_\alpha g = \rho_{-\alpha} \Longrightarrow g \rho_\alpha = \rho_{-\alpha} g
\end{equation}

Ma se $g$ è nel centro, allora deve commutare con tutti gli elementi di $G$, in particolare:

\begin{equation}
	g\rho_\alpha = \rho_\alpha g
\end{equation}

Quindi deve risultare:

\begin{gather}
	\rho_{-\alpha} = \rho_\alpha \\
	\alpha \in \{0, \pi\}
\end{gather}

Quindi basta scegliere un $\alpha \not\in \{0, \pi\}$ e ho un elemento che non commuta con $g$. Quindi \emph{le riflessioni non appartengono al centro}.

\textbf{Caso 2: $g \in R$}: è una rotazione.

Una rotazione commuta con tutte le altre rotazioni. Devo però trovare le rotazioni che commutano anche con le riflessioni. Chiamando $g = \rho_\alpha$ rifaccio il ragionamento di prima:

\begin{gather}
	\iota \rho_\alpha \iota = \rho_{-\alpha} \Longrightarrow \iota \rho_\alpha = \rho_{-\alpha} \iota \\
	\rho_\alpha \in Z(G) \Longrightarrow \iota\rho_\alpha = \rho_\alpha \iota \\
	\rho_{-\alpha} = \rho_\alpha \\
	\alpha \in \{0, \pi\}
\end{gather}

Quindi gli unici elementi che stanno in $Z(G)$ sono l'identità e la rotazione $\rho_\pi$:

\begin{equation}
	\label{eqn:Isometrie_Z_G}
	Z(G) = \gen{\rho_\pi} \isomorfo C_2
\end{equation}

\section[Gruppi diedrali]{Gruppi diedrali\protect\footnote{\cite[7 novembre 2021]{lucchini}}}

Prendiamo un pentagono regolare e costruiamo il gruppo delle isometrie che mandano il pentagono in se stesso (figura~\ref{fig:Isometrie_Pentagono}).
\footnote{
	Solitamente si disegna il poligono con un vertice sul semiasse positivo delle ascisse (e il centro nell'origine degli assi). In questo caso la riflessione "base" $\iota$ rappresenta una riflessione che tiene fermo tale vertice e scambia quelli del primo e secondo quadrante con quelli del terzo e quarto quadrante.
	
	Tuttavia è possibile disegnare il poligono anche in altre posizioni.
	Se $n$ è dispari, si può disegnare il poligono con un vertice sul semiasse negativo delle ascisse. Mentre, se $n$ è dispari, si può disegnare il poligono ovviamente simmetrico rispetto all'asse delle ascisse ma con i lati che tagliano l'asse delle $x$. In questo caso la riflessione $\iota$ va a scambiare tutti i vertici: quelli di ordinata positiva con quelli di ordinata negativa.
}

\begin{figure}[tp]
	\centering
	\begin{tikzpicture}[line cap=round,line join=round,>=triangle 45,x=1cm,y=1cm]
		\begin{axis}[
			x=2cm,y=2cm,
			axis lines=middle,
			xmin=-2.5,
			xmax=2.5,
			ymin=-1.5,
			ymax=1.5,
			xtick={-3,3},
			ytick={-2,2},]
			\clip(-2.5,-1.5) rectangle (2.5,1.5);
			\fill[line width=2pt,color=figura,fill=figura,fill opacity=0.1] (1,0) -- (0.30901699437494745,0.9510565162951535) -- (-0.8090169943749471,0.5877852522924731) -- (-0.8090169943749472,-0.5877852522924729) -- (0.30901699437494723,-0.9510565162951534) -- cycle;
			%\draw [line width=2pt] (0,0) circle (2cm);
			\draw [line width=2pt,color=figura] (1,0)-- (0.30901699437494745,0.9510565162951535);
			\draw [line width=2pt,color=figura] (0.30901699437494745,0.9510565162951535)-- (-0.8090169943749471,0.5877852522924731);
			\draw [line width=2pt,color=figura] (-0.8090169943749471,0.5877852522924731)-- (-0.8090169943749472,-0.5877852522924729);
			\draw [line width=2pt,color=figura] (-0.8090169943749472,-0.5877852522924729)-- (0.30901699437494723,-0.9510565162951534);
			\draw [line width=2pt,color=figura] (0.30901699437494723,-0.9510565162951534)-- (1,0);
			\begin{scriptsize}
				\draw (2.4,-.1) node {$x$};
				\draw (-0.1,1.4) node {$y$};
			\end{scriptsize}
		\end{axis}
	\end{tikzpicture}
	\caption{Pentagono regolare di riferimento}
	\label{fig:Isometrie_Pentagono}
\end{figure}

Quali isometrie ci sono?

Abbiamo le rotazioni pari a $\frac{2\pi}{5}$, $2\frac{2\pi}{5}$, $\dots$, $4\frac{2\pi}{5}$:

\begin{gather}
	\rho = \rho_{\frac{2\pi}{5}} \\
	\ordine{\rho} = 5
\end{gather}

Inoltre abbiamo le riflessioni. Chiamiamo $\iota$ la riflessione lungo l'asse delle ascisse.

\begin{gather}
	D_5 = \gen{\rho, \iota} \\
	\ordine{D_5} = 10 \\
	D_5 \isomorfo \gen{\rho}\gen{\iota}
\end{gather}

Vale con qualunque poligono di $n$ lati: otteniamo il \textbf{gruppo diedrale} $D_n$, di grado $n$ e ordine $2n$ (attenzione alle notazioni che cambiano di autore in autore).

Possiamo battezzare i vertici del pentagono con i numeri (figura~\ref{fig:Isometrie_Pentagono_con_numeri}).

\begin{figure}[tp]
	\centering
	\begin{tikzpicture}[line cap=round,line join=round,>=triangle 45,x=1cm,y=1cm]
		\begin{axis}[
			x=2cm,y=2cm,
			axis lines=middle,
			xmin=-2.5,
			xmax=2.5,
			ymin=-1.5,
			ymax=1.5,
			xtick={-3,3},
			ytick={-2,2},]
			\clip(-2.5,-1.5) rectangle (2.5,1.5);
			\fill[line width=2pt,color=figura,fill=figura,fill opacity=0.1] (1,0) -- (0.30901699437494745,0.9510565162951535) -- (-0.8090169943749471,0.5877852522924731) -- (-0.8090169943749472,-0.5877852522924729) -- (0.30901699437494723,-0.9510565162951534) -- cycle;
			%\draw [line width=2pt] (0,0) circle (2cm);
			\draw [line width=2pt,color=figura] (1,0)-- (0.30901699437494745,0.9510565162951535);
			\draw [line width=2pt,color=figura] (0.30901699437494745,0.9510565162951535)-- (-0.8090169943749471,0.5877852522924731);
			\draw [line width=2pt,color=figura] (-0.8090169943749471,0.5877852522924731)-- (-0.8090169943749472,-0.5877852522924729);
			\draw [line width=2pt,color=figura] (-0.8090169943749472,-0.5877852522924729)-- (0.30901699437494723,-0.9510565162951534);
			\draw [line width=2pt,color=figura] (0.30901699437494723,-0.9510565162951534)-- (1,0);
			\begin{scriptsize}
				\draw (1.05,0.15) node {1};
				\draw (0.4,1.1) node {2};
				\draw (-0.9,0.7) node {3};
				\draw (-0.9,-0.7) node {4};
				\draw (0.4,-1.1) node {5};
				\draw (2.4,-.1) node {$x$};
				\draw (-0.1,1.4) node {$y$};
			\end{scriptsize}
		\end{axis}
	\end{tikzpicture}
	\caption{Pentagono regolare di riferimento con vertici numerati}
	\label{fig:Isometrie_Pentagono_con_numeri}
\end{figure}


Posso descrivere ogni isometria rispetto al comportamento sui vertici.

Se considero la rotazione di $\frac{2\pi}{5}$ ottengo la permutazione $(12345)$, mentre se considero la rotazione $\iota$ ottengo la permutazione $(25)(34)$. Quindi possiamo dire:

\begin{equation}
	D_5 \isomorfo \gen{(12345), (25)(34)} \le S_5
\end{equation}

\section[Centro dei gruppi diedrali]{Centro dei gruppi diedrali\footnote{\cite[8 novembre 2021]{lucchini}}}

\begin{esercizio}
	Descrivere $Z(D_n)$.
\end{esercizio}

	Ricordando~\eqref{eqn:Isometrie_Z_G} possiamo dire:
	
	\begin{equation}
		Z(D_n) \subseteq Z(G) = \gen{\rho_\pi}
	\end{equation}

	Ma $\rho_\pi$ potrebbe non appartenere a $D_n$.
	
	Se $\rho_\pi \in D_n$, allora in $D_n$ c'è una rotazione di ordine 2. Ma se $n$ è dispari non ci sono rotazioni di ordine 2. Quindi succede quando $n$ è pari:
	
	\begin{equation}
		\rho_\pi \in D_n \Longleftrightarrow n \text{ è pari}
	\end{equation}

\section[Sottogruppi normali dei gruppi diedrali]{Sottogruppi normali dei gruppi diedrali\footnote{\cite[8 novembre 2021]{lucchini}}}

\begin{esercizio}
	Descrivere i sottogruppi normali di $D_n$.
\end{esercizio}

	Dimostriamo innanzitutto che:
	
	\begin{equation}
		H \le \gen{\rho} \Longrightarrow H \normale D_n
	\end{equation}

	Prendiamo un elemento
	
	\begin{equation}
		g = \rho^\alpha \iota^\beta
	\end{equation}

	con $\alpha$ ridotto a modulo $n$ e $\beta$ ridotto a modulo 2.
	
	$H$ è sottogruppo di un gruppo ciclico, quindi è ciclico:
	
	\begin{equation}
		H = \gen{\rho^k}
	\end{equation}

	Facciamo il coniugio del generatore con l'elemento $g$:
	
	\begin{align}
		g \rho^k g^{-1} 
		& = \rho^\alpha \iota^\beta \rho^k (\rho^\alpha \iota^\beta)^{-1} \\
		& = \rho^\alpha \iota^\beta \rho^k \iota^\beta \rho^{-\alpha}
	\end{align}

	Il valore di $\iota^\beta \rho^k \iota^\beta$ dipende dal valore di $\beta$:
	
	\begin{equation}
		\iota^\beta \rho^k \iota^\beta = 
		\begin{cases}
			\rho^k & \text{se } \beta = 0 \\
			\rho^{-k} & \text{se } \beta = 1
		\end{cases}
	\end{equation}

	Quindi:
	
	\begin{equation}
		g \rho^k g^{-1} = \rho^\alpha \rho^{\pm k} \rho^{-\alpha} 
	\end{equation}

	E poiché le rotazioni commutano:
	
	\begin{equation}
		g \rho^k g^{-1} = \rho^{\pm k}
	\end{equation}
	
	Quindi se coniugo $\rho^k$ o trovo $\rho^k$ oppure trovo il suo inverso $\rho^{-k}$. Quindi $H$ è normale in $D_n$.
	
	Quindi tutti i sottogruppi di $\gen{\rho}$ sono normali.
	
	Verifichiamo ora se ce ne sono altri.
	
	Sia:
	
	\begin{equation}
		H \normale D_n \text{ con } H \not\subseteq \gen{\rho}
	\end{equation}

	Ovvero in $H$ c'è almeno una riflessione:
	
	\begin{equation}
		\exists k : \rho^k \iota = \iota^* \in H
	\end{equation}

	Consideriamo ora l'isometria $\rho \iota^* \rho^{-1} \iota^*$. La componente $\rho \iota^* \rho^{-1}$ appartiene ad $H$ perché $\iota^* \in H \normale D_n$. Anche $\iota^*$ appartiene ad $H$, quindi:
	
	\begin{gather}
		\rho \iota^* \rho^{-1} \iota^* \in H \\
		\rho (\iota^* \rho^{-1} \iota^*) \in H \\
		\rho \rho \in H \\
		\rho^2 \in H \\
		\gen{\rho^2} \le H
	\end{gather}

	Dobbiamo ora distinguere i due casi: $n$ dispari e $n$ pari.
	
	\textbf{Caso 1: $n$ dispari}
	
	Se $n$ è dispari allora
	
	\begin{equation}
		(n; 2) = 1
	\end{equation}

	quindi anche $\rho^2$ è generatore di $\gen{\rho}$:
	
	\begin{equation}
		\ordine{\rho^2} = \ordine{\rho} \quad\Longleftrightarrow\quad \gen{\rho^2} = \gen{\rho}
	\end{equation}
	 
	Quindi $H$ contiene $\rho$ e siccome contiene anche la riflessione $\iota^* = \rho^k \iota$ allora contiene anche $\iota$. Quindi $H = D_n$.

	Quindi l'unico sottogruppo normale che non ho ancora trovato e che non è contenuto in $\gen{\rho}$ è $D_n$. 

	\textbf{Caso 2: $n$ pari}

	Chiamo:
	
	\begin{equation}
		N = \gen{\rho^2}
	\end{equation}

	Usiamo il \textbf{Teorema di corrispondenza} (vedi~\ref{thr:Isomorfismi_corrispondenza}),
	il quale ci dice che esiste una corrispondenza tra i sottogruppi normali di $D_n$ che contengono $N$ e i sottogruppi normali di $\dfrac{D_n}{N}$ (figura~\ref{fig:Isometrie_diedrali_normali}).
	
	\begin{figure}[tp]
		\centering
		\tikz {
			\node (a) at (2,4) {$D_n = \gen{\rho, \iota}$};
			\node (b) at (0,2) {?};
			\node (c) at (2,2) {?};
			\node (d) at (4,2) {$\dots$};
			\node (e) at (2,0) {$N = \gen{\rho^2}$};
			\draw (a) edge[->] (b) 
			(a) edge[->] (c)
			(a) edge[->] (d) 
			(b) edge[->] (e)
			(c) edge[->] (e)
			(d) edge[->] (e);
		}
		$\qquad$
		\tikz {
			\node (a) at (2,4) {$D_n/N$};
			\node (b) at (0,2) {?};
			\node (c) at (2,2) {?};
			\node (d) at (4,2) {$\dots$};
			\node (e) at (2,0) {$1 = N/N$};
			\draw (a) edge[->] (b) 
			(a) edge[->] (c)
			(a) edge[->] (d) 
			(b) edge[->] (e)
			(c) edge[->] (e)
			(d) edge[->] (e);
		}
		\caption{Corrispondenze tra i sottogruppi normali dei gruppi diedrali (1).}
		\label{fig:Isometrie_diedrali_normali}
	\end{figure}

	Ora, dal momento che $D_n = \gen{\rho, \iota}$ abbiamo che:
	\footnote{
		Se $n$ è pari, il sottogruppo normale $\gen{\rho^2}$ partiziona il gruppo $D_n$ nei seguenti laterali:
		\begin{align*}
			1\gen{\rho^2} & = \{1, \rho^2, \rho^4, \dots, \rho^{n-2}\} \\
			& = 1\gen{\rho^2} = \rho^2\gen{\rho^2} = \dots = \rho^{n-2}\gen{\rho^2} \\
			& = \text{ insieme delle rotazioni pari} \\
			\rho\gen{\rho^2} & = \{\rho, \rho^3, \dots, \rho^{n-1}\} \\
			& = \rho\gen{\rho^2} = \rho^3\gen{\rho^2} = \dots = \rho^{n-1}\gen{\rho^2} \\
			& = \text{ insieme delle rotazioni dispari} \\
			\iota\gen{\rho^2} & = \{\iota, \rho^2\iota, \rho^4\iota, \dots, \rho^{n-2}\iota\} \\
			& = \iota\gen{\rho^2} = \rho^2\iota\gen{\rho^2} = \dots = \rho^n\iota\gen{\rho^2} \\
			& = \text{ insieme delle riflessioni pari} \\
			\rho\iota\gen{\rho^2} & = \{\rho\iota, \rho^3\iota, \rho^5\iota, \dots, \rho^{n-1}\iota\} \\
			& = \rho\iota\gen{\rho^2} = \rho^3\iota\gen{\rho^2} = \dots = \rho^{n-1}\iota\gen{\rho^2} \\
			& = \text{ insieme delle riflessioni dispari}
		\end{align*}
		Se $n$ è dispari, si hanno i seguenti laterali:
		\begin{align*}
			1\gen{\rho^2} & = \{\rho, \rho^2, \dots, \rho^n\} \\
			& = 1\gen{\rho^2} = \rho\gen{\rho^2} = \dots = \rho^{n-1}\gen{\rho^2} \\
			& = \text{ insieme delle rotazioni} \\
			\iota\gen{\rho^2} & = \{\iota, \rho\iota, \rho^2\iota, \dots, \rho^{n-1}\iota\} \\
			& = \iota\gen{\rho^2} = \rho\iota\gen{\rho^2} = \dots = \rho^{n-1}\iota\gen{\rho^2} \\
			& = \text{ insieme delle riflessioni} \\
		\end{align*}
	}
	
	\begin{gather}
		\dfrac{D_n}{N} = \gen{\rho N, \iota N} \\
		\ordine{\rho N} = 2 \\
		\ordine{\iota N} = 2
	\end{gather}

	Quindi $\dfrac{D_n}{N}$ è di ordine 4,
	\footnote{In \cite[p. 48]{garonzi} c'è un'altra interessante dimostrazione dell'ordine di $D_n/N$: \\
	\[\ordine{D_n : \gen{\rho^2}} = \ordine{D_n : \gen{\rho}} \cdot \ordine{\gen{\rho} : \gen{\rho^2}} = 4\]}
	abeliano \textbf{(TODO: perché?)}, non ciclico (infatti contiene 2 elementi di ordine 2), quindi:
	
	\begin{equation}
		\dfrac{D_n}{N} \isomorfo C_2 \times C_2
	\end{equation}
	
	$C_2 \times C_2$ ha 4 sottogruppi, tutti normali perché abeliano \textbf{(TODO: perché?)}
	
	Quindi $\dfrac{D_n}{N}$ ha 4 sottogruppi: l'identità di ordine 1 e 3 sottogruppi di ordine 2, che corrispondono a:
	\footnote{
		$\rho N$ è l'insieme delle rotazioni dispari, e la composizione di due rotazioni dispari fa una rotazione pari, ovvero un elemento del laterale $1 N$ che in $D_n/N$ si comporta da unità. 
		
		$\iota N$ è l'insieme delle riflessioni pari, e la composizione di due riflessioni pari è una rotazione pari, ovvero un elemento del laterale $1 N$ che in $D_n/N$ si comporta da unità. 
		
		$\rho\iota N$ è l'insieme delle riflessioni dispari, e la composizione di due riflessioni dispari è una rotazione pari, ovvero un elemento del laterale $1 N$ che in $D_n/N$ si comporta da unità.
	}
	
	\begin{equation}
		\gen{\rho N}, \gen{\iota N}, \gen{\rho\iota N}
	\end{equation}

	Tornando indietro con il teorema di corrispondenza trovo che i tre sottogruppi normali che contengono $\gen{\rho^2}$ sono (figura~\ref{fig:Isometrie_diedrali_normali_c2_c2}):
	
	\begin{equation}
		\gen{\rho^2, \rho} = \gen{\rho}, \gen{\rho^2, \iota}, \gen{\rho^2, \rho\iota}
	\end{equation} 
	
	\begin{sidewaysfigure}
		\centering
		\tikz {
			\node (a) at (2,4) {$D_n = \gen{\rho, \iota}$};
			\node (b) at (0,2) {$\gen{\rho}$};
			\node (c) at (2,2) {$\gen{\rho^2, \iota}$};
			\node (d) at (4,2) {$\gen{\rho^2, \rho\iota}$};
			\node (e) at (2,0) {$N = \gen{\rho^2}$};
			\draw (a) edge[->] (b) 
			(a) edge[->] (c)
			(a) edge[->] (d) 
			(b) edge[->] (e)
			(c) edge[->] (e)
			(d) edge[->] (e);
		}
		$\qquad$
		\tikz {
			\node (a) at (2,4) {$D_n/N$};
			\node (b) at (0,2) {$\gen{\rho N}$};
			\node (c) at (2,2) {$\gen{\iota N}$};
			\node (d) at (4,2) {$\gen{\rho\iota N}$};
			\node (e) at (2,0) {$1 = \gen{1N} = N/N$};
			\draw (a) edge[->] (b) 
			(a) edge[->] (c)
			(a) edge[->] (d) 
			(b) edge[->] (e)
			(c) edge[->] (e)
			(d) edge[->] (e);
		}
		$\qquad$
		\tikz {
			\node (a) at (2,4) {$C_2 \times C_2$};
			\node (b) at (0,2) {$\gen{a, 1}$};
			\node (c) at (2,2) {$\gen{1, b}$};
			\node (d) at (4,2) {$\gen{a, b}$};
			\node (e) at (2,0) {$1$};
			\draw (a) edge[->] (b) 
			(a) edge[->] (c)
			(a) edge[->] (d) 
			(b) edge[->] (e)
			(c) edge[->] (e)
			(d) edge[->] (e);
		}
		\caption{Corrispondenze tra i sottogruppi normali dei gruppi diedrali (2).}
		\label{fig:Isometrie_diedrali_normali_c2_c2}
	\end{sidewaysfigure}

	Quindi i sottogruppi normali di $D_n$ sono:
	
	\begin{enumerate}
		\item i sottogruppi di $\gen{\rho}$
		\item il gruppo $D_n$
		\item \emph{se $n$ è pari:} i sottogruppi $\gen{\rho}$, $\gen{\rho^2, \iota}$, $\gen{\rho^2, \rho\iota}$
	\end{enumerate}

	I gruppi $\gen{\rho^2, \iota}$ e $\gen{\rho^2, \rho\iota}$ sono gruppi diedrali di grado $\frac{n}{2}$ e corrispondono alle isometrie di poligoni con un numero dimezzato di lati.

\section[Sottogruppi di $D_6$]{Sottogruppi di $D_6$\footnote{\cite[8 novembre 2021]{lucchini}}}

\begin{esercizio}
	Elencare i sottogruppi di $D_6$
\end{esercizio}

Sottogruppo delle rotazioni:

\begin{equation}
	R = \gen{\rho} \isomorfo C_6
\end{equation}

Sottogruppi di $R$, in corrispondenza ai divisori di 6:

\begin{equation}
	1, \quad \gen{\rho^3} \isomorfo C_2, \quad \gen{\rho^2} \isomorfo C_2
\end{equation}

Cerchiamo poi i sottogruppi $H \le G$ con $H \not\subseteq R$.

Sappiamo che $R$ è normale, quindi $HR$ è un sottogruppo di $D_6$ (vedi~\ref{thr:prodotto_sottogruppi}).

$R$ è di indice 2 e $H$ non è contenuto in $R$, quindi $HR = D_6$.
\footnote{
Non trovando spiegazioni migliori, io la giustifico così.
$R$ è di indice 2. $H$ non è contenuto in $R$, quindi contiene almeno un elemento che non appartiene ad $R$. 
Questo elemento è presente anche in $HR$, quindi $HR$ ha cardinalità maggiore di $R$, ovvero ha indice minore (visto che cardinalità e indici sono inversamente proporzionali). L'unico indice minore di 2 è 1, quindi $HR$ coincide con $D_6$.
}

Quindi per il teorema dei due sottogruppi (vedi~\ref{thr:due_sottogruppi}) sappiamo che (figura~\ref{fig:Isometrie_due_sottogruppi}):

\begin{figure}[tp]
	\centering
	\tikz {
		\node (a) at (2,4) {$D_6 = HR$};
		\node (b) at (0,2) {$H$};
		\node (d) at (4,2) {$R$};
		\node (e) at (2,0) {$H \cap R$};
		\draw (a) edge[->] (b) 
		(a) edge[->] (d) 
		(b) edge[->] (e)
		(d) edge[->] (e);
	}
	\caption{Teorema dei due sottogruppi applicato a $D_6$.}
	\label{fig:Isometrie_due_sottogruppi}
\end{figure}


\begin{equation}
	\dfrac{H}{H \cap R} \isomorfo \dfrac{HR}{R} \isomorfo C_2 
\end{equation}

Quindi $H$ è generato da $H \cap R$ e da un altro elemento che sta fuori. Visto che $H \cap R$ contiene solo rotazioni, affinché $H$ non sia un sottogruppo delle rotazioni è necessario che l'elemento da aggiungere come generatore sia un'opportuna rotazione. Quindi per conoscere $H$ partiamo dalla varie possibilità di $H \cap R$
	
\[
	\begin{array}{cc}
		\toprule
		H \cap R & H \\
		\midrule
		\gen{\rho} \isomorfo C_6 & \gen{\rho, \iota} = D_6 \\
		\midrule
		\gen{\rho^2} \isomorfo C_3 & \gen{\rho^2, \iota} \isomorfo D_3 \isomorfo S_3 \\
			& \gen{\rho^2, \iota\rho} \isomorfo D_3 \isomorfo S_3 \\
		\midrule
		\gen{\rho^3} \isomorfo C_2 & \gen{\rho^3, \iota} \isomorfo C_2 \times C_2 \\
			& \gen{\rho^3, \iota\rho} \isomorfo C_2 \times C_2 \\
			& \gen{\rho^3, \iota\rho^2} \isomorfo C_2 \times C_2 \\
		\midrule
		1 & \gen{\iota} \isomorfo C_2 \\
			& \gen{\iota\rho} \isomorfo C_2 \\
			& \gen{\iota\rho^2} \isomorfo C_2 \\
			& \gen{\iota\rho^3} \isomorfo C_2 \\
			& \gen{\iota\rho^4} \isomorfo C_2 \\
			& \gen{\iota\rho^5} \isomorfo C_2 \\
		\bottomrule
	\end{array}
\]

Nota che $\iota\rho$ non è contenuto in $\gen{\rho^2, \iota}$.

Invece non devo aggiungere $\gen{\rho^2, \iota\rho^2}$ perché $\iota\rho^2 \in \gen{\rho^2, \iota}$.
