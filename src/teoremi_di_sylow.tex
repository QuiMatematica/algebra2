\chapter{Teoremi di Sylow}
\label{ch:sylow}

\begin{teorema}[Primo teorema di Sylow]
    \label{th:sylow_1}
    Sia $G$ un gruppo finito di ordine $n$.
    Sia $p$ un numero primo divisore di $n$, tale che $n = p^a \cdot m$ con $\MCD{m}{p} = 1$.
    Allora $G$ contiene un sottogruppo di ordine $p^k$.
\end{teorema}
\begin{dimostrazione}
    Dimostriamo il teorema per induzione.

    Se $\ordine{G} = 1$, il risultato è evidente.

    Consideriamo ora che il teorema sia vero per ogni gruppo di ordine $< \ordine{G}$.

    Consideriamo il sottogruppo $Z = Z(G)$ centro di $G$ e distinguiamo due situazioni.

    \bigskip
    \textbf{Caso 1: $p$ divide $\ordine{Z}$.}

    Per il teorema di Cauchy~\ref{thr:cauchy} esiste un elemento $z \in Z$ con $\ordine{z} = p$.

    Il sottogruppo $N = \gen{z}$ è normale perché sottogruppo del centro: $N \normale G$.

    Possiamo allora lavorare sul gruppo quoziente $G/N$ e abbiamo:
    \begin{equation*}
        \ordine{G/N} = \dfrac{\ordine{G}}{\ordine{N}} = \dfrac{p^a \cdot m}{p} = p^{a-1} \cdot m
    \end{equation*}

    Per induzione il gruppo $G/N$ contiene un sottogruppo $P/N$ di ordine $p^{a-1}$.
    Quindi:
    \begin{equation*}
        \ordine{P} = \ordine{P/N} \cdot \ordine{N} = p^{a-1} \cdot p = p^a
    \end{equation*}
    
    \bigskip
    \textbf{Caso 2: $p$ non divide $ordine{Z}$.}

    Consideriamo l'equazione delle classi~\ref{eq:equazione_delle_classi}:
    \begin{equation*}
        \ordine{G} = \ordine{Z} + \ordine{C_1} + \dots + \ordine{C_t}
    \end{equation*}

    $p$ divide l'ordine di $G$:
    \begin{equation*}
        \ordine{G} \equiv 0 \mod p
    \end{equation*}

    $p$ non divide l'ordine di $Z$:
    \begin{equation*}
        \ordine{Z} \not\equiv 0 \mod p
    \end{equation*}

    Quindi tra le classi di coniugio $C_i$ ne esiste almeno una che non è divisibile per $p$:
    \begin{equation*}
        \exists i \taleche \ordine{C_i} \not\equiv 0 \mod p
    \end{equation*}

    La classe di coniugio $C_i$ ha tanti elementi quanto l'indice del centralizzanti di un suo elemento $g_i \in C_i$:
    \begin{equation*}
        \ordine{C_i} = \indice{G}{C_G(g_i)}
    \end{equation*}

    $G$ ha ordine $p^a \cdot m$; l'indice di $C_G(g_i)$ è coprimo con $p$.
    Quindi $C_G(g_i)$ ha ordine $p^a \cdot m^*$,
    con $m^* < m$.

    Per induzione il gruppo $C_G(g_i)$ (che ha ordine minore del gruppo di $G$) contiene un sottogruppo $P$ di ordine
    $p^a$, allora $P$ è sottogruppo di $G$.
    Quindi $P$ è il sottogruppo di ordine $p^a$ che stiamo cercando.
\end{dimostrazione}

Sia $p^a$ la massima potenza di $p$ che divide $\ordine{G}$.
Il primo teorema di Sylow dimostra che esistono sottogruppi di $G$ di ordine $p^a$.
Questi sottogruppi sono detti \textbf{$p$-sottogruppi di Sylow} di $G$, o più semplicemente \textbf{$p$-Sylow}.
Il loro insieme si indica con:
\begin{equation*}
    \Syl_p(G)
\end{equation*}

Sia $p^b$ una qualunque potenza (anche non massima) di $p$ che divide $\ordine{G}$.
Un sottogruppo di ordine $p^b$ si dice \textbf{$p$-sottogruppo} di $G$.

\begin{lemma}
    Sia:
    \begin{enumerate}
        \item $P \in \Syl_p(G)$;
        \item $Q$ un $p$-sottogruppo di $G$ con $\ordine{Q} = p^b$;
        \item $Q \subseteq N_G(P)$.
    \end{enumerate}

    Allora: $Q \subseteq P$.
\end{lemma}
\begin{dimostrazione}
    Prendiamo un elemento $x \in Q$.
    Dal momento che $x$ appartiene anche al normalizzante $N_G(P)$ allora:
    \begin{equation*}
        xPx^{-1} = P \allora xP = Px \allora QP = PQ \allora QP \le G
    \end{equation*}

    Inoltre anche il sottogruppo $Q \cap P$ avrà come ordine una potenza di $p$: chiamiamola $p^c$ con $c \le b \le a$.

    Possiamo allora calcolare l'ordine di $QP$:
    \begin{equation*}
        \ordine{QP} = \dfrac{\ordine{Q} \cdot \ordine{P}}{\ordine{Q \cap P}} = \dfrac{p^b \cdot p^a}{p^c} = p^{a+(b-c)}
    \end{equation*}

    Per il teorema di Lagrange l'ordine di $QP$ deve dividere l'ordine di $G$:
    \begin{equation*}
        p^{a + (b-c)} \divisore p^a \cdot m \allora b = c \allora \ordine{QP} = \ordine{P}
    \end{equation*}

    Ma $P \subset QP$, quindi:
    \begin{equation*}
        P = QP \allora Q \subseteq P
    \end{equation*}

    Siccome $P$ è massimale nella sua proprietà di essere un $p$-sottogruppo, quando aggiungo $Q$ non ingrandisco $P$.
\end{dimostrazione}

\begin{teorema}[Secondo teorema di Sylow]
    Ogni due $p$-Sylow di $G$ sono coniugati in $G$:
    \begin{equation*}
        P_1, P_2 \in \Syl_p(G) \allora \exists a \in G \taleche P_1 = a P_2 a^{-a}
    \end{equation*}

    Il numero $n_p(G)$ dei $p$-Sylow di un gruppo $G$ è tale che:
    \begin{gather*}
        n_p(G) \text{ divide } \indice{G}{P} \quad \text{con } P \in \Syl_p(G) \\
        n_p(G) \equiv 1 \mod p
    \end{gather*}

    Ogni $p$-sottogruppo è contenuto in un $p$-Sylow:
    \begin{equation*}
        P \in \Syl_p(G) \land \ordine{Q} = p^b \allora \exists g \in G \taleche Q \subseteq gPg^{-1}
    \end{equation*}
\end{teorema}
\begin{dimostrazione}
    Chiamo $\Sigma = \Syl_p(G)$.

    Faccio agire $G$ per coniugio di sottogruppi su $\Sigma$.
    Siano $\Sigma_1$, \dots, $\Sigma_t$ le $G$-orbite.
    Abbiamo decomposto $\Sigma$ come unione disgiunta di orbite:
    \begin{equation*}
        \Sigma = \Sigma_1 \overset{\circ}{\cup} \dots \overset{\circ}{\cup} \Sigma_t
    \end{equation*}

    Mi concentro su $\Sigma_1$ e prendo un suo elemento $P \in \Sigma_1$.
    Dal momento che $\Sigma$ è l'insieme dei $p$-Sylow, $P$ è un $p$-Sylow di $G$.

    Faccio agire $P$ su $\Sigma_1$ per coniugio.
    $\Sigma_1$ si decompone nelle $P$-orbite:
    \begin{equation*}
        \Sigma_1 = \Omega_1 \overset{\circ}{\cup} \dots \overset{\circ}{\cup} \Omega_u
    \end{equation*}

    Ogni $P$-orbita ha cardinalità un divisore di $\ordine{P} = p^a$, perché è uguale all'indice $\indice{P}{N_P(Q)}$
    dove $Q$ è un qualunque elemento dell'orbita.
    Quindi le $P$-orbite hanno cardinalità 1 o una potenza di $p$.

    Se coniughiamo $P$ con un qualunque elemento di $P$, otteniamo ancora $P$.
    Quindi esiste una $P$-orbita che contiene un solo elemento, e questa $P$-orbita è $\{P\}$.

    L'orbita $\{P\}$ è l'unica di cardinalità 1.
    Se ne esistesse un'altra, detto $Q$ un elemento di tale orbita, avremmo che l'indice del normalizzante di $Q$
    sarebbe uguale ad 1, quindi $P = N_P(Q)$.
    In particolare $P \subseteq N_P(Q)$.
    $Q$ è un $p$-Sylow, quindi per il lemma (a lettere invertite) $P \subseteq Q$.
    Quindi abbiamo $Q \normale N_P(Q) = P$ e $P \subseteq Q$ quindi tutto annichilisce in un unico sottogruppo:
    $P = Q = N_P(Q)$.
    Ovvero ritroviamo l'orbita $\{P\}$.

    Quinti tutte le orbite hanno cardinalità potenza di $p$ diversa da 1, tranne una: quella che contiene $P$.
    Quindi:
    \begin{equation*}
        \ordine{\Sigma_1} \equiv 1 \mod p
    \end{equation*}

    Se $\Sigma_1$ coincidesse con $\Sigma$, avrei finito.

    Ipotizziamo per assurdo che $\Sigma_1 \ne \Sigma$.
    Allora prendo un $Q \in \Sigma - \Sigma_1$.
    Decompongo $\Sigma_1$ tramite le $Q$-orbite:
    \begin{equation*}
        \Sigma_1 = \Lambda_1 \overset{\circ}{\cup} \dots \overset{\circ}{\cup} \Lambda_v
    \end{equation*}

    Anche $Q$ è un $p$-Sylow, quindi le orbite hanno cardinalità una potenza di $p$.
    Per quanto visto sopra, anche tra queste orbite deve essercene una di cardinalità 1.
    Questa orbita contiene quindi un $p$-Sylow che chiamiamo $R$ e si ha che $\indice{Q}{N_Q(R)} = 1$,
    per cui $Q = N_Q(R)$, in particolare $Q \subseteq N_Q(R)$.
    Quindi per il lemma $Q \subseteq R$, ovvero $Q$ dovrebbe appartenere a $\Sigma_1$, il che contraddice l'ipotesi
    per assurdo.

    Abbiamo verificato, quindi, che $\Sigma = \Sigma_1$.

    L'azione di $G$ su $\Sigma = \Syl_p(G)$ genera quindi un'unica orbita, pertanto i $p$-Sylow sono sottogruppi
    coniugati.

    La cardinalità dell'orbita è uguale all'indice del normalizzante, il quale è un divisore dell'indice del sottogruppo:
    \begin{equation*}
        \ordine{\Syl_p(G)} \text{ divide } \indice{G}{P}
    \end{equation*}

    Inoltre la cardinalità di $\Sigma = \Syl_p(G)$ è uguale alla cardinalità di $\Sigma_1$, quindi:
    \begin{equation*}
        \ordine{\Syl_p(G)} \equiv 1 \mod p
    \end{equation*}

    Per dimostrare l'ultima parte, facciamo agire il $p$-sottogruppo $Q$ per coniugio di sottogruppi su $\Syl_p(G)$.

    Anche le $H$-orbite devono avere cardinalità una potenza di $p$.
    Ma $\ordine{Syl_p(G)} \equiv 1 \mod p$.
    Pertanto esiste un'unica orbita $O = \{R\}$ fatta da un unico elemento $R$.
    Quindi $\indice{Q}{N_Q(R)} = 1$, ovvero $Q \subseteq N_G(R)$, quindi per il lemma $Q \subseteq R$.
    $R$ è un $p$-Sylow, quindi $Q$ appartiene a un $p$-Sylow che è coniugato a qualunque altro $p$-Sylow.
\end{dimostrazione}

\begin{corollario}
    Se un $p$-Sylow è unico, è un sottogruppo normale.
\end{corollario}
\begin{dimostrazione}
    Se ho un unico $p$-Sylow, quando lo coniugo con un qualunque elemento di $G$, ottengo sempre lo stesso sottogruppo, quindi il
    sottogruppo è normale.
\end{dimostrazione}

\begin{esercizio}[{\cite[es. 2 pag. 82]{jacobson}}]
    Dimostra che non esistono gruppi semplici di ordine 148.
\end{esercizio}
\begin{soluzione}
    Il numero 148 si scompone in $2^2 \cdot 37$.

    Determiniamo quanti possono essere i 37-Sylow.
    Per il secondo teorema di Sylow il numero di 37-Sylow deve essere congruo ad 1 modulo 37 e deve dividere l'indice
    di $P$, che è 4.
    \begin{gather*}
        n_{37}(G) \equiv 1 \mod 37 \\
        n_{37}(G) \divisore 4
    \end{gather*}

    L'unico numero che rispetta entrambi le condizione è 1.
    Quindi avremo un unico 37-Sylow.
    Per il corollario al secondo teorema di Sylow, il sottogruppo è normale.
    E il gruppo $G$ non è semplice.
\end{soluzione}

\begin{esercizio}[{\cite[es. 2 pag. 82]{jacobson}}]
    Dimostra che non esistono gruppi semplici di ordine 56.
\end{esercizio}
\begin{soluzione}
    Il numero 56 si scompone in $7 \cdot 2^3$.

    Verifichiamo quanti possono essere i 7-Sylow:
    \begin{gather*}
        n_7(G) \equiv 1 \mod 7 \\
        n_7(G) \divisore 8
    \end{gather*}

    Quindi il numero di 7-Sylow può essere 1 o 8.

    Per il corollario al secondo teorema di Sylow, se abbiamo un solo 7-Sylow, tale sottogruppo è normale.
    Quindi il gruppo $G$ non è semplice.

    Se $G$ contiene 8 7-Sylow, ciascuno di questi sottogruppi ha ordine 7, quindi per i corollario~\ref{cor:gruppo_ordine_primo} sono ciclici.
    Inoltre, per il teorema~\ref{thr:generatori_gruppi_ciclici_finiti} questi sottogruppi contengono l'identità e
    6 elementi di ordine 7.

    Se interseco due di questi 7-Sylow ottengo un sottogruppo il cui ordine deve dividere l'ordine dei 7-Sylow.
    Quindi l'ordine dell'intersezione è 1: non può essere 7 altrimenti i due 7-Sylow sarebbero lo stesso.

    Quindi i 6 elementi di ordine 7 di ciascun 7-Sylow sono distinti, ovvero il gruppo $G$ contiene $8 \cdot 6 = 48$
    elementi di ordine 7.
    Quindi in $G$ ci sono $56-48 = 8$ elementi che non hanno ordine 7.

    I 2-Sylow del gruppo di ordine 56 deve avere 8 elementi di ordine potenza 2.
    Ma abbiamo appena visto che il gruppo contiene 8 elementi di ordine diverso da 7.
    Quindi questi 8 elementi appartengono al 2-Sylow.
    Abbiamo quindi un unico 2-Sylow, che, per il corollario al secondo teorema di Sylow, è normale.

    In sintesi: se non è normale il 7-Sylow, è normale il 2-Sylow.
\end{soluzione}

\begin{esercizio}[{\cite[es. 3 pag. 82]{jacobson}}]
    Dimostra che non esiste un gruppo semplice di ordine $pq$, con $p$ e $q$ primi.
\end{esercizio}
\begin{soluzione}
    Abbiamo due casi.

    \bigskip
    \textbf{Caso 1: $p = q$.}

    Per il teorema~\ref{thr:ordine_potenza_di_primo}, il gruppo $G$ contiene un centro che non è banale:
    \begin{equation*}
        1 \ne Z(G) \normale G
    \end{equation*}

    \emph{Secondo la registrazione del 15/10/2021, il centro ha ordine $p$, pertanto è un sottogruppo non banale normale.}
    \emph{Ma credo che il centro possa anche essere di ordine $p^2$, quando il gruppo $G$ è abeliano.}
    \emph{In questo caso come faccio a dimostrare che il sottogruppo di ordine $p$ è normale?}

    \emph{Nelle schede settimanali c'è un'altra spiegazione.}
    \emph{IL gruppo $G$ risulta essere abeliano, per cui ogni sottogruppo (in particolare quello di ordine $p$) è normale.}
    \emph{Ma in questo caso mi sfugge come mai $G$ sia necessariamente abeliano.}

    \bigskip
    \textbf{Caso 2: $p \ne q$.}

    Fissiamo l'ordine tra i due primi e poniamo $p > q$.
    Per il secondo teorema di Sylow abbiamo:
    \begin{gather*}
        n_p(G) \equiv 1 \mod p \\
        n_p(G) \divisore q
    \end{gather*}

    L'unico numero che rispetta le due condizioni è 1:
    \begin{equation*}
        n_p(G) = 1
    \end{equation*}

    Quindi, per il corollario al secondo teorema di Sylow, il $p$-Sylow è normale e il gruppo $G$ non è semplice.
\end{soluzione}

\begin{esercizio}
    Trovare tutti i sottogruppi di $S_4$.
\end{esercizio}
\begin{soluzione}
    Per il teorema di Lagrange~\ref{thr:Laterali_Lagrange}, i sottogruppi possono avere ordine 1, 2, 3, 4, 8, 12 e 24.

    \bigskip
    \textbf{Sottogruppi di ordine 1.}

    L'unico sottogruppo di ordine 1 è il gruppo identico 1.

    \bigskip
    \textbf{Sottogruppi di ordine 24.}

    L'unico sottogruppo di ordine 24 è il gruppo stesso $S_4$.

    \bigskip
    \textbf{Sottogruppi di ordine 12.}

    I sottogruppi di ordine 12 hanno indice 2, pertanto sono normali.
    Abbiamo già verificato nell'esercizio~\ref{ex:sottogruppi_normali_s4} che l'unico sottogruppo normale di ordine 12 è $A_4$.

    \bigskip
    \textbf{Sottogruppi di ordine 2.}

    I sottogruppi di ordine 2 sono ciclici.
    Ci basta contare gli elementi di $S_4$ di ordine 2.

    Abbiamo 6 elementi del tipo $(ab)$ e 3 elementi del tipo $(ab)(cd)$.
    Quindi in totale abbiamo 9 sottogruppi ciclici di ordine 2.

    \bigskip
    \textbf{Sottogruppi di ordine 3.}

    I sottogruppi di ordine 3 sono ciclici.
    Abbiamo 8 3-cicli, ma uno è inverso dell'altro, quindi abbiamo 4 copie.

    Abbiamo 4 sottogruppi ciclici di ordine 3.

    4 è compatibile con il secondo teorema di Sylow, perché:
    \begin{gather*}
        n_3(S_4) \equiv 1 \mod 3 \sse 4 \equiv 1 \mod 3 \\
        n_3(S_4) \divisore \ordine(S_4) \sse 4 \divisore 24
    \end{gather*}

    \bigskip
    \textbf{Sottogruppi di ordine 4.}

    I sottogruppi di ordine 4 sono abeliani.

    Conosciamo già il gruppo di Klein $V$.

    Poi abbiamo i sottogruppi ciclici generati da un 4-ciclo.
    Abbiamo $3! = 6$ 4-cicli;
    li accoppiamo per inversi.
    Abbiamo 3 sottogruppi ciclici di ordine 4 isomorfi a $C_4$.

    Ipotizziamo di avere altri sottogruppi di ordine 4: $\ordine{H} = 4$ con $H$ non ciclico e $H \ne V$.

    $H$ deve contenere elementi di ordine 2 che non siano doppi scambi: $(ij) \in H$ (\emph{perché?}).

    I sottogruppi di ordine 4 sono abeliani, quindi $H$ deve essere abeliano.
    Quindi gli elementi di $H$ devono commutare con $(ij)$.
    Quindi $H$ deve essere contenuto nel centralizzante di $(ij)$:
    \begin{equation*}
        H \le C_{S_4}((ij))
    \end{equation*}

    Quindi $H$ è il prodotto diretto di $(ij)$ con lo scambio complementare.
    Abbiamo allora 3 sottogruppi isomorfi a $C_2 \times C_2$:
    \begin{equation*}
        \gen{(12),(34)}, \quad \gen{(13),(24)}, \quad \gen{(14),(23)}
    \end{equation*}

    \bigskip
    \textbf{Sottogruppi di ordine 6.}

    Cerchiamo i sottogruppi $H$ di ordine 6.

    Per il primo teorema di Sylow esiste un 3-Sylow $P$ di $H$, ed ha indice 2.
    Quindi $P$ è normale n $H$.
    Quindi $N$ è contenuto nel normalizzante di $P$:
    \begin{equation*}
        H \subseteq N_{S_4}(P)
    \end{equation*}

    Determiniamo quanto è grande il normalizzante $N_{S_4}(P)$.

    $P$ è un 3-Sylow anche di $S_4$.
    Abbiamo già determinato che $S_4$ ha quattro 3-Sylow.
    \begin{equation*}
        \indice{S_4}{N_{S_4}(P)} = n_3(S_4) = 4
    \end{equation*}

    Quindi il normalizzante $N_{S_4}(P)$ ha ordine 6.

    Quindi $H$ coincide con il normalizzante.

    Allora per determinare i sottogruppi di ordine 6 parto dai sottogruppi di ordine 3 e che faccio i centralizzanti.

    Ottengo 4 sottogruppi di ordine 6: sono 4 sottogruppi isomorfi a $S_3$.

    \bigskip
    \textbf{Sottogruppi di ordine 8.}

    Sono i 2-Sylow.

    Ne conosciamo già uno: $D_4 \le S_4$.

    I sottogruppi di ordine 8 sono isomorfi a $D_4$.
    Il loro numero è dato da $\indice{S_4}{N_{S_4}(D_4)}$.
    L'indice $\indice{S_4}{D_4} = 3$ quindi o $D_4$ coincide con il normalizzante oppure $D_4$ è normale.

    Se è normale, vuol dire che è l'unico 2-Sylow quindi tutti gli elementi il cui è ordine è una potenza di 2 sono qui
    dentro.
    Ma questo è falso perché in $S_4$ ci sono più di 8 elementi di ordine potenza di 2.

    Quindi abbiamo 3 sottogruppi di ordine 8.
    Essi sono tanti quanti i gruppi ciclici di ordine 4, ognuno del quale produce un diedrale.
\end{soluzione}

Possiamo quindi verificare che $A_4$ non ha sottogruppi di ordine 6.
Se ne avesse, questi sarebbero sottogruppi anche di $S_4$, quindi sarebbero i sottogruppi isomorfi ad $S_3$.
Ma questi sottogruppi contengono delle permutazioni dispari, quindi non possono stare in $A_4$.