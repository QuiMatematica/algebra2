\chapter{Teoremi di Sylow}
\label{ch:sylow}

\begin{teorema}[Primo teorema di Sylow]
    \label{th:sylow_1}
    Sia $G$ un gruppo finito di ordine $n$.
    Sia $p$ un numero primo divisore di $n$, tale che $n = p^a \cdot m$ con $\MCD{m}{p} = 1$.
    Allora $G$ contiene un sottogruppo di ordine $p^k$.
\end{teorema}
\begin{dimostrazione}
    Dimostriamo il teorema per induzione.

    Se $\ordine{G} = 1$, il risultato è evidente.

    Consideriamo ora che il teorema sia vero per ogni gruppo di ordine $< \ordine{G}$.

    Consideriamo il sottogruppo $Z = Z(G)$ centro di $G$ e distinguiamo due situazioni.

    \bigskip
    \textbf{Caso 1: $p$ divide $\ordine{Z}$.}

    Per il teorema di Cauchy~\ref{thr:cauchy} esiste un elemento $z \in Z$ con $\ordine{z} = p$.

    Il sottogruppo $N = \gen{z}$ è normale perché sottogruppo del centro: $N \normale G$.

    Possiamo allora lavorare sul gruppo quoziente $G/N$ e abbiamo:
    \begin{equation*}
        \ordine{G/N} = \dfrac{\ordine{G}}{\ordine{N}} = \dfrac{p^a \cdot m}{p} = p^{a-1} \cdot m
    \end{equation*}

    Per induzione il gruppo $G/N$ contiene un sottogruppo $P/N$ di ordine $p^{a-1}$.
    Quindi:
    \begin{equation*}
        \ordine{P} = \ordine{P/N} \cdot \ordine{N} = p^{a-1} \cdot p = p^a
    \end{equation*}
    
    \bigskip
    \textbf{Caso 2: $p$ non divide $ordine{Z}$.}

    Consideriamo l'equazione delle classi~\ref{eq:equazione_delle_classi}:
    \begin{equation*}
        \ordine{G} = \ordine{Z} + \ordine{C_1} + \dots + \ordine{C_t}
    \end{equation*}

    $p$ divide l'ordine di $G$:
    \begin{equation*}
        \ordine{G} \equiv 0 \mod p
    \end{equation*}

    $p$ non divide l'ordine di $Z$:
    \begin{equation*}
        \ordine{Z} \not\equiv 0 \mod p
    \end{equation*}

    Quindi tra le classi di coniugio $C_i$ ne esiste almeno una che non è divisibile per $p$:
    \begin{equation*}
        \exists i \taleche \ordine{C_i} \not\equiv 0 \mod p
    \end{equation*}

    La classe di coniugio $C_i$ ha tanti elementi quanto l'indice del centralizzanti di un suo elemento $g_i \in C_i$:
    \begin{equation*}
        \ordine{C_i} = \indice{G}{C_G(g_i)}
    \end{equation*}

    $G$ ha ordine $p^a \cdot m$; l'indice di $C_G(g_i)$ è coprimo con $p$. Quindi $C_G(g_i)$ ha ordine $p^a \cdot m^*$,
    con $m^* < m$.

    Per induzione il gruppo $C_G(g_i)$ (che ha ordine minore del gruppo di $G$) contiene un sottogruppo $P$ di ordine
    $p^a$, allora $P$ è sottogruppo di $G$.
    Quindi $P$ è il sottogruppo di ordine $p^a$ che stiamo cercando.
\end{dimostrazione}

Sia $p^a$ la massima potenza di $p$ che divide $\ordine{G}$.
Il primo teorema di Sylow dimostra che esistono sottogruppi di $G$ di ordine $p^a$.
Questi sottogruppi sono detti \textbf{$p$-sottogruppi di Sylow} di $G$, o più semplicemente \textbf{$p$-Sylow}.
Il loro insieme si indica con:
\begin{equation*}
    \Syl_p(G)
\end{equation*}

Sia $p^b$ una qualunque potenza (anche non massima) di $p$ che divide $\ordine{G}$.
Un sottogruppo di ordine $p^b$ si dice \textbf{$p$-sottogruppo} di $G$.

\begin{lemma}
    Sia:
    \begin{enumerate}
        \item $P \in \Syl_p(G)$;
        \item $Q$ un $p$-sottogruppo di $G$ con $\ordine{Q} = p^b$;
        \item $Q \subseteq N_G(P)$.
    \end{enumerate}

    Allora: $Q \subseteq P$.
\end{lemma}
\begin{dimostrazione}
    Prendiamo un elemento $x \in Q$.
    Dal momento che $x$ appartiene anche al normalizzante $N_G(P)$ allora:
    \begin{equation*}
        xPx^{-1} = P \allora xP = Px \allora QP = PQ \allora QP \le G
    \end{equation*}

    Inoltre anche il sottogruppo $Q \cap P$ avrà come ordine una potenza di $p$: chiamiamola $p^c$ con $c \le b \le a$.

    Possiamo allora calcolare l'ordine di $QP$:
    \begin{equation*}
        \ordine{QP} = \dfrac{\ordine{Q} \cdot \ordine{P}}{\ordine{Q \cap P}} = \dfrac{p^b \cdot p^a}{p^c} = p^{a+(b-c)}
    \end{equation*}

    Per il teorema di Lagrange l'ordine di $QP$ deve dividere l'ordine di $G$:
    \begin{equation*}
        p^{a + (b-c)} \divisore p^a \cdot m \allora b = c \allora \ordine{QP} = \ordine{P}
    \end{equation*}

    Ma $P \subset QP$, quindi:
    \begin{equation*}
        P = QP \allora Q \subseteq P
    \end{equation*}

    Siccome $P$ è massimale nella sua proprietà di essere un $p$-sottogruppo, quando aggiungo $Q$ non ingrandisco $P$.
\end{dimostrazione}

\begin{teorema}[Secondo teorema di Sylow]
    Ogni due $p$-Sylow di $G$ sono coniugati in $G$:
    \begin{equation*}
        P_1, P_2 \in \Syl_p(G) \allora \exists a \in G \taleche P_1 = a P_2 a^{-a}
    \end{equation*}

    Il numero $n_p(G)$ dei $p$-Sylow di un gruppo $G$ è tale che:
    \begin{gather*}
        n_p(G) \text{ divide } \indice{G}{P} \quad \text{con } P \in \Syl_p(G) \\
        n_p(G) \equiv 1 \mod p
    \end{gather*}

    Ogni $p$-sottogruppo è contenuto in un $p$-Sylow:
    \begin{equation*}
        P \in \Syl_p(G) \land \ordine{Q} = p^b \allora \exists g \in G \taleche Q \subseteq gPg^{-1}
    \end{equation*}
\end{teorema}
\begin{dimostrazione}
    Chiamo $\Sigma = \Syl_p(G)$.

    Faccio agire $G$ per coniugio su $Sigma$.
    Siano $\Sigma_1$, \dots, $\Sigma_t$ le $G$-orbite.
    Abbiamo smembrato $\Sigma$ come unione disgiunta di orbite:
    \begin{equation*}
        \Sigma = \Sigma_1 \overset{\circ}{\cup} \dots \overset{\circ}{\cup} \Sigma_t
    \end{equation*}

    Mi concentro su $\Sigma_1$;
    prendo un suo elemento $P \in \Sigma_1$.
    Faccio agire $P$ su $\Sigma_1$ per coniugio.
    Non è detto che $\Sigma_1$ sia l'orbita per $P$, quindi $\Sigma_1$ può smembrarsi in più $P$-orbite:
    \begin{equation*}
        \Sigma_1 = \Omega_1 \overset{\circ}{\cup} \dots \overset{\circ}{\cup} \Omega_u
    \end{equation*}

    Calcoliamo il numero di elementi in $\Omega_1$;
    sia $Q_i \in \Omega_i$:
    \begin{equation*}
        \ordine{\Omega_i} = \indice{P}{P \cap N_G(Q_i)} = \indice{P}{N_P(Q_i)}
    \end{equation*}

    $P$ è un $p$-sottogruppo, quindi:
    \begin{equation*}
        \indice{P}{N_P(Q_i)} = p^{b_i}
    \end{equation*}
\end{dimostrazione}